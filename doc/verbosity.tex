\subsection{verbosity}
\label{labverbosity}
\noindent Name: \textbf{verbosity}\\
global variable controlling the quantity of information displayed by commands.\\

\noindent Description: \begin{itemize}

\item \textbf{verbosity} accepts any integer value. At level 0, commands do not display anything
   on standard out. Note that very critical information may however be displayed on
   standard err.

\item Default level is 1. It displays important informations such as warnings when 
   roundings happen.

\item For higher levels more informations are displayed depending on the command.
\end{itemize}
\noindent Example 1: 
\begin{center}\begin{minipage}{15cm}\begin{Verbatim}[frame=single]
> verbosity=0!;
> 1.2+"toto";
error
> verbosity=1!;
> 1.2+"toto";
Warning: at least one of the given expressions or a subexpression is not correct
ly typed
or its evaluation has failed because of some error on a side-effect.
error
> verbosity=2!;
> 1.2+"toto";
Warning: at least one of the given expressions or a subexpression is not correct
ly typed
or its evaluation has failed because of some error on a side-effect.
Information: the expression or a partial evaluation of it has been the following
:
(0.120000000000000000000000000000000000000000000000003e1) + ("toto")
error
\end{Verbatim}
\end{minipage}\end{center}
See also: \textbf{suppressroundingwarnings} (\ref{labsuppresswarnings})
