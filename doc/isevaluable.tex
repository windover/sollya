\subsection{isevaluable}
\label{labisevaluable}
\noindent Name: \textbf{isevaluable}\\
tests whether a function can be evaluated at a point \\
\noindent Usage: 
\begin{center}
\textbf{isevaluable}(\emph{function}, \emph{constant}) : (\textsf{function}, \textsf{constant}) $\rightarrow$ \textsf{boolean}\\
\end{center}
Parameters: 
\begin{itemize}
\item \emph{function} represents a function
\item \emph{constant} represents a constant point
\end{itemize}
\noindent Description: \begin{itemize}

\item \\textbf{isevaluable} applied to function \\emph{function} and a constant \\emph{constant} returns\n   a boolean indicating whether or not a subsequent call to \\textbf{evaluate} on the\n   same function \\emph{function} and constant \\emph{constant} will produce a numerical\n   result or NaN. This means \\textbf{isevaluable} returns false iff \\textbf{evaluate} will return NaN.\n\end{itemize}
\noindent Example 1: 
\begin{center}\begin{minipage}{15cm}\begin{Verbatim}[frame=single]
\end{Verbatim}
\end{minipage}\end{center}
\noindent Example 2: 
\begin{center}\begin{minipage}{15cm}\begin{Verbatim}[frame=single]
\end{Verbatim}
\end{minipage}\end{center}
\noindent Example 3: 
\begin{center}\begin{minipage}{15cm}\begin{Verbatim}[frame=single]
\end{Verbatim}
\end{minipage}\end{center}
See also: \textbf{evaluate} (\ref{labevaluate})
