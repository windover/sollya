\subsection{ isevaluable }
\noindent Name: \textbf{isevaluable}\\
tests whether a function can be evaluated at a point \\

\noindent Usage: 
\begin{center}
\textbf{isevaluable}(\emph{function}, \emph{constant}) : (\textsf{function}, \textsf{constant}) $\rightarrow$ \textsf{boolean}\\
\end{center}
Parameters: 
\begin{itemize}
\item \emph{function} represents a function
\item \emph{constant} represents a constant point
\end{itemize}
\noindent Description: \begin{itemize}

\item \textbf{isevaluable} applied to function \emph{function} and a constant \emph{constant} returns
   a boolean indicating whether or not a subsequent call to \textbf{evaluate} on the
   same function \emph{function} and constant \emph{constant} will produce a numerical
   result or NaN. I.e. \textbf{isevaluable} returns false iff \textbf{evaluate} will return NaN.
\end{itemize}
\noindent Example 1: 
\begin{center}\begin{minipage}{15cm}\begin{Verbatim}[frame=single]
> isevaluable(sin(pi * 1/x), 0.75);
true
> print(evaluate(sin(pi * 1/x), 0.75));
-0.866025403784438646763723170752936183471402626905185165
\end{Verbatim}
\end{minipage}\end{center}
\noindent Example 2: 
\begin{center}\begin{minipage}{15cm}\begin{Verbatim}[frame=single]
> isevaluable(sin(pi * 1/x), 0.5);
true
> print(evaluate(sin(pi * 1/x), 0.5));
[-0.172986452514381269516508615031098129542836767991679e-12714;0.759411982011879
631450695643145256617060390843900679e-12715]
\end{Verbatim}
\end{minipage}\end{center}
\noindent Example 3: 
\begin{center}\begin{minipage}{15cm}\begin{Verbatim}[frame=single]
> isevaluable(sin(pi * 1/x), 0);
false
> print(evaluate(sin(pi * 1/x), 0));
@NaN@
\end{Verbatim}
\end{minipage}\end{center}
See also: \textbf{evaluate}
