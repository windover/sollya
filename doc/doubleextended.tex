\subsection{ doubleextended }
\noindent Names: \textbf{doubleextended}, \textbf{DE}\\
computes the nearest number with 64 bits of mantissa.\\

\noindent Description: \begin{itemize}

\item \textbf{doubleextended} is a function that computes the nearest floating-point number with
   64 bits of mantissa to a given number. Since it is a function, it can be
   composed with other functions of Sollya such as \textbf{exp}, \textbf{sin}, etc.

\item It does not handle subnormal numbers. The range of possible exponents is the 
   range used for all numbers represented in Sollya (e.g. basically the range 
   used in the library MPFR).

\item Since it is a function and not a command, its behavior is a bit different from 
   the behavior of \textbf{round}(x,64,RN) even if the result is exactly the same.
   \textbf{round}(x,64,RN) is immediately evaluated whereas \textbf{doubleextended}(x) can be composed 
   with other functions (and thus be plotted and so on).

\item Be aware that \textbf{doubleextended} cannot be used as a constant to represent a format in the
   commands \textbf{roundcoefficients} and \textbf{implementpoly} (contrary to \textbf{D}, \textbf{DD},and \textbf{TD}).
\end{itemize}
\noindent Example 1: 
\begin{center}\begin{minipage}{14.8cm}\begin{Verbatim}[frame=single]
   > display=binary!;
   > DE(0.1);
   1.100110011001100110011001100110011001100110011001100110011001101_2 * 2^(-4)
   > round(0.1,64,RN);
   1.100110011001100110011001100110011001100110011001100110011001101_2 * 2^(-4)
\end{Verbatim}
\end{minipage}\end{center}
\noindent Example 2: 
\begin{center}\begin{minipage}{14.8cm}\begin{Verbatim}[frame=single]
   > D(2^(-2000));
   0
   > DE(2^(-2000));
   0.870980981621721667557619549477887229585910374270538862e-602
\end{Verbatim}
\end{minipage}\end{center}
\noindent Example 3: 
\begin{center}\begin{minipage}{14.8cm}\begin{Verbatim}[frame=single]
   > verbosity=1!;
   > f = sin(DE(x));
   > f(pi);
   Warning: rounding has happened. The value displayed is a faithful rounding of the true result.
   -0.501655761266833202355732708033075701383156167025495e-19
   > g = sin(round(x,64,RN));
   Warning: at least one of the given expressions or a subexpression is not correctly typed
   or its evaluation has failed because of some error on a side-effect.
\end{Verbatim}
\end{minipage}\end{center}
See also: \textbf{double}, \textbf{doubledouble}, \textbf{tripledouble}, \textbf{round}
