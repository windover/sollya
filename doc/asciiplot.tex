\subsection{asciiplot}
\label{labasciiplot}
\noindent Name: \textbf{asciiplot}\\
plots a function in a range using ASCII characters\\
\noindent Usage: 
\begin{center}
\textbf{asciiplot}(\emph{function}, \emph{range}) : (\textsf{function}, \textsf{range}) $\rightarrow$ \textsf{void}
\\ 
\end{center}
Parameters: 
\begin{itemize}
\item \emph{function} represents a function to be plotted
\item \emph{range} represents a range the function is to be plotted in 
\end{itemize}
\noindent Description: \begin{itemize}

\item \textbf{asciiplot} plots the function \emph{function} in range \emph{range} using ASCII
   characters.  On systems that provide the necessary 
   \texttt{TIOCGWINSZ ioctl}, \sollya determines the size of the
   terminal for the plot size if connected to a terminal. If it is not
   connected to a terminal or if the test is not possible, the plot is of
   fixed size $77\times25$ characters.  The function is
   evaluated on a number of points equal to the number of columns
   available. Its value is rounded to the next integer in the range of
   lines available. A letter \texttt{x} is written at this place. If zero is in
   the hull of the image domain of the function, an x-axis is
   displayed. If zero is in range, a y-axis is displayed.  If the
   function is constant or if the range is reduced to one point, the
   function is evaluated to a constant and the constant is displayed
   instead of a plot.
\end{itemize}
\noindent Example 1: 
\begin{center}\begin{minipage}{15cm}\begin{Verbatim}[frame=single]
> asciiplot(exp(x),[1;2]);
                                                                           x
                                                                         xx 
                                                                      xxx   
                                                                    xx      
                                                                  xx        
                                                               xxx          
                                                             xx             
                                                          xxx               
                                                        xx                  
                                                     xxx                    
                                                  xxx                       
                                               xxx                          
                                            xxx                             
                                         xxx                                
                                      xxx                                   
                                  xxxx                                      
                              xxxx                                          
                           xxx                                              
                       xxxx                                                 
                  xxxxx                                                     
             xxxxx                                                          
         xxxx                                                               
    xxxxx                                                                   
xxxx                                                                        
\end{Verbatim}
\end{minipage}\end{center}
\noindent Example 2: 
\begin{center}\begin{minipage}{15cm}\begin{Verbatim}[frame=single]
> asciiplot(expm1(x),[-1;2]);
                         |                                                 x
                         |                                                x 
                         |                                               x  
                         |                                              x   
                         |                                             x    
                         |                                           xx     
                         |                                          x       
                         |                                         x        
                         |                                       xx         
                         |                                     xx           
                         |                                   xx             
                         |                                 xx               
                         |                                x                 
                         |                             xxx                  
                         |                           xx                     
                         |                        xxx                       
                         |                     xxx                          
                         |                 xxxx                             
                         |             xxxx                                 
                         |         xxxx                                     
                         |   xxxxxx                                         
---------------------xxxxxxxx-----------------------------------------------
         xxxxxxxxxxxx    |                                                  
xxxxxxxxx                |                                                  
\end{Verbatim}
\end{minipage}\end{center}
\noindent Example 3: 
\begin{center}\begin{minipage}{15cm}\begin{Verbatim}[frame=single]
> asciiplot(5,[-1;1]);
5
\end{Verbatim}
\end{minipage}\end{center}
\noindent Example 4: 
\begin{center}\begin{minipage}{15cm}\begin{Verbatim}[frame=single]
> asciiplot(exp(x),[1;1]);
2.71828182845904523536028747135266249775724709369998
\end{Verbatim}
\end{minipage}\end{center}
See also: \textbf{plot} (\ref{labplot})
