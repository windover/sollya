\subsection{ neq }
\noindent Name: \textbf{!$=$}\\
negated equality test operator\\

\noindent Usage: 
\begin{center}
\emph{expr1} \textbf{!$=$} \emph{expr2} : (\textsf{any type}, \textsf{any type}) $\rightarrow$ \textsf{boolean}\\
\end{center}
Parameters: 
\begin{itemize}
\item \emph{expr1} and \emph{expr2} represent expressions
\end{itemize}
\noindent Description: \begin{itemize}

\item The operator \textbf{!$=$} evaluates to true iff its operands \emph{expr1} and
   \emph{expr2} are syntactically unequal and both different from \textbf{error} or
   constant expressions that evaluate to two different floating-point
   number with the global precision \textbf{prec}. The user should be aware of
   the fact that because of floating-point evaluation, the operator
   \textbf{!$=$} is not exactly the same as the negation of the mathematical
   equality. 
   Note that the expressions \textbf{!}(\emph{expr1} \textbf{!$=$} \emph{expr2}) and \emph{expr1} \textbf{$==$}
   \emph{expr2} do not evaluate to the same boolean value. See \textbf{error} for
   details.
\end{itemize}
\noindent Example 1: 
\begin{center}\begin{minipage}{15cm}\begin{Verbatim}[frame=single]
> "Hello" != "Hello";
false
> "Hello" != "Salut";
true
> "Hello" != 5;
true
> 5 + x != 5 + x;
false
\end{Verbatim}
\end{minipage}\end{center}
\noindent Example 2: 
\begin{center}\begin{minipage}{15cm}\begin{Verbatim}[frame=single]
> 1 != exp(0);
false
> asin(1) * 2 != pi;
false
> exp(5) != log(4);
true
\end{Verbatim}
\end{minipage}\end{center}
\noindent Example 3: 
\begin{center}\begin{minipage}{15cm}\begin{Verbatim}[frame=single]
> prec = 12;
The precision has been set to 12 bits.
> 16384 != 16385;
false
\end{Verbatim}
\end{minipage}\end{center}
\noindent Example 4: 
\begin{center}\begin{minipage}{15cm}\begin{Verbatim}[frame=single]
> error != error;
false
\end{Verbatim}
\end{minipage}\end{center}
See also: \textbf{$==$}, \textbf{$>$}, \textbf{$>=$}, \textbf{$<=$}, \textbf{$<$}, \textbf{!}, \textbf{$\&\&$}, \textbf{$||$}, \textbf{error}, \textbf{prec}
