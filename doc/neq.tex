\subsection{!$=$}
\label{labneq}
\noindent Name: \textbf{!$=$}\\
negated equality test operator\\
\noindent Usage: 
\begin{center}
\emph{expr1} \textbf{!$=$} \emph{expr2} : (\textsf{any type}, \textsf{any type}) $\rightarrow$ \textsf{boolean}\\
\end{center}
Parameters: 
\begin{itemize}
\item \emph{expr1} and \emph{expr2} represent expressions
\end{itemize}
\noindent Description: \begin{itemize}

\item The operator \\textbf{!$=$} evaluates to true iff its operands \\emph{expr1} and\n   \\emph{expr2} are syntactically unequal and both different from \\textbf{error} or\n   constant expressions that are not constants and that evaluate to two\n   different floating-point number with the global precision \\textbf{prec}. The\n   user should be aware of the fact that because of floating-point\n   evaluation, the operator \\textbf{!$=$} is not exactly the same as the\n   negation of the mathematical equality.\n     \n   Note that the expressions \\textbf{!}(\\emph{expr1} \\textbf{!$=$} \\emph{expr2}) and \\emph{expr1}\n   \\textbf{$==$} \\emph{expr2} do not evaluate to the same boolean value. See \\textbf{error}\n   for details.\n\end{itemize}
\noindent Example 1: 
\begin{center}\begin{minipage}{15cm}\begin{Verbatim}[frame=single]
\end{Verbatim}
\end{minipage}\end{center}
\noindent Example 2: 
\begin{center}\begin{minipage}{15cm}\begin{Verbatim}[frame=single]
\end{Verbatim}
\end{minipage}\end{center}
\noindent Example 3: 
\begin{center}\begin{minipage}{15cm}\begin{Verbatim}[frame=single]
\end{Verbatim}
\end{minipage}\end{center}
\noindent Example 4: 
\begin{center}\begin{minipage}{15cm}\begin{Verbatim}[frame=single]
\end{Verbatim}
\end{minipage}\end{center}
\noindent Example 5: 
\begin{center}\begin{minipage}{15cm}\begin{Verbatim}[frame=single]
\end{Verbatim}
\end{minipage}\end{center}
See also: \textbf{$==$} (\ref{labequal}), \textbf{$>$} (\ref{labgt}), \textbf{$>=$} (\ref{labge}), \textbf{$<=$} (\ref{lable}), \textbf{$<$} (\ref{lablt}), \textbf{!} (\ref{labnot}), \textbf{$\&\&$} (\ref{laband}), \textbf{$||$} (\ref{labor}), \textbf{error} (\ref{laberror}), \textbf{prec} (\ref{labprec})
