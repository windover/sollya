\subsection{externalproc}
\label{labexternalproc}
\noindent Name: \textbf{externalproc}\\
binds an external code to a \sollya procedure\\
\noindent Usage: 
\begin{center}
\textbf{externalproc}(\emph{identifier}, \emph{filename}, \emph{argumenttype} $->$ \emph{resulttype}) : (\textsf{identifier type}, \textsf{string}, \textsf{type type}, \textsf{type type}) $\rightarrow$ \textsf{void}\\
\end{center}
Parameters: 
\begin{itemize}
\item \emph{identifier} represents the identifier the code is to be bound to
\item \emph{filename} of type \textsf{string} represents the name of the object file where the code of procedure can be found
\item \emph{argumenttype} represents a definition of the types of the arguments of the \sollya procedure and the external code
\item \emph{resulttype} represents a definition of the result type of the external code
\end{itemize}
\noindent Description: \begin{itemize}

\item \\textbf{externalproc} allows for binding the \\sollya identifier\n   \\emph{identifier} to an external code.  After this binding, when \\sollya\n   encounters \\emph{identifier} applied to a list of actual parameters, it\n   will evaluate these parameters and call the external code with these\n   parameters. If the external code indicated success, it will receive\n   the result produced by the external code, transform it to \\sollya's\n   internal representation and return it.\n    \n   In order to allow correct evaluation and typing of the data in\n   parameter and in result to be passed to and received from the external\n   code, \\textbf{externalproc} has a third parameter \\emph{argumenttype} $->$ \\emph{resulttype}.\n   Both \\emph{argumenttype} and \\emph{resulttype} are one of \\textbf{void}, \\textbf{constant},\n   \\textbf{function}, \\textbf{range}, \\textbf{integer}, \\textbf{string}, \\textbf{boolean}, \\textbf{list of} \\textbf{constant}, \\textbf{list of} \\textbf{function}, \n   \\textbf{list of} \\textbf{range}, \\textbf{list of} \\textbf{integer}, \\textbf{list of} \\textbf{string}, \\textbf{list of} \\textbf{boolean}.\n    \n   If upon a usage of a procedure bound to an external procedure the type\n   of the actual parameters given or its number is not correct, \\sollya\n   produces a type error. An external function not applied to arguments\n   represents itself and prints out with its argument and result types.\n    \n   The external function is supposed to return an integer indicating\n   success.  It returns its result depending on its \\sollya result type\n   as follows. Here, the external procedure is assumed to be implemented\n   as a C function.\n    \n   If the \\sollya result type is void, the C function has no pointer\n   argument for the result.  If the \\sollya result type is \\textbf{constant}, the\n   first argument of the C function is of C type \\texttt{mpfr\\_t *}, the result is\n   returned by affecting the MPFR variable.  If the \\sollya result type\n   is \\textbf{function}, the first argument of the C function is of C type \\texttt{node **},\n   the result is returned by the \\texttt{node *} pointed with a new \\texttt{node *}.\n   If the \\sollya result type is \\textbf{range}, the first argument of the C\n   function is of C type \\texttt{mpfi\\_t *}, the result is returned by affecting\n   the MPFI variable.  If the \\sollya result type is \\textbf{integer}, the first\n   argument of the C function is of C type \\texttt{int *}, the result is returned\n   by affecting the int variable.  If the \\sollya result type is \\textbf{string},\n   the first argument of the C function is of C type \\texttt{char **}, the result\n   is returned by the \\texttt{char *} pointed with a new \\texttt{char *}.  If the \\sollya\n   result type is \\textbf{boolean}, the first argument of the C function is of C\n   type \\texttt{int *}, the result is returned by affecting the int variable with\n   a boolean value.  If the \\sollya result type is \\textbf{list of} type, the\n   first argument of the C function is of C type \\texttt{chain **}, the result is\n   returned by the \\texttt{chain *} pointed with a new \\texttt{chain *}.  This chain\n   contains for \\sollya type \\textbf{constant} pointers \\texttt{mpfr\\_t *} to new MPFR\n   variables, for \\sollya type \\textbf{function} pointers \\texttt{node *} to new nodes, for\n   \\sollya type \\textbf{range} pointers \\texttt{mpfi\\_t *}  to new MPFI variables, for\n   \\sollya type \\textbf{integer} pointers \\texttt{int *} to new int variables for \\sollya\n   type \\textbf{string} pointers \\texttt{char *} to new \\texttt{char *} variables and for \\sollya\n   type \\textbf{boolean} pointers \\texttt{int *} to new int variables representing boolean\n   values.\n    	       \n   The external procedure affects its possible pointer argument if and\n   only if it succeeds.  This means, if the function returns an integer\n   indicating failure, it does not leak any memory to the encompassing\n   environment.\n    \n   The external procedure receives its arguments as follows: If the\n   \\sollya argument type is \\textbf{void}, no argument array is given.  Otherwise\n   the C function receives a C \\texttt{void **} argument representing an array of\n   size equal to the arity of the function where each entry (of C type\n   \\texttt{void *}) represents a value with a C type depending on the\n   corresponding \\sollya type. If the \\sollya type is \\textbf{constant}, the C\n   type the \\texttt{void *} is to be casted to is \\texttt{mpfr\\_t *}.  If the \\sollya type\n   is \\textbf{function}, the C type the \\texttt{void *} is to be casted to is \\texttt{node *}.  If\n   the \\sollya type is \\textbf{range}, the C type the \\texttt{void *} is to be casted to is\n   \\texttt{mpfi\\_t *}.  If the \\sollya type is \\textbf{integer}, the C type the \\texttt{void *} is to\n   be casted to is \\texttt{int *}.  If the \\sollya type is \\textbf{string}, the C type the\n   \\texttt{void *} is to be casted to is \\texttt{char *}.  If the \\sollya type is \\textbf{boolean},\n   the C type the \\texttt{void *} is to be casted to is \\texttt{int *}.  If the \\sollya\n   type is \\textbf{list of} type, the C type the \\texttt{void *} is to be casted to is\n   \\texttt{chain *}.  Here depending on type, the values in the chain are to be\n   casted to \\texttt{mpfr\\_t *}  for \\sollya type \\textbf{constant}, \\texttt{node *} for \\sollya type\n   \\textbf{function}, \\texttt{mpfi\\_t *} for \\sollya type \\textbf{range}, \\texttt{int *} for \\sollya type\n   \\textbf{integer}, \\texttt{char *} for \\sollya type \\textbf{string} and \\texttt{int *} for \\sollya type\n   \\textbf{boolean}.\n    \n   The external procedure is not supposed to alter the memory pointed by\n   its array argument \\texttt{void **}.\n    \n   In both directions (argument and result values), empty lists are\n   represented by \\texttt{chain * NULL} pointers.\n    \n   In contrast to internal procedures, externally bounded procedures can\n   be considered to be objects inside \\sollya that can be assigned to other\n   variables, stored in list etc.\n\end{itemize}
\noindent Example 1: 
\begin{center}\begin{minipage}{15cm}\begin{Verbatim}[frame=single]
\end{Verbatim}
\end{minipage}\end{center}
See also: \textbf{library} (\ref{lablibrary}), \textbf{externalplot} (\ref{labexternalplot}), \textbf{bashexecute} (\ref{labbashexecute}), \textbf{void} (\ref{labvoid}), \textbf{constant} (\ref{labconstant}), \textbf{function} (\ref{labfunction}), \textbf{range} (\ref{labrange}), \textbf{integer} (\ref{labinteger}), \textbf{string} (\ref{labstring}), \textbf{boolean} (\ref{labboolean}), \textbf{list of} (\ref{lablistof})
