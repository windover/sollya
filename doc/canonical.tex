\subsection{canonical}
\label{labcanonical}
\noindent Name: \textbf{canonical}\\
brings all polynomial subexpressions of an expression to canonical form or activates, deactivates or checks canonical form printing\\

\noindent Usage: 
\begin{center}
\textbf{canonical}(\emph{function}) : \textsf{function} $\rightarrow$ \textsf{function}\\
\textbf{canonical} = \emph{activation value} : \textsf{on$|$off} $\rightarrow$ \textsf{void}\\
\textbf{canonical} = \emph{activation value} ! : \textsf{on$|$off} $\rightarrow$ \textsf{void}\\
\textbf{canonical} = ? : \textsf{void} $\rightarrow$ \textsf{on$|$off}\\
\end{center}
Parameters: 
\begin{itemize}
\item \emph{function} represents the expression to be rewritten in canonical form
\item \emph{activation value} represents \textbf{on} or \textbf{off}, i.e. activation or deactivation
\end{itemize}
\noindent Description: \begin{itemize}

\item The command \textbf{canonical} rewrites the expression representing the function
   \emph{function} in a way such that all polynomial subexpressions (or the
   whole expression itself, if it is a polynomial) are written in
   canonical form, i.e. as a sum of monomials in the canonical base. The
   canonical base is the base of the integer powers of the global free
   variable. The command \textbf{canonical} does not endanger the safety of
   computations even in Sollya's floating-point environment: the
   function returned is mathematically equal to the function \emph{function}.

\item An assignment \textbf{canonical} = \emph{activation value}, where \emph{activation value}
   is one of \textbf{on} or \textbf{off}, activates respectively deactivates the
   automatic printing of polynomial expressions in canonical form,
   i.e. as a sum of monomials in the canonical base. If automatic
   printing in canonical form is deactivated, automatic printing yield to
   displaying polynomial subexpressions in Horner form.
   If the assignment \textbf{canonical} = \emph{activation value} is followed by an
   exclamation mark, no message indicating the new state is
   displayed. Otherwise the user is informed of the new state of the
   global mode by an indication.

\item The expression \textbf{canonical} = ? evaluates to a variable of type
   \textsf{on$|$off}, indicating whether or not the automatic printing of
   subexpressions in canonical form is activated. If automatic printing
   in canonical form is deactivated, automatic printing yield to
   displaying polynomial subexpressions in Horner form.
\end{itemize}
\noindent Example 1: 
\begin{center}\begin{minipage}{15cm}\begin{Verbatim}[frame=single]
> print(canonical(1 + x * (x + 3 * x^2));
> print(canonical((x + 1)^7));
1 + 7 * x + 21 * x^2 + 35 * x^3 + 35 * x^4 + 21 * x^5 + 7 * x^6 + x^7
\end{Verbatim}
\end{minipage}\end{center}
\noindent Example 2: 
\begin{center}\begin{minipage}{15cm}\begin{Verbatim}[frame=single]
> print(canonical(exp((x + 1)^5) - log(asin(((x + 2) + x)^4 * (x + 1)) + x)));
exp(1 + 5 * x + 10 * x^2 + 10 * x^3 + 5 * x^4 + x^5) - log(asin(16 + 80 * x + 16
0 * x^2 + 160 * x^3 + 80 * x^4 + 16 * x^5) + x)
\end{Verbatim}
\end{minipage}\end{center}
\noindent Example 3: 
\begin{center}\begin{minipage}{15cm}\begin{Verbatim}[frame=single]
> canonical = ?;
off
> (x + 2)^9;
512 + x * (2304 + x * (4608 + x * (5376 + x * (4032 + x * (2016 + x * (672 + x *
 (144 + x * (18 + x))))))))
> canonical = on;
Canonical automatic printing output has been activated.
> (x + 2)^9;
512 + 2304 * x + 4608 * x^2 + 5376 * x^3 + 4032 * x^4 + 2016 * x^5 + 672 * x^6 +
 144 * x^7 + 18 * x^8 + x^9
> canonical = ?;
on
> canonical = off!;
> (x + 2)^9;
512 + x * (2304 + x * (4608 + x * (5376 + x * (4032 + x * (2016 + x * (672 + x *
 (144 + x * (18 + x))))))))
\end{Verbatim}
\end{minipage}\end{center}
See also: \textbf{horner} (\ref{labhorner}), \textbf{print} (\ref{labprint})
