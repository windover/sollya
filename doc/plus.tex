\subsection{$+$}
\label{labplus}
\noindent Name: \textbf{$+$}\\
addition function\\
\noindent Usage: 
\begin{center}
\emph{function1} \textbf{$+$} \emph{function2} : (\textsf{function}, \textsf{function}) $\rightarrow$ \textsf{function}\\
\end{center}
Parameters: 
\begin{itemize}
\item \emph{function1} and \emph{function2} represent functions
\end{itemize}
\noindent Description: \begin{itemize}

\item \textbf{$+$} represents the addition (function) on reals. 
   The expression \emph{function1} \textbf{$+$} \emph{function2} stands for
   the function composed of the addition function and the two
   functions \emph{function1} and \emph{function2}.
\end{itemize}
\noindent Example 1: 
\begin{center}\begin{minipage}{15cm}\begin{Verbatim}[frame=single]
> 1 + 2;
3
\end{Verbatim}
\end{minipage}\end{center}
\noindent Example 2: 
\begin{center}\begin{minipage}{15cm}\begin{Verbatim}[frame=single]
> x + 2;
2 + x
\end{Verbatim}
\end{minipage}\end{center}
\noindent Example 3: 
\begin{center}\begin{minipage}{15cm}\begin{Verbatim}[frame=single]
> x + x;
x * 2
\end{Verbatim}
\end{minipage}\end{center}
\noindent Example 4: 
\begin{center}\begin{minipage}{15cm}\begin{Verbatim}[frame=single]
> diff(sin(x) + exp(x));
cos(x) + exp(x)
\end{Verbatim}
\end{minipage}\end{center}
See also: \textbf{$-$} (\ref{labminus}), \textbf{$*$} (\ref{labmult}), \textbf{/} (\ref{labdivide}), \textbf{$\mathbf{\hat{~}}$} (\ref{labpower})
