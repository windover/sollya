\subsection{ simplify }
\noindent Name: \textbf{simplify}\\
simplifies an expression representing a function\\

\noindent Usage: 
\begin{center}
\textbf{simplify}(\emph{function}) : \textsf{function} $\rightarrow$ \textsf{function}\\
\end{center}
Parameters: 
\emph{function} represents the expression to be simplified\\

\noindent Description: \begin{itemize}

\item The command \textbf{simplify} simplifies constant subexpressions of the
   expression given in argument representing the function
   \emph{function}. Those constant subexpressions are evaluated in using
   floating-point arithmetic with the global precision \textbf{prec}.
\end{itemize}
\noindent Example 1: 
\begin{center}\begin{minipage}{14.8cm}\begin{Verbatim}[frame=single]
   > print(simplify(sin(pi * x)));
   sin(0.31415926535897932384626433832795028841971693993750801e1 * x)
   > print(simplify(erf(exp(3) + x * log(4))));
   erf(0.200855369231876677409285296545817178969879078385543785e2 + x * 0.138629436111989061883446424291635313615100026872049663e1)
\end{Verbatim}
\end{minipage}\end{center}
\noindent Example 2: 
\begin{center}\begin{minipage}{14.8cm}\begin{Verbatim}[frame=single]
   > prec = 20!;
   > t = erf(0.5);
   > s = simplify(erf(0.5));
   > prec = 200!;
   > t;
   0.52049987781304653768274665389196452873645157575796370005880583
   > s;
   0.52050018310546875
\end{Verbatim}
\end{minipage}\end{center}
See also: \textbf{simplifysafe}, \textbf{autosymplify}, \textbf{prec}, \textbf{evaluate}
