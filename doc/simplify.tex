\subsection{simplify}
\label{labsimplify}
\noindent Name: \textbf{simplify}\\
simplifies an expression representing a function\\
\noindent Usage: 
\begin{center}
\textbf{simplify}(\emph{function}) : \textsf{function} $\rightarrow$ \textsf{function}\\
\end{center}
Parameters: 
\begin{itemize}
\item \emph{function} represents the expression to be simplified
\end{itemize}
\noindent Description: \begin{itemize}

\item The command \\textbf{simplify} simplifies constant subexpressions of the\n   expression given in argument representing the function\n   \\emph{function}. Those constant subexpressions are evaluated using\n   floating-point arithmetic with the global precision \\textbf{prec}.\n\end{itemize}
\noindent Example 1: 
\begin{center}\begin{minipage}{15cm}\begin{Verbatim}[frame=single]
\end{Verbatim}
\end{minipage}\end{center}
\noindent Example 2: 
\begin{center}\begin{minipage}{15cm}\begin{Verbatim}[frame=single]
\end{Verbatim}
\end{minipage}\end{center}
See also: \textbf{simplifysafe} (\ref{labsimplifysafe}), \textbf{autosimplify} (\ref{labautosimplify}), \textbf{prec} (\ref{labprec}), \textbf{evaluate} (\ref{labevaluate})
