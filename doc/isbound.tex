\subsection{isbound}
\label{labisbound}
\noindent Name: \textbf{isbound}\\
indicates whether a variable is bound or not.\\
\noindent Usage: 
\begin{center}
\textbf{isbound}(\emph{ident}) : \textsf{boolean}\\
\end{center}
Parameters: 
\begin{itemize}
\item \emph{ident} is a name.
\end{itemize}
\noindent Description: \begin{itemize}

\item \\textbf{isbound}(\\emph{ident}) returns a boolean value indicating whether the name \\emph{ident}\n   is used or not to represent a variable. It returns true when \\emph{ident} is the \n   name used to represent the global variable or if the name is currently used\n   to refer to a (possibly local) variable.\n
\item When a variable is defined in a block and has not been defined outside, \n   \\textbf{isbound} returns true when called inside the block, and false outside.\n   Note that \\textbf{isbound} returns true as soon as a variable has been declared with \n   \\textbf{var}, even if no value is actually stored in it.\n
\item If \\emph{ident1} is bound to a variable and if \\emph{ident2} refers to the global \n   variable, the command \\textbf{rename}(\\emph{ident2}, \\emph{ident1}) hides the value of \\emph{ident1}\n   which becomes the global variable. However, if the global variable is again\n   renamed, \\emph{ident1} gets its value back. In this case, \\textbf{isbound}(\\emph{ident1}) returns\n   true. If \\emph{ident1} was not bound before, \\textbf{isbound}(\\emph{ident1}) returns false after\n   that \\emph{ident1} has been renamed.\n\end{itemize}
\noindent Example 1: 
\begin{center}\begin{minipage}{15cm}\begin{Verbatim}[frame=single]
\end{Verbatim}
\end{minipage}\end{center}
\noindent Example 2: 
\begin{center}\begin{minipage}{15cm}\begin{Verbatim}[frame=single]
\end{Verbatim}
\end{minipage}\end{center}
\noindent Example 3: 
\begin{center}\begin{minipage}{15cm}\begin{Verbatim}[frame=single]
\end{Verbatim}
\end{minipage}\end{center}
\noindent Example 4: 
\begin{center}\begin{minipage}{15cm}\begin{Verbatim}[frame=single]
\end{Verbatim}
\end{minipage}\end{center}
See also: \textbf{rename} (\ref{labrename})
