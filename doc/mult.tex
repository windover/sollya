\subsection{ mult }
\noindent Name: \textbf{$*$}\\
multiplication function\\

\noindent Usage: 
\begin{center}
\emph{function1} \textbf{$*$} \emph{function2} : (\textsf{function}, \textsf{function}) $\rightarrow$ \textsf{function}\\
\end{center}
Parameters: 
\emph{function1} and \emph{function2} represent functions\\

\noindent Description: \begin{itemize}

\item \textbf{$*$} represents the multiplication (function) on reals. 
   The expression \emph{function1} \textbf{$*$} \emph{function2} stands for
   the function composed of the multiplication function and the two
   functions \emph{function1} and \emph{function2}.
\end{itemize}
\noindent Example 1: 
\begin{center}\begin{minipage}{14.8cm}\begin{Verbatim}[frame=single]
   > 5 * 2;
   10
\end{Verbatim}
\end{minipage}\end{center}
\noindent Example 2: 
\begin{center}\begin{minipage}{14.8cm}\begin{Verbatim}[frame=single]
   > x * 2;
   x * 2
\end{Verbatim}
\end{minipage}\end{center}
\noindent Example 3: 
\begin{center}\begin{minipage}{14.8cm}\begin{Verbatim}[frame=single]
   > x * x;
   x^2
\end{Verbatim}
\end{minipage}\end{center}
\noindent Example 4: 
\begin{center}\begin{minipage}{14.8cm}\begin{Verbatim}[frame=single]
   > diff(sin(x) * exp(x));
   sin(x) * exp(x) + exp(x) * cos(x)
\end{Verbatim}
\end{minipage}\end{center}
See also: \textbf{$+$}, \textbf{$-$}, \textbf{/}, \textbf{\^}
