\subsection{numberroots}
\label{labnumberroots}
\noindent Name: \textbf{numberroots}\\
Computes the number of roots of a polynomial in a given range\\
\noindent Usage: 
\begin{center}
\textbf{numberroots}(\emph{p}, \emph{I}) : (\textsf{function}, \textsf{range}) $\rightarrow$ \textsf{integer}\\
\end{center}
Parameters: 
\begin{itemize}
\item \emph{p} is the polynomials to be analyzed.
\item \emph{I} is the interval over which the polynomial is to be analyzed.
\end{itemize}
\noindent Description: \begin{itemize}

\item Nothing.
\end{itemize}
\noindent Example 1: 
\begin{center}\begin{minipage}{15cm}\begin{Verbatim}[frame=single]
> numberroots(0.25*x^3 - x^2 + 1,[1;7]);
2
> findzeros(0.25*x^3 - x^2 + 1,[1;7]);
[|[1.193359375;1.194091796875], [3.709228515625;3.7099609375]|]
\end{Verbatim}
\end{minipage}\end{center}
See also: \textbf{dirtyfindzeros} (\ref{labdirtyfindzeros}), \textbf{findzeros} (\ref{labfindzeros}), \textbf{autodiff} (\ref{labautodiff}), \textbf{taylorform} (\ref{labtaylorform})
