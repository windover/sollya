\subsection{roundcoefficients}
\label{labroundcoefficients}
\noindent Name: \textbf{roundcoefficients}\\
rounds the coefficients of a polynomial to classical formats.\\
\noindent Usage: 
\begin{center}
\textbf{roundcoefficients}(\emph{p},\emph{L}) : (\textsf{function}, \textsf{list}) $\rightarrow$ \textsf{function}\\
\end{center}
Parameters: 
\begin{itemize}
\item \emph{p} is a function. Usually a polynomial.
\item \emph{L} is a list of formats.
\end{itemize}
\noindent Description: \begin{itemize}

\item If \\emph{p} is a polynomial and \\emph{L} a list of floating-point formats, \n   \\textbf{roundcoefficients}(\\emph{p},\\emph{L}) rounds each coefficient of \\emph{p} to the corresponding format\n   in \\emph{L}.\n
\item If \\emph{p} is not a polynomial, \\textbf{roundcoefficients} does not do anything.\n
\item If \\emph{L} contains other elements than \\textbf{D}, \\textbf{double}, \\textbf{DD}, \\textbf{doubledouble}, \\textbf{TD} and\n   \\textbf{tripledouble}, an error occurs.\n
\item The coefficients in \\emph{p} corresponding to $X^i$ is rounded to the \n   format L[i]. If \\emph{L} does not contain enough elements\n   (e.g. if \\textbf{length}(L) $<$ \\textbf{degree}(p)+1), a warning is displayed. However, the\n   coefficients corresponding to an element of \\emph{L} are rounded. The trailing \n   coefficients (that do not have a corresponding element in \\emph{L}) are kept with\n   their own precision.\n   If \\emph{L} contains too much elements, the trailing useless elements are ignored.\n   In particular \\emph{L} may be end-elliptic in which case \\textbf{roundcoefficients} has the \n   natural behavior.\n\end{itemize}
\noindent Example 1: 
\begin{center}\begin{minipage}{15cm}\begin{Verbatim}[frame=single]
\end{Verbatim}
\end{minipage}\end{center}
\noindent Example 2: 
\begin{center}\begin{minipage}{15cm}\begin{Verbatim}[frame=single]
\end{Verbatim}
\end{minipage}\end{center}
\noindent Example 3: 
\begin{center}\begin{minipage}{15cm}\begin{Verbatim}[frame=single]
\end{Verbatim}
\end{minipage}\end{center}
See also: \textbf{single} (\ref{labsingle}), \textbf{double} (\ref{labdouble}), \textbf{doubledouble} (\ref{labdoubledouble}), \textbf{tripledouble} (\ref{labtripledouble})
