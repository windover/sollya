\subsection{autodiff}
\label{labautodiff}
\noindent Name: \textbf{autodiff}\\
nothing\\
\noindent Usage: 
\begin{center}
\textbf{autodiff}(\emph{f}, \emph{n}, \emph{I}) : (\textsf{function}, \textsf{integer}, \textsf{range}) $\rightarrow$ \textsf{list}\\
\end{center}
Parameters: 
\begin{itemize}
\item \emph{f} is the function to be approximated.
\item \emph{n} is the order of differentiation.
\item \emph{I} is the interval over which the function is differentiated.
\end{itemize}
\noindent Description: \begin{itemize}

\item Nothing.
\end{itemize}
\noindent Example 1: 
\begin{center}\begin{minipage}{15cm}\begin{Verbatim}[frame=single]
> L = autodiff(exp(x), 5, 0);
\end{Verbatim}
\end{minipage}\end{center}
See also: \textbf{diff} (\ref{labdiff}), \textbf{taylorform} (\ref{labtaylorform}), \textbf{numberroots} (\ref{labnumberroots}), \textbf{evaluate} (\ref{labevaluate})
