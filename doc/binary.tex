\subsection{binary}
\label{labbinary}
\noindent Name: \textbf{binary}\\
special value for global state \textbf{display}\\
\noindent Description: \begin{itemize}

\item \\textbf{binary} is a special value used for the global state \\textbf{display}.  If\n   the global state \\textbf{display} is equal to \\textbf{binary}, all data will be\n   output in binary notation.\n    \n   As any value it can be affected to a variable and stored in lists.\n\end{itemize}
See also: \textbf{decimal} (\ref{labdecimal}), \textbf{dyadic} (\ref{labdyadic}), \textbf{powers} (\ref{labpowers}), \textbf{hexadecimal} (\ref{labhexadecimal})
