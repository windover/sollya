\subsection{hopitalrecursions}
\label{labhopitalrecursions}
\noindent Name: \textbf{hopitalrecursions}\\
controls the number of recursion steps when applying L'Hopital's rule.\\
\noindent Usage: 
\begin{center}
\textbf{hopitalrecursions} = \emph{n} : \textsf{integer} $\rightarrow$ \textsf{void}\\
\textbf{hopitalrecursions} = \emph{n} ! : \textsf{integer} $\rightarrow$ \textsf{void}\\
\textbf{hopitalrecursions} : \textsf{integer}\\
\end{center}
Parameters: 
\begin{itemize}
\item \emph{n} represents the number of recursions
\end{itemize}
\noindent Description: \begin{itemize}

\item \\textbf{hopitalrecursions} is a global variable. Its value represents the number of steps of\n   recursion that are tried when applying L'Hopital's rule. This rule is applied\n   by the interval evaluator present in the core of \\sollya (and particularly\n   visible in commands like \\textbf{infnorm}).\n
\item If an expression of the form $f/g$ has to be evaluated by interval \n   arithmetic on an interval $I$ and if $f$ and $g$ have a common zero\n   in $I$, a direct evaluation leads to NaN.\n   \\sollya implements a safe heuristic to avoid this, based on L'Hopital's rule: in \n   such a case, it can be shown that $(f/g)(I) \\subseteq (f'/g')(I)$. Since\n   the same problem may exist for $f'/g'$, the rule is applied recursively.\n   The number of step in this recursion process is controlled by \\textbf{hopitalrecursions}.\n
\item Setting \\textbf{hopitalrecursions} to 0 makes \\sollya use this rule only once;\n   setting it to 1 makes \\sollya use the rule twice, and so on.\n   In particular: the rule is always applied at least once, if necessary.\n\end{itemize}
\noindent Example 1: 
\begin{center}\begin{minipage}{15cm}\begin{Verbatim}[frame=single]
\end{Verbatim}
\end{minipage}\end{center}
