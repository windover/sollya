\subsection{ denominator }
\noindent Name: \textbf{denominator}\\
gives the denominator of an expression\\

\noindent Usage: 
\begin{center}
\textbf{denominator}(\emph{expr}) : \textsf{function} $\rightarrow$ \textsf{function}\\
\end{center}
Parameters: 
\emph{expr} represents an expression\\

\noindent Description: \begin{itemize}

\item If \emph{expr} represents a fraction \emph{expr1}/\emph{expr2}, \textbf{denominator}(\emph{expr})
   returns the denominator of this fraction, i.e. \emph{expr2}.
   If \emph{expr} represents something else, \textbf{denominator}(\emph{expr}) 
   returns 1.
   Note that for all expressions \emph{expr}, \textbf{numerator}(\emph{expr}) \textbf{/} \textbf{denominator}(\emph{expr})
   is equal to \emph{expr}.
\end{itemize}
\noindent Example 1: 
\begin{center}\begin{minipage}{14.8cm}\begin{Verbatim}[frame=single]
   > denominator(5/3);
   3
\end{Verbatim}
\end{minipage}\end{center}
\noindent Example 2: 
\begin{center}\begin{minipage}{14.8cm}\begin{Verbatim}[frame=single]
   > denominator(exp(x));
   1
\end{Verbatim}
\end{minipage}\end{center}
\noindent Example 3: 
\begin{center}\begin{minipage}{14.8cm}\begin{Verbatim}[frame=single]
   > a = 5/3;
   > b = numerator(a)/denominator(a);
   > print(a);
   5 / 3
   > print(b);
   5 / 3
\end{Verbatim}
\end{minipage}\end{center}
\noindent Example 4: 
\begin{center}\begin{minipage}{14.8cm}\begin{Verbatim}[frame=single]
   > a = exp(x/3);
   > b = numerator(a)/denominator(a);
   > print(a);
   exp(x / 3)
   > print(b);
   exp(x / 3)
\end{Verbatim}
\end{minipage}\end{center}
See also: \textbf{numerator}
