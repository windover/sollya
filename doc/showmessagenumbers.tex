\subsection{showmessagenumbers}
\label{labshowmessagenumbers}
\noindent Name: \textbf{showmessagenumbers}\\
\phantom{aaa}activates, deactivates or inspects the state variable controlling the displaying of numbers for messages\\[0.2cm]
\noindent Usage: 
\begin{center}
\textbf{showmessagenumbers} = \emph{activation value} : \textsf{on$|$off} $\rightarrow$ \textsf{void}\\
\textbf{showmessagenumbers} = \emph{activation value} ! : \textsf{on$|$off} $\rightarrow$ \textsf{void}\\
\textbf{showmessagenumbers} : \textsf{on$|$off}\\
\end{center}
Parameters: 
\begin{itemize}
\item \emph{activation value} represents \textbf{on} or \textbf{off}, i.e. activation or deactivation
\end{itemize}
\noindent Description: \begin{itemize}

\item An assignment \textbf{showmessagenumbers} = \emph{activation value}, where \emph{activation value}
   is one of \textbf{on} or \textbf{off}, activates respectively deactivates the
   displaying of numbers for warning and information messages. All
   \sollya warning or information messages (that are not fatal to the
   tool's execution) have message numbers. By default, these numbers are
   not displayed when a message is output. When message number displaying
   is activated, the message numbers are displayed together with the
   message. This allows the user to recover the number of a particular
   message in order to suppress resp. unsuppress the displaying of this
   particular message (see \textbf{suppressmessage} and \textbf{unsuppressmessage}).

\item The user should be aware of the fact that message number display
   activation resp. deactivation through \textbf{showmessagenumbers} does not affect message
   displaying in general. For instance, even with message number
   displaying activated, messages with only displayed when general
   verbosity and rounding warning mode are set accordingly.

\item If the assignment \textbf{showmessagenumbers} = \emph{activation value} is followed by an
   exclamation mark, no message indicating the new state is
   displayed. Otherwise the user is informed of the new state of the
   global mode by an indication.
\end{itemize}
\noindent Example 1: 
\begin{center}\begin{minipage}{15cm}\begin{Verbatim}[frame=single]
> verbosity = 1;
The verbosity level has been set to 1.
> 0.1;
Warning: Rounding occurred when converting the constant "0.1" to floating-point 
with 165 bits.
If safe computation is needed, try to increase the precision.
0.1
> showmessagenumbers = on;
Displaying of message numbers has been activated.
> 0.1;
Warning (174): Rounding occurred when converting the constant "0.1" to floating-
point with 165 bits.
If safe computation is needed, try to increase the precision.
0.1
> showmessagenumbers;
on
> showmessagenumbers = off!;
> 0.1;
Warning: Rounding occurred when converting the constant "0.1" to floating-point 
with 165 bits.
If safe computation is needed, try to increase the precision.
0.1
\end{Verbatim}
\end{minipage}\end{center}
\noindent Example 2: 
\begin{center}\begin{minipage}{15cm}\begin{Verbatim}[frame=single]
> showmessagenumbers = on;
Displaying of message numbers has been activated.
> verbosity = 1;
The verbosity level has been set to 1.
> diff(0.1 * x + 1.5 * x^2);
Warning (174): Rounding occurred when converting the constant "0.1" to floating-
point with 165 bits.
If safe computation is needed, try to increase the precision.
0.1 + 3 * x
> verbosity = 0;
The verbosity level has been set to 0.
> diff(0.1 * x + 1.5 * x^2);
0.1 + 3 * x
> verbosity = 12;
The verbosity level has been set to 12.
> diff(0.1 * x + 1.5 * x^2);
Warning (174): Rounding occurred when converting the constant "0.1" to floating-
point with 165 bits.
If safe computation is needed, try to increase the precision.
Information (196): formally differentiating a function.
Information (197): differentiating the expression '0.1 * x + 1.5 * x^2'
Information (205): extraction of coefficient terms from a polynomial uses a spec
ial algorithm for canonical forms.
Information (195): expression '0.1 + 2 * 1.5 * x' has been simplified to express
ion '0.1 + 3 * x'.
Information (207): no Horner simplification will be performed because the given 
tree is already in Horner form.
0.1 + 3 * x
\end{Verbatim}
\end{minipage}\end{center}
See also: \textbf{getsuppressedmessages} (\ref{labgetsuppressedmessages}), \textbf{suppressmessage} (\ref{labsuppressmessage}), \textbf{unsuppressmessage} (\ref{labunsuppressmessage}), \textbf{verbosity} (\ref{labverbosity}), \textbf{roundingwarnings} (\ref{labroundingwarnings})
