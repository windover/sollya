\subsection{horner}
\label{labhorner}
\noindent Name: \textbf{horner}\\
brings all polynomial subexpressions of an expression to Horner form\\
\noindent Usage: 
\begin{center}
\textbf{horner}(\emph{function}) : \textsf{function} $\rightarrow$ \textsf{function}\\
\end{center}
Parameters: 
\begin{itemize}
\item \emph{function} represents the expression to be rewritten in Horner form
\end{itemize}
\noindent Description: \begin{itemize}

\item The command \\textbf{horner} rewrites the expression representing the function\n   \\emph{function} in a way such that all polynomial subexpressions (or the\n   whole expression itself, if it is a polynomial) are written in Horner\n   form.  The command \\textbf{horner} does not endanger the safety of\n   computations even in \\sollya's floating-point environment: the\n   function returned is mathematically equal to the function \\emph{function}.\n\end{itemize}
\noindent Example 1: 
\begin{center}\begin{minipage}{15cm}\begin{Verbatim}[frame=single]
\end{Verbatim}
\end{minipage}\end{center}
\noindent Example 2: 
\begin{center}\begin{minipage}{15cm}\begin{Verbatim}[frame=single]
\end{Verbatim}
\end{minipage}\end{center}
See also: \textbf{canonical} (\ref{labcanonical}), \textbf{print} (\ref{labprint})
