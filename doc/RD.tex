\subsection{RD}
\label{labRD}
\noindent Name: \textbf{RD}\\
constant representing rounding-downwards mode.\\

\noindent Description: \begin{itemize}

\item \textbf{RD} is used in command \textbf{round} to specify that the value $x$ must be rounded
   to the greatest floating-point number $y$ such that $y \le x$.
\end{itemize}
\noindent Example 1: 
\begin{center}\begin{minipage}{15cm}\begin{Verbatim}[frame=single]
> display=binary!;
> round(Pi,20,RD);
1.1001001000011111101_2 * 2^(1)
\end{Verbatim}
\end{minipage}\end{center}
See also: \textbf{RZ} (\ref{labrz}), \textbf{RU} (\ref{labru}), \textbf{RN} (\ref{labrn}), \textbf{round} (\ref{labround})
