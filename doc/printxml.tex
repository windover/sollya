\subsection{printxml}
\label{labprintxml}
\noindent Name: \textbf{printxml}\\
prints an expression as an MathML-Content-Tree\\
\noindent Usage: 
\begin{center}
\textbf{printxml}(\emph{expr}) : \textsf{function} $\rightarrow$ \textsf{void}\\
\textbf{printxml}(\emph{expr}) $>$ \emph{filename} : (\textsf{function}, \textsf{string}) $\rightarrow$ \textsf{void}\\
\textbf{printxml}(\emph{expr}) $>$ $>$ \emph{filename} : (\textsf{function}, \textsf{string}) $\rightarrow$ \textsf{void}\\
\end{center}
Parameters: 
\begin{itemize}
\item \emph{expr} represents a functional expression
\item \emph{filename} represents a character sequence indicating a file name
\end{itemize}
\noindent Description: \begin{itemize}

\item \\textbf{printxml}(\\emph{expr}) prints the functional expression \\emph{expr} as a tree of\n   MathML Content Definition Markups. This XML tree can be re-read in\n   external tools or by usage of the \\textbf{readxml} command.\n    \n   If a second argument \\emph{filename} is given after a single $>$, the\n   MathML tree is not output on the standard output of \\sollya but if in\n   the file \\emph{filename} that get newly created or overwritten. If a double\n   $>$ $>$ is given, the output will be appended to the file \\emph{filename}.\n\end{itemize}
\noindent Example 1: 
\begin{center}\begin{minipage}{15cm}\begin{Verbatim}[frame=single]
\end{Verbatim}
\end{minipage}\end{center}
\noindent Example 2: 
\begin{center}\begin{minipage}{15cm}\begin{Verbatim}[frame=single]
\end{Verbatim}
\end{minipage}\end{center}
\noindent Example 3: 
\begin{center}\begin{minipage}{15cm}\begin{Verbatim}[frame=single]
\end{Verbatim}
\end{minipage}\end{center}
See also: \textbf{readxml} (\ref{labreadxml}), \textbf{print} (\ref{labprint}), \textbf{write} (\ref{labwrite})
