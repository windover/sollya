\subsection{dirtyintegral}
\label{labdirtyintegral}
\noindent Name: \textbf{dirtyintegral}\\
computes a numerical approximation of the integral of a function on an interval.\\
\noindent Usage: 
\begin{center}
\textbf{dirtyintegral}(\emph{f},\emph{I}) : (\textsf{function}, \textsf{range}) $\rightarrow$ \textsf{constant}\\
\end{center}
Parameters: 
\begin{itemize}
\item \emph{f} is a function.
\item \emph{I} is an interval.
\end{itemize}
\noindent Description: \begin{itemize}

\item \\textbf{dirtyintegral}(\\emph{f},\\emph{I}) computes an approximation of the integral of \\emph{f} on \\emph{I}.\n
\item The interval must be bound. If the interval contains one of -Inf or +Inf, the \n   result of \\textbf{dirtyintegral} is NaN, even if the integral has a meaning.\n
\item The result of this command depends on the global variables \\textbf{prec} and \\textbf{points}.\n   The method used is the trapezium rule applied at $n$ evenly distributed\n   points in the interval, where $n$ is the value of global variable \\textbf{points}.\n
\item This command computes a numerical approximation of the exact value of the \n   integral. It should not be used if safety is critical. In this case, use\n   command \\textbf{integral} instead.\n
\item Warning: this command is currently known to be unsatisfactory. If you really\n   need to compute integrals, think of using an other tool or report a feature\n   request to sylvain.chevillard@ens-lyon.org.\n\end{itemize}
\noindent Example 1: 
\begin{center}\begin{minipage}{15cm}\begin{Verbatim}[frame=single]
\end{Verbatim}
\end{minipage}\end{center}
See also: \textbf{prec} (\ref{labprec}), \textbf{points} (\ref{labpoints}), \textbf{integral} (\ref{labintegral})
