\subsection{time}
\label{labtime}
\noindent Name: \textbf{time}\\
procedure for timing \sollya code.\\
\noindent Usage: 
\begin{center}
\textbf{time}(\emph{code}) : \textsf{code} $\rightarrow$ \textsf{constant}\\
\end{center}
Parameters: 
\begin{itemize}
\item \emph{code} is the code to be timed.
\end{itemize}
\noindent Description: \begin{itemize}

\item \\textbf{time} permits timing a \\sollya instruction, resp. a begin-end block\n   of \\sollya instructions. The timing value, measured in seconds, is returned\n   as a \\sollya constant (and not merely displayed as for \\textbf{timing}). This \n   permits performing computations of the timing measurement value inside \\sollya.\n
\item The extended \\textbf{nop} command permits executing a defined number of\n   useless instructions. Taking the ratio of the time needed to execute a\n   certain \\sollya instruction and the time for executing a \\textbf{nop}\n   therefore gives a way to abstract from the speed of a particular \n   machine when evaluating an algorithm's performance.\n\end{itemize}
\noindent Example 1: 
\begin{center}\begin{minipage}{15cm}\begin{Verbatim}[frame=single]
\end{Verbatim}
\end{minipage}\end{center}
\noindent Example 2: 
\begin{center}\begin{minipage}{15cm}\begin{Verbatim}[frame=single]
\end{Verbatim}
\end{minipage}\end{center}
\noindent Example 3: 
\begin{center}\begin{minipage}{15cm}\begin{Verbatim}[frame=single]
\end{Verbatim}
\end{minipage}\end{center}
\noindent Example 4: 
\begin{center}\begin{minipage}{15cm}\begin{Verbatim}[frame=single]
\end{Verbatim}
\end{minipage}\end{center}
See also: \textbf{timing} (\ref{labtiming}), \textbf{nop} (\ref{labnop})
