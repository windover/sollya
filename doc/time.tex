\subsection{time}
\label{labtime}
\noindent Name: \textbf{time}\\
\phantom{aaa}procedure for timing \sollya code.\\[0.2cm]
\noindent Usage: 
\begin{center}
\textbf{time}(\emph{code}) : \textsf{code} $\rightarrow$ \textsf{constant}\\
\end{center}
Parameters: 
\begin{itemize}
\item \emph{code} is the code to be timed.
\end{itemize}
\noindent Description: \begin{itemize}

\item \textbf{time} permits timing a \sollya instruction, resp. a begin-end block
   of \sollya instructions. The timing value, measured in seconds, is returned
   as a \sollya constant (and not merely displayed as for \textbf{timing}). This 
   permits performing computations of the timing measurement value inside \sollya.

\item The extended \textbf{nop} command permits executing a defined number of
   useless instructions. Taking the ratio of the time needed to execute a
   certain \sollya instruction and the time for executing a \textbf{nop}
   therefore gives a way to abstract from the speed of a particular 
   machine when evaluating an algorithm's performance.
\end{itemize}
\noindent Example 1: 
\begin{center}\begin{minipage}{15cm}\begin{Verbatim}[frame=single]
> t = time(p=remez(sin(x),10,[-1;1]));
> write(t,"s were spent computing p = ",p,"\n");
0.254056000000000000034992841957404152708477340638638s were spent computing p = 
5.0282378350010211058128384123578806139307237364901e-15 * x^10 + 2.6876259495430
3046842518220248962109634016728684035e-6 * x^9 + -1.3789116451286674170531616441
916183590574659143235e-14 * x^8 + -1.9834486301827741649326815515415892442200429
0362704e-4 * x^7 + 1.33797221389218815884112341005509833434214820600635e-14 * x^
6 + 8.3333037186548537651002133031675072810009327877946e-3 * x^5 + -5.3734444911
159112186289355138557505858711113724049e-15 * x^4 + -0.1666666613860130370329129
821967413856804986981073 * x^3 + 7.880275187730278668449934379904773444604978762
3868e-16 * x^2 + 0.9999999997362835995537201146471312100344298816769 * x + (-3.3
426550293345171908513995127407123097704733272204e-17)
\end{Verbatim}
\end{minipage}\end{center}
\noindent Example 2: 
\begin{center}\begin{minipage}{15cm}\begin{Verbatim}[frame=single]
> write(time({ p=remez(sin(x),10,[-1;1]); write("The error is 2^(", log2(dirtyin
fnorm(p-sin(x),[-1;1])), ")\n"); }), " s were spent\n");
The error is 2^(log2(2.3960246769563172784864176818665931373861902892867e-11))
0.458635999999999999948187279219524725704104639589787 s were spent
\end{Verbatim}
\end{minipage}\end{center}
\noindent Example 3: 
\begin{center}\begin{minipage}{15cm}\begin{Verbatim}[frame=single]
> t = time(bashexecute("sleep 10"));
> write(~(t-10),"s of execution overhead.\n");
2.88499999999999867661415464681340381503105163574219e-3s of execution overhead.
\end{Verbatim}
\end{minipage}\end{center}
\noindent Example 4: 
\begin{center}\begin{minipage}{15cm}\begin{Verbatim}[frame=single]
> ratio := time(p=remez(sin(x),10,[-1;1]))/time(nop(10));
> write("This ratio = ", ratio, " should somehow be independent of the type of m
achine.\n");
This ratio = 6.3938755300510372493675510161026826242249805467218 should somehow 
be independent of the type of machine.
\end{Verbatim}
\end{minipage}\end{center}
See also: \textbf{timing} (\ref{labtiming}), \textbf{nop} (\ref{labnop})
