\subsection{printexpansion}
\label{labprintexpansion}
\noindent Name: \textbf{printexpansion}\\
prints a polynomial in Horner form with its coefficients written as a expansions of double precision numbers\\
\noindent Usage: 
\begin{center}
\textbf{printexpansion}(\emph{polynomial}) : \textsf{function} $\rightarrow$ \textsf{void}\\
\end{center}
Parameters: 
\begin{itemize}
\item \emph{polynomial} represents the polynomial to be printed
\end{itemize}
\noindent Description: \begin{itemize}

\item The command \\textbf{printexpansion} prints the polynomial \\emph{polynomial} in Horner form\n   writing its coefficients as expansions of double precision\n   numbers. The double precision numbers themselves are displayed in\n   hexadecimal memory notation (see \\textbf{printhexa}). \n    \n   If some of the coefficients of the polynomial \\emph{polynomial} are not\n   floating-point constants but constant expressions, they are evaluated\n   to floating-point constants using the global precision \\textbf{prec}.  If a\n   rounding occurs in this evaluation, a warning is displayed.\n    \n   If the exponent range of double precision is not sufficient to display\n   all the mantissa bits of a coefficient, the coefficient is displayed\n   rounded and a warning is displayed.\n    \n   If the argument \\emph{polynomial} does not a polynomial, nothing but a\n   warning or a newline is displayed. Constants can be displayed using\n   \\textbf{printexpansion} since they are polynomials of degree $0$.\n\end{itemize}
\noindent Example 1: 
\begin{center}\begin{minipage}{15cm}\begin{Verbatim}[frame=single]
\end{Verbatim}
\end{minipage}\end{center}
\noindent Example 2: 
\begin{center}\begin{minipage}{15cm}\begin{Verbatim}[frame=single]
\end{Verbatim}
\end{minipage}\end{center}
\noindent Example 3: 
\begin{center}\begin{minipage}{15cm}\begin{Verbatim}[frame=single]
\end{Verbatim}
\end{minipage}\end{center}
See also: \textbf{printhexa} (\ref{labprinthexa}), \textbf{horner} (\ref{labhorner}), \textbf{print} (\ref{labprint}), \textbf{prec} (\ref{labprec}), \textbf{remez} (\ref{labremez}), \textbf{taylor} (\ref{labtaylor}), \textbf{roundcoefficients} (\ref{labroundcoefficients})
