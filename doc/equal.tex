\subsection{$==$}
\label{labequal}
\noindent Name: \textbf{$==$}\\
equality test operator\\
\noindent Usage: 
\begin{center}
\emph{expr1} \textbf{$==$} \emph{expr2} : (\textsf{any type}, \textsf{any type}) $\rightarrow$ \textsf{boolean}\\
\end{center}
Parameters: 
\begin{itemize}
\item \emph{expr1} and \emph{expr2} represent expressions
\end{itemize}
\noindent Description: \begin{itemize}

\item The operator \textbf{$==$} evaluates to true iff its operands \emph{expr1} and
   \emph{expr2} are syntactically equal and different from \textbf{error} or constant
   expressions that are not constants and that evaluate to the same
   floating-point number with the global precision \textbf{prec}. The user should
   be aware of the fact that because of floating-point evaluation, the
   operator \textbf{$==$} is not exactly the same as the mathematical
   equality. Further remark that according to IEEE 754-2008 floating-point
   rules, which \sollya emulates, floating-point data which are NaN do not
   compare equal to any other floating-point datum, including NaN. 
\end{itemize}
\noindent Example 1: 
\begin{center}\begin{minipage}{15cm}\begin{Verbatim}[frame=single]
> "Hello" == "Hello";
true
> "Hello" == "Salut";
false
> "Hello" == 5;
false
> 5 + x == 5 + x;
true
\end{Verbatim}
\end{minipage}\end{center}
\noindent Example 2: 
\begin{center}\begin{minipage}{15cm}\begin{Verbatim}[frame=single]
> 1 == exp(0);
true
> asin(1) * 2 == pi;
true
> exp(5) == log(4);
false
\end{Verbatim}
\end{minipage}\end{center}
\noindent Example 3: 
\begin{center}\begin{minipage}{15cm}\begin{Verbatim}[frame=single]
> sin(pi/6) == 1/2 * sqrt(3);
false
\end{Verbatim}
\end{minipage}\end{center}
\noindent Example 4: 
\begin{center}\begin{minipage}{15cm}\begin{Verbatim}[frame=single]
> prec = 12;
The precision has been set to 12 bits.
> 16384.1 == 16385.1;
true
\end{Verbatim}
\end{minipage}\end{center}
\noindent Example 5: 
\begin{center}\begin{minipage}{15cm}\begin{Verbatim}[frame=single]
> error == error;
false
\end{Verbatim}
\end{minipage}\end{center}
\noindent Example 6: 
\begin{center}\begin{minipage}{15cm}\begin{Verbatim}[frame=single]
> a = "Biba";
> b = NaN;
> a == a;
true
> b == b;
false
\end{Verbatim}
\end{minipage}\end{center}
See also: \textbf{!$=$} (\ref{labneq}), \textbf{$>$} (\ref{labgt}), \textbf{$>=$} (\ref{labge}), \textbf{$<=$} (\ref{lable}), \textbf{$<$} (\ref{lablt}), \textbf{in} (\ref{labin}), \textbf{!} (\ref{labnot}), \textbf{$\&\&$} (\ref{laband}), \textbf{$||$} (\ref{labor}), \textbf{error} (\ref{laberror}), \textbf{prec} (\ref{labprec})
