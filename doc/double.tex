\subsection{double}
\label{labdouble}
\noindent Names: \textbf{double}, \textbf{D}\\
rounding to the nearest IEEE 754 double (binary64).\\
\noindent Description: \begin{itemize}

\item \\textbf{double} is both a function and a constant.\n
\item As a function, it rounds its argument to the nearest IEEE 754 double precision (i.e. IEEE754-2008 binary64) number.\n   Subnormal numbers are supported as well as standard numbers: it is the real\n   rounding described in the standard.\n
\item As a constant, it symbolizes the double precision format. It is used in \n   contexts when a precision format is necessary, e.g. in the commands \n   \\textbf{round}, \\textbf{roundcoefficients} and \\textbf{implementpoly}.\n   See the corresponding help pages for examples.\n\end{itemize}
\noindent Example 1: 
\begin{center}\begin{minipage}{15cm}\begin{Verbatim}[frame=single]
\end{Verbatim}
\end{minipage}\end{center}
See also: \textbf{single} (\ref{labsingle}), \textbf{printhexa} (\ref{labprinthexa}), \textbf{doubleextended} (\ref{labdoubleextended}), \textbf{doubledouble} (\ref{labdoubledouble}), \textbf{tripledouble} (\ref{labtripledouble}), \textbf{roundcoefficients} (\ref{labroundcoefficients}), \textbf{implementpoly} (\ref{labimplementpoly}), \textbf{round} (\ref{labround})
