\subsection{fpminimax}
\label{labfpminimax}
\noindent Name: \textbf{fpminimax}\\
computes a good polynomial approximation with fixed-point or floating-point coefficients\\

\noindent Usage: 
\begin{center}
\textbf{fpminimax}(\emph{f}, \emph{n}, \emph{formats}, \emph{range}, \emph{indic1}, \emph{indic2}, \emph{indic3}, \emph{P}) : (\textsf{function}, \textsf{integer}, \textsf{list}, \textsf{range}, \textsf{absolute$|$relative} $|$ \textsf{fixed$|$floating} $|$ \textsf{function}, \textsf{absolute$|$relative} $|$ \textsf{fixed$|$floating} $|$ \textsf{function}, \textsf{absolute$|$relative} $|$ \textsf{fixed$|$floating} $|$ \textsf{function}, \textsf{function}) $\rightarrow$ \textsf{function}\\
\textbf{fpminimax}(\emph{f}, \emph{monomials}, \emph{formats}, \emph{range}, \emph{indic1}, \emph{indic2}, \emph{indic3}, \emph{P}) : (\textsf{function}, \textsf{list}, \textsf{list}, \textsf{range}, \textsf{absolute$|$relative} $|$ \textsf{fixed$|$floating} $|$ \textsf{function}, \textsf{absolute$|$relative} $|$ \textsf{fixed$|$floating} $|$ \textsf{function}, \textsf{absolute$|$relative} $|$ \textsf{fixed$|$floating} $|$ \textsf{function}, \textsf{function}) $\rightarrow$ \textsf{function}\\
\textbf{fpminimax}(\emph{f}, \emph{n}, \emph{formats}, \emph{L}, \emph{indic1}, \emph{indic2}, \emph{indic3}, \emph{P}) : (\textsf{function}, \textsf{integer}, \textsf{list}, \textsf{list}, \textsf{absolute$|$relative} $|$ \textsf{fixed$|$floating} $|$ \textsf{function}, \textsf{absolute$|$relative} $|$ \textsf{fixed$|$floating} $|$ \textsf{function}, \textsf{absolute$|$relative} $|$ \textsf{fixed$|$floating} $|$ \textsf{function}, \textsf{function}) $\rightarrow$ \textsf{function}\\
\textbf{fpminimax}(\emph{f}, \emph{monomials}, \emph{formats}, \emph{L}, \emph{indic1}, \emph{indic2}, \emph{indic3}, \emph{P}) : (\textsf{function}, \textsf{list}, \textsf{list}, \textsf{list}, \textsf{absolute$|$relative} $|$ \textsf{fixed$|$floating} $|$ \textsf{function}, \textsf{absolute$|$relative} $|$ \textsf{fixed$|$floating} $|$ \textsf{function}, \textsf{absolute$|$relative} $|$ \textsf{fixed$|$floating} $|$ \textsf{function}, \textsf{function}) $\rightarrow$ \textsf{function}\\
\end{center}
Parameters: 
\begin{itemize}
\item \emph{f} is the function to be approximated
\item \emph{n} is the degree of the polynomial that must approximate\emph{f}
\item \emph{monomials} is the list of monomials that must be used to represent the polynomial that approximates~\emph{f}
\item \emph{formats} is a list indicating the formats that the coefficients of the polynomial must have
\item \emph{range} is the interval where the function must be approximated
\item \emph{L} is a list of interpolation points used by the method
\item \emph{indic1} (optional) is one of the optional indication parameters. See the detailed description below.
\item \emph{indic2} (optional) is one of the optional indication parameters. See the detailed description below.
\item \emph{indic3} (optional) is one of the optional indication parameters. See the detailed description below.
\item \emph{P} (optional) is the minimax polynomial to be considered for solving the problem.
\end{itemize}
\noindent Description: \begin{itemize}

\item \textbf{fpminimax} uses a heuristical (but practically efficient) method to find a good
   polynomial approximation of a function \emph{f} on an interval \emph{range}. It 
   implements the method published in the article:\\
   Efficient polynomial $L^\infty$-approximations\\ 
   Nicolas Brisebarre and Sylvain Chevillard\\
   Proceedings of the 18th IEEE Symposium on Computer Arithmetic (ARITH 18)\\
   pp. 169-176

\item The basic usage of this command is \textbf{fpminimax}(\emph{f}, \emph{n}, \emph{formats}, \emph{range}).
   It computes a polynomial approximation of $f$ with degree at most $n$
   on the interval \emph{range}. \emph{formats} is a list of integers or format types 
   (such as \textbf{double}, \textbf{doubledouble}, etc.). The polynomial returned by the
   command has its coefficients that fits the formats indication. For 
   instance, if formats[0] is 35, the coefficient of degree 0 of the 
   polynomial will fit a floating-point format of 35 bits. If formats[1] 
   is D, the coefficient of degree 1 will be representable by a IEEE double
   precision number, etc.

\item The second argument may be either an integer or a list of integers
   interpreted as the list of desired monomials. For instance, the list
   $[|0,\,2,\,4,\,6|]$ indicates that the polynomial must be even and of
   degree at most 6. Giving an integer $n$ as second argument is equivalent
   as giving $[|0,\,\dots,\,n|]$.

\item The list of formats may contain either integers or format types (\textbf{double},
   \textbf{doubledouble}, \textbf{tripledouble} and \textbf{doubleextended}). The list may be too big
   or even infinite. Only the first indications will be considered. For 
   instance, for a degree $n$ polynomial, $\mathrm{formats}[n+1]$ and above will
   be discarded. This lets one use elliptical indications for the last
   coefficients.

\item The fourth argument may be a range or a list. Lists are for advanced users
   that know what they are doing. The core of the  method is a kind of
   approximated interpolation. The list given here is a list of points that
   must be considered for the interpolation. It must contain at least as 
   many points as unknown coefficients. If you give a list, it is also 
   recommended that you provide the minimax polynomial as last argument.
   If you give a range, the list of points will be automatically computed.

\item The fivth, sixth and seventh arguments are optional. By default, \textbf{fpminimax}
   will approximate $f$ optimizing the relative error, and interpreting
   the list of formats as a list of floating-point formats.\\
   This default behavior may be changed with these optional arguments. You
   may provide zero, one, two or three of the arguments and in any order.
   This lets the user indicate only the non-default arguments.\\
   The three possible arguments are: \begin{itemize}
   \item \textbf{relative} or \textbf{absolute}: the error to be optimized;
   \item \textbf{floating} or \textbf{fixed}: formats of the coefficients;
   \item a constrained part $q$.
   \end{itemize}
   The constrained part lets the user assign in advance some of the
   coefficients. For instance, for approximating $\exp(x)$, it may
   be interesting to search for a polynomial $p$ of the form
                   $$p = 1 + x + \frac{x^2}{2} + a_3 x^3 + a_4 x^4.$$
   Thus, there is a constrained part $q = 1 + x + x^2/2$ and the unknown
   polynomial should be considered in the monomial basis $[|3, 4|]$.
   Calling \textbf{fpminimax} with monomial basis $[|3,\,4|]$ and constrained
   part $q$, will return a polynomial with the right form.

\item The last argument is for advanced users. It is the minimax polynomial that
   approximates the function $f$ in the monomial basis. If it is not given
   this polynomial will be automatically computed by \textbf{fpminimax}.
   \\
   This minimax polynomial is used to compute the list of interpolation
   points required by the method. In general, you do not have to provide this
   argument. But if you want to obtain several polynomials of the same degree
   that approximate the same function on the same range, just changing the
   formats, you should probably consider computing only once the minimax
   polynomial and the list of points instead of letting \textbf{fpminimax} recompute
   them each time.
   \\
   Note that in the case when a constrained part is given, the minimax 
   polynomial must take it into account. For instance, in the previous
   example, the minimax would be obtained by the following command:
          \begin{center}\verb~P = remez(1-(1+x+x^2/2)/exp(x), [|3,4|], range, 1/exp(x));~\end{center}
   Note that the constrained part is not to be added to $P$.
\end{itemize}
\noindent Example 1: 
\begin{center}\begin{minipage}{15cm}\begin{Verbatim}[frame=single]
> P = fpminimax(cos(x),6,[|DD, DD, D...|],[-1b-5;1b-5]);
> printexpansion(P);
(0x3ff0000000000000 + 0xbc09fda20235c100) + x * ((0x3b29ecd485d34781 + 0xb7c1cbc
971529754) + x * (0xbfdfffffffffff98 + x * (0xbbfa6e0b3183cb0d + x * (0x3fa55555
55145337 + x * (0x3ca3540480618939 + x * 0xbf56c138142d8c3b)))))
\end{Verbatim}
\end{minipage}\end{center}
\noindent Example 2: 
\begin{center}\begin{minipage}{15cm}\begin{Verbatim}[frame=single]
> P = fpminimax(sin(x),6,[|32...|],[-1b-5;1b-5], fixed, absolute);
> display = powers!;
> P;
x * (1 + x^2 * ((-357913941 * 2^(-31)) + x^2 * 35789873 * 2^(-32)))
\end{Verbatim}
\end{minipage}\end{center}
\noindent Example 3: 
\begin{center}\begin{minipage}{15cm}\begin{Verbatim}[frame=single]
> P = fpminimax(exp(x), [|3,4|], [|D,24|], [-1/256; 1/246], 1+x+x^2/2);
> display = powers!;
> P;
1 + x * (1 + x * (1 * 2^(-1) + x * (375300225001191 * 2^(-51) + x * 5592621 * 2^
(-27))))
\end{Verbatim}
\end{minipage}\end{center}
\noindent Example 4: 
\begin{center}\begin{minipage}{15cm}\begin{Verbatim}[frame=single]
> f = cos(exp(x));
> pstar = remez(f, 5, [-1b-7;1b-7]);
> listpoints = dirtyfindzeros(f-pstar, [-1b-7; 1b-7]);
> P1 = fpminimax(f, 5, [|DD...|], listpoints, absolute, default, default, pstar)
;
> P2 = fpminimax(f, 5, [|D...|], listpoints, absolute, default, default, pstar);

> P3 = fpminimax(f, 5, [|D, D, D, 24...|], listpoints, absolute, default, defaul
t, pstar);
> print("Error of pstar: ", dirtyinfnorm(f-pstar, [-1b-7; 1b-7]));
Error of pstar:  7.9048482198097924881613709250085234503882055369255e-16
> print("Error of P1:    ", dirtyinfnorm(f-P1, [-1b-7; 1b-7]));
Error of P1:     7.9048482198097925145227335488934227938190853343743e-16
> print("Error of P2:    ", dirtyinfnorm(f-P2, [-1b-7; 1b-7]));
Error of P2:     8.2477144579950903511320656591431998721972862465026e-16
> print("Error of P3:    ", dirtyinfnorm(f-P3, [-1b-7; 1b-7]));
Error of P3:     1.08589225157946674530174198018927191510501967039589e-15
\end{Verbatim}
\end{minipage}\end{center}
See also: \textbf{remez} (\ref{labremez}), \textbf{dirtyfindzeros} (\ref{labdirtyfindzeros}), \textbf{absolute} (\ref{lababsolute}), \textbf{relative} (\ref{labrelative}), \textbf{fixed} (\ref{labfixed}), \textbf{floating} (\ref{labfloating}), \textbf{default} (\ref{labdefault})
