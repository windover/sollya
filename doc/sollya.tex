\documentclass[a4paper]{article}

\usepackage[english]{babel}
\usepackage[naturalnames]{hyperref}
\usepackage{fullpage}
\usepackage{xspace}
\usepackage{amssymb}
\usepackage{fancyvrb}

\newcommand{\com}[1]{\texttt{#1}}
\newcommand{\key}[1]{\texttt{#1}}
\newcommand{\sollya}{\texttt{Sollya}\xspace}
\newcommand{\rlwrap}{\texttt{rlwrap}\xspace}

\newcommand{\code}[1]{
\begin{center}
\begin{tabular}{|p{14.8cm}|}
\hline
#1
\hline
\end{tabular}
\end{center}
}

\newcommand{\ligne}[1]{\texttt{#1}\\}

\title{Users' manual for the \sollya tool \\ {\large Release 1.0} \\ ~ \\ {\large Laboratoire de l'Informatique du Parall\'elisme \\ UMR CNRS - ENS Lyon - UCB Lyon 1 - INRIA 5668}}

\author{Sylvain Chevillard \\ \small{\url{sylvain.chevillard@ens-lyon.fr}} \and Christoph Lauter \\ \small{\url{sylvain.chevillard@ens-lyon.fr}} \and Nicolas Jourdan \\ \small{\url{sylvain.chevillard@ens-lyon.fr}}}

\date{\today}


\begin{document}

\maketitle

\section*{License}

The \sollya tool is copyright \copyright~ 2007 Laboratoire de
l'Informatique du Parall\'elisme - UMR CNRS - ENS Lyon - UCB Lyon 1 -
INRIA 5668.  

The \sollya tool is open software. It is distributed and can be used,
modified and redistributed under the terms of the CeCILL-C licence
available at \url{http://www.cecill.info/} and reproduced in the
\texttt{COPYING} file of the distribution. The distribution contains
parts of other libraries as a support for but not integral part of
\sollya. These libraries are reigned by the GNU Lesser General Public
License that is available at \url{http://www.gnu.org/licenses/} and
reproduced in the \texttt{COPYING} file of the distribution.

\tableofcontents

\section{Compilation and installation of the \sollya tool}

The \sollya distribution can be compiled and installed using the usual
\texttt{./configure}, \texttt{make}, \texttt{make install}
procedure. Besides a \texttt{C} compiler, \sollya needs the following
software libraries and tools to be installed. The \texttt{./configure}
script checks for the installation of the libraries. However \sollya
will build without error if some of its external tools is not
installed. In this case an error will be produced at runtime.
\begin{itemize}
\item \texttt{MPFR}
\item \texttt{MPFI}
\item \texttt{PARI} version 2.3.0
\item \texttt{libxml2}
\item \texttt{gnuplot}
\end{itemize}
The use of the external tool \texttt{rlwrap} is highly recommended but
not indispensable.


\section{Introduction}
\sollya is an interactive tool for handling numerical functions and working with arbitrary precision. It can evaluate functions accurately, compute polynomial approximations of functions, automatically implement polynomials for use in math libraries, plot functions, compute infinite norms, etc. The language \sollya comes with is a full-featured script programming language with support for procedures etc. 

Let us begin this manual with an example. \sollya does not allow command line edition; since that may quickly become uncomfortable, we highly suggest to use the software \rlwrap with \sollya:

\code{
\ligne{~/\%rlwrap sollya}
\ligne{>}
}

\sollya manipulates only univariate functions. The first time that an unbound variable is used, this name is fixed. It will be used to refer to the free variable. For instance, try

\code{
\ligne{~/\% rlwrap sollya}
\ligne{> f = sin(x)/x;}
\ligne{> g = cos(y)-1;}
\ligne{Warning: the identifier "y" is neither assigned to, nor bound to a library }
\ligne{function nor equal to the current free variable.}
\ligne{Will interpret "y" as "x".}
\ligne{> g;}
\ligne{cos(x) - 1}
\ligne{>} 
}

Now, the name $x$ can only be used to refer to the free variable:

\code{
\ligne{> x=3;}
\ligne{Warning: the identifier "x" is already bound to the free variable or to a }
\ligne{library function}
\ligne{The command will have no effect.}
\ligne{Warning: the last assignment will have no effect.}
\ligne{>}
}

If you really want to unbound $x$, you can use the \com{rename} command and change the name of the free variable:

\code{
\ligne{> rename(x,y);}
\ligne{Information: the free variable has been renamed from "x" to "y".}
\ligne{> g;}
\ligne{cos(y) - 1}
\ligne{> x=3;}
\ligne{> x;}
\ligne{3}
\ligne{> }
}

As you have seen, you can name functions and easily work with. The basic thing to do with a function is to evaluate it at some point:

\code{
\ligne{> evaluate f(-2);}
\ligne{Warning: rounding has happened. The value displayed is a faithful rounding of }
\ligne{the true result.}
\ligne{0.454648713412840847698009932955872421351127485723941}
}

The printed value is generally a faithful rounding of the exact value at the working precision. The working precision is controlled by the global variable \com{prec}:

\code{
\ligne{> prec=?;}
\ligne{165}
\ligne{> prec=200;}
\ligne{The precision has been set to 200 bits.}
\ligne{> prec=?;}
\ligne{200}
\ligne{> f(-2);}
\ligne{Warning: rounding has happened. The value displayed is a faithful rounding of}
\ligne{the true result.}
\ligne{0.45464871341284084769800993295587242135112748572394513418948652}
}

Sometimes, a faithful rounding cannot easily be computed. In such a case, an approximated value is printed:

\code{
\ligne{> sin(pi);}
\ligne{Warning: rounding has happened. The value displayed is not a faithful rounding}
\ligne{of the true result.}
\ligne{-0.379705991005939815725347821572628308530195421950339e-12715}
}

The philosophy of \sollya is: whenever something is not exact, print a warning. This explains the warnings in the previous examples. If the result can be shown to be exact, there is no warning:

\code{
\ligne{> sin(0);}
\ligne{0}
}

Let us finish this Section with a small complete example that shows a bit of what can be done with \sollya:

\code{
\ligne{>  restart;}
\ligne{The tool has been restarted.}
\ligne{> prec=50;}
\ligne{The precision has been set to 50 bits.}
\ligne{> f=cos(2*exp(x));}
\ligne{> d=[-1/8;1/8];}
\ligne{> p=remez(f,2,d);}
\ligne{> derivativeZeros = dirtyfindzeros(diff(p-f),d);}
\ligne{> derivativeZeros = inf(d).:derivativeZeros:.sup(d);}
\ligne{> max=0; for t in derivativeZeros do}
\ligne{\{}
\ligne{  r = evaluate(abs(p-f), t);}
\ligne{  if r > max then {max=r; argmax=t;};}
\ligne{\};}
\ligne{> print("The infinite norm of", p-f, "is", max, "and is reached at", argmax);}
\ligne{The infinite norm of (-0.41626557294429078481812212) + x * ((-0.1798067204872539}
\ligne{9037039096583612263e1) + x * (-0.38971068364047456444865247249254026e-1)) - cos(}
\ligne{2 * exp(x)) is 0.86306625059183635084725239e-3 and is reached at -0.580167296300}
\ligne{62879863317e-1}
\ligne{>}
}

In this example, we define a function $f$, an interval $d$ and we compute the best degree-4 polynomial approximation of $f$ on $d$ with respect to the infinite norm. In other words, $\max_{x \in d} \{|p(x)-f(x)|\}$ is minimal amongst polynomials with degree not greater than $4$. Then, we compute the list of the zeros of the derivative of $p-f$ and add the bounds of $d$ to this list. Finally, we evaluate $|p-f|$ for each point in the list and store the maximum and the point where it is reached. We conclude by printing the result in a formatted way.

Note that you do not really need to use such a script for computing infinite norm; as we will see, the command \com{dirtyinfnorm} does this for you.

\section{General principles}
The first goal of \sollya is to help people using numerical functions and numerical algorithm in a safe way. It is first designed to be used interactively but it can also be used in scripts\footnote{Remark: some of the behaviours of \sollya slightly change when used in scripts. For example, no prompt is printed.}.

One of the originalities of \sollya is to work with multi-precision arithmetic (it uses the \texttt{MPFR} library). For safety purposes, \sollya knows how to use interval arithmetic. It uses the interval arithmetic to produce tight and safe results with the precision required by the user.

The general philosophy of \sollya is: \emph{When you can make a computation exactly and sufficiently quickly, do it; when you cannot, do not, unless you have been explicitely asked for.}

The precision of the tools is set by the global variable \key{prec}. It indicates the number of bits used to represent the constants in \sollya. In general, the variable \key{prec} determines the precision of the outputs of commands: more precisely, the command will internally determine what precision should be used during the computations in order to ensure that the output is a faithful result with \key{prec} bits.

For decidability and efficiency reasons, this general principle cannot be applied everytime, so be careful. Moreover certain commands are known to be unsafe: they give in general excellent results and give almost \key{prec} correct bits in output for everyday examples. However they are just heuristic and should not be used when the result must be safe. See the documentation of each command to know precisely how confident you can be with its result.

A second principle (that comes together with the first one) is: \emph{When a computation leads to inexact results, inform the user with a warning}. This can be quite irritating in some circumstances: in particular if you are using \sollya within other scripts. The global variable \key{verbosity} lets you change the level of verbosity of \sollya. When set to $0$, \sollya becomes completely silent on stdout and prints only very important messages on stderr. Increase \key{verbosity} if you want more informations about what \sollya is doing. Note that when you affect a value to a global variable, a message is always printed even if \com{verbosity} is set to $0$. In order to silently affect a global variable, use \texttt{!}:

\code{
\ligne{> prec=30;}
\ligne{The precision has been set to 30 bits.}
\ligne{> prec=30!;}
\ligne{>}
}

For conviviality reasons, values are displayed in decimal by default. This lets a normal human being understand the numbers he or she manipulates. But since constants are internally represented in binary, this causes permanent conversions that are sources of roundings. Thus you are loosing in accuracy and \sollya is always complaining about inexact results. If you just want to store or communicate your results (to another tools for instance) you can use bit-exact representations avaliable in \sollya. The global variable \key{display} defines the way constants are displayed. Here is an example of the five available modes:


\code{
\ligne{> prec=30!;}
\ligne{> a = 17.25;}
\ligne{> display=decimal;}
\ligne{Display mode is decimal numbers.}
\ligne{> a;}
\ligne{0.1725e2}
\ligne{> display=binary;}
\ligne{Display mode is binary numbers.}
\ligne{> a;}
\ligne{1.000101\_2 * 2\^{}(4)}
\ligne{> display=powers;}
\ligne{Display mode is dyadic numbers in integer-power-of-2 notation.}
\ligne{> a;}
\ligne{69 * 2\^{}(-2)}
\ligne{> display=dyadic;}
\ligne{Display mode is dyadic numbers.}
\ligne{> a;}
\ligne{69b-2}
\ligne{> display=hexadecimal;}
\ligne{Display mode is hexadecimal numbers.}
\ligne{> a;}
\ligne{0x1.14p4}
\ligne{> }
}

As always, the symbol \texttt{e} means $\times 10^\square $. The same way the symbol \texttt{b} means  $\times 2^\square $. The symbol \texttt{p} means $\times 16^\square$ and is used only with the following \texttt{0x} prefix. The prefix \texttt{0x} indicates that the digits of the following number until 
a symbol \texttt{p} or whitespace are hexadecimal. The suffix \texttt{\_2} indicates to \sollya that the previous number has been written in binary. \sollya can parse these notations even if you are not in the corresponding \key{display} mode, so you can always use them.

You can also use memory-dump hexadecimal notation frequently used to represent IEEE 754 \texttt{double} and \texttt{single} precision numbers. Since this notation does not allow for exactly representing numbers with arbitrary precision, there is no corresponding \key{display} mode. However, the commands \com{printhexa} respectively \com{printfloat} round the value to the nearest \texttt{double} respectively \texttt{single}. The number is then printed in hexadecimal as the integer number corresponding to the memory representation of the IEEE 754 \texttt{double} or \texttt{single} number:

\code{
\ligne{> printhexa(a);}
\ligne{0x4031400000000000}
\ligne{> printfloat(a);}
\ligne{0x418a0000}
}

\sollya can parse these memory-dump hexadecimal notation back in any \key{display} mode.

\section{Data types}
\sollya has a (very) basic system of types. If you try to perform an illicit operation (such as adding a number and a string, for instance), you will get a type error. Let us see the available data types.

\subsection{Booleans}
There are two special values \key{true} and \key{false}. Boolean expressions can be constructed using the boolean connectors \key{\&\&} (and), \key{||} (or), \key{!} (not), and comparisons.

The comparison operators \key{<}, \key{<=}, \key{>} and \key{>=} can only be used between two numbers or constant expressions.

The comparison operators \key{==} and \key{!=} are polymorphic. You can use it to compare any two objects, like two strings, two intervals, etc. Note that testing the equality between two functions will return \key{true} if and only if the expression trees representing the two functions are exactly the same. See \ref{laberror} for an exception concerning the special object \key{error}. Example:

\code{
\ligne{> 1+x==1+x;}
\ligne{true}
\ligne{> 1+x==x+1;}
\ligne{false}
}

\subsection{Numbers}
As seen above, \sollya represents numbers as floating-point values with the current precision \com{prec}. A number in an expression is rounded to the precision \com{prec} when the expression gets evaluated:

\code{
\ligne{> prec=12!;}
\ligne{> 4097;}
\ligne{Warning: Rounding occured when converting the constant "4097" to floating-point}
\ligne{with 12 bits.}
\ligne{If safe computation is needed, try to increase the precision.}
\ligne{4096}
\ligne{> 4098;}
\ligne{4098}
\ligne{> 4097+1;}
\ligne{Warning: Rounding occured when converting the constant "4097" to floating-point}
\ligne{with 12 bits.}
\ligne{If safe computation is needed, try to increase the precision.}
\ligne{Warning: rounding has happened. The value displayed is a faithful rounding of}
\ligne{the true result.}
\ligne{4096}
}

Note that each variable has its own precision that corresponds to the value of \com{prec} when the variable was set. Thus you can work with variables having a precision bigger than the current precision.

The same way, if you define a function that refers to some constant, this constant is stored in the function with the current precision and will keep this value in the future, even if \com{prec} becomes smaller.

If you define a function that refers to some variable, the precision of the variable is kept, independently of the current precision:

\code{
\ligne{> prec=50!;}
\ligne{> a = 4097;}
\ligne{> prec=12!;}
\ligne{> f = x+a;}
\ligne{> g = x+4097;}
\ligne{Warning: Rounding occured when converting the constant "4097" to floating-point}
\ligne{with 12 bits.}
\ligne{If safe computation is needed, try to increase the precision.}
\ligne{> prec=50!;}
\ligne{> f;}
\ligne{4097 + x}
\ligne{> g;}
\ligne{4096 + x}
}

\subsection{Intervals}
Intervals are composed of two numbers or constant expressions representing the lower and the upper bound. These values are separated either by commas or semi-colons:

\code{
\ligne{> d=[1;2];}
\ligne{> d2=[1,1+1];}
\ligne{> d==d2;}
\ligne{true}
}

If bounds are defined by constant expressions, these are evaluated to floating-point numbers using the current precision. Numbers or variables containing numbers keep their precision for the interval bounds. Interval bound evaluation is performed in a way that ensures the inclusion property: all points
in the original, unevaluated interval will be contained in the interval with its bounds evaluated to floating-point numbers. Remark that 
evaluation bounds defined by constant expressions includes $\pi$:

\code{
\ligne{> prec=30!;}
\ligne{> a=4097;}
\ligne{> prec=12!;}
\ligne{> d=[4096; a];}
\ligne{> prec=30!;}
\ligne{> d;}
\ligne{[4096;4097]}
\ligne{> [-pi;pi];}
\ligne{[-0.31415926591e1;0.31415926591e1]}
}

You can get the upper-bound (respectively the lower-bound)) of an interval with the function \com{sup} (respectively \com{inf}). The middle of the interval is got with the function \com{mid}. Note that these functions can also be used on numbers (in that case, the number is interpreted as an interval containing only one single point. Thus the functions \com{inf}, \com{mid} and \com{sup} are just the identity):

\code{
\ligne{> d=[1;3];}
\ligne{> inf(d);}
\ligne{1}
\ligne{> mid(d);}
\ligne{2}
\ligne{> sup(4);}
\ligne{4}
}

\subsection{Functions}
\sollya knows only functions with one single variable. The first time in a session that an unbound name is used (without being assigned) it determines the name used to refer to the free variable.

The basic functions available in \sollya are the following:
\begin{itemize}
\item \com{+}, \com{-}, \com{*}, \com{/}, \com{\^{}}
\item \com{sqrt}
\item \com{abs}
\item \com{sin}, \com{cos}, \com{tan}, \com{sinh}, \com{cosh}, \com{tanh}
\item \com{asin}, \com{acos}, \com{atan}, \com{asinh}, \com{atanh}
\item \com{exp}, \com{expm1} (defined as $\mathrm{expm1}(x) = \exp(x)-1$)
\item \com{log} (neperian logarithm), \com{log2} (binary logarithm), \com{log10} (decimal logarithm), \com{log1p} (defined as $\mathrm{log1p}(x) = \log(1+x)$)
\item \com{erf}, \com{erfc}
\end{itemize}

The constant $\pi$ is available through the keword \key{pi} as a $0$-ary function: its behavior is exactly the same as if it were a constant with an infinite precision:

\code{
\ligne{> display=binary!;}
\ligne{> prec=12!;}
\ligne{> a=pi;}
\ligne{> a;}
\ligne{Warning: rounding has happened. The value displayed is a faithful rounding of}
\ligne{the true result.}
\ligne{1.10010010001\_2 * 2\^{}(1)}
\ligne{> prec=30!;}
\ligne{> a;}
\ligne{Warning: rounding has happened. The value displayed is a faithful rounding of}
\ligne{the true result.}
\ligne{1.10010010000111111011010101001\_2 * 2\^{}(1)}
}


\subsection{Strings}
Anything written between quotes is interpreted as a string. The infix operator \com{@} concatenates two strings. To get the length of a string, use the \com{length} function. You can access the $i$-th character of a string using brackets (see the example above). There is no character type in \sollya: the $i$-th character of a string is returned as a string itself.

\code{
\ligne{> s1 = "Hello "; s2 = "World!";}
\ligne{> s = s1@s2;}
\ligne{> length(s);}
\ligne{12}
\ligne{> s[0];}
\ligne{H}
\ligne{> s[11];}
\ligne{!}
}

Strings may contain the following escape sequences:
\texttt{$\backslash\backslash$}, \texttt{$\backslash$\"},
\texttt{$\backslash$?}, \texttt{$\backslash$\'},
\texttt{$\backslash$n}, \texttt{$\backslash$t},
\texttt{$\backslash$a}, \texttt{$\backslash$b},
\texttt{$\backslash$f}, \texttt{$\backslash$r},
\texttt{$\backslash$v}, \texttt{$\backslash$x}[hexadecimal number] and
\texttt{$\backslash$}[octal number]. See the C99 standard for their
meaning.

\subsection{Particular values}
\sollya knows some particular values. These values do not really have a type but they can be stored in variables and in lists. A (possibly not exhaustive) list of such values is the following:

\begin{itemize}
\item \com{on}, \com{off} (see sections \ref{labon} and \ref{laboff})
\item \com{dyadic}, \com{powers}, \com{binary}, \com{decimal}, \com{hexadecimal} (see sections \ref{labdyadic}, \ref{labpowers}, \ref{labbinary}, \ref{labdecimal} and \ref{labhexadecimal})
\item \com{file}, \com{postscript}, \com{postscriptfile} (see sections \ref{labfile}, \ref{labpostscript} and \ref{labpostscriptfile})
\item \com{RU}, \com{RD}, \com{RN}, \com{RZ}
\item \com{absolute}, \com{relative} (see sections \ref{lababsolute} and \ref{labrelative})
\item \com{double}, \com{doubleextended}, \com{doubledouble}, \com{tripledouble} (see sections \ref{labdouble}, \ref{labdoubleextended}, \ref{labdoubledouble} and \ref{labtripledouble})
\item \com{D}, \com{DE}, \com{DD}, \com{TD} (see sections \ref{labdouble}, \ref{labdoubleextended}, \ref{labdoubledouble} and \ref{labtripledouble})
\item \com{perturb} (see section \ref{labperturb})
\item \com{honorcoeffprec} (see section \ref{labhonorcoeffprec})
\item \com{default} (see section \ref{labdefault})
\item \com{error} (see section \ref{laberror})
\item \com{void} (see section \ref{labvoid})
\end{itemize}

\subsection{Lists}
Objects can be grouped into lists. A list can contain elements with different types. As for strings, you can concatenate two lists with \com{@}. The function \com{length} gives also the length of a list.

You can prepend an element to a list using \com{.:} (in $\mathcal{O}(1)$) and you can append an element to a list using \com{:.} (in $\mathcal{O}(n)$). The following example illustrates some features:

\code{
\ligne{> list = [| "foo" |];}
\ligne{> list = list:.1;}
\ligne{> list = "bar".:list;}
\ligne{> list;}
\ligne{[|bar, foo, 1|]}
\ligne{> list[1];}
\ligne{foo}
\ligne{> list@list;}
\ligne{[|bar, foo, 1, bar, foo, 1|]}
}

Lists can be considered as arrays and elements of lists can be
referenced using brackets. Possible indices start at $0$. The
following example illustrates this point:

\code{
\ligne{> l = [|1,2,3,4,5|];}
\ligne{> l;}
\ligne{[|1, 2, 3, 4, 5|]}
\ligne{> l[3];}
\ligne{4}
}

Remark that the complexity for accessing an element of the list using
indices is $\mathcal{O}(n)$.

Lists may contain ellipses indicated by \texttt{,...,} between
elements that are constant and evaluate to integers that are
incrementally ordered. \sollya translates such ellipses to the full
list upon evaluation. Using ellipses between elements that are not
constants is not allowed. This feature is provided for ease of
programmation; remark that the complexity of expanding such lists is
high. For illustration, see the following example:

\code{
\ligne{> [|1,...,5|];}
\ligne{[|1, 2, 3, 4, 5|]}
\ligne{> [|-5,...,5|];}
\ligne{[|-5, -4, -3, -2, -1, 0, 1, 2, 3, 4, 5|]}
\ligne{> [|3,...,1|];}
\ligne{Warning: at least one of the given expressions or a subexpression is not correctly typed}
\ligne{or its evaluation has failed because of some error on a side-effect.}
\ligne{error}
\ligne{> [|true,...,false|];}
\ligne{Warning: at least one of the given expressions or a subexpression is not correctly typed}
\ligne{or its evaluation has failed because of some error on a side-effect.}
\ligne{error}
}

Lists may be continued to infinity by means of the \texttt{...}
indicator after the last element given. At least one element must
explicitly be given. If the last element given is a constant
expression that evaluates to an integer, the list is considered as
continued to infinity by all integers greater than that last
element. If the last element is another object, the list is considered
as continued to infinity by reduplicating this last element. Remark
that bracket notation is supported for such end-elliptic lists even
for implicitly given elements. However, evaluation complexity is
high. Combinations of ellipses inside a list and in its end are
possible. The usage of lists described here is best illustrated by the
following examples:

\code{
\ligne{> l = [|1,2,true,3...|];}
\ligne{> l;}
\ligne{[|1, 2, true, 3...|]}
\ligne{> l[2];}
\ligne{true}
\ligne{> l[3];}
\ligne{3}
\ligne{> l[4];}
\ligne{4}
\ligne{> l[1200];}
\ligne{1200}
\ligne{> l = [|1,...,5,true...|];}
\ligne{> l;}
\ligne{[|1, 2, 3, 4, 5, true...|]}
\ligne{> l[1200];}
\ligne{true}
}



\section{Iterative language elements: assignments, conditional statements and loops}

\subsection{Assignments}

\sollya has two different assignment operators, \texttt{=} and
\texttt{:=}. The assigment operator \texttt{=} assigns its
right-hand-object ``as is'', i.e. without evaluating functional
expressions. For instance, \texttt{i = i + 1;} will dereferentiate the
identifier \texttt{i} with some content, notate it $y$, build up the
expression (function) $y + 1$ and assign this expression back to
\texttt{i}. In the example, if \texttt{i} stood for the value $1000$,
the statement \texttt{i = i + 1;} will assign $1000 + 1$ -- and not
$1001$ -- to \texttt{i}. The assignment operator \texttt{:=} evaluates
constant functional expressions before assigning them. On other
expressions it behaves like \texttt{=}. Still in the example, the
statement \texttt{i := i + 1;} really assigns $1001$ to \texttt{i}.

Both \sollya assignment operators support indexing of lists or strings
elements using brackets on the left-hand-side of the assignment
operator. The indexed element of the list or string gets replaced by
the right-hand-side of the assignment operator.  When indexing strings
this way, that right-hand side must evaluate to a string of length
$1$. End-elliptic lists are supported with their usual semantic for
this kind of assignment.  When referencing and assigning a value in
the implicit part of the end-elliptic list, the list gets expanded to
the corresponding length. The indexing of lists on left-hand sides of
assignments is reduced to the first order. Multiple indexing of lists
of lists is not supported for complexity reasons. 

The following examples well illustrate the behaviour of assignment
statements:
\code{
\ligne{> autosimplify = off;}
\ligne{Automatic pure tree simplification has been deactivated.}
\ligne{> i = 1000;}
\ligne{> i = i + 1;}
\ligne{> print(i);}
\ligne{1000 + 1}
\ligne{> i := i + 1;}
\ligne{> print(i);}
\ligne{1002}
\ligne{> l = [|1,...,5|];}
\ligne{> print(l);}
\ligne{[|1, 2, 3, 4, 5|]}
\ligne{> l[3] = l[3] + 1;}
\ligne{> l[4] := l[4] + 1;}
\ligne{> print(l);}
\ligne{[|1, 2, 3, 4 + 1, 6|]}
\ligne{> l[5] = true;}
\ligne{> l;}
\ligne{[|1, 2, 3, 5, 6, true|]}
\ligne{> s = "Hello world";}
\ligne{> s;}
\ligne{Hello world}
\ligne{> s[1] = "a";}
\ligne{> s;}
\ligne{Hallo world}
\ligne{> l = [|true,1,...,5,9...|];}
\ligne{> l;}
\ligne{[|true, 1, 2, 3, 4, 5, 9...|]}
\ligne{> l[13] = "Hello";}
\ligne{> l;}
\ligne{[|true, 1, 2, 3, 4, 5, 9, 10, 11, 12, 13, 14, 15, Hello...|]}
}

\subsection{Conditional statements}

TODO

If, If-Else, syntaxe bizarre du if-Else.

\subsection{Loops}

For from by, for in, while 

TODO

\section{Functional language elements: procedures}

procedures commes objets, declaration, recursion et identifiants,
variables locales, return, procedures en argument, evaluation retardee
de constantes etc.

TODO

\sollya also supports external procedures, i.e. procedures written in
\texttt{C} (or some other language) and dynamically bound to \sollya
identifiers. See \ref{labexternalproc} for details.

\section{Commands and functions}

\subsection{ absolute }
\noindent Name: \textbf{perturb}\\
indicates an absolute error for \textbf{externalplot}\\

\noindent Usage: 
\begin{center}
\textbf{perturb} : \textsf{absolute|relative}\\
\end{center}
\noindent Description: \begin{itemize}

\item The use of \textbf{perturb} in the command \textbf{externalplot} indicates that during
   plotting in \textbf{externalplot} an absolute error is to be considered.
   See \textbf{externalplot} for details.
\end{itemize}
\noindent Example 1: 
\begin{center}\begin{minipage}{14.8cm}\begin{Verbatim}[frame=single]
   > bashexecute("gcc -fPIC -c externalplotexample.c");
   > bashexecute("gcc -shared -o externalplotexample externalplotexample.o -lgmp -lmpfr");
   > externalplot("./externalplotexample",absolute,exp(x),[-1/2;1/2],12,perturb);
\end{Verbatim}
\end{minipage}\end{center}
See also: \textbf{externalplot}, \textbf{relative}, \textbf{bashexecute}

\subsection{ abs }
\noindent Name: \textbf{abs}\\
the absolute value.\\

\noindent Description: \begin{itemize}

\item \textbf{abs} is the absolute value function. \textbf{abs}(x)=x if x>0 and -x otherwise.
\end{itemize}

\subsection{accurateinfnorm}
\label{labaccurateinfnorm}
\noindent Name: \textbf{accurateinfnorm}\\
computes a faithful rounding of the infinity norm of a function \\
\noindent Usage: 
\begin{center}
\textbf{accurateinfnorm}(\emph{function},\emph{range},\emph{constant}) : (\textsf{function}, \textsf{range}, \textsf{constant}) $\rightarrow$ \textsf{constant}\\
\textbf{accurateinfnorm}(\emph{function},\emph{range},\emph{constant},\emph{exclusion range 1},...,\emph{exclusion range n}) : (\textsf{function}, \textsf{range}, \textsf{constant}, \textsf{range}, ..., \textsf{range}) $\rightarrow$ \textsf{constant}\\
\end{center}
Parameters: 
\begin{itemize}
\item \emph{function} represents the function whose infinity norm is to be computed
\item \emph{range} represents the infinity norm is to be considered on
\item \emph{constant} represents the number of bits in the significant of the result
\item \emph{exclusion range 1} through \emph{exclusion range n} represent ranges to be excluded 
\end{itemize}
\noindent Description: \begin{itemize}

\item The command \\textbf{accurateinfnorm} computes an upper bound to the infinity norm of\n   function \\emph{function} in \\emph{range}. This upper bound is the least\n   floating-point number greater than the value of the infinity norm that\n   lies in the set of dyadic floating point numbers having \\emph{constant}\n   significant mantissa bits. This means the value \\textbf{accurateinfnorm} evaluates to\n   is at the time an upper bound and a faithful rounding to \\emph{constant}\n   bits of the infinity norm of function \\emph{function} on range \\emph{range}.\n    \n   If given, the fourth and further arguments of the command \\textbf{accurateinfnorm},\n   \\emph{exclusion range 1} through \\emph{exclusion range n} the infinity norm of\n   the function \\emph{function} is not to be considered on.\n\end{itemize}
\noindent Example 1: 
\begin{center}\begin{minipage}{15cm}\begin{Verbatim}[frame=single]
\end{Verbatim}
\end{minipage}\end{center}
\noindent Example 2: 
\begin{center}\begin{minipage}{15cm}\begin{Verbatim}[frame=single]
\end{Verbatim}
\end{minipage}\end{center}
See also: \textbf{infnorm} (\ref{labinfnorm}), \textbf{dirtyinfnorm} (\ref{labdirtyinfnorm}), \textbf{checkinfnorm} (\ref{labcheckinfnorm}), \textbf{remez} (\ref{labremez}), \textbf{diam} (\ref{labdiam})

\subsection{acosh}
\label{labacosh}
\noindent Name: \textbf{acosh}\\
\phantom{aaa}the arg-hyperbolic cosine function.\\[0.2cm]
\noindent Library names:\\
\verb|   sollya_obj_t sollya_lib_acosh(sollya_obj_t)|\\
\verb|   sollya_obj_t sollya_lib_build_function_acosh(sollya_obj_t)|\\
\verb|   #define SOLLYA_ACOSH(x) sollya_lib_build_function_acosh(x)|\\[0.2cm]
\noindent Description: \begin{itemize}

\item \textbf{acosh} is the inverse of the function \textbf{cosh}: \textbf{acosh}(y) is the unique number 
   $x \in [0; +\infty]$ such that \textbf{cosh}(x)=y.

\item It is defined only for $y \in [0;+\infty]$.
\end{itemize}
See also: \textbf{cosh} (\ref{labcosh})

\subsection{acos}
\label{labacos}
\noindent Name: \textbf{acos}\\
\phantom{aaa}the arccosine function.\\[0.2cm]
\noindent Library names:\\
\verb|   sollya_obj_t sollya_lib_acos(sollya_obj_t)|\\
\verb|   sollya_obj_t sollya_lib_build_function_acos(sollya_obj_t)|\\
\verb|   #define SOLLYA_ACOS(x) sollya_lib_build_function_acos(x)|\\[0.2cm]
\noindent Description: \begin{itemize}

\item \textbf{acos} is the inverse of the function \textbf{cos}: \textbf{acos}($y$) is the unique number 
   $x \in [0; \pi]$ such that \textbf{cos}($x$)=$y$.

\item It is defined only for $y \in [-1;1]$.
\end{itemize}
See also: \textbf{cos} (\ref{labcos})

\subsection{$\&\&$}
\label{laband}
\noindent Name: \textbf{$\&\&$}\\
boolean AND operator\\
\noindent Usage: 
\begin{center}
\emph{expr1} \textbf{$\&\&$} \emph{expr2} : (\textsf{boolean}, \textsf{boolean}) $\rightarrow$ \textsf{boolean}\\
\end{center}
Parameters: 
\begin{itemize}
\item \emph{expr1} and \emph{expr2} represent boolean expressions
\end{itemize}
\noindent Description: \begin{itemize}

\item \\textbf{$\\&\\&$} evaluates to the boolean AND of the two\n   boolean expressions \\emph{expr1} and \\emph{expr2}. \\textbf{$\\&\\&$} evaluates to \n   true iff both \\emph{expr1} and \\emph{expr2} evaluate to true.\n\end{itemize}
\noindent Example 1: 
\begin{center}\begin{minipage}{15cm}\begin{Verbatim}[frame=single]
\end{Verbatim}
\end{minipage}\end{center}
\noindent Example 2: 
\begin{center}\begin{minipage}{15cm}\begin{Verbatim}[frame=single]
\end{Verbatim}
\end{minipage}\end{center}
See also: \textbf{$||$} (\ref{labor}), \textbf{!} (\ref{labnot})

\input{append}
\subsection{$\sim$}
\label{labapprox}
\noindent Name: \textbf{$\sim$}\\
floating-point evaluation of a constant expression\\
\noindent Usage: 
\begin{center}
\textbf{$\sim$} \emph{expression} : \textsf{function} $\rightarrow$ \textsf{constant}\\
\textbf{$\sim$} \emph{something} : \textsf{any type} $\rightarrow$ \textsf{any type}\\
\end{center}
Parameters: 
\begin{itemize}
\item \emph{expression} stands for an expression that is a constant
\item \emph{something} stands for some language element that is not a constant expression
\end{itemize}
\noindent Description: \begin{itemize}

\item \\textbf{$\\sim$} \\emph{expression} evaluates the \\emph{expression} that is a constant\n   term to a floating-point constant. The evaluation may involve a\n   rounding. If \\emph{expression} is not a constant, the evaluated constant is\n   a faithful rounding of \\emph{expression} with \\textbf{precision} bits, unless the\n   \\emph{expression} is exactly $0$ as a result of cancellation. In the\n   latter case, a floating-point approximation of some (unknown) accuracy\n   is returned.\n
\item \\textbf{$\\sim$} does not do anything on all language elements that are not a\n   constant expression.  In other words, it behaves like the identity\n   function on any type that is not a constant expression. It can hence\n   be used in any place where one wants to be sure that expressions are\n   simplified using floating-point computations to constants of a known\n   precision, regardless of the type of actual language elements.\n
\item \\textbf{$\\sim$} \\textbf{error} evaluates to error and provokes a warning.\n
\item \\textbf{$\\sim$} is a prefix operator not requiring parentheses. Its\n   precedence is the same as for the unary $+$ and $-$\n   operators. It cannot be repeatedly used without brackets.\n\end{itemize}
\noindent Example 1: 
\begin{center}\begin{minipage}{15cm}\begin{Verbatim}[frame=single]
\end{Verbatim}
\end{minipage}\end{center}
\noindent Example 2: 
\begin{center}\begin{minipage}{15cm}\begin{Verbatim}[frame=single]
\end{Verbatim}
\end{minipage}\end{center}
\noindent Example 3: 
\begin{center}\begin{minipage}{15cm}\begin{Verbatim}[frame=single]
\end{Verbatim}
\end{minipage}\end{center}
\noindent Example 4: 
\begin{center}\begin{minipage}{15cm}\begin{Verbatim}[frame=single]
\end{Verbatim}
\end{minipage}\end{center}
\noindent Example 5: 
\begin{center}\begin{minipage}{15cm}\begin{Verbatim}[frame=single]
\end{Verbatim}
\end{minipage}\end{center}
\noindent Example 6: 
\begin{center}\begin{minipage}{15cm}\begin{Verbatim}[frame=single]
\end{Verbatim}
\end{minipage}\end{center}
See also: \textbf{evaluate} (\ref{labevaluate}), \textbf{prec} (\ref{labprec}), \textbf{error} (\ref{laberror})

\subsection{asciiplot}
\label{labasciiplot}
\noindent Name: \textbf{asciiplot}\\
plots a function in a range using ASCII characters\\
\noindent Usage: 
\begin{center}
\textbf{asciiplot}(\emph{function}, \emph{range}) : (\textsf{function}, \textsf{range}) $\rightarrow$ \textsf{void}\\
\end{center}
Parameters: 
\begin{itemize}
\item \emph{function} represents a function to be plotted
\item \emph{range} represents a range the function is to be plotted in 
\end{itemize}
\noindent Description: \begin{itemize}

\item \\textbf{asciiplot} plots the function \\emph{function} in range \\emph{range} using ASCII\n   characters.  On systems that provide the necessary \n   \\texttt{TIOCGWINSZ ioctl}, \\sollya determines the size of the\n   terminal for the plot size if connected to a terminal. If it is not\n   connected to a terminal or if the test is not possible, the plot is of\n   fixed size $77\\times25$ characters.  The function is\n   evaluated on a number of points equal to the number of columns\n   available. Its value is rounded to the next integer in the range of\n   lines available. A letter \\texttt{x} is written at this place. If zero is in\n   the hull of the image domain of the function, an x-axis is\n   displayed. If zero is in range, a y-axis is displayed.  If the\n   function is constant or if the range is reduced to one point, the\n   function is evaluated to a constant and the constant is displayed\n   instead of a plot.\n\end{itemize}
\noindent Example 1: 
\begin{center}\begin{minipage}{15cm}\begin{Verbatim}[frame=single]
\end{Verbatim}
\end{minipage}\end{center}
\noindent Example 2: 
\begin{center}\begin{minipage}{15cm}\begin{Verbatim}[frame=single]
\end{Verbatim}
\end{minipage}\end{center}
\noindent Example 3: 
\begin{center}\begin{minipage}{15cm}\begin{Verbatim}[frame=single]
\end{Verbatim}
\end{minipage}\end{center}
\noindent Example 4: 
\begin{center}\begin{minipage}{15cm}\begin{Verbatim}[frame=single]
\end{Verbatim}
\end{minipage}\end{center}
See also: \textbf{plot} (\ref{labplot})

\subsection{asinh}
\label{labasinh}
\noindent Name: \textbf{asinh}\\
the arg-hyperbolic sine function.\\
\noindent Description: \begin{itemize}

\item \\textbf{asinh} is the inverse of the function \\textbf{sinh}: \\textbf{asinh}($y$) is the unique number \n   $x \\in [-\\infty; +\\infty]$ such that \\textbf{sinh}($x$)=$y$.\n
\item It is defined for every real number y.\n\end{itemize}
See also: \textbf{sinh} (\ref{labsinh})

\subsection{asin}
\label{labasin}
\noindent Name: \textbf{asin}\\
the arcsine function.\\
\noindent Description: \begin{itemize}

\item \\textbf{asin} is the inverse of the function \\textbf{sin}: \\textbf{asin}($y$) is the unique number \n   $x \\in [-\\pi/2; \\pi/2]$ such that \\textbf{sin}($x$)=$y$.\n
\item It is defined only for $y \\in [-1;1]$.\n\end{itemize}
See also: \textbf{sin} (\ref{labsin})

\subsection{atanh}
\label{labatanh}
\noindent Name: \textbf{atanh}\\
\phantom{aaa}the hyperbolic arctangent function.\\[0.2cm]
\noindent Library names:\\
\verb|   sollya_obj_t sollya_lib_atanh(sollya_obj_t)|\\
\verb|   sollya_obj_t sollya_lib_build_function_atanh(sollya_obj_t)|\\
\verb|   #define SOLLYA_ATANH(x) sollya_lib_build_function_atanh(x)|\\[0.2cm]
\noindent Description: \begin{itemize}

\item \textbf{atanh} is the inverse of the function \textbf{tanh}: \textbf{atanh}($y$) is the unique number 
   $x \in [-\infty; +\infty]$ such that \textbf{tanh}($x$)=$y$.

\item It is defined only for $y \in [-1; 1]$.
\end{itemize}
See also: \textbf{tanh} (\ref{labtanh})

\subsection{atan}
\label{labatan}
\noindent Name: \textbf{atan}\\
the arctangent function.\\
\noindent Description: \begin{itemize}

\item \\textbf{atan} is the inverse of the function \\textbf{tan}: \\textbf{atan}($y$) is the unique number \n   $x \\in [-\\pi/2; +\\pi/2]$ such that \\textbf{tan}($x$)=$y$.\n
\item It is defined for every real number y.\n\end{itemize}
See also: \textbf{tan} (\ref{labtan})

\subsection{autodiff}
\label{labautodiff}
\noindent Name: \textbf{autodiff}\\
nothing\\
\noindent Usage: 
\begin{center}
\textbf{autodiff}(\emph{f}, \emph{n}, \emph{I}) : (\textsf{function}, \textsf{integer}, \textsf{range}) $\rightarrow$ \textsf{list}\\
\end{center}
Parameters: 
\begin{itemize}
\item \emph{f} is the function to be approximated.
\item \emph{n} is the order of differentiation.
\item \emph{I} is the interval over which the function is differentiated.
\end{itemize}
\noindent Description: \begin{itemize}

\item Nothing.
\end{itemize}
\noindent Example 1: 
\begin{center}\begin{minipage}{15cm}\begin{Verbatim}[frame=single]
> L = autodiff(exp(x), 5, 0);
\end{Verbatim}
\end{minipage}\end{center}
See also: \textbf{diff} (\ref{labdiff}), \textbf{taylorform} (\ref{labtaylorform}), \textbf{numberroots} (\ref{labnumberroots}), \textbf{evaluate} (\ref{labevaluate})

\subsection{ autosimplify }
\noindent Name: \textbf{autosimplify}\\
activates, deactivates or inspects the value of the automatic simplification state variable\\

\noindent Usage: 
\begin{center}
\textbf{autosimplify} = \emph{activation value} : \textsf{on$|$off} $\rightarrow$ \textsf{void}\\
\textbf{autosimplify} = \emph{activation value} ! : \textsf{on$|$off} $\rightarrow$ \textsf{void}\\
\textbf{autosimplify} = ? : \textsf{void} $\rightarrow$ \textsf{on$|$off}\\
\end{center}
Parameters: 
\begin{itemize}
\item \emph{activation value} represents \textbf{on} or \textbf{off}, i.e. activation or deactivation
\end{itemize}
\noindent Description: \begin{itemize}

\item An assignment \textbf{autosimplify} = \emph{activation value}, where \emph{activation value}
   is one of \textbf{on} or \textbf{off}, activates respectively deactivates the
   automatic safe simplification of expressions of functions generated by
   the evaluation of commands or in argument of other commands.
   Sollya commands like \textbf{remez}, \textbf{taylor} or \textbf{rationalapprox} sometimes
   produce expressions that can be simplified. Constant subexpressions
   can be evaluated to dyadic floating-point numbers, monomials with
   coefficients $0$ can be eliminated. Further, expressions
   indicated by the user perform better in many commands when simplified
   before being passed in argument to a commans. When the automatic
   simplification of expressions is activated, Sollya automatically
   performs a safe (not value changing) simplification process on such
   expression.
   The automatic generation of subexpressions can be annoying, in
   particular if it takes too much time for not enough usage. Further the
   user might want to inspect the structure of the expression tree
   returned by a command. In this case, the automatic simplification
   should be deactivated.
   If the assignment \textbf{autosimplify} = \emph{activation value} is followed by an
   exclamation mark, no message indicating the new state is
   displayed. Otherwise the user is informed of the new state of the
   global mode by an indication.

\item The expression \textbf{autosimplify} = ? evaluates to a variable of type
   \textsf{on|off}, indicating whether or not the automatic simplifications
   of expressions of functions is activated.
\end{itemize}
\noindent Example 1: 
\begin{center}\begin{minipage}{15cm}\begin{Verbatim}[frame=single]
> autosimplify = on !;
> print(x - x);
0
> autosimplify = off ;
Automatic pure tree simplification has been deactivated.
> print(x - x);
x - x
\end{Verbatim}
\end{minipage}\end{center}
\noindent Example 2: 
\begin{center}\begin{minipage}{15cm}\begin{Verbatim}[frame=single]
> autosimplify = on !; 
> print(rationalapprox(sin(pi/5.9),7));
0.5
> autosimplify = off !; 
> print(rationalapprox(sin(pi/5.9),7));
1 / 2
\end{Verbatim}
\end{minipage}\end{center}
\noindent Example 3: 
\begin{center}\begin{minipage}{15cm}\begin{Verbatim}[frame=single]
> autosimplify = ?; 
on
\end{Verbatim}
\end{minipage}\end{center}
See also: \textbf{print}, \textbf{prec}, \textbf{points}, \textbf{diam}, \textbf{display}, \textbf{verbosity}, \textbf{canonical}, \textbf{taylorrecursions}, \textbf{timing}, \textbf{fullparentheses}, \textbf{midpointmode}, \textbf{hopitalrecursions}, \textbf{remez}, \textbf{rationalapprox}, \textbf{taylor}

\subsection{bashexecute}
\label{labbashexecute}
\noindent Name: \textbf{bashexecute}\\
executes a shell command.\\
\noindent Usage: 
\begin{center}
\textbf{bashexecute}(\emph{command}) : \textsf{string} $\rightarrow$ \textsf{void}\\
\end{center}
Parameters: 
\begin{itemize}
\item \emph{command} is a command to be interpreted by the shell.
\end{itemize}
\noindent Description: \begin{itemize}

\item \textbf{bashexecute}(\emph{command}) lets the shell interpret \emph{command}. It is useful to execute
   some external code within \sollya.

\item \textbf{bashexecute} does not return anything. It just executes its argument. However, if
   \emph{command} produces an output in a file, this result can be imported in \sollya
   with help of commands like \textbf{execute}, \textbf{readfile} and \textbf{parse}.
\end{itemize}
\noindent Example 1: 
\begin{center}\begin{minipage}{15cm}\begin{Verbatim}[frame=single]
> bashexecute("LANG=C date");
Wed Mar 24 15:17:50 CET 2010
\end{Verbatim}
\end{minipage}\end{center}
See also: \textbf{execute} (\ref{labexecute}), \textbf{readfile} (\ref{labreadfile}), \textbf{parse} (\ref{labparse})

\subsection{binary}
\label{labbinary}
\noindent Name: \textbf{hexadecimal}\\
special value for global state \textbf{display}\\

\noindent Description: \begin{itemize}

\item \textbf{hexadecimal} is a special value used for the global state \textbf{display}.  If
   the global state \textbf{display} is equal to \textbf{hexadecimal}, all data will be
   output in binary notation.
    
   As any value it can be affected to a variable and stored in lists.
\end{itemize}
See also: \textbf{decimal} (\ref{labdecimal}), \textbf{dyadic} (\ref{labdyadic}), \textbf{powers} (\ref{labpowers}), \textbf{hexadecimal} (\ref{labhexadecimal})

\subsection{boolean}
\label{labboolean}
\noindent Name: \textbf{boolean}\\
keyword representing a \textsf{boolean} type \\
\noindent Usage: 
\begin{center}
\textbf{boolean} : \textsf{type type}\\
\end{center}
\noindent Description: \begin{itemize}

\item \\textbf{boolean} represents the \\textsf{boolean} type for declarations\n   of external procedures by means of \\textbf{externalproc}.\n    \n   Remark that in contrast to other indicators, type indicators like\n   \\textbf{boolean} cannot be handled outside the \\textbf{externalproc} context.  In\n   particular, they cannot be assigned to variables.\n\end{itemize}
See also: \textbf{externalproc} (\ref{labexternalproc}), \textbf{constant} (\ref{labconstant}), \textbf{function} (\ref{labfunction}), \textbf{integer} (\ref{labinteger}), \textbf{list of} (\ref{lablistof}), \textbf{range} (\ref{labrange}), \textbf{string} (\ref{labstring})

\subsection{canonical}
\label{labcanonical}
\noindent Name: \textbf{canonical}\\
brings all polynomial subexpressions of an expression to canonical form or activates, deactivates or checks canonical form printing\\
\noindent Usage: 
\begin{center}
\textbf{canonical}(\emph{function}) : \textsf{function} $\rightarrow$ \textsf{function}\\
\textbf{canonical} = \emph{activation value} : \textsf{on$|$off} $\rightarrow$ \textsf{void}\\
\textbf{canonical} = \emph{activation value} ! : \textsf{on$|$off} $\rightarrow$ \textsf{void}\\
\end{center}
Parameters: 
\begin{itemize}
\item \emph{function} represents the expression to be rewritten in canonical form
\item \emph{activation value} represents \textbf{on} or \textbf{off}, i.e. activation or deactivation
\end{itemize}
\noindent Description: \begin{itemize}

\item The command \\textbf{canonical} rewrites the expression representing the function\n   \\emph{function} in a way such that all polynomial subexpressions (or the\n   whole expression itself, if it is a polynomial) are written in\n   canonical form, i.e. as a sum of monomials in the canonical base. The\n   canonical base is the base of the integer powers of the global free\n   variable. The command \\textbf{canonical} does not endanger the safety of\n   computations even in \\sollya's floating-point environment: the\n   function returned is mathematically equal to the function \\emph{function}.\n
\item An assignment \\textbf{canonical} = \\emph{activation value}, where \\emph{activation value}\n   is one of \\textbf{on} or \\textbf{off}, activates respectively deactivates the\n   automatic printing of polynomial expressions in canonical form,\n   i.e. as a sum of monomials in the canonical base. If automatic\n   printing in canonical form is deactivated, automatic printing yields to\n   displaying polynomial subexpressions in Horner form.\n    \n   If the assignment \\textbf{canonical} = \\emph{activation value} is followed by an\n   exclamation mark, no message indicating the new state is\n   displayed. Otherwise the user is informed of the new state of the\n   global mode by an indication.\n\end{itemize}
\noindent Example 1: 
\begin{center}\begin{minipage}{15cm}\begin{Verbatim}[frame=single]
\end{Verbatim}
\end{minipage}\end{center}
\noindent Example 2: 
\begin{center}\begin{minipage}{15cm}\begin{Verbatim}[frame=single]
\end{Verbatim}
\end{minipage}\end{center}
\noindent Example 3: 
\begin{center}\begin{minipage}{15cm}\begin{Verbatim}[frame=single]
\end{Verbatim}
\end{minipage}\end{center}
See also: \textbf{horner} (\ref{labhorner}), \textbf{print} (\ref{labprint})

\subsection{ceil}
\label{labceil}
\noindent Name: \textbf{ceil}\\
\phantom{aaa}the usual function ceil.\\[0.2cm]
\noindent Library names:\\
\verb|   sollya_obj_t sollya_lib_ceil(sollya_obj_t)|\\
\verb|   sollya_obj_t sollya_lib_build_function_ceil(sollya_obj_t)|\\
\verb|   #define SOLLYA_CEIL(x) sollya_lib_build_function_ceil(x)|\\[0.2cm]
\noindent Description: \begin{itemize}

\item \textbf{ceil} is defined as usual: \textbf{ceil}($x$) is the smallest integer $y$ such that $y \ge x$.

\item It is defined for every real number $x$.
\end{itemize}
See also: \textbf{floor} (\ref{labfloor}), \textbf{nearestint} (\ref{labnearestint}), \textbf{round} (\ref{labround}), \textbf{RU} (\ref{labru})

\subsection{checkinfnorm}
\label{labcheckinfnorm}
\noindent Name: \textbf{checkinfnorm}\\
checks whether the infinite norm of a function is bounded by a value\\

\noindent Usage: 
\begin{center}
\textbf{checkinfnorm}(\emph{function},\emph{range},\emph{constant}) : (\textsf{function}, \textsf{range}, \textsf{constant}) $\rightarrow$ \textsf{boolean}\\
\end{center}
Parameters: 
\begin{itemize}
\item \emph{function} represents the function whose infinite norm is to be checked
\item \emph{range} represents the infinite norm is to be considered on
\item \emph{constant} represents the upper bound the infinite norm is to be checked to
\end{itemize}
\noindent Description: \begin{itemize}

\item The command \textbf{checkinfnorm} checks whether the infinite norm of the given
   function \emph{function} in the range \emph{range} can be proven (by Sollya) to
   be less than the given bound \emph{bound}. This means, if \textbf{checkinfnorm}
   evaluates to \textbf{true}, the infinite norm has been proven (by Sollya's
   interval arithmetic) to be less than the bound. If \textbf{checkinfnorm} evaluates
   to \textbf{false}, there are two possibilities: either the bound is less than
   or equal to the infinite norm of the function or the bound is greater
   than the infinite norm but Sollya could not conclude using its
   internal interval arithmetic.
    
   \textbf{checkinfnorm} is sensitive to the global variable \textbf{diam}. The smaller \textbf{diam},
   the more time Sollya will spend on the evaluation of \textbf{checkinfnorm} in
   order to prove the bound before returning \textbf{false} although the infinite
   is bounded by the bound. If \textbf{diam} is equal to $0$, Sollya will
   eventually spend infinite time on instances where the given bound
   \emph{bound} is less or equal to the infinite norm of the function
   \emph{function} in range \emph{range}. In contrast, with \textbf{diam} being zero,
   \textbf{checkinfnorm} evaluates to \textbf{true} iff the infinite norm of the function in
   the range is bounded by the given bound.
\end{itemize}
\noindent Example 1: 
\begin{center}\begin{minipage}{15cm}\begin{Verbatim}[frame=single]
> checkinfnorm(sin(x),[0;1.75], 1);
true
> checkinfnorm(sin(x),[0;1.75], 1/2); checkinfnorm(sin(x),[0;20/39],
false
> 1/2);
true
\end{Verbatim}
\end{minipage}\end{center}
\noindent Example 2: 
\begin{center}\begin{minipage}{15cm}\begin{Verbatim}[frame=single]
> p = remez(exp(x), 5, [-1;1]);
> b = dirtyinfnorm(p - exp(x), [-1;1]);
> checkinfnorm(p - exp(x), [-1;1], b);
false
> b1 = round(b, 20, RU);
> checkinfnorm(p - exp(x), [-1;1], b1);
false
> b2 = round(b, 25, RU);
> checkinfnorm(p - exp(x), [-1;1], b2);
false
> diam = 1b-20!;
> checkinfnorm(p - exp(x), [-1;1], b2);
true
\end{Verbatim}
\end{minipage}\end{center}
See also: \textbf{infnorm} (\ref{labinfnorm}), \textbf{dirtyinfnorm} (\ref{labdirtyinfnorm}), \textbf{accurateinfnorm} (\ref{labaccurateinfnorm}), \textbf{remez} (\ref{labremez}), \textbf{diam} (\ref{labdiam})

\input{coeff}
\subsection{@}
\label{labconcat}
\noindent Name: \textbf{@}\\
concatenates two lists or strings or applies a list as arguments to a procedure\\
\noindent Usage: 
\begin{center}
\emph{L1}\textbf{@}\emph{L2} : (\textsf{list}, \textsf{list}) $\rightarrow$ \textsf{list}\\
\emph{string1}\textbf{@}\emph{string2} : (\textsf{string}, \textsf{string}) $\rightarrow$ \textsf{string}\\
\emph{proc}\textbf{@}\emph{L1} : (\textsf{procedure}, \textsf{list}) $\rightarrow$ \textsf{any type}\\
\end{center}
Parameters: 
\begin{itemize}
\item \emph{L1} and \emph{L2} are two lists.
\item \emph{string1} and \emph{string2} are two strings.
\item \emph{proc} is a procedure.
\end{itemize}
\noindent Description: \begin{itemize}

\item In its first usage form, \\textbf{@} concatenates two lists or strings.\n
\item In its second usage form, \\textbf{@} applies the elements of a list as\n   arguments to a procedure. In the case when \\emph{proc} is a procedure \n   with a fixed number of arguments, a check is done if the number of\n   elements in the list corresponds to the number of formal parameters\n   of the procedure. An empty list can therefore applied only to a \n   procedure that does not take any argument. In the case of a \n   procedure with an arbitrary number of arguments, no such check is \n   performed.\n\end{itemize}
\noindent Example 1: 
\begin{center}\begin{minipage}{15cm}\begin{Verbatim}[frame=single]
\end{Verbatim}
\end{minipage}\end{center}
\noindent Example 2: 
\begin{center}\begin{minipage}{15cm}\begin{Verbatim}[frame=single]
\end{Verbatim}
\end{minipage}\end{center}
\noindent Example 3: 
\begin{center}\begin{minipage}{15cm}\begin{Verbatim}[frame=single]
\end{Verbatim}
\end{minipage}\end{center}
\noindent Example 4: 
\begin{center}\begin{minipage}{15cm}\begin{Verbatim}[frame=single]
\end{Verbatim}
\end{minipage}\end{center}
\noindent Example 5: 
\begin{center}\begin{minipage}{15cm}\begin{Verbatim}[frame=single]
\end{Verbatim}
\end{minipage}\end{center}
See also: \textbf{.:} (\ref{labprepend}), \textbf{:.} (\ref{labappend}), \textbf{procedure} (\ref{labprocedure}), \textbf{proc} (\ref{labproc})

\subsection{ constant }
\noindent Name: \textbf{constant}\\
keyword representing a \textsf{constant} type \\

\noindent Usage: 
\begin{center}
\textbf{constant} : \textsf{type type}\\
\end{center}
\noindent Description: \begin{itemize}

\item \textbf{constant} represents the \textsf{constant} type for declarations
   of external procedures by means of \textbf{externalproc}.
   Remark that in contrast to other indicators, type indicators like
   \textbf{constant} cannot be handled outside the \textbf{externalproc} context.  In
   particular, they cannot be assigned to variables.
\end{itemize}
See also: \textbf{externalproc}, \textbf{boolean}, \textbf{function}, \textbf{integer}, \textbf{list of}, \textbf{range}, \textbf{string}

\subsection{cosh}
\label{labcosh}
\noindent Name: \textbf{cosh}\\
\phantom{aaa}the hyperbolic cosine function.\\[0.2cm]
\noindent Library names:\\
\verb|   sollya_obj_t sollya_lib_cosh(sollya_obj_t)|\\
\verb|   sollya_obj_t sollya_lib_build_function_cosh(sollya_obj_t)|\\
\verb|   #define SOLLYA_COSH(x) sollya_lib_build_function_cosh(x)|\\[0.2cm]
\noindent Description: \begin{itemize}

\item \textbf{cosh} is the usual hyperbolic function: $\cosh(x) = \frac{e^x + e^{-x}}{2}$.

\item It is defined for every real number $x$.
\end{itemize}
See also: \textbf{acosh} (\ref{labacosh}), \textbf{sinh} (\ref{labsinh}), \textbf{tanh} (\ref{labtanh}), \textbf{exp} (\ref{labexp})

\subsection{cos}
\label{labcos}
\noindent Name: \textbf{cos}\\
\phantom{aaa}the cosine function.\\[0.2cm]
\noindent Library names:\\
\verb|   sollya_obj_t sollya_lib_cos(sollya_obj_t)|\\
\verb|   sollya_obj_t sollya_lib_build_function_cos(sollya_obj_t)|\\
\verb|   #define SOLLYA_COS(x) sollya_lib_build_function_cos(x)|\\[0.2cm]
\noindent Description: \begin{itemize}

\item \textbf{cos} is the usual cosine function.

\item It is defined for every real number $x$.
\end{itemize}
See also: \textbf{acos} (\ref{labacos}), \textbf{sin} (\ref{labsin}), \textbf{tan} (\ref{labtan})

\subsection{decimal}
\label{labdecimal}
\noindent Name: \textbf{decimal}\\
special value for global state \textbf{display}\\
\noindent Description: \begin{itemize}

\item \\textbf{decimal} is a special value used for the global state \\textbf{display}.\n   If the global state \\textbf{display} is equal to \\textbf{decimal}, all data will\n   be output in decimal notation.\n    \n   As any value it can be affected to a variable and stored in lists.\n\end{itemize}
See also: \textbf{dyadic} (\ref{labdyadic}), \textbf{powers} (\ref{labpowers}), \textbf{hexadecimal} (\ref{labhexadecimal}), \textbf{binary} (\ref{labbinary})

\subsection{default}
\label{labdefault}
\noindent Name: \textbf{default}\\
default value for some commands.\\
\noindent Description: \begin{itemize}

\item \\textbf{default} is a special value and is replaced by something depending on the \n   context where it is used. It can often be used as a joker, when you want to \n   specify one of the optional parameters of a command and not the others: set \n   the value of uninteresting parameters to \\textbf{default}.\n
\item Global variables can be reset by affecting them the special value \\textbf{default}.\n\end{itemize}
\noindent Example 1: 
\begin{center}\begin{minipage}{15cm}\begin{Verbatim}[frame=single]
\end{Verbatim}
\end{minipage}\end{center}
\noindent Example 2: 
\begin{center}\begin{minipage}{15cm}\begin{Verbatim}[frame=single]
\end{Verbatim}
\end{minipage}\end{center}

\subsection{degree}
\label{labdegree}
\noindent Name: \textbf{degree}\\
gives the degree of a polynomial.\\
\noindent Usage: 
\begin{center}
\textbf{degree}(\emph{f}) : \textsf{function} $\rightarrow$ \textsf{integer}\\
\end{center}
Parameters: 
\begin{itemize}
\item \emph{f} is a function (usually a polynomial).
\end{itemize}
\noindent Description: \begin{itemize}

\item If \\emph{f} is a polynomial, \\textbf{degree}(\\emph{f}) returns the degree of \\emph{f}.\n
\item Contrary to the usage, \\sollya considers that the degree of the null polynomial\n   is 0.\n
\item If \\emph{f} is a function that is not a polynomial, \\textbf{degree}(\\emph{f}) returns -1.\n\end{itemize}
\noindent Example 1: 
\begin{center}\begin{minipage}{15cm}\begin{Verbatim}[frame=single]
\end{Verbatim}
\end{minipage}\end{center}
\noindent Example 2: 
\begin{center}\begin{minipage}{15cm}\begin{Verbatim}[frame=single]
\end{Verbatim}
\end{minipage}\end{center}
See also: \textbf{coeff} (\ref{labcoeff})

\subsection{denominator}
\label{labdenominator}
\noindent Name: \textbf{denominator}\\
gives the denominator of an expression\\
\noindent Usage: 
\begin{center}
\textbf{denominator}(\emph{expr}) : \textsf{function} $\rightarrow$ \textsf{function}\\
\end{center}
Parameters: 
\begin{itemize}
\item \emph{expr} represents an expression
\end{itemize}
\noindent Description: \begin{itemize}

\item If \\emph{expr} represents a fraction \\emph{expr1}/\\emph{expr2}, \\textbf{denominator}(\\emph{expr})\n   returns the denominator of this fraction, i.e. \\emph{expr2}.\n    \n   If \\emph{expr} represents something else, \\textbf{denominator}(\\emph{expr}) \n   returns 1.\n    \n   Note that for all expressions \\emph{expr}, \\textbf{numerator}(\\emph{expr}) \\textbf{/} \\textbf{denominator}(\\emph{expr})\n   is equal to \\emph{expr}.\n\end{itemize}
\noindent Example 1: 
\begin{center}\begin{minipage}{15cm}\begin{Verbatim}[frame=single]
\end{Verbatim}
\end{minipage}\end{center}
\noindent Example 2: 
\begin{center}\begin{minipage}{15cm}\begin{Verbatim}[frame=single]
\end{Verbatim}
\end{minipage}\end{center}
\noindent Example 3: 
\begin{center}\begin{minipage}{15cm}\begin{Verbatim}[frame=single]
\end{Verbatim}
\end{minipage}\end{center}
\noindent Example 4: 
\begin{center}\begin{minipage}{15cm}\begin{Verbatim}[frame=single]
\end{Verbatim}
\end{minipage}\end{center}
See also: \textbf{numerator} (\ref{labnumerator})

\subsection{diam}
\label{labdiam}
\noindent Name: \textbf{diam}\\
parameter used in safe algorithms of Sollya and controlling the maximal length of the involved intervals.\\

\noindent Description: \begin{itemize}

\item \textbf{diam} is a global variable. Its value represents the maximal length allowed
   for intervals involved in safe algorithms of Sollya (namely \textbf{infnorm},
   \textbf{checkinfnorm}, \textbf{accurateinfnorm}, \textbf{integral}, \textbf{findzeros}).

\item More precisely, \textbf{diam} is relative to the diameter of the input interval of
   the command. For instance, suppose that \textbf{diam}=1e-5: if \textbf{infnorm} is called
   on interval $[0,\,1]$, the maximal length of an interval will be 1e-5. But if it
   is called on interval $[0,\,1\mathrm{e}{-3}]$, the maximal length will be 1e-8.
\end{itemize}
See also: \textbf{infnorm} (\ref{labinfnorm}), \textbf{checkinfnorm} (\ref{labcheckinfnorm}), \textbf{accurateinfnorm} (\ref{labaccurateinfnorm}), \textbf{integral} (\ref{labintegral}), \textbf{findzeros} (\ref{labfindzeros})

\input{dieonerrormode}
\subsection{diff}
\label{labdiff}
\noindent Name: \textbf{diff}\\
differentiation operator\\
\noindent Usage: 
\begin{center}
\textbf{diff}(\emph{function}) : \textsf{function} $\rightarrow$ \textsf{function}\\
\end{center}
Parameters: 
\begin{itemize}
\item \emph{function} represents a function
\end{itemize}
\noindent Description: \begin{itemize}

\item \\textbf{diff}(\\emph{function}) returns the symbolic derivative of the function\n   \\emph{function} by the global free variable.\n    \n   If \\emph{function} represents a function symbol that is externally bound\n   to some code by \\textbf{library}, the derivative is performed as a symbolic\n   annotation to the returned expression tree.\n\end{itemize}
\noindent Example 1: 
\begin{center}\begin{minipage}{15cm}\begin{Verbatim}[frame=single]
\end{Verbatim}
\end{minipage}\end{center}
\noindent Example 2: 
\begin{center}\begin{minipage}{15cm}\begin{Verbatim}[frame=single]
\end{Verbatim}
\end{minipage}\end{center}
\noindent Example 3: 
\begin{center}\begin{minipage}{15cm}\begin{Verbatim}[frame=single]
\end{Verbatim}
\end{minipage}\end{center}
See also: \textbf{library} (\ref{lablibrary})

\subsection{dirtyfindzeros}
\label{labdirtyfindzeros}
\noindent Name: \textbf{dirtyfindzeros}\\
gives a list of numerical values listing the zeros of a function on an interval.\\
\noindent Usage: 
\begin{center}
\textbf{dirtyfindzeros}(\emph{f},\emph{I}) : (\textsf{function}, \textsf{range}) $\rightarrow$ \textsf{list}\\
\end{center}
Parameters: 
\begin{itemize}
\item \emph{f} is a function.
\item \emph{I} is an interval.
\end{itemize}
\noindent Description: \begin{itemize}

\item \\textbf{dirtyfindzeros}(\\emph{f},\\emph{I}) returns a list containing some zeros of \\emph{f} in the\n   interval \\emph{I}. The values in the list are numerical approximation of the exact\n   zeros. The precision of these approximations is approximately the precision\n   stored in \\textbf{prec}. If \\emph{f} does not have two zeros very close to each other, it \n   can be expected that all zeros are listed. However, some zeros may be\n   forgotten. This command should be considered as a numerical algorithm and\n   should not be used if safety is critical.\n
\item More precisely, the algorithm relies on global variables \\textbf{prec} and \\textbf{points} and it performs the following steps: \n   let $n$ be the value of variable \\textbf{points} and $t$ be the value\n   of variable \\textbf{prec}.\n   \\begin{itemize}\n   \\item Evaluate $|f|$ at $n$ evenly distributed points in the interval $I$.\n     The working precision to be used is automatically chosen in order to ensure that the sign\n     is correct.\n   \\item Whenever $f$ changes its sign for two consecutive points,\n     find an approximation $x$ of its zero with precision $t$ using\n     Newton's algorithm. The number of steps in Newton's iteration depends on $t$:\n     the precision of the approximation is supposed to be doubled at each step.\n   \\item Add this value to the list.\n   \\end{itemize}\n\end{itemize}
\noindent Example 1: 
\begin{center}\begin{minipage}{15cm}\begin{Verbatim}[frame=single]
\end{Verbatim}
\end{minipage}\end{center}
\noindent Example 2: 
\begin{center}\begin{minipage}{15cm}\begin{Verbatim}[frame=single]
\end{Verbatim}
\end{minipage}\end{center}
See also: \textbf{prec} (\ref{labprec}), \textbf{points} (\ref{labpoints}), \textbf{findzeros} (\ref{labfindzeros})

\subsection{dirtyinfnorm}
\label{labdirtyinfnorm}
\noindent Name: \textbf{dirtyinfnorm}\\
computes a numerical approximation of the infinity norm of a function on an interval.\\
\noindent Usage: 
\begin{center}
\textbf{dirtyinfnorm}(\emph{f},\emph{I}) : (\textsf{function}, \textsf{range}) $\rightarrow$ \textsf{constant}\\
\end{center}
Parameters: 
\begin{itemize}
\item \emph{f} is a function.
\item \emph{I} is an interval.
\end{itemize}
\noindent Description: \begin{itemize}

\item \\textbf{dirtyinfnorm}(\\emph{f},\\emph{I}) computes an approximation of the infinity norm of the \n   given function $f$ on the interval $I$, e.g. $\\max_{x \\in I} \\{|f(x)|\\}$.\n
\item The interval must be bound. If the interval contains one of -Inf or +Inf, the \n   result of \\textbf{dirtyinfnorm} is NaN.\n
\item The result of this command depends on the global variables \\textbf{prec} and \\textbf{points}.\n   Therefore, the returned result is generally a good approximation of the exact\n   infinity norm, with precision \\textbf{prec}. However, the result is generally \n   underestimated and should not be used when safety is critical.\n   Use \\textbf{infnorm} instead.\n
\item The following algorithm is used: let $n$ be the value of variable \\textbf{points}\n   and $t$ be the value of variable \\textbf{prec}.\n   \\begin{itemize}\n   \\item Evaluate $|f|$ at $n$ evenly distributed points in the\n     interval $I$. The evaluation are faithful roundings of the exact\n     results at precision $t$.\n   \\item Whenever the derivative of $f$ changes its sign for two consecutive \n     points, find an approximation $x$ of its zero with precision $t$.\n     Then compute a faithful rounding of $|f(x)|$ at precision $t$.\n   \\item Return the maximum of all computed values.\n   \\end{itemize}\n\end{itemize}
\noindent Example 1: 
\begin{center}\begin{minipage}{15cm}\begin{Verbatim}[frame=single]
\end{Verbatim}
\end{minipage}\end{center}
\noindent Example 2: 
\begin{center}\begin{minipage}{15cm}\begin{Verbatim}[frame=single]
\end{Verbatim}
\end{minipage}\end{center}
\noindent Example 3: 
\begin{center}\begin{minipage}{15cm}\begin{Verbatim}[frame=single]
\end{Verbatim}
\end{minipage}\end{center}
See also: \textbf{prec} (\ref{labprec}), \textbf{points} (\ref{labpoints}), \textbf{infnorm} (\ref{labinfnorm}), \textbf{checkinfnorm} (\ref{labcheckinfnorm})

\subsection{dirtyintegral}
\label{labdirtyintegral}
\noindent Name: \textbf{dirtyintegral}\\
computes a numerical approximation of the integral of a function on an interval.\\

\noindent Usage: 
\begin{center}
\textbf{dirtyintegral}(\emph{f},\emph{I}) : (\textsf{function}, \textsf{range}) $\rightarrow$ \textsf{constant}\\
\end{center}
Parameters: 
\begin{itemize}
\item \emph{f} is a function.
\item \emph{I} is an interval.
\end{itemize}
\noindent Description: \begin{itemize}

\item \textbf{dirtyintegral}(\emph{f},\emph{I}) computes an approximation of the integral of \emph{f} on \emph{I}.

\item The interval must be bound. If the interval contains one of -Inf or +Inf, the 
   result of \textbf{dirtyintegral} is NaN, even if the integral has a meaning.

\item The result of this command depends on the global variables \textbf{prec} and \textbf{points}.
   The method used is the trapezium rule applied at $n$ evenly distributed
   points in the interval, where $n$ is the value of global variable \textbf{points}.

\item This command computes a numerical approximation of the exact value of the 
   integral. It should not be used if safety is critical. In this case, use
   command \textbf{integral} instead.

\item Warning: this command is known to be currently unsatisfactory. If you really
   need to compute integrals, think of using an other tool or report a feature
   request to sylvain.chevillard@ens-lyon.fr.
\end{itemize}
\noindent Example 1: 
\begin{center}\begin{minipage}{15cm}\begin{Verbatim}[frame=single]
> sin(10);
-0.544021110889369813404747661851377281683643012916219
> dirtyintegral(cos(x),[0;10]);
-0.544003049051526298224480588824753820365362983562818375241
> points=2000!;
> dirtyintegral(cos(x),[0;10]);
-0.544019977511583219722226973125831990359958379268927638295
\end{Verbatim}
\end{minipage}\end{center}
See also: \textbf{prec} (\ref{labprec}), \textbf{points} (\ref{labpoints}), \textbf{integral} (\ref{labintegral})

\subsection{display}
\label{labdisplay}
\noindent Name: \textbf{display}\\
sets or inspects the global variable specifying number notation\\
\noindent Usage: 
\begin{center}
\textbf{display} = \emph{notation value} : \textsf{decimal$|$binary$|$dyadic$|$powers$|$hexadecimal} $\rightarrow$ \textsf{void}\\
\textbf{display} = \emph{notation value} ! : \textsf{decimal$|$binary$|$dyadic$|$powers$|$hexadecimal} $\rightarrow$ \textsf{void}\\
\textbf{display} : \textsf{decimal$|$binary$|$dyadic$|$powers$|$hexadecimal}\\
\end{center}
Parameters: 
\begin{itemize}
\item \emph{notation value} represents a variable of type \textsf{decimal$|$binary$|$dyadic$|$powers$|$hexadecimal}
\end{itemize}
\noindent Description: \begin{itemize}

\item An assignment \\textbf{display} = \\emph{notation value}, where \\emph{notation value} is\n   one of \\textbf{decimal}, \\textbf{dyadic}, \\textbf{powers}, \\textbf{binary} or \\textbf{hexadecimal}, activates\n   the corresponding notation for output of values in \\textbf{print}, \\textbf{write} or\n   at the \\sollya prompt.\n    \n   If the global notation variable \\textbf{display} is \\textbf{decimal}, all numbers will\n   be output in scientific decimal notation.  If the global notation\n   variable \\textbf{display} is \\textbf{dyadic}, all numbers will be output as dyadic\n   numbers with Gappa notation.  If the global notation variable \\textbf{display}\n   is \\textbf{powers}, all numbers will be output as dyadic numbers with a\n   notation compatible with Maple and PARI/GP.  If the global notation\n   variable \\textbf{display} is \\textbf{binary}, all numbers will be output in binary\n   notation.  If the global notation variable \\textbf{display} is \\textbf{hexadecimal},\n   all numbers will be output in C99/ IEEE754-2008 notation.  All output\n   notations can be parsed back by \\sollya, inducing no error if the input\n   and output precisions are the same (see \\textbf{prec}).\n    \n   If the assignment \\textbf{display} = \\emph{notation value} is followed by an\n   exclamation mark, no message indicating the new state is\n   displayed. Otherwise the user is informed of the new state of the\n   global mode by an indication.\n\end{itemize}
\noindent Example 1: 
\begin{center}\begin{minipage}{15cm}\begin{Verbatim}[frame=single]
\end{Verbatim}
\end{minipage}\end{center}
See also: \textbf{print} (\ref{labprint}), \textbf{write} (\ref{labwrite}), \textbf{decimal} (\ref{labdecimal}), \textbf{dyadic} (\ref{labdyadic}), \textbf{powers} (\ref{labpowers}), \textbf{binary} (\ref{labbinary}), \textbf{hexadecimal} (\ref{labhexadecimal}), \textbf{prec} (\ref{labprec})

\subsection{\/}
\label{labdivide}
\noindent Name: \textbf{/}\\
division function\\
\noindent Usage: 
\begin{center}
\emph{function1} \textbf{/} \emph{function2} : (\textsf{function}, \textsf{function}) $\rightarrow$ \textsf{function}\\
\emph{interval1} \textbf{/} \emph{interval2} : (\textsf{range}, \textsf{range}) $\rightarrow$ \textsf{range}\\
\emph{interval1} \textbf{/} \emph{constant} : (\textsf{range}, \textsf{constant}) $\rightarrow$ \textsf{range}\\
\emph{interval1} \textbf{/} \emph{constant} : (\textsf{constant}, \textsf{range}) $\rightarrow$ \textsf{range}\\
\end{center}
Parameters: 
\begin{itemize}
\item \emph{function1} and \emph{function2} represent functions
\item \emph{interval1} and \emph{interval2} represent intervals (ranges)
\item \emph{constant} represents a constant or constant expression
\end{itemize}
\noindent Description: \begin{itemize}

\item \\textbf{/} represents the division (function) on reals. \n   The expression \\emph{function1} \\textbf{/} \\emph{function2} stands for\n   the function composed of the division function and the two\n   functions \\emph{function1} and \\emph{function2}, where \\emph{function1} is\n   the numerator and \\emph{function2} the denominator.\n
\item \\textbf{/} can be used for interval arithmetic on intervals\n   (ranges). \\textbf{/} will evaluate to an interval that safely\n   encompasses all images of the division function with arguments\n   varying in the given intervals. If the intervals given contain points\n   where the division function is not defined, infinities and NaNs will be\n   produced in the output interval.  Any combination of intervals with\n   intervals or constants (resp. constant expressions) is\n   supported. However, it is not possible to represent families of\n   functions using an interval as one argument and a function (varying in\n   the free variable) as the other one.\n\end{itemize}
\noindent Example 1: 
\begin{center}\begin{minipage}{15cm}\begin{Verbatim}[frame=single]
\end{Verbatim}
\end{minipage}\end{center}
\noindent Example 2: 
\begin{center}\begin{minipage}{15cm}\begin{Verbatim}[frame=single]
\end{Verbatim}
\end{minipage}\end{center}
\noindent Example 3: 
\begin{center}\begin{minipage}{15cm}\begin{Verbatim}[frame=single]
\end{Verbatim}
\end{minipage}\end{center}
\noindent Example 4: 
\begin{center}\begin{minipage}{15cm}\begin{Verbatim}[frame=single]
\end{Verbatim}
\end{minipage}\end{center}
\noindent Example 5: 
\begin{center}\begin{minipage}{15cm}\begin{Verbatim}[frame=single]
\end{Verbatim}
\end{minipage}\end{center}
\noindent Example 6: 
\begin{center}\begin{minipage}{15cm}\begin{Verbatim}[frame=single]
\end{Verbatim}
\end{minipage}\end{center}
See also: \textbf{$+$} (\ref{labplus}), \textbf{$-$} (\ref{labminus}), \textbf{$*$} (\ref{labmult}), \textbf{$\mathbf{\hat{~}}$} (\ref{labpower})

\subsection{doubledouble}
\label{labdoubledouble}
\noindent Names: \textbf{doubledouble}, \textbf{DD}\\
represents a number as the sum of two IEEE doubles.\\
\noindent Description: \begin{itemize}

\item \\textbf{doubledouble} is both a function and a constant.\n
\item As a function, it rounds its argument to the nearest number that can be written\n   as the sum of two double precision numbers.\n
\item The algorithm used to compute \\textbf{doubledouble}($x$) is the following: let $x_h$ = \\textbf{double}($x$)\n   and let $x_l$ = \\textbf{double}($x-x_h$). Return the number $x_h+x_l$. Note that if the current \n   precision is not sufficient to exactly represent $x_h + x_l$, a rounding will occur\n   and the result of \\textbf{doubledouble}($x$) will be useless.\n
\item As a constant, it symbolizes the double-double precision format. It is used in \n   contexts when a precision format is necessary, e.g. in the commands \n   \\textbf{round}, \\textbf{roundcoefficients} and \\textbf{implementpoly}.\n   See the corresponding help pages for examples.\n\end{itemize}
\noindent Example 1: 
\begin{center}\begin{minipage}{15cm}\begin{Verbatim}[frame=single]
\end{Verbatim}
\end{minipage}\end{center}
See also: \textbf{single} (\ref{labsingle}), \textbf{double} (\ref{labdouble}), \textbf{doubleextended} (\ref{labdoubleextended}), \textbf{tripledouble} (\ref{labtripledouble}), \textbf{roundcoefficients} (\ref{labroundcoefficients}), \textbf{implementpoly} (\ref{labimplementpoly}), \textbf{round} (\ref{labround})

\subsection{doubleextended}
\label{labdoubleextended}
\noindent Names: \textbf{doubleextended}, \textbf{DE}\\
computes the nearest number with 64 bits of mantissa.\\
\noindent Description: \begin{itemize}

\item \\textbf{doubleextended} is a function that computes the nearest floating-point number with\n   64 bits of mantissa to a given number. Since it is a function, it can be\n   composed with other \\sollya functions such as \\textbf{exp}, \\textbf{sin}, etc.\n
\item It does not handle subnormal numbers. The range of possible exponents is the \n   range used for all numbers represented in \\sollya (e.g. basically the range \n   used in the library MPFR).\n
\item Since it is a function and not a command, its behavior is a bit different from \n   the behavior of \\textbf{round}(x,64,RN) even if the result is exactly the same.\n   \\textbf{round}(x,64,RN) is immediately evaluated whereas \\textbf{doubleextended}(x) can be composed \n   with other functions (and thus be plotted and so on).\n
\item Be aware that \\textbf{doubleextended} cannot be used as a constant to represent a format in the\n   commands \\textbf{roundcoefficients} and \\textbf{implementpoly} (contrary to \\textbf{D}, \\textbf{DD},and \\textbf{TD}). However, it\n   can be used in \\textbf{round}.\n\end{itemize}
\noindent Example 1: 
\begin{center}\begin{minipage}{15cm}\begin{Verbatim}[frame=single]
\end{Verbatim}
\end{minipage}\end{center}
\noindent Example 2: 
\begin{center}\begin{minipage}{15cm}\begin{Verbatim}[frame=single]
\end{Verbatim}
\end{minipage}\end{center}
\noindent Example 3: 
\begin{center}\begin{minipage}{15cm}\begin{Verbatim}[frame=single]
\end{Verbatim}
\end{minipage}\end{center}
See also: \textbf{roundcoefficients} (\ref{labroundcoefficients}), \textbf{single} (\ref{labsingle}), \textbf{double} (\ref{labdouble}), \textbf{doubledouble} (\ref{labdoubledouble}), \textbf{tripledouble} (\ref{labtripledouble}), \textbf{round} (\ref{labround})

\subsection{double}
\label{labdouble}
\noindent Names: \textbf{double}, \textbf{D}\\
rounding to the nearest IEEE 754 double (binary64).\\
\noindent Description: \begin{itemize}

\item \\textbf{double} is both a function and a constant.\n
\item As a function, it rounds its argument to the nearest IEEE 754 double precision (i.e. IEEE754-2008 binary64) number.\n   Subnormal numbers are supported as well as standard numbers: it is the real\n   rounding described in the standard.\n
\item As a constant, it symbolizes the double precision format. It is used in \n   contexts when a precision format is necessary, e.g. in the commands \n   \\textbf{round}, \\textbf{roundcoefficients} and \\textbf{implementpoly}.\n   See the corresponding help pages for examples.\n\end{itemize}
\noindent Example 1: 
\begin{center}\begin{minipage}{15cm}\begin{Verbatim}[frame=single]
\end{Verbatim}
\end{minipage}\end{center}
See also: \textbf{single} (\ref{labsingle}), \textbf{printhexa} (\ref{labprinthexa}), \textbf{doubleextended} (\ref{labdoubleextended}), \textbf{doubledouble} (\ref{labdoubledouble}), \textbf{tripledouble} (\ref{labtripledouble}), \textbf{roundcoefficients} (\ref{labroundcoefficients}), \textbf{implementpoly} (\ref{labimplementpoly}), \textbf{round} (\ref{labround})

\subsection{ dyadic }
\noindent Name: \textbf{dyadic}\\
special value for global state \textbf{display}\\

\noindent Description: \begin{itemize}

\item \textbf{dyadic} is a special value used for the global state \textbf{display}.
   If the global state \textbf{display} is equal to \textbf{dyadic}, all data will
   be output in dyadic notation with numbers displayed in Gappa format.
   As any value it can be affected to a variable and stored in lists.
\end{itemize}
See also: \textbf{decimal}, \textbf{powers}, \textbf{hexadecimal}, \textbf{binary}

\subsection{$==$}
\label{labequal}
\noindent Name: \textbf{$==$}\\
equality test operator\\
\noindent Usage: 
\begin{center}
\emph{expr1} \textbf{$==$} \emph{expr2} : (\textsf{any type}, \textsf{any type}) $\rightarrow$ \textsf{boolean}\\
\end{center}
Parameters: 
\begin{itemize}
\item \emph{expr1} and \emph{expr2} represent expressions
\end{itemize}
\noindent Description: \begin{itemize}

\item The operator \\textbf{$==$} evaluates to true iff its operands \\emph{expr1} and\n   \\emph{expr2} are syntactically equal and different from \\textbf{error} or constant\n   expressions that are not constants and that evaluate to the same\n   floating-point number with the global precision \\textbf{prec}. The user should\n   be aware of the fact that because of floating-point evaluation, the\n   operator \\textbf{$==$} is not exactly the same as the mathematical\n   equality. Further remark that according to IEEE 754-2008 floating-point\n   rules, which \\sollya emulates, floating-point data which are NaN do not\n   compare equal to any other floating-point datum, including NaN. \n\end{itemize}
\noindent Example 1: 
\begin{center}\begin{minipage}{15cm}\begin{Verbatim}[frame=single]
\end{Verbatim}
\end{minipage}\end{center}
\noindent Example 2: 
\begin{center}\begin{minipage}{15cm}\begin{Verbatim}[frame=single]
\end{Verbatim}
\end{minipage}\end{center}
\noindent Example 3: 
\begin{center}\begin{minipage}{15cm}\begin{Verbatim}[frame=single]
\end{Verbatim}
\end{minipage}\end{center}
\noindent Example 4: 
\begin{center}\begin{minipage}{15cm}\begin{Verbatim}[frame=single]
\end{Verbatim}
\end{minipage}\end{center}
\noindent Example 5: 
\begin{center}\begin{minipage}{15cm}\begin{Verbatim}[frame=single]
\end{Verbatim}
\end{minipage}\end{center}
\noindent Example 6: 
\begin{center}\begin{minipage}{15cm}\begin{Verbatim}[frame=single]
\end{Verbatim}
\end{minipage}\end{center}
See also: \textbf{!$=$} (\ref{labneq}), \textbf{$>$} (\ref{labgt}), \textbf{$>=$} (\ref{labge}), \textbf{$<=$} (\ref{lable}), \textbf{$<$} (\ref{lablt}), \textbf{!} (\ref{labnot}), \textbf{$\&\&$} (\ref{laband}), \textbf{$||$} (\ref{labor}), \textbf{error} (\ref{laberror}), \textbf{prec} (\ref{labprec})

\subsection{erfc}
\label{laberfc}
\noindent Name: \textbf{erfc}\\
\phantom{aaa}the complementary error function.\\[0.2cm]
\noindent Library names:\\
\verb|   sollya_obj_t sollya_lib_erfc(sollya_obj_t)|\\
\verb|   sollya_obj_t sollya_lib_build_function_erfc(sollya_obj_t)|\\
\verb|   #define SOLLYA_ERFC(x) sollya_lib_build_function_erfc(x)|\\[0.2cm]
\noindent Description: \begin{itemize}

\item \textbf{erfc} is the complementary error function defined by $\mathrm{erfc}(x) = 1 - \mathrm{erf}(x)$.

\item It is defined for every real number $x$.
\end{itemize}
See also: \textbf{erf} (\ref{laberf})

\subsection{erf}
\label{laberf}
\noindent Name: \textbf{erf}\\
the error function.\\
\noindent Description: \begin{itemize}

\item \\textbf{erf} is the error function defined by:\n   $$\\mathrm{erf}(x) = \\frac{2}{\\sqrt{\\pi}} \\int_0^x e^{-t^2} {\\rm d}t.$$\n
\item It is defined for every real number x.\n\end{itemize}
See also: \textbf{erfc} (\ref{laberfc}), \textbf{exp} (\ref{labexp})

\subsection{error}
\label{laberror}
\noindent Name: \textbf{error}\\
expression representing an input that is wrongly typed or that cannot be executed\\
\noindent Usage: 
\begin{center}
\textbf{error} : \textsf{error}\\
\end{center}
\noindent Description: \begin{itemize}

\item The variable \\textbf{error} represents an input during the evaluation of\n   which a type or execution error has been detected or is to be\n   detected. Inputs that are syntactically correct but wrongly typed\n   evaluate to \\textbf{error} at some stage.  Inputs that are correctly typed\n   but containing commands that depend on side-effects that cannot be\n   performed or inputs that are wrongly typed at meta-level (cf. \\textbf{parse}),\n   evaluate to \\textbf{error}.\n    \n   Remark that in contrast to all other elements of the \\sollya language,\n   \\textbf{error} compares neither equal nor unequal to itself. This provides a\n   means of detecting syntax errors inside the \\sollya language itself\n   without introducing issues of two different wrongly typed inputs being\n   equal.\n\end{itemize}
\noindent Example 1: 
\begin{center}\begin{minipage}{15cm}\begin{Verbatim}[frame=single]
\end{Verbatim}
\end{minipage}\end{center}
\noindent Example 2: 
\begin{center}\begin{minipage}{15cm}\begin{Verbatim}[frame=single]
\end{Verbatim}
\end{minipage}\end{center}
\noindent Example 3: 
\begin{center}\begin{minipage}{15cm}\begin{Verbatim}[frame=single]
\end{Verbatim}
\end{minipage}\end{center}
\noindent Example 4: 
\begin{center}\begin{minipage}{15cm}\begin{Verbatim}[frame=single]
\end{Verbatim}
\end{minipage}\end{center}
See also: \textbf{void} (\ref{labvoid}), \textbf{parse} (\ref{labparse})

\subsection{evaluate}
\label{labevaluate}
\noindent Name: \textbf{evaluate}\\
evaluates a function at a constant point or in a range\\
\noindent Usage: 
\begin{center}
\textbf{evaluate}(\emph{function}, \emph{constant}) : (\textsf{function}, \textsf{constant}) $\rightarrow$ \textsf{constant} $|$ \textsf{range}\\
\textbf{evaluate}(\emph{function}, \emph{range}) : (\textsf{function}, \textsf{range}) $\rightarrow$ \textsf{range}\\
\textbf{evaluate}(\emph{function}, \emph{function2}) : (\textsf{function}, \textsf{function}) $\rightarrow$ \textsf{function}\\
\end{center}
Parameters: 
\begin{itemize}
\item \emph{function} represents a function
\item \emph{constant} represents a constant point
\item \emph{range} represents a range
\item \emph{function2} represents a function that is not constant
\end{itemize}
\noindent Description: \begin{itemize}

\item If its second argument is a constant \\emph{constant}, \\textbf{evaluate} evaluates\n   its first argument \\emph{function} at the point indicated by\n   \\emph{constant}. This evaluation is performed in a way that the result is a\n   faithful rounding of the real value of the \\emph{function} at \\emph{constant} to\n   the current global precision. If such a faithful rounding is not\n   possible, \\textbf{evaluate} returns a range surely encompassing the real value\n   of the function \\emph{function} at \\emph{constant}. If even interval evaluation\n   is not possible because the expression is undefined or numerically\n   unstable, NaN will be produced.\n
\item If its second argument is a range \\emph{range}, \\textbf{evaluate} evaluates its\n   first argument \\emph{function} by interval evaluation on this range\n   \\emph{range}. This ensures that the image domain of the function \\emph{function}\n   on the preimage domain \\emph{range} is surely enclosed in the returned\n   range.\n
\item In the case when the second argument is a range that is reduced to a\n   single point (such that $[1;\\,1]$ for instance), the evaluation\n   is performed in the same way as when the second argument is a constant but\n   it produces a range as a result: \\textbf{evaluate} automatically adjusts the precision\n   of the intern computations and returns a range that contains at most three floating-point\n   consecutive numbers in precision \\textbf{prec}. This correponds to the same accuracy\n   as a faithful rounding of the actual result. If such a faithful rounding\n   is not possible, \\textbf{evaluate} has the same behavior as in the case when the\n   second argument is a constant.\n
\item If its second argument is a function \\emph{function2} that is not a\n   constant, \\textbf{evaluate} replaces all occurrences of the free variable in\n   function \\emph{function} by function \\emph{function2}.\n\end{itemize}
\noindent Example 1: 
\begin{center}\begin{minipage}{15cm}\begin{Verbatim}[frame=single]
\end{Verbatim}
\end{minipage}\end{center}
\noindent Example 2: 
\begin{center}\begin{minipage}{15cm}\begin{Verbatim}[frame=single]
\end{Verbatim}
\end{minipage}\end{center}
\noindent Example 3: 
\begin{center}\begin{minipage}{15cm}\begin{Verbatim}[frame=single]
\end{Verbatim}
\end{minipage}\end{center}
\noindent Example 4: 
\begin{center}\begin{minipage}{15cm}\begin{Verbatim}[frame=single]
\end{Verbatim}
\end{minipage}\end{center}
\noindent Example 5: 
\begin{center}\begin{minipage}{15cm}\begin{Verbatim}[frame=single]
\end{Verbatim}
\end{minipage}\end{center}
See also: \textbf{isevaluable} (\ref{labisevaluable})

\subsection{execute}
\label{labexecute}
\noindent Name: \textbf{execute}\\
executes the content of a file\\
\noindent Usage: 
\begin{center}
\textbf{execute}(\emph{filename}) : \textsf{string} $\rightarrow$ \textsf{void}\\
\end{center}
Parameters: 
\begin{itemize}
\item \emph{filename} is a string representing a file name
\end{itemize}
\noindent Description: \begin{itemize}

\item \\textbf{execute} opens the file indicated by \\emph{filename}, and executes the sequence of \n   commands it contains. This command is evaluated at execution time: this way you\n   can modify the file \\emph{filename} (for instance using \\textbf{bashexecute}) and execute it\n   just after.\n
\item If \\emph{filename} contains a command \\textbf{execute}, it will be executed recursively.\n
\item If \\emph{filename} contains a call to \\textbf{restart}, it will be neglected.\n
\item If \\emph{filename} contains a call to \\textbf{quit}, the commands following \\textbf{quit}\n   in \\emph{filename} will be neglected.\n\end{itemize}
\noindent Example 1: 
\begin{center}\begin{minipage}{15cm}\begin{Verbatim}[frame=single]
\end{Verbatim}
\end{minipage}\end{center}
\noindent Example 2: 
\begin{center}\begin{minipage}{15cm}\begin{Verbatim}[frame=single]
\end{Verbatim}
\end{minipage}\end{center}
\noindent Example 3: 
\begin{center}\begin{minipage}{15cm}\begin{Verbatim}[frame=single]
\end{Verbatim}
\end{minipage}\end{center}
See also: \textbf{parse} (\ref{labparse}), \textbf{readfile} (\ref{labreadfile}), \textbf{write} (\ref{labwrite}), \textbf{print} (\ref{labprint}), \textbf{bashexecute} (\ref{labbashexecute})

\subsection{expand}
\label{labexpand}
\noindent Name: \textbf{expand}\\
expands polynomial subexpressions\\
\noindent Usage: 
\begin{center}
\textbf{expand}(\emph{function}) : \textsf{function} $\rightarrow$ \textsf{function}\\
\end{center}
Parameters: 
\begin{itemize}
\item \emph{function} represents a function
\end{itemize}
\noindent Description: \begin{itemize}

\item \\textbf{expand}(\\emph{function}) expands all polynomial subexpressions in function\n   \\emph{function} as far as possible. Factors of sums are multiplied out,\n   power operators with constant positive integer exponents are replaced\n   by multiplications and divisions are multiplied out, i.e. denomiators\n   are brought at the most interior point of expressions.\n\end{itemize}
\noindent Example 1: 
\begin{center}\begin{minipage}{15cm}\begin{Verbatim}[frame=single]
\end{Verbatim}
\end{minipage}\end{center}
\noindent Example 2: 
\begin{center}\begin{minipage}{15cm}\begin{Verbatim}[frame=single]
\end{Verbatim}
\end{minipage}\end{center}
\noindent Example 3: 
\begin{center}\begin{minipage}{15cm}\begin{Verbatim}[frame=single]
\end{Verbatim}
\end{minipage}\end{center}
See also: \textbf{simplify} (\ref{labsimplify}), \textbf{simplifysafe} (\ref{labsimplifysafe}), \textbf{horner} (\ref{labhorner})

\subsection{expm1}
\label{labexpm1}
\noindent Name: \textbf{expm1}\\
\phantom{aaa}shifted exponential function.\\[0.2cm]
\noindent Library names:\\
\verb|   sollya_obj_t sollya_lib_expm1(sollya_obj_t)|\\
\verb|   sollya_obj_t sollya_lib_build_function_expm1(sollya_obj_t)|\\
\verb|   #define SOLLYA_EXPM1(x) sollya_lib_build_function_expm1(x)|\\[0.2cm]
\noindent Description: \begin{itemize}

\item \textbf{expm1} is defined by ${\rm expm1}(x) = \exp(x)-1$.

\item It is defined for every real number $x$.
\end{itemize}
See also: \textbf{exp} (\ref{labexp})

\input{exponent}
\subsection{exp}
\label{labexp}
\noindent Name: \textbf{exp}\\
\phantom{aaa}the exponential function.\\[0.2cm]
\noindent Library names:\\
\verb|   sollya_obj_t sollya_lib_exp(sollya_obj_t)|\\
\verb|   sollya_obj_t sollya_lib_build_function_exp(sollya_obj_t)|\\
\verb|   #define SOLLYA_EXP(x) sollya_lib_build_function_exp(x)|\\[0.2cm]
\noindent Description: \begin{itemize}

\item \textbf{exp} is the usual exponential function defined as the solution of the
   ordinary differential equation $y'=y$ with $y(0)=1$.

\item \textbf{exp}(x) is defined for every real number $x$.
\end{itemize}
See also: \textbf{exp} (\ref{labexp}), \textbf{log} (\ref{lablog})

\subsection{externalplot}
\label{labexternalplot}
\noindent Name: \textbf{externalplot}\\
plots the error of an external code with regard to a function\\
\noindent Usage: 
\begin{center}
\textbf{externalplot}(\emph{filename}, \emph{mode}, \emph{function}, \emph{range}, \emph{precision}) : (\textsf{string}, \textsf{absolute$|$relative}, \textsf{function}, \textsf{range}, \textsf{integer}) $\rightarrow$ \textsf{void}\\
\textbf{externalplot}(\emph{filename}, \emph{mode}, \emph{function}, \emph{range}, \emph{precision}, \emph{perturb}) : (\textsf{string}, \textsf{absolute$|$relative}, \textsf{function}, \textsf{range}, \textsf{integer}, \textsf{perturb}) $\rightarrow$ \textsf{void}\\
\textbf{externalplot}(\emph{filename}, \emph{mode}, \emph{function}, \emph{range}, \emph{precision}, \emph{plot mode}, \emph{result filename}) : (\textsf{string}, \textsf{absolute$|$relative}, \textsf{function}, \textsf{range}, \textsf{integer}, \textsf{file$|$postscript$|$postscriptfile}, \textsf{string}) $\rightarrow$ \textsf{void}\\
\textbf{externalplot}(\emph{filename}, \emph{mode}, \emph{function}, \emph{range}, \emph{precision}, \emph{perturb}, \emph{plot mode}, \emph{result filename}) : (\textsf{string}, \textsf{absolute$|$relative}, \textsf{function}, \textsf{range}, \textsf{integer}, \textsf{perturb}, \textsf{file$|$postscript$|$postscriptfile}, \textsf{string}) $\rightarrow$ \textsf{void}\\
\end{center}
\noindent Description: \begin{itemize}

\item The command \\textbf{externalplot} plots the error of an external function\n   evaluation code sequence implemented in the object file named\n   \\emph{filename} with regard to the function \\emph{function}.  If \\emph{mode}\n   evaluates to \\emph{absolute}, the difference of both functions is\n   considered as an error function; if \\emph{mode} evaluates to \\emph{relative},\n   the difference is divided by the function \\emph{function}. The resulting\n   error function is plotted on all floating-point numbers with\n   \\emph{precision} significant mantissa bits in the range \\emph{range}. \n    \n   If the sixth argument of the command \\textbf{externalplot} is given and evaluates to\n   \\textbf{perturb}, each of the floating-point numbers the function is evaluated at gets perturbed by a\n   random value that is uniformly distributed in $\\pm1$ ulp\n   around the original \\emph{precision} bit floating-point variable.\n    \n   If a sixth and seventh argument, respectively a seventh and eighth\n   argument in the presence of \\textbf{perturb} as a sixth argument, are given\n   that evaluate to a variable of type \\textsf{file$|$postscript$|$postscriptfile} respectively to a\n   character sequence of type \\textsf{string}, \\textbf{externalplot} will plot\n   (additionally) to a file in the same way as the command \\textbf{plot}\n   does. See \\textbf{plot} for details.\n    \n   The external function evaluation code given in the object file name\n   \\emph{filename} is supposed to define a function name \\texttt{f} as\n   follows (here in C syntax): \\texttt{void f(mpfr\\_t rop, mpfr\\_ op)}. \n   This function is supposed to evaluate \\texttt{op} with an accuracy corresponding\n   to the precision of \\texttt{rop} and assign this value to\n   \\texttt{rop}.\n\end{itemize}
\noindent Example 1: 
\begin{center}\begin{minipage}{15cm}\begin{Verbatim}[frame=single]
\end{Verbatim}
\end{minipage}\end{center}
See also: \textbf{plot} (\ref{labplot}), \textbf{asciiplot} (\ref{labasciiplot}), \textbf{perturb} (\ref{labperturb}), \textbf{absolute} (\ref{lababsolute}), \textbf{relative} (\ref{labrelative}), \textbf{file} (\ref{labfile}), \textbf{postscript} (\ref{labpostscript}), \textbf{postscriptfile} (\ref{labpostscriptfile}), \textbf{bashexecute} (\ref{labbashexecute}), \textbf{externalproc} (\ref{labexternalproc}), \textbf{library} (\ref{lablibrary})

\subsection{externalproc}
\label{labexternalproc}
\noindent Name: \textbf{externalproc}\\
binds an external code to a Sollya procedure\\

\noindent Usage: 
\begin{center}
\textbf{externalproc}(\emph{identifier}, \emph{filename}, \emph{argumenttype} $->$ \emph{resulttype}) : (\textsf{identifier type}, \textsf{string}, \textsf{type type}, \textsf{type type}) $\rightarrow$ \textsf{void}\\
\end{center}
Parameters: 
\begin{itemize}
\item \emph{identifier} represents the identifier the code is to be bound to
\item \emph{filename} of type \textsf{string} represents the name of the object file where the code of procedure can be found
\item \emph{argumenttype} represents a definition of the types of the arguments of the Sollya procedure and the external code
\item \emph{resulttype} represents a definition of the result type of the external code
\end{itemize}
\noindent Description: \begin{itemize}

\item \textbf{externalproc} allows for binding the Sollya identifier
   \emph{identifier} to an external code.  After this binding, when Sollya
   encounters \emph{identifier} applied to a list of actual parameters, it
   will evaluate these parameters and call the external code with these
   parameters. If the external code indicated success, it will receive
   the result produced by the external code, transform it to Sollya's
   iternal representation and return it.
    
   In order to allow correct evaluation and typing of the data in
   parameter and in result to be passed to and received from the external
   code, \textbf{externalproc} has a third parameter \emph{argumenttype} $->$ \emph{resulttype}.
   Both \emph{argumenttype} and \emph{resulttype} are one of void, constant,
   function, range, integer, string, boolean, list of constant, list of
   function, list of range, list of integer, list of string, list of
   boolean.
    
   If upon a usage of a procedure bound to an external procedure the type
   of the actual parameters given or its number is not correct, Sollya
   produces a type error. An external function not applied to arguments
   represents itself and prints out with its argument and result types.
    
   The external function is supposed to return an integer indicating
   success.  It returns its result depending on its Sollya result type
   as follows. Here, the external procedure is assumed to be implemented
   as a C function.
    
   If the Sollya result type is void, the C function has no pointer
   argument for the result.  If the Sollya result type is constant, the
   first argument of the C function is of C type \texttt{mpfr\_t *}, the result is
   returned by affecting the MPFR variable.  If the Sollya result type
   is function, the first argument of the C function is of C type §§node
   **§\texttt{node **}§§, the result is returned by the \texttt{node *} pointed with a new \texttt{node *}.
   If the Sollya result type is range, the first argument of the C
   function is of C type \texttt{mpfi\_t *}, the result is returned by affecting
   the MPFI variable.  If the Sollya result type is integer, the first
   argument of the C function is of C type \texttt{int *}, the result is returned
   by affecting the int variable.  If the Sollya result type is string,
   the first argument of the C function is of C type \texttt{char **}, the result
   is returned by the \texttt{char *} pointed with a new \texttt{char *}.  If the Sollya
   result type is boolean, the first argument of the C function is of C
   type \texttt{int *}, the result is returned by affecting the int variable with
   a boolean value.  If the Sollya result type is list of type, the
   first argument of the C function is of C type \texttt{chain **}, the result is
   returned by the \texttt{chain *} pointed with a new \texttt{chain *}.  This chain
   contains for Sollya type constant pointers \texttt{mpfr\_t *} to new MPFR
   variables, for Sollya type function pointers \texttt{node *} to new nodes, for
   Sollya type range pointers \texttt{mpfi\_t *}  to new MPFI variables, for
   Sollya type integer pointers \texttt{int *} to new int variables for Sollya
   type string pointers \texttt{char *} to new \texttt{char *} variables and for Sollya
   type boolean pointers \texttt{int *} to new int variables representing boolean
   values.
    	       
   The external procedure affects its possible pointer argument if and
   only if it succeeds.  This means, if the function returns an integer
   indicating failure, it does not leak any memory to the encompassing
   environment.
    
   The external procedure receives its arguments as follows: If the
   Sollya argument type is void, no argument array is given.  Otherwise
   the C function receives a C \texttt{void **} argument representing an array of
   size equal to the arity of the function where each entry (of C type
   \texttt{void *}) represents a value with a C type depending on the
   corresponding Sollya type. If the Sollya type is constant, the C
   type the \texttt{void *} is to be casted to is \texttt{mpfr\_t *}.  If the Sollya type
   is function, the C type the \texttt{void *} is to be casted to is \texttt{node *}.  If
   the Sollya type is range, the C type the \texttt{void *} is to be casted to is
   \texttt{mpfi\_t *}.  If the Sollya type is integer, the C type the \texttt{void *} is to
   be casted to is \texttt{int *}.  If the Sollya type is string, the C type the
   \texttt{void *} is to be casted to is \texttt{char *}.  If the Sollya type is boolean,
   the C type the \texttt{void *} is to be casted to is \texttt{int *}.  If the Sollya
   type is list of type, the C type the \texttt{void *} is to be casted to is
   \texttt{chain *}.  Here depending on type, the values in the chain are to be
   casted to \texttt{mpfr\_t *}  for Sollya type constant, \texttt{node *} for Sollya type
   function, \texttt{mpfi\_t *} for Sollya type range, \texttt{int *} for Sollya type
   integer, \texttt{char *} for Sollya type string and \texttt{int *} for Sollya type
   boolean.
    
   The external procedure is not supposed to alter the memory pointed by
   its array argument \texttt{void **}.
    
   In both directions (argument and result values), empty lists are
   represented by \texttt{chain * NULL} pointers.
    
   In contrast to internal procedures, externally bounded procedures can
   be considered as objects inside Sollya that can be assigned to other
   variables, stored in list etc.
\end{itemize}
\noindent Example 1: 
\begin{center}\begin{minipage}{15cm}\begin{Verbatim}[frame=single]
> bashexecute("gcc -fPIC -Wall -c externalprocexample.c");
> bashexecute("gcc -fPIC -shared -o externalprocexample externalprocexample.o");

> externalproc(foo, "./externalprocexample", (integer, integer) -> integer);
> foo;
foo(integer, integer) -> integer
> foo(5, 6);
11
> verbosity = 1!;
> foo();
Warning: at least one of the given expressions or a subexpression is not correct
ly typed
or its evaluation has failed because of some error on a side-effect.
error
> a = foo;
> a(5,6);
11
\end{Verbatim}
\end{minipage}\end{center}
See also: \textbf{library} (\ref{lablibrary}), \textbf{externalplot} (\ref{labexternalplot}), \textbf{bashexecute} (\ref{labbashexecute}), \textbf{void} (\ref{labvoid}), \textbf{constant} (\ref{labconstant}), \textbf{function} (\ref{labfunction}), \textbf{range} (\ref{labrange}), \textbf{integer} (\ref{labinteger}), \textbf{string} (\ref{labstring}), \textbf{boolean} (\ref{labboolean}), \textbf{list of} (\ref{lablistof})

\subsection{false}
\label{labfalse}
\noindent Name: \textbf{false}\\
\phantom{aaa}the boolean value representing the false.\\[0.2cm]
\noindent Library names:\\
\verb|   sollya_obj_t sollya_lib_false()|\\
\verb|   int sollya_lib_is_false(sollya_obj_t)|\\[0.2cm]
\noindent Description: \begin{itemize}

\item \textbf{false} is the usual boolean value.
\end{itemize}
\noindent Example 1: 
\begin{center}\begin{minipage}{15cm}\begin{Verbatim}[frame=single]
> true && false;
false
> 2<1;
false
\end{Verbatim}
\end{minipage}\end{center}
See also: \textbf{true} (\ref{labtrue}), \textbf{$\&\&$} (\ref{laband}), \textbf{$||$} (\ref{labor})

\subsection{file}
\label{labfile}
\noindent Name: \textbf{file}\\
\phantom{aaa}special value for commands \textbf{plot} and \textbf{externalplot}\\[0.2cm]
\noindent Library names:\\
\verb|   sollya_obj_t sollya_lib_file()|\\
\verb|   int sollya_lib_is_file(sollya_obj_t)|\\[0.2cm]
\noindent Description: \begin{itemize}

\item \textbf{file} is a special value used in commands \textbf{plot} and \textbf{externalplot} to save
   the result of the command in a data file.

\item As any value it can be affected to a variable and stored in lists.
\end{itemize}
\noindent Example 1: 
\begin{center}\begin{minipage}{15cm}\begin{Verbatim}[frame=single]
> savemode=file;
> name="plotSinCos";
> plot(sin(x),0,cos(x),[-Pi,Pi],savemode, name);
\end{Verbatim}
\end{minipage}\end{center}
See also: \textbf{externalplot} (\ref{labexternalplot}), \textbf{plot} (\ref{labplot}), \textbf{postscript} (\ref{labpostscript}), \textbf{postscriptfile} (\ref{labpostscriptfile})

\subsection{ findzeros }
\noindent Name: \textbf{findzeros}\\
gives a list of intervals containing all zeros of a function on an interval.\\

\noindent Usage: 
\begin{center}
\textbf{findzeros}(\emph{f},\emph{I}) : (\textsf{function}, \textsf{range}) $\rightarrow$ \textsf{list}\\
\end{center}
Parameters: 
\begin{itemize}
\item \emph{f} is a function.
\item \emph{I} is an interval.
\end{itemize}
\noindent Description: \begin{itemize}

\item \textbf{findzeros}(\emph{f},\emph{I}) returns a list of intervals \emph{I1}, ... ,\emph{In} such that, for 
   every zero $z$ of $f$, there exists some $k$ such that $z \in I_k$.

\item The list may contain intervals \emph{Ik} that do not contain any zero of \emph{f}.
   An interval \emph{Ik} may contain many zeros of \emph{f}.

\item This command is ment for cases when safety is critical. If you want to be sure
   not to forget any zero, use \textbf{findzeros}. However, if you just want to know 
   numerical values for the zeros of \emph{f}, \textbf{dirtyfindzeros} should be quite 
   satisfactory and a lot faster.

\item If $\delta$ denotes the value of global variable \textbf{diam}, the algorithm ensures
   that for each $k$, $|I_k| \le \delta \cdot |I|$.

\item The algorithm used is basically a bisection algorithm. It is the same algorithm
   that the one used for \textbf{infnorm}. See the help page of this command for more 
   details. In short, the behavior of the algorithm depends on global variables
   \textbf{prec}, \textbf{diam}, \textbf{taylorrecursions} and \textbf{hopitalrecursions}.
\end{itemize}
\noindent Example 1: 
\begin{center}\begin{minipage}{15cm}\begin{Verbatim}[frame=single]
> findzeros(sin(x),[-5;5]);
[|[-0.314208984375e1;-0.3140869140625e1], [-0.1220703125e-2;0.1220703125e-2], [0
.3140869140625e1;0.314208984375e1]|]
> diam=1e-10!;
> findzeros(sin(x),[-5;5]);
[|[-0.314159265370108187198638916015625e1;-0.3141592652536928653717041015625e1],
 [-0.116415321826934814453125e-8;0.116415321826934814453125e-8], [0.314159265253
6928653717041015625e1;0.314159265370108187198638916015625e1]|]
\end{Verbatim}
\end{minipage}\end{center}
See also: \textbf{dirtyfindzeros}, \textbf{infnorm}, \textbf{prec}, \textbf{diam}, \textbf{taylorrecursions}, \textbf{hopitalrecursions}

\subsection{fixed}
\label{labfixed}
\noindent Name: \textbf{fixed}\\
indicates that fixed-point formats should be used for \textbf{fpminimax}\\
\noindent Usage: 
\begin{center}
\textbf{fixed} : \textsf{fixed$|$floating}\\
\end{center}
\noindent Description: \begin{itemize}

\item The use of \\textbf{fixed} in the command \\textbf{fpminimax} indicates that the list of\n   formats given as argument is to be considered to be a list of fixed-point\n   formats.\n   See \\textbf{fpminimax} for details.\n\end{itemize}
\noindent Example 1: 
\begin{center}\begin{minipage}{15cm}\begin{Verbatim}[frame=single]
\end{Verbatim}
\end{minipage}\end{center}
See also: \textbf{fpminimax} (\ref{labfpminimax}), \textbf{floating} (\ref{labfloating})

\subsection{floating}
\label{labfloating}
\noindent Name: \textbf{floating}\\
indicates that floating-point formats should be used for \textbf{fpminimax}\\
\noindent Usage: 
\begin{center}
\textbf{floating} : \textsf{fixed$|$floating}\\
\end{center}
\noindent Description: \begin{itemize}

\item The use of \\textbf{floating} in the command \\textbf{fpminimax} indicates that the list of\n   formats given as argument is to be considered to be a list of floating-point\n   formats.\n   See \\textbf{fpminimax} for details.\n\end{itemize}
\noindent Example 1: 
\begin{center}\begin{minipage}{15cm}\begin{Verbatim}[frame=single]
\end{Verbatim}
\end{minipage}\end{center}
See also: \textbf{fpminimax} (\ref{labfpminimax}), \textbf{fixed} (\ref{labfixed})

\subsection{floor}
\label{labfloor}
\noindent Name: \textbf{floor}\\
\phantom{aaa}the usual function floor.\\[0.2cm]
\noindent Library names:\\
\verb|   sollya_obj_t sollya_lib_floor(sollya_obj_t)|\\
\verb|   sollya_obj_t sollya_lib_build_function_floor(sollya_obj_t)|\\
\verb|   #define SOLLYA_FLOOR(x) sollya_lib_build_function_floor(x)|\\[0.2cm]
\noindent Description: \begin{itemize}

\item \textbf{floor} is defined as usual: \textbf{floor}($x$) is the greatest integer y such that $y \le x$.

\item It is defined for every real number $x$.
\end{itemize}
See also: \textbf{ceil} (\ref{labceil}), \textbf{nearestint} (\ref{labnearestint}), \textbf{round} (\ref{labround}), \textbf{RD} (\ref{labrd})

\subsection{fpminimax}
\label{labfpminimax}
\noindent Name: \textbf{fpminimax}\\
computes a good polynomial approximation with fixed-point or floating-point coefficients\\
\noindent Usage: 
\begin{center}
\textbf{fpminimax}(\emph{f}, \emph{n}, \emph{formats}, \emph{range}, \emph{indic1}, \emph{indic2}, \emph{indic3}, \emph{P}) : (\textsf{function}, \textsf{integer}, \textsf{list}, \textsf{range}, \textsf{absolute$|$relative} $|$ \textsf{fixed$|$floating} $|$ \textsf{function}, \textsf{absolute$|$relative} $|$ \textsf{fixed$|$floating} $|$ \textsf{function}, \textsf{absolute$|$relative} $|$ \textsf{fixed$|$floating} $|$ \textsf{function}, \textsf{function}) $\rightarrow$ \textsf{function}\\
\textbf{fpminimax}(\emph{f}, \emph{monomials}, \emph{formats}, \emph{range}, \emph{indic1}, \emph{indic2}, \emph{indic3}, \emph{P}) : (\textsf{function}, \textsf{list}, \textsf{list}, \textsf{range},  \textsf{absolute$|$relative} $|$ \textsf{fixed$|$floating} $|$ \textsf{function}, \textsf{absolute$|$relative} $|$ \textsf{fixed$|$floating} $|$ \textsf{function}, \textsf{absolute$|$relative} $|$ \textsf{fixed$|$floating} $|$ \textsf{function}, \textsf{function}) $\rightarrow$ \textsf{function}\\
\textbf{fpminimax}(\emph{f}, \emph{n}, \emph{formats}, \emph{L}, \emph{indic1}, \emph{indic2}, \emph{indic3}, \emph{P}) : (\textsf{function}, \textsf{integer}, \textsf{list}, \textsf{list},  \textsf{absolute$|$relative} $|$ \textsf{fixed$|$floating} $|$ \textsf{function}, \textsf{absolute$|$relative} $|$ \textsf{fixed$|$floating} $|$ \textsf{function}, \textsf{absolute$|$relative} $|$ \textsf{fixed$|$floating} $|$ \textsf{function}, \textsf{function}) $\rightarrow$ \textsf{function}\\
\textbf{fpminimax}(\emph{f}, \emph{monomials}, \emph{formats}, \emph{L}, \emph{indic1}, \emph{indic2}, \emph{indic3}, \emph{P}) : (\textsf{function}, \textsf{list}, \textsf{list}, \textsf{list},  \textsf{absolute$|$relative} $|$ \textsf{fixed$|$floating} $|$ \textsf{function}, \textsf{absolute$|$relative} $|$ \textsf{fixed$|$floating} $|$ \textsf{function}, \textsf{absolute$|$relative} $|$ \textsf{fixed$|$floating} $|$ \textsf{function}, \textsf{function}) $\rightarrow$ \textsf{function}\\
\end{center}
Parameters: 
\begin{itemize}
\item \emph{f} is the function to be approximated
\item \emph{n} is the degree of the polynomial that must approximate \emph{f}
\item \emph{monomials} is the list of monomials that must be used to represent the polynomial that approximates~\emph{f}
\item \emph{formats} is a list indicating the formats that the coefficients of the polynomial must have
\item \emph{range} is the interval where the function must be approximated
\item \emph{L} is a list of interpolation points used by the method
\item \emph{indic1} (optional) is one of the optional indication parameters. See the detailed description below.
\item \emph{indic2} (optional) is one of the optional indication parameters. See the detailed description below.
\item \emph{indic3} (optional) is one of the optional indication parameters. See the detailed description below.
\item \emph{P} (optional) is the minimax polynomial to be considered for solving the problem.
\end{itemize}
\noindent Description: \begin{itemize}

\item \\textbf{fpminimax} uses a heuristic (but practically efficient) method to find a good\n   polynomial approximation of a function \\emph{f} on an interval \\emph{range}. It \n   implements the method published in the article:\\\\\n   Efficient polynomial $L^\\infty$-approximations\\\\ \n   Nicolas Brisebarre and Sylvain Chevillard\\\\\n   Proceedings of the 18th IEEE Symposium on Computer Arithmetic (ARITH 18)\\\\\n   pp. 169-176\n
\item The basic usage of this command is \\textbf{fpminimax}(\\emph{f}, \\emph{n}, \\emph{formats}, \\emph{range}).\n   It computes a polynomial approximation of $f$ with degree at most $n$\n   on the interval \\emph{range}. \\emph{formats} is a list of integers or format types \n   (such as \\textbf{double}, \\textbf{doubledouble}, etc.). The polynomial returned by the\n   command has its coefficients that fit the formats indications. For \n   instance, if formats[0] is 35, the coefficient of degree 0 of the \n   polynomial will fit a floating-point format of 35 bits. If formats[1] \n   is D, the coefficient of degree 1 will be representable by a floating-point\n   number with a precision of 53 bits (which is not necessarily an IEEE 754 double\n   precision number. See the remark below), etc.\n
\item The second argument may be either an integer or a list of integers\n   interpreted as the list of desired monomials. For instance, the list\n   $[|0,\\,2,\\,4,\\,6|]$ indicates that the polynomial must be even and of\n   degree at most 6. Giving an integer $n$ as second argument is equivalent\n   as giving $[|0,\\,\\dots,\\,n|]$.\\\\\n   The list of formats is interpreted with respect to the list of monomials. For\n   instance, if the list of monomials is $[|0,\\,2,\\,4,\\,6|]$ and the list\n   of formats is $[|161,\\,107,\\,53,\\,24|]$, the coefficients of degree 0 is \n   searched as a floating-point number with precision 161, the coefficient of \n   degree 2 is searched as a number of precision 107, and so on.\n
\item The list of formats may contain either integers or format types (\\textbf{double},\n   \\textbf{doubledouble}, \\textbf{tripledouble} and \\textbf{doubleextended}). The list may be too large\n   or even infinite. Only the first indications will be considered. For \n   instance, for a degree $n$ polynomial, $\\mathrm{formats}[n+1]$ and above will\n   be discarded. This lets one use elliptical indications for the last\n   coefficients.\n
\item The floating-point coefficients considered by \\textbf{fpminimax} do not have an\n   exponent range. In particular, in the format list, \\textbf{double} or 53 does not\n   imply that the corresponding coefficient is an IEEE-754 double.\n
\item By default, the list of formats is interpreted as a list of floating-point\n   formats. This may be changed by passing \\textbf{fixed} as an optional argument (see\n   below). Let us take an example: \\textbf{fpminimax}(f, 2, [107, DD, 53], [0;1]).\n   Here the optional argument is missing (we could have set it to \\textbf{floating}).\n   Thus, \\textbf{fpminimax} will search for a polynomial of degree 2 with a constant \n   coefficient that is a 107 bits floating-point number, etc.\\\\\n   Currently, \\textbf{doubledouble} is just a synonym for 107 and \\textbf{tripledouble} a \n   synonym for 161. This behavior may change in the future (taking into\n   account the fact that some double-doubles are not representable with\n   107 bits).\\\\\n   Second example: \\textbf{fpminimax}(f, 2, [25, 18, 30], [0;1], \\textbf{fixed}).\n   In this case, \\textbf{fpminimax} will search for a polynomial of degree 2 with a\n   constant coefficient of the form $m/2^{25}$ where $m$ is an\n   integer. In other words, it is a fixed-point number with 25 bits after\n   the point. Note that even with argument \\textbf{fixed}, the formats list is \n   allowed to contain \\textbf{double}, \\textbf{doubledouble} or \\textbf{tripledouble}. In this this\n   case, it is just a synonym for 53, 107 or 161. This is deprecated and may\n   change in the future.\n
\item The fourth argument may be a range or a list. Lists are for advanced users\n   that know what they are doing. The core of the  method is a kind of\n   approximated interpolation. The list given here is a list of points that\n   must be considered for the interpolation. It must contain at least as \n   many points as unknown coefficients. If you give a list, it is also \n   recommended that you provide the minimax polynomial as last argument.\n   If you give a range, the list of points will be automatically computed.\n
\item The fifth, sixth and seventh arguments are optional. By default, \\textbf{fpminimax}\n   will approximate $f$ while optimizing the relative error, and interpreting\n   the list of formats as a list of floating-point formats.\\\\\n   This default behavior may be changed with these optional arguments. You\n   may provide zero, one, two or three of the arguments in any order.\n   This lets the user indicate only the non-default arguments.\\\\\n   The three possible arguments are: \\begin{itemize}\n   \\item \\textbf{relative} or \\textbf{absolute}: the error to be optimized;\n   \\item \\textbf{floating} or \\textbf{fixed}: formats of the coefficients;\n   \\item a constrained part $q$.\n   \\end{itemize}\n   The constrained part lets the user assign in advance some of the\n   coefficients. For instance, for approximating $\\exp(x)$, it may\n   be interesting to search for a polynomial $p$ of the form\n                   $$p = 1 + x + \\frac{x^2}{2} + a_3 x^3 + a_4 x^4.$$\n   Thus, there is a constrained part $q = 1 + x + x^2/2$ and the unknown\n   polynomial should be considered in the monomial basis $[|3, 4|]$.\n   Calling \\textbf{fpminimax} with monomial basis $[|3,\\,4|]$ and constrained\n   part $q$, will return a polynomial with the right form.\n
\item The last argument is for advanced users. It is the minimax polynomial that\n   approximates the function $f$ in the monomial basis. If it is not given\n   this polynomial will be automatically computed by \\textbf{fpminimax}.\n   \\\\\n   This minimax polynomial is used to compute the list of interpolation\n   points required by the method. In general, you do not have to provide this\n   argument. But if you want to obtain several polynomials of the same degree\n   that approximate the same function on the same range, just changing the\n   formats, you should probably consider computing only once the minimax\n   polynomial and the list of points instead of letting \\textbf{fpminimax} recompute\n   them each time.\n   \\\\\n   Note that in the case when a constrained part is given, the minimax \n   polynomial must take that into account. For instance, in the previous\n   example, the minimax would be obtained by the following command:\n          \\begin{center}\\verb~P = remez(1-(1+x+x^2/2)/exp(x), [|3,4|], range, 1/exp(x));~\\end{center}\n   Note that the constrained part is not to be added to $P$.\n
\item Note that \\textbf{fpminimax} internally computes a minimax polynomial (using\n   the same algorithm as \\textbf{remez} command). Thus \\textbf{fpminimax} may encounter\n   the same problems as \\textbf{remez}. In particular, it may be very slow \n   when Haar condition is not fulfilled. Another consequence is that\n   currently \\textbf{fpminimax} has to be run with a sufficiently high working precision.\n\end{itemize}
\noindent Example 1: 
\begin{center}\begin{minipage}{15cm}\begin{Verbatim}[frame=single]
\end{Verbatim}
\end{minipage}\end{center}
\noindent Example 2: 
\begin{center}\begin{minipage}{15cm}\begin{Verbatim}[frame=single]
\end{Verbatim}
\end{minipage}\end{center}
\noindent Example 3: 
\begin{center}\begin{minipage}{15cm}\begin{Verbatim}[frame=single]
\end{Verbatim}
\end{minipage}\end{center}
\noindent Example 4: 
\begin{center}\begin{minipage}{15cm}\begin{Verbatim}[frame=single]
\end{Verbatim}
\end{minipage}\end{center}
See also: \textbf{remez} (\ref{labremez}), \textbf{dirtyfindzeros} (\ref{labdirtyfindzeros}), \textbf{absolute} (\ref{lababsolute}), \textbf{relative} (\ref{labrelative}), \textbf{fixed} (\ref{labfixed}), \textbf{floating} (\ref{labfloating}), \textbf{default} (\ref{labdefault})

\subsection{fullparentheses}
\label{labfullparentheses}
\noindent Name: \textbf{fullparentheses}\\
activates, deactivates or inspects the state variable controlling output with full parenthesising\\
\noindent Usage: 
\begin{center}
\textbf{fullparentheses} = \emph{activation value} : \textsf{on$|$off} $\rightarrow$ \textsf{void}\\
\textbf{fullparentheses} = \emph{activation value} ! : \textsf{on$|$off} $\rightarrow$ \textsf{void}\\
\end{center}
Parameters: 
\begin{itemize}
\item \emph{activation value} represents \textbf{on} or \textbf{off}, i.e. activation or deactivation
\end{itemize}
\noindent Description: \begin{itemize}

\item An assignment \\textbf{fullparentheses} = \\emph{activation value}, where \\emph{activation value}\n   is one of \\textbf{on} or \\textbf{off}, activates respectively deactivates the output\n   of expressions with full parenthesising. In full parenthesising mode,\n   \\sollya commands like \\textbf{print}, \\textbf{write} and the implicit command when an\n   expression is given at the prompt will output expressions with\n   parenthesising at all places where it is necessary for expressions\n   containing infix operators to be parsed back with the same\n   result. Otherwise parentheses around associative operators are\n   omitted.\n    \n   If the assignment \\textbf{fullparentheses} = \\emph{activation value} is followed by an\n   exclamation mark, no message indicating the new state is\n   displayed. Otherwise the user is informed of the new state of the\n   global mode by an indication.\n\end{itemize}
\noindent Example 1: 
\begin{center}\begin{minipage}{15cm}\begin{Verbatim}[frame=single]
\end{Verbatim}
\end{minipage}\end{center}
See also: \textbf{print} (\ref{labprint}), \textbf{write} (\ref{labwrite}), \textbf{autosimplify} (\ref{labautosimplify})

\subsection{ function }
\noindent Name: \textbf{function}\\
keyword representing a \textsf{function} type \\

\noindent Usage: 
\begin{center}
\textbf{function} : \textsf{type type}\\
\end{center}
\noindent Description: \begin{itemize}

\item \textbf{function} represents the \textsf{function} type for declarations
   of external procedures by means of \textbf{externalproc}.
   Remark that in contrast to other indicators, type indicators like
   \textbf{function} cannot be handled outside the \textbf{externalproc} context.  In
   particular, they cannot be assigned to variables.
\end{itemize}
See also: \textbf{externalproc}, \textbf{boolean}, \textbf{constant}, \textbf{integer}, \textbf{list of}, \textbf{range}, \textbf{string}

\subsection{$>=$}
\label{labge}
\noindent Name: \textbf{$>=$}\\
greater-than-or-equal-to operator\\
\noindent Usage: 
\begin{center}
\emph{expr1} \textbf{$>=$} \emph{expr2} : (\textsf{constant}, \textsf{constant}) $\rightarrow$ \textsf{boolean}\\
\end{center}
Parameters: 
\begin{itemize}
\item \emph{expr1} and \emph{expr2} represent constant expressions
\end{itemize}
\noindent Description: \begin{itemize}

\item The operator \\textbf{$>=$} evaluates to true iff its operands \\emph{expr1} and\n   \\emph{expr2} evaluate to two floating-point numbers $a_1$\n   respectively $a_2$ with the global precision \\textbf{prec} and\n   $a_1$ is greater than or equal to $a_2$. The user should\n   be aware of the fact that because of floating-point evaluation, the\n   operator \\textbf{$>=$} is not exactly the same as the mathematical\n   operation \\emph{greater-than-or-equal-to}.\n\end{itemize}
\noindent Example 1: 
\begin{center}\begin{minipage}{15cm}\begin{Verbatim}[frame=single]
\end{Verbatim}
\end{minipage}\end{center}
\noindent Example 2: 
\begin{center}\begin{minipage}{15cm}\begin{Verbatim}[frame=single]
\end{Verbatim}
\end{minipage}\end{center}
See also: \textbf{$==$} (\ref{labequal}), \textbf{!$=$} (\ref{labneq}), \textbf{$>$} (\ref{labgt}), \textbf{$<=$} (\ref{lable}), \textbf{$<$} (\ref{lablt}), \textbf{!} (\ref{labnot}), \textbf{$\&\&$} (\ref{laband}), \textbf{$||$} (\ref{labor}), \textbf{prec} (\ref{labprec})

\input{gt}
\subsection{ guessdegree }
\noindent Name: \textbf{guessdegree}\\
returns the minimal degree needed for a polynomial to approximate a function with a certain error on an interval.\\

\noindent Usage: 
\begin{center}
\textbf{guessdegree}(\emph{f},\emph{I},\emph{eps},\emph{w}) : (\textsf{function}, \textsf{range}, \textsf{constant}, \textsf{function}) $\rightarrow$ \textsf{range}\\
\end{center}
Parameters: 
\begin{itemize}
\item \emph{f} is the function to be approximated.
\item \emph{I} is the interval where the function must be approximated.
\item \emph{eps} is the maximal acceptable error.
\item \emph{w} (optional) is a weight function. Default is 1.
\end{itemize}
\noindent Description: \begin{itemize}

\item \textbf{guessdegree} tries to find the minimal degree needed to approximate \emph{f}
   on \emph{I} by a polynomial with an infinite error not greater than \emph{eps}.
   More precisely, it finds $n$ minimal such that there exists a
   polynomial $p$ of degree $n$ such that $\|pw-f\|_{\infty} < \mathrm{eps}$.

\item \textbf{guessdegree} returns an interval: for common cases, this interval is reduced to a 
   single number (e.g. the minimal degree). But in certain cases, \textbf{guessdegree} does
   not succeed in finding the minimal degree. In such cases the returned interval
   is of the form $[n,\,p]$ such that:
   \begin{itemize}
   \item no polynomial of degree $n-1$ gives an error less than \emph{eps}.
   \item there exists a polynomial of degree $p$ giving an error less than \emph{eps}. 
   \end{itemize}
\end{itemize}
\noindent Example 1: 
\begin{center}\begin{minipage}{15cm}\begin{Verbatim}[frame=single]
> guessdegree(exp(x),[-1;1],1e-10);
[10;10]
\end{Verbatim}
\end{minipage}\end{center}
\noindent Example 2: 
\begin{center}\begin{minipage}{15cm}\begin{Verbatim}[frame=single]
> guessdegree(1, [-1;1], 1e-8, 1/exp(x));
[8;9]
\end{Verbatim}
\end{minipage}\end{center}
See also: \textbf{dirtyinfnorm}, \textbf{remez}

\subsection{head}
\label{labhead}
\noindent Name: \textbf{head}\\
gives the first element of a list.\\
\noindent Usage: 
\begin{center}
\textbf{head}(\emph{L}) : \textsf{list} $\rightarrow$ \textsf{any type}\\
\end{center}
Parameters: 
\begin{itemize}
\item \emph{L} is a list.
\end{itemize}
\noindent Description: \begin{itemize}

\item \\textbf{head}(\\emph{L}) returns the first element of the list \\emph{L}. It is equivalent\n   to L[0].\n
\item If \\emph{L} is empty, the command will fail with an error.\n\end{itemize}
\noindent Example 1: 
\begin{center}\begin{minipage}{15cm}\begin{Verbatim}[frame=single]
\end{Verbatim}
\end{minipage}\end{center}
See also: \textbf{tail} (\ref{labtail})

\subsection{hexadecimal}
\label{labhexadecimal}
\noindent Name: \textbf{hexadecimal}\\
\phantom{aaa}special value for global state \textbf{display}\\[0.2cm]
\noindent Library names:\\
\verb|   sollya_obj_t sollya_lib_hexadecimal()|\\
\verb|   int sollya_lib_is_hexadecimal(sollya_obj_t)|\\[0.2cm]
\noindent Description: \begin{itemize}

\item \textbf{hexadecimal} is a special value used for the global state \textbf{display}.  If
   the global state \textbf{display} is equal to \textbf{hexadecimal}, all data will be
   output in hexadecimal C99/ IEEE 754-2008 notation.
    
   As any value it can be affected to a variable and stored in lists.
\end{itemize}
See also: \textbf{decimal} (\ref{labdecimal}), \textbf{dyadic} (\ref{labdyadic}), \textbf{powers} (\ref{labpowers}), \textbf{binary} (\ref{labbinary}), \textbf{display} (\ref{labdisplay})

\subsection{honorcoeffprec}
\label{labhonorcoeffprec}
\noindent Name: \textbf{honorcoeffprec}\\
indicates the (forced) honoring the precision of the coefficients in \textbf{implementpoly}\\
\noindent Usage: 
\begin{center}
\textbf{honorcoeffprec} : \textsf{honorcoeffprec}\\
\end{center}
\noindent Description: \begin{itemize}

\item Used with command \\textbf{implementpoly}, \\textbf{honorcoeffprec} makes \\textbf{implementpoly} honor\n   the precision of the given polynomial. This means if a coefficient\n   needs a double-double or a triple-double to be exactly stored,\n   \\textbf{implementpoly} will allocate appropriate space and use a double-double\n   or triple-double operation even if the automatic (heuristic)\n   determination implemented in command \\textbf{implementpoly} indicates that the\n   coefficient could be stored on less precision or, respectively, the\n   operation could be performed with less precision. See \\textbf{implementpoly}\n   for details.\n\end{itemize}
\noindent Example 1: 
\begin{center}\begin{minipage}{15cm}\begin{Verbatim}[frame=single]
\end{Verbatim}
\end{minipage}\end{center}
See also: \textbf{implementpoly} (\ref{labimplementpoly}), \textbf{printexpansion} (\ref{labprintexpansion})

\subsection{hopitalrecursions}
\label{labhopitalrecursions}
\noindent Name: \textbf{hopitalrecursions}\\
controls the number of recursion steps when applying L'Hopital's rule.\\
\noindent Usage: 
\begin{center}
\textbf{hopitalrecursions} = \emph{n} : \textsf{integer} $\rightarrow$ \textsf{void}\\
\textbf{hopitalrecursions} = \emph{n} ! : \textsf{integer} $\rightarrow$ \textsf{void}\\
\textbf{hopitalrecursions} : \textsf{integer}\\
\end{center}
Parameters: 
\begin{itemize}
\item \emph{n} represents the number of recursions
\end{itemize}
\noindent Description: \begin{itemize}

\item \\textbf{hopitalrecursions} is a global variable. Its value represents the number of steps of\n   recursion that are tried when applying L'Hopital's rule. This rule is applied\n   by the interval evaluator present in the core of \\sollya (and particularly\n   visible in commands like \\textbf{infnorm}).\n
\item If an expression of the form $f/g$ has to be evaluated by interval \n   arithmetic on an interval $I$ and if $f$ and $g$ have a common zero\n   in $I$, a direct evaluation leads to NaN.\n   \\sollya implements a safe heuristic to avoid this, based on L'Hopital's rule: in \n   such a case, it can be shown that $(f/g)(I) \\subseteq (f'/g')(I)$. Since\n   the same problem may exist for $f'/g'$, the rule is applied recursively.\n   The number of step in this recursion process is controlled by \\textbf{hopitalrecursions}.\n
\item Setting \\textbf{hopitalrecursions} to 0 makes \\sollya use this rule only once;\n   setting it to 1 makes \\sollya use the rule twice, and so on.\n   In particular: the rule is always applied at least once, if necessary.\n\end{itemize}
\noindent Example 1: 
\begin{center}\begin{minipage}{15cm}\begin{Verbatim}[frame=single]
\end{Verbatim}
\end{minipage}\end{center}

\subsection{horner}
\label{labhorner}
\noindent Name: \textbf{horner}\\
brings all polynomial subexpressions of an expression to Horner form\\
\noindent Usage: 
\begin{center}
\textbf{horner}(\emph{function}) : \textsf{function} $\rightarrow$ \textsf{function}\\
\end{center}
Parameters: 
\begin{itemize}
\item \emph{function} represents the expression to be rewritten in Horner form
\end{itemize}
\noindent Description: \begin{itemize}

\item The command \\textbf{horner} rewrites the expression representing the function\n   \\emph{function} in a way such that all polynomial subexpressions (or the\n   whole expression itself, if it is a polynomial) are written in Horner\n   form.  The command \\textbf{horner} does not endanger the safety of\n   computations even in \\sollya's floating-point environment: the\n   function returned is mathematically equal to the function \\emph{function}.\n\end{itemize}
\noindent Example 1: 
\begin{center}\begin{minipage}{15cm}\begin{Verbatim}[frame=single]
\end{Verbatim}
\end{minipage}\end{center}
\noindent Example 2: 
\begin{center}\begin{minipage}{15cm}\begin{Verbatim}[frame=single]
\end{Verbatim}
\end{minipage}\end{center}
See also: \textbf{canonical} (\ref{labcanonical}), \textbf{print} (\ref{labprint})

\input{implementconstant}
\subsection{implementpoly}
\label{labimplementpoly}
\noindent Name: \textbf{implementpoly}\\
implements a polynomial using double, double-double and triple-double arithmetic and generates a Gappa proof\\
\noindent Usage: 
\begin{center}
\textbf{implementpoly}(\emph{polynomial}, \emph{range}, \emph{error bound}, \emph{format}, \emph{functionname}, \emph{filename}) : (\textsf{function}, \textsf{range}, \textsf{constant}, \textsf{D$|$double$|$DD$|$doubledouble$|$TD$|$tripledouble}, \textsf{string}, \textsf{string}) $\rightarrow$ \textsf{function}\\
\textbf{implementpoly}(\emph{polynomial}, \emph{range}, \emph{error bound}, \emph{format}, \emph{functionname}, \emph{filename}, \emph{honor coefficient precisions}) : (\textsf{function}, \textsf{range}, \textsf{constant}, \textsf{D$|$double$|$DD$|$doubledouble$|$TD$|$tripledouble}, \textsf{string}, \textsf{string}, \textsf{honorcoeffprec}) $\rightarrow$ \textsf{function}\\
\textbf{implementpoly}(\emph{polynomial}, \emph{range}, \emph{error bound}, \emph{format}, \emph{functionname}, \emph{filename}, \emph{proof filename}) : (\textsf{function}, \textsf{range}, \textsf{constant}, \textsf{D$|$double$|$DD$|$doubledouble$|$TD$|$tripledouble}, \textsf{string}, \textsf{string}, \textsf{string}) $\rightarrow$ \textsf{function}\\
\textbf{implementpoly}(\emph{polynomial}, \emph{range}, \emph{error bound}, \emph{format}, \emph{functionname}, \emph{filename}, \emph{honor coefficient precisions}, \emph{proof filename}) : (\textsf{function}, \textsf{range}, \textsf{constant}, \textsf{D$|$double$|$DD$|$doubledouble$|$TD$|$tripledouble}, \textsf{string}, \textsf{string}, \textsf{honorcoeffprec}, \textsf{string}) $\rightarrow$ \textsf{function}\\
\end{center}
\noindent Description: \begin{itemize}

\item The command \\textbf{implementpoly} implements the polynomial \\emph{polynomial} in range\n   \\emph{range} as a function called \\emph{functionname} in \\texttt{C} code\n   using double, double-double and triple-double arithmetic in a way that\n   the rounding error (estimated at its first order) is bounded by \\emph{error bound}. \n   The produced code is output in a file named \\emph{filename}. The\n   argument \\emph{format} indicates the double, double-double or triple-double\n   format of the variable in which the polynomial varies, influencing\n   also in the signature of the \\texttt{C} function.\n    \n   If a seventh or eighth argument \\emph{proof filename} is given and if this\n   argument evaluates to a variable of type \\textsf{string}, the command\n   \\textbf{implementpoly} will produce a \\texttt{Gappa} proof that the\n   rounding error is less than the given bound. This proof will be output\n   in \\texttt{Gappa} syntax in a file name \\emph{proof filename}.\n    \n   The command \\textbf{implementpoly} returns the polynomial that has been\n   implemented. As the command \\textbf{implementpoly} tries to adapt the precision\n   needed in each evaluation step to its strict minimum and as it applies\n   renormalization to double-double and triple-double precision\n   coefficients to bring them to a round-to-nearest expansion form, the\n   returned polynomial may differ from the polynomial\n   \\emph{polynomial}. Nevertheless the difference will be small enough that\n   the rounding error bound with regard to the polynomial \\emph{polynomial}\n   (estimated at its first order) will be less than the given error\n   bound.\n    \n   If a seventh argument \\emph{honor coefficient precisions} is given and\n   evaluates to a variable \\textbf{honorcoeffprec} of type \\textsf{honorcoeffprec},\n   \\textbf{implementpoly} will honor the precision of the given polynomial\n   \\emph{polynomials}. This means if a coefficient needs a double-double or a\n   triple-double to be exactly stored, \\textbf{implementpoly} will allocate appropriate\n   space and use a double-double or triple-double operation even if the\n   automatic (heuristic) determination implemented in command \\textbf{implementpoly}\n   indicates that the coefficient could be stored on less precision or,\n   respectively, the operation could be performed with less\n   precision. The use of \\textbf{honorcoeffprec} has advantages and\n   disadvantages. If the polynomial \\emph{polynomial} given has not been\n   determined by a process considering directly polynomials with\n   floating-point coefficients, \\textbf{honorcoeffprec} should not be\n   indicated. The \\textbf{implementpoly} command can then determine the needed\n   precision using the same error estimation as used for the\n   determination of the precisions of the operations. Generally, the\n   coefficients will get rounded to double, double-double and\n   triple-double precision in a way that minimizes their number and\n   respects the rounding error bound \\emph{error bound}.  Indicating\n   \\textbf{honorcoeffprec} may in this case short-circuit most precision\n   estimations leading to sub-optimal code. On the other hand, if the\n   polynomial \\emph{polynomial} has been determined with floating-point\n   precisions in mind, \\textbf{honorcoeffprec} should be indicated because such\n   polynomials often are very sensitive in terms of error propagation with\n   regard to their coefficients' values. Indicating \\textbf{honorcoeffprec}\n   prevents the \\textbf{implementpoly} command from rounding the coefficients and\n   altering by many orders of magnitude the approximation error of the\n   polynomial with regard to the function it approximates.\n    \n   The implementer behind the \\textbf{implementpoly} command makes some assumptions on\n   its input and verifies them. If some assumption cannot be verified,\n   the implementation will not succeed and \\textbf{implementpoly} will evaluate to a\n   variable \\textbf{error} of type \\textsf{error}. The same behaviour is observed if\n   some file is not writable or some other side-effect fails, e.g. if\n   the implementer runs out of memory.\n    \n   As error estimation is performed only on the first order, the code\n   produced by the \\textbf{implementpoly} command should be considered valid iff a\n   \\texttt{Gappa} proof has been produced and successfully run\n   in \\texttt{Gappa}.\n\end{itemize}
\noindent Example 1: 
\begin{center}\begin{minipage}{15cm}\begin{Verbatim}[frame=single]
\end{Verbatim}
\end{minipage}\end{center}
\noindent Example 2: 
\begin{center}\begin{minipage}{15cm}\begin{Verbatim}[frame=single]
\end{Verbatim}
\end{minipage}\end{center}
\noindent Example 3: 
\begin{center}\begin{minipage}{15cm}\begin{Verbatim}[frame=single]
\end{Verbatim}
\end{minipage}\end{center}
\noindent Example 4: 
\begin{center}\begin{minipage}{15cm}\begin{Verbatim}[frame=single]
\end{Verbatim}
\end{minipage}\end{center}
See also: \textbf{honorcoeffprec} (\ref{labhonorcoeffprec}), \textbf{roundcoefficients} (\ref{labroundcoefficients}), \textbf{double} (\ref{labdouble}), \textbf{doubledouble} (\ref{labdoubledouble}), \textbf{tripledouble} (\ref{labtripledouble}), \textbf{readfile} (\ref{labreadfile}), \textbf{printexpansion} (\ref{labprintexpansion}), \textbf{error} (\ref{laberror})

\subsection{infnorm}
\label{labinfnorm}
\noindent Name: \textbf{infnorm}\\
computes an interval bounding the infinity norm of a function on an interval.\\
\noindent Usage: 
\begin{center}
\textbf{infnorm}(\emph{f},\emph{I},\emph{filename},\emph{Ilist}) : (\textsf{function}, \textsf{range}, \textsf{string}, \textsf{list}) $\rightarrow$ \textsf{range}\\
\end{center}
Parameters: 
\begin{itemize}
\item \emph{f} is a function.
\item \emph{I} is an interval.
\item \emph{filename} (optional) is the name of the file into a proof will be saved.
\item \emph{IList} (optional) is a list of intervals to be excluded.
\end{itemize}
\noindent Description: \begin{itemize}

\item \\textbf{infnorm}(\\emph{f},\\emph{range}) computes an interval bounding the infinity norm of the \n   given function $f$ on the interval $I$, e.g. computes an interval $J$\n   such that $\\max_{x \\in I} \\{|f(x)|\\} \\subseteq J$.\n
\item If \\emph{filename} is given, a proof in English will be produced (and stored in file\n   called \\emph{filename}) proving that  $\\max_{x \\in I} \\{|f(x)|\\} \\subseteq J$.\n
\item If a list \\emph{IList} of intervals $I_1, \\dots, I_n$ is given, the infinity norm will\n   be computed on $I \\backslash (I_1 \\cup \\dots \\cup I_n)$.\n
\item The function \\emph{f} is assumed to be at least twice continuous on \\emph{I}. More \n   generally, if \\emph{f} is $\\mathcal{C}^k$, global variables \\textbf{hopitalrecursions} and\n   \\textbf{taylorrecursions} must have values not greater than $k$.  \n
\item If the interval is reduced to a single point, the result of \\textbf{infnorm} is an \n   interval containing the exact absolute value of \\emph{f} at this point.\n
\item If the interval is not bound, the result will be $[0,\\,+\\infty]$ \n   which is correct but perfectly useless. \\textbf{infnorm} is not meant to be used with \n   infinite intervals.\n
\item The result of this command depends on the global variables \\textbf{prec}, \\textbf{diam},\n   \\textbf{taylorrecursions} and \\textbf{hopitalrecursions}. The contribution of each variable is \n   not easy even to analyse.\n   \\begin{itemize}\n   \\item  The algorithm uses interval arithmetic with precision \\textbf{prec}. The\n     precision should thus be set high enough to ensure that no critical\n     cancellation will occur.\n   \\item  When an evaluation is performed on an interval $[a,\\,b]$, if the result\n     is considered being too large, the interval is split into $[a,\\,\\frac{a+b}{2}]$\n     and $[\\frac{a+b}{2},\\,b]$ and so on recursively. This recursion step\n     is  not performed if the $(b-a) < \\delta \\cdot |I|$ where $\\delta$ is the value\n     of variable \\textbf{diam}. In other words, \\textbf{diam} controls the minimum length of an\n     interval during the algorithm.\n   \\item  To perform the evaluation of a function on an interval, Taylor's rule is\n     applied, e.g. $f([a,b]) \\subseteq f(m) + [a-m,\\,b-m] \\cdot f'([a,\\,b])$\n     where $m=\\frac{a+b}{2}$. This rule is recursively applied $n$ times\n     where $n$ is the value of variable \\textbf{taylorrecursions}. Roughly speaking,\n     the evaluations will avoid decorrelation up to order $n$.\n   \\item  When a function of the form $\\frac{g}{h}$ has to be evaluated on an\n     interval $[a,\\,b]$ and when $g$ and $h$ vanish at a same point\n     $z$ of the interval, the ratio may be defined even if the expression\n     $\\frac{g(z)}{h(z)}=\\frac{0}{0}$ does not make any sense. In this case, L'Hopital's rule\n     may be used and $\\left(\\frac{g}{h}\\right)([a,\\,b]) \\subseteq \\left(\\frac{g'}{h'}\\right)([a,\\,b])$.\n     Since the same can occur with the ratio $\\frac{g'}{h'}$, the rule is applied\n     recursively. The variable \\textbf{hopitalrecursions} controls the number of \n     recursion steps.\n   \\end{itemize}\n
\item The algorithm used for this command is quite complex to be explained here. \n   Please find a complete description in the following article:\\\\\n        S. Chevillard and C. Lauter\\\\\n        A certified infinity norm for the implementation of elementary functions\\\\\n        LIP Research Report number RR2007-26\\\\\n        http://prunel.ccsd.cnrs.fr/ensl-00119810\\\\\n\end{itemize}
\noindent Example 1: 
\begin{center}\begin{minipage}{15cm}\begin{Verbatim}[frame=single]
\end{Verbatim}
\end{minipage}\end{center}
\noindent Example 2: 
\begin{center}\begin{minipage}{15cm}\begin{Verbatim}[frame=single]
\end{Verbatim}
\end{minipage}\end{center}
\noindent Example 3: 
\begin{center}\begin{minipage}{15cm}\begin{Verbatim}[frame=single]
\end{Verbatim}
\end{minipage}\end{center}
\noindent Example 4: 
\begin{center}\begin{minipage}{15cm}\begin{Verbatim}[frame=single]
\end{Verbatim}
\end{minipage}\end{center}
\noindent Example 5: 
\begin{center}\begin{minipage}{15cm}\begin{Verbatim}[frame=single]
\end{Verbatim}
\end{minipage}\end{center}
\noindent Example 6: 
\begin{center}\begin{minipage}{15cm}\begin{Verbatim}[frame=single]
\end{Verbatim}
\end{minipage}\end{center}
See also: \textbf{prec} (\ref{labprec}), \textbf{diam} (\ref{labdiam}), \textbf{hopitalrecursions} (\ref{labhopitalrecursions}), \textbf{dirtyinfnorm} (\ref{labdirtyinfnorm}), \textbf{checkinfnorm} (\ref{labcheckinfnorm})

\subsection{inf}
\label{labinf}
\noindent Name: \textbf{inf}\\
\phantom{aaa}gives the lower bound of an interval.\\[0.2cm]
\noindent Library name:\\
\verb|   sollya_obj_t sollya_lib_inf(sollya_obj_t)|\\[0.2cm]
\noindent Usage: 
\begin{center}
\textbf{inf}(\emph{I}) : \textsf{range} $\rightarrow$ \textsf{constant}\\
\textbf{inf}(\emph{x}) : \textsf{constant} $\rightarrow$ \textsf{constant}\\
\end{center}
Parameters: 
\begin{itemize}
\item \emph{I} is an interval.
\item \emph{x} is a real number.
\end{itemize}
\noindent Description: \begin{itemize}

\item Returns the lower bound of the interval \emph{I}. Each bound of an interval has its 
   own precision, so this command is exact, even if the current precision is too 
   small to represent the bound.

\item When called on a real number \emph{x}, \textbf{inf} behaves like the identity.
\end{itemize}
\noindent Example 1: 
\begin{center}\begin{minipage}{15cm}\begin{Verbatim}[frame=single]
> inf([1;3]);
1
> inf(0);
0
\end{Verbatim}
\end{minipage}\end{center}
\noindent Example 2: 
\begin{center}\begin{minipage}{15cm}\begin{Verbatim}[frame=single]
> display=binary!;
> I=[0.111110000011111_2; 1];
> inf(I);
1.11110000011111_2 * 2^(-1)
> prec=12!;
> inf(I);
1.11110000011111_2 * 2^(-1)
\end{Verbatim}
\end{minipage}\end{center}
See also: \textbf{mid} (\ref{labmid}), \textbf{sup} (\ref{labsup}), \textbf{max} (\ref{labmax}), \textbf{min} (\ref{labmin})

\subsection{ integer }
\noindent Name: \textbf{integer}\\
keyword representing a machine integer type \\

\noindent Usage: 
\begin{center}
\textbf{integer} : \textsf{type type}\\
\end{center}
\noindent Description: \begin{itemize}

\item \textbf{integer} represents the machine integer type for declarations
   of external procedures by means of \textbf{externalproc}.
   Remark that in contrast to other indicators, type indicators like
   \textbf{integer} cannot be handled outside the \textbf{externalproc} context.  In
   particular, they cannot be assigned to variables.
\end{itemize}
See also: \textbf{externalproc}, \textbf{boolean}, \textbf{constant}, \textbf{function}, \textbf{list of}, \textbf{range}, \textbf{string}

\subsection{integral}
\label{labintegral}
\noindent Name: \textbf{integral}\\
computes an interval bounding the integral of a function on an interval.\\
\noindent Usage: 
\begin{center}
\textbf{integral}(\emph{f},\emph{I}) : (\textsf{function}, \textsf{range}) $\rightarrow$ \textsf{range}\\
\end{center}
Parameters: 
\begin{itemize}
\item \emph{f} is a function.
\item \emph{I} is an interval.
\end{itemize}
\noindent Description: \begin{itemize}

\item \\textbf{integral}(\\emph{f},\\emph{I}) returns an interval $J$ such that the exact value of \n   the integral of \\emph{f} on \\emph{I} lies in $J$.\n
\item This command is safe but very inefficient. Use \\textbf{dirtyintegral} if you just want\n   an approximate value.\n
\item The result of this command depends on the global variable \\textbf{diam}.\n   The method used is the following: \\emph{I} is cut into intervals of length not \n   greater then $\\delta \\cdot |I|$ where $\\delta$ is the value\n   of global variable \\textbf{diam}.\n   On each small interval \\emph{J}, an evaluation of \\emph{f} by interval is\n   performed. The result is multiplied by the length of \\emph{J}. Finally all values \n   are summed.\n\end{itemize}
\noindent Example 1: 
\begin{center}\begin{minipage}{15cm}\begin{Verbatim}[frame=single]
\end{Verbatim}
\end{minipage}\end{center}
See also: \textbf{diam} (\ref{labdiam}), \textbf{dirtyintegral} (\ref{labdirtyintegral})

\input{in}
\subsection{isbound}
\label{labisbound}
\noindent Name: \textbf{isbound}\\
indicates whether a variable is bound or not.\\
\noindent Usage: 
\begin{center}
\textbf{isbound}(\emph{ident}) : \textsf{boolean}\\
\end{center}
Parameters: 
\begin{itemize}
\item \emph{ident} is a name.
\end{itemize}
\noindent Description: \begin{itemize}

\item \\textbf{isbound}(\\emph{ident}) returns a boolean value indicating whether the name \\emph{ident}\n   is used or not to represent a variable. It returns true when \\emph{ident} is the \n   name used to represent the global variable or if the name is currently used\n   to refer to a (possibly local) variable.\n
\item When a variable is defined in a block and has not been defined outside, \n   \\textbf{isbound} returns true when called inside the block, and false outside.\n   Note that \\textbf{isbound} returns true as soon as a variable has been declared with \n   \\textbf{var}, even if no value is actually stored in it.\n
\item If \\emph{ident1} is bound to a variable and if \\emph{ident2} refers to the global \n   variable, the command \\textbf{rename}(\\emph{ident2}, \\emph{ident1}) hides the value of \\emph{ident1}\n   which becomes the global variable. However, if the global variable is again\n   renamed, \\emph{ident1} gets its value back. In this case, \\textbf{isbound}(\\emph{ident1}) returns\n   true. If \\emph{ident1} was not bound before, \\textbf{isbound}(\\emph{ident1}) returns false after\n   that \\emph{ident1} has been renamed.\n\end{itemize}
\noindent Example 1: 
\begin{center}\begin{minipage}{15cm}\begin{Verbatim}[frame=single]
\end{Verbatim}
\end{minipage}\end{center}
\noindent Example 2: 
\begin{center}\begin{minipage}{15cm}\begin{Verbatim}[frame=single]
\end{Verbatim}
\end{minipage}\end{center}
\noindent Example 3: 
\begin{center}\begin{minipage}{15cm}\begin{Verbatim}[frame=single]
\end{Verbatim}
\end{minipage}\end{center}
\noindent Example 4: 
\begin{center}\begin{minipage}{15cm}\begin{Verbatim}[frame=single]
\end{Verbatim}
\end{minipage}\end{center}
See also: \textbf{rename} (\ref{labrename})

\subsection{isevaluable}
\label{labisevaluable}
\noindent Name: \textbf{isevaluable}\\
tests whether a function can be evaluated at a point \\
\noindent Usage: 
\begin{center}
\textbf{isevaluable}(\emph{function}, \emph{constant}) : (\textsf{function}, \textsf{constant}) $\rightarrow$ \textsf{boolean}\\
\end{center}
Parameters: 
\begin{itemize}
\item \emph{function} represents a function
\item \emph{constant} represents a constant point
\end{itemize}
\noindent Description: \begin{itemize}

\item \\textbf{isevaluable} applied to function \\emph{function} and a constant \\emph{constant} returns\n   a boolean indicating whether or not a subsequent call to \\textbf{evaluate} on the\n   same function \\emph{function} and constant \\emph{constant} will produce a numerical\n   result or NaN. This means \\textbf{isevaluable} returns false iff \\textbf{evaluate} will return NaN.\n\end{itemize}
\noindent Example 1: 
\begin{center}\begin{minipage}{15cm}\begin{Verbatim}[frame=single]
\end{Verbatim}
\end{minipage}\end{center}
\noindent Example 2: 
\begin{center}\begin{minipage}{15cm}\begin{Verbatim}[frame=single]
\end{Verbatim}
\end{minipage}\end{center}
\noindent Example 3: 
\begin{center}\begin{minipage}{15cm}\begin{Verbatim}[frame=single]
\end{Verbatim}
\end{minipage}\end{center}
See also: \textbf{evaluate} (\ref{labevaluate})

\subsection{length}
\label{lablength}
\noindent Name: \textbf{length}\\
computes the length of a list or string.\\
\noindent Usage: 
\begin{center}
\textbf{length}(\emph{L}) : \textsf{list} $\rightarrow$ \textsf{integer}\\
\textbf{length}(\emph{s}) : \textsf{string} $\rightarrow$ \textsf{integer}\\
\end{center}
Parameters: 
\begin{itemize}
\item \emph{L} is a list.
\item \emph{s} is a string.
\end{itemize}
\noindent Description: \begin{itemize}

\item \\textbf{length} returns the length of a list or a string, e.g. the number of elements\n   or letters.\n
\item The empty list or string have length 0.\n   If \\emph{L} is an end-elliptic list, \\textbf{length} returns +Inf.\n\end{itemize}
\noindent Example 1: 
\begin{center}\begin{minipage}{15cm}\begin{Verbatim}[frame=single]
\end{Verbatim}
\end{minipage}\end{center}
\noindent Example 2: 
\begin{center}\begin{minipage}{15cm}\begin{Verbatim}[frame=single]
\end{Verbatim}
\end{minipage}\end{center}
\noindent Example 3: 
\begin{center}\begin{minipage}{15cm}\begin{Verbatim}[frame=single]
\end{Verbatim}
\end{minipage}\end{center}
\noindent Example 4: 
\begin{center}\begin{minipage}{15cm}\begin{Verbatim}[frame=single]
\end{Verbatim}
\end{minipage}\end{center}

\input{le}
\input{libraryconstant}
\subsection{library}
\label{lablibrary}
\noindent Name: \textbf{library}\\
binds an external mathematical function to a variable in \sollya\\
\noindent Usage: 
\begin{center}
\textbf{library}(\emph{path}) : \textsf{string} $\rightarrow$ \textsf{function}\\
\end{center}
\noindent Description: \begin{itemize}

\item The command \\textbf{library} lets you extend the set of mathematical\n   functions known to \\sollya.\n   By default, \\sollya knows the most common mathematical functions such\n   as \\textbf{exp}, \\textbf{sin}, \\textbf{erf}, etc. Within \\sollya, these functions may be\n   composed. This way, \\sollya should satisfy the needs of a lot of\n   users. However, for particular applications, one may want to\n   manipulate other functions such as Bessel functions, or functions\n   defined by an integral or even a particular solution of an ODE.\n
\item \\textbf{library} makes it possible to let \\sollya know about new functions. In\n   order to let it know, you have to provide an implementation of the\n   function you are interested in. This implementation is a C file containing\n   a function of the form:\n   \\begin{verbatim} int my_ident(mpfi_t result, mpfi_t op, int n)\\end{verbatim}\n   The semantic of this function is the following: it is an implementation of\n   the function and its derivatives in interval arithmetic.\n   \\verb|my_ident(result, I, n)| shall store in \\verb|result| an enclosure \n   of the image set of the $n$-th derivative\n   of the function f over \\verb|I|: $f^{(n)}(I) \\subseteq \\mathrm{result}$.\n
\item The integer value returned by the function implementation currently has no meaning.\n
\item You do not need to provide a working implementation for any \\verb|n|. Most functions\n   of \\sollya requires a relevant implementation only for $f$, $f'$ and $f''$. For higher \n   derivatives, its is not so critical and the implementation may just store \n   $[-\\infty,\\,+\\infty]$ in result whenever $n>2$.\n
\item Note that you should respect somehow MPFI standards in your implementation:\n   \\verb|result| has its own precision and you should perform the \n   intermediate computations so that \\verb|result| is as tight as possible.\n
\item You can include sollya.h in your implementation and use library \n   functionnalities of \\sollya for your implementation. However, this requires to have compiled\n   \\sollya with \\texttt{-fPIC} in order to make the \\sollya executable code position \n   independent and to use a system on with programs, using \\texttt{dlopen} to open\n   dynamic routines can dynamically open themselves.\n
\item To bind your function into \\sollya, you must use the same identifier as the\n   function name used in your implementation file (\\verb|my_ident| in the previous\n   example). Once the function code has been bound to an identifier, you can use a simple assignment\n   to assign the bound identifier to yet another identifier. This way, you may use convenient\n   names inside \\sollya even if your implementation environment requires you to use a less\n   convenient name.\n\end{itemize}
\noindent Example 1: 
\begin{center}\begin{minipage}{15cm}\begin{Verbatim}[frame=single]
\end{Verbatim}
\end{minipage}\end{center}
See also: \textbf{bashexecute} (\ref{labbashexecute}), \textbf{externalproc} (\ref{labexternalproc}), \textbf{externalplot} (\ref{labexternalplot})

\subsection{ listof }
\noindent Name: \textbf{list of}\\
keyword used in combination with a type keyword\\

\noindent Description: \begin{itemize}

\item \textbf{list of} is used in combination with one of the following keywords for
   indicating lists of the respective type in declarations of external
   procedures using \textbf{externalproc}: \textbf{boolean}, \textbf{constant}, \textbf{function},
   \textbf{integer}, \textbf{range} and \textbf{string}.
\end{itemize}
See also: \textbf{externalproc}, \textbf{boolean}, \textbf{constant}, \textbf{function}, \textbf{integer}, \textbf{range}, \textbf{string}

\subsection{log10}
\label{lablog10}
\noindent Name: \textbf{log10}\\
\phantom{aaa}decimal logarithm.\\[0.2cm]
\noindent Library names:\\
\verb|   sollya_obj_t sollya_lib_log10(sollya_obj_t)|\\
\verb|   sollya_obj_t sollya_lib_build_function_log10(sollya_obj_t)|\\
\verb|   #define SOLLYA_LOG10(x) sollya_lib_build_function_log10(x)|\\[0.2cm]
\noindent Description: \begin{itemize}

\item \textbf{log10} is the decimal logarithm defined by: ${\rm log10}(x) = \log(x)/\log(10)$.

\item It is defined only for $x \in [0; +\infty]$.
\end{itemize}
See also: \textbf{log} (\ref{lablog}), \textbf{log2} (\ref{lablog2})

\subsection{log1p}
\label{lablog1p}
\noindent Name: \textbf{log1p}\\
\phantom{aaa}translated logarithm.\\[0.2cm]
\noindent Library names:\\
\verb|   sollya_obj_t sollya_lib_log1p(sollya_obj_t)|\\
\verb|   sollya_obj_t sollya_lib_build_function_log1p(sollya_obj_t)|\\
\verb|   #define SOLLYA_LOG1P(x) sollya_lib_build_function_log1p(x)|\\[0.2cm]
\noindent Description: \begin{itemize}

\item \textbf{log1p} is the function defined by ${\rm log1p}(x) = \log(1+x)$.

\item It is defined only for $x \in [-1; +\infty]$.
\end{itemize}
See also: \textbf{log} (\ref{lablog})

\subsection{log2}
\label{lablog2}
\noindent Name: \textbf{log2}\\
\phantom{aaa}binary logarithm.\\[0.2cm]
\noindent Library names:\\
\verb|   sollya_obj_t sollya_lib_log2(sollya_obj_t)|\\
\verb|   sollya_obj_t sollya_lib_build_function_log2(sollya_obj_t)|\\
\verb|   #define SOLLYA_LOG2(x) sollya_lib_build_function_log2(x)|\\[0.2cm]
\noindent Description: \begin{itemize}

\item \textbf{log2} is the binary logarithm defined by: ${\rm log2}(x) = \log(x)/\log(2)$.

\item It is defined only for $x \in [0; +\infty]$.
\end{itemize}
See also: \textbf{log} (\ref{lablog}), \textbf{log10} (\ref{lablog10})

\subsection{log}
\label{lablog}
\noindent Name: \textbf{log}\\
\phantom{aaa}natural logarithm.\\[0.2cm]
\noindent Library names:\\
\verb|   sollya_obj_t sollya_lib_log(sollya_obj_t)|\\
\verb|   sollya_obj_t sollya_lib_build_function_log(sollya_obj_t)|\\
\verb|   #define SOLLYA_LOG(x) sollya_lib_build_function_log(x)|\\[0.2cm]
\noindent Description: \begin{itemize}

\item \textbf{log} is the natural logarithm defined as the inverse of the exponential
   function: $\log(y)$ is the unique real number $x$ such that $\exp(x)=y$.

\item It is defined only for $y \in [0; +\infty]$.
\end{itemize}
See also: \textbf{exp} (\ref{labexp}), \textbf{log2} (\ref{lablog2}), \textbf{log10} (\ref{lablog10})

\subsection{ lt }
\noindent Name: \textbf{$<$}\\
less-than operator\\

\noindent Usage: 
\begin{center}
\emph{expr1} \textbf{$<$} \emph{expr2} : (\textsf{constant}, \textsf{constant}) $\rightarrow$ \textsf{boolean}\\
\end{center}
Parameters: 
\emph{expr1} and \emph{expr2} represent constant expressions\\

\noindent Description: \begin{itemize}

\item The operator \textbf{$<$} evaluates to true iff its operands \emph{expr1} and
   \emph{expr2} evaluate to two floating-point numbers $a_1$
   respectively $a_2$ with the global precision \textbf{prec} and
   $a_1$ is less than $a_2$. The user should
   be aware of the fact that because of floating-point evaluation, the
   operator \textbf{$<$} is not exactly the same as the mathematical
   operation \emph{less-than}.
\end{itemize}
\noindent Example 1: 
\begin{center}\begin{minipage}{14.8cm}\begin{Verbatim}[frame=single]
> 5 < 4;
   false
> 5 < 5;
   false
> 5 < 6;
   true
> exp(2) < exp(1);
   false
> log(1) < exp(2);
   true
\end{Verbatim}
\end{minipage}\end{center}
\noindent Example 2: 
\begin{center}\begin{minipage}{14.8cm}\begin{Verbatim}[frame=single]
> prec = 12;
   The precision has been set to 12 bits.
> 16384 < 16385;
   false
\end{Verbatim}
\end{minipage}\end{center}
See also: \textbf{$==$}, \textbf{!$=$}, \textbf{$>=$}, \textbf{$>$}, \textbf{$<=$}, \textbf{!}, \textbf{$\&\&$}, \textbf{$||$}, \textbf{prec}

\subsection{mantissa}
\label{labmantissa}
\noindent Name: \textbf{mantissa}\\
returns the integer mantissa of a number.\\
\noindent Usage: 
\begin{center}
\textbf{mantissa}(\emph{x}) : \textsf{constant} $\rightarrow$ \textsf{integer}\\
\end{center}
Parameters: 
\begin{itemize}
\item \emph{x} is a dyadic number.
\end{itemize}
\noindent Description: \begin{itemize}

\item \\textbf{mantissa}($x$) is by definition $x$ if $x$ equals 0, NaN, or Inf.\n
\item If \\emph{x} is not zero, it can be uniquely written as $x = m \\cdot 2^e$ where\n   $m$ is an odd integer and $e$ is an integer. \\textbf{mantissa}(x) returns $m$. \n\end{itemize}
\noindent Example 1: 
\begin{center}\begin{minipage}{15cm}\begin{Verbatim}[frame=single]
\end{Verbatim}
\end{minipage}\end{center}
See also: \textbf{exponent} (\ref{labexponent}), \textbf{precision} (\ref{labprecision})

\input{max}
\subsection{midpointmode}
\label{labmidpointmode}
\noindent Name: \textbf{midpointmode}\\
global variable controlling the way intervals are displayed.\\
\noindent Description: \begin{itemize}

\item \\textbf{midpointmode} is a global variable. When its value is \\textbf{off}, intervals are displayed\n   as usual (in the form $\\left[ a;b\\right]$).\n   When its value is \\textbf{on}, and if $a$ and $b$ have the same first significant digits,\n   the interval in displayed in a way that lets one immediately see the common\n   digits of the two bounds.\n
\item This mode is supported only with \\textbf{display} set to \\textbf{decimal}. In other modes of \n   display, \\textbf{midpointmode} value is simply ignored.\n\end{itemize}
\noindent Example 1: 
\begin{center}\begin{minipage}{15cm}\begin{Verbatim}[frame=single]
\end{Verbatim}
\end{minipage}\end{center}
See also: \textbf{on} (\ref{labon}), \textbf{off} (\ref{laboff}), \textbf{roundingwarnings} (\ref{labroundingwarnings})

\subsection{mid}
\label{labmid}
\noindent Name: \textbf{mid}\\
\phantom{aaa}gives the middle of an interval.\\[0.2cm]
\noindent Library name:\\
\verb|   sollya_obj_t sollya_lib_mid(sollya_obj_t)|\\[0.2cm]
\noindent Usage: 
\begin{center}
\textbf{mid}(\emph{I}) : \textsf{range} $\rightarrow$ \textsf{constant}\\
\textbf{mid}(\emph{x}) : \textsf{constant} $\rightarrow$ \textsf{constant}\\
\end{center}
Parameters: 
\begin{itemize}
\item \emph{I} is an interval.
\item \emph{x} is a real number.
\end{itemize}
\noindent Description: \begin{itemize}

\item Returns the middle of the interval \emph{I}. If the middle is not exactly
   representable at the current precision, the value is returned as an
   unevaluated expression.

\item When called on a real number \emph{x}, \textbf{mid} behaves like the identity.
\end{itemize}
\noindent Example 1: 
\begin{center}\begin{minipage}{15cm}\begin{Verbatim}[frame=single]
> mid([1;3]);
2
> mid(17);
17
\end{Verbatim}
\end{minipage}\end{center}
See also: \textbf{inf} (\ref{labinf}), \textbf{sup} (\ref{labsup})

\subsection{minus}
\label{labminus}
\noindent Name: \textbf{$-$}\\
subtraction function\\
\noindent Usage: 
\begin{center}
\emph{function1} \textbf{$-$} \emph{function2} : (\textsf{function}, \textsf{function}) $\rightarrow$ \textsf{function}
\\ 
\end{center}
Parameters: 
\begin{itemize}
\item \emph{function1} and \emph{function2} represent functions
\end{itemize}
\noindent Description: \begin{itemize}

\item \textbf{$-$} represents the subtraction (function) on reals. 
   The expression \emph{function1} \textbf{$-$} \emph{function2} stands for
   the function composed of the subtraction function and the two
   functions \emph{function1} and \emph{function2}, where \emph{function1} is 
   the subtrahend and \emph{function2} the subtractor.
\end{itemize}
\noindent Example 1: 
\begin{center}\begin{minipage}{15cm}\begin{Verbatim}[frame=single]
> 5 - 2;
3
\end{Verbatim}
\end{minipage}\end{center}
\noindent Example 2: 
\begin{center}\begin{minipage}{15cm}\begin{Verbatim}[frame=single]
> x - 2;
-2 + x
\end{Verbatim}
\end{minipage}\end{center}
\noindent Example 3: 
\begin{center}\begin{minipage}{15cm}\begin{Verbatim}[frame=single]
> x - x;
0
\end{Verbatim}
\end{minipage}\end{center}
\noindent Example 4: 
\begin{center}\begin{minipage}{15cm}\begin{Verbatim}[frame=single]
> diff(sin(x) - exp(x));
cos(x) - exp(x)
\end{Verbatim}
\end{minipage}\end{center}
See also: \textbf{$+$} (\ref{labplus}), \textbf{$*$} (\ref{labmult}), \textbf{/} (\ref{labdivide}), \textbf{\^} (\ref{labpower})

\input{min}
\subsection{mult}
\label{labmult}
\noindent Name: \textbf{$*$}\\
multiplication function\\
\noindent Usage: 
\begin{center}
\emph{function1} \textbf{$*$} \emph{function2} : (\textsf{function}, \textsf{function}) $\rightarrow$ \textsf{function}
\end{center}
Parameters: 
\begin{itemize}
\item \emph{function1} and \emph{function2} represent functions
\end{itemize}
\noindent Description: \begin{itemize}

\item \textbf{$*$} represents the multiplication (function) on reals. 
   The expression \emph{function1} \textbf{$*$} \emph{function2} stands for
   the function composed of the multiplication function and the two
   functions \emph{function1} and \emph{function2}.
\end{itemize}
\noindent Example 1: 
\begin{center}\begin{minipage}{15cm}\begin{Verbatim}[frame=single]
> 5 * 2;
10
\end{Verbatim}
\end{minipage}\end{center}
\noindent Example 2: 
\begin{center}\begin{minipage}{15cm}\begin{Verbatim}[frame=single]
> x * 2;
x * 2
\end{Verbatim}
\end{minipage}\end{center}
\noindent Example 3: 
\begin{center}\begin{minipage}{15cm}\begin{Verbatim}[frame=single]
> x * x;
x^2
\end{Verbatim}
\end{minipage}\end{center}
\noindent Example 4: 
\begin{center}\begin{minipage}{15cm}\begin{Verbatim}[frame=single]
> diff(sin(x) * exp(x));
sin(x) * exp(x) + exp(x) * cos(x)
\end{Verbatim}
\end{minipage}\end{center}
See also: \textbf{$+$} (\ref{labplus}), \textbf{$-$} (\ref{labminus}), \textbf{/} (\ref{labdivide}), \textbf{\^} (\ref{labpower})

\subsection{nearestint}
\label{labnearestint}
\noindent Name: \textbf{nearestint}\\
\phantom{aaa}the function mapping the reals to the integers nearest to them.\\[0.2cm]
\noindent Library names:\\
\verb|   sollya_obj_t sollya_lib_nearestint(sollya_obj_t)|\\
\verb|   sollya_obj_t sollya_lib_build_function_nearestint(sollya_obj_t)|\\
\verb|   #define SOLLYA_NEARESTINT(x) sollya_lib_build_function_nearestint(x)|\\[0.2cm]
\noindent Description: \begin{itemize}

\item \textbf{nearestint} is defined as usual: \textbf{nearestint}($x$) is the integer nearest to $x$, with the
   special rule that the even integer is chosen if there exist two integers equally near to $x$.

\item It is defined for every real number $x$.
\end{itemize}
See also: \textbf{ceil} (\ref{labceil}), \textbf{floor} (\ref{labfloor}), \textbf{round} (\ref{labround}), \textbf{RN} (\ref{labrn})

\subsection{!$=$}
\label{labneq}
\noindent Name: \textbf{!$=$}\\
negated equality test operator\\
\noindent Usage: 
\begin{center}
\emph{expr1} \textbf{!$=$} \emph{expr2} : (\textsf{any type}, \textsf{any type}) $\rightarrow$ \textsf{boolean}\\
\end{center}
Parameters: 
\begin{itemize}
\item \emph{expr1} and \emph{expr2} represent expressions
\end{itemize}
\noindent Description: \begin{itemize}

\item The operator \\textbf{!$=$} evaluates to true iff its operands \\emph{expr1} and\n   \\emph{expr2} are syntactically unequal and both different from \\textbf{error} or\n   constant expressions that are not constants and that evaluate to two\n   different floating-point number with the global precision \\textbf{prec}. The\n   user should be aware of the fact that because of floating-point\n   evaluation, the operator \\textbf{!$=$} is not exactly the same as the\n   negation of the mathematical equality.\n     \n   Note that the expressions \\textbf{!}(\\emph{expr1} \\textbf{!$=$} \\emph{expr2}) and \\emph{expr1}\n   \\textbf{$==$} \\emph{expr2} do not evaluate to the same boolean value. See \\textbf{error}\n   for details.\n\end{itemize}
\noindent Example 1: 
\begin{center}\begin{minipage}{15cm}\begin{Verbatim}[frame=single]
\end{Verbatim}
\end{minipage}\end{center}
\noindent Example 2: 
\begin{center}\begin{minipage}{15cm}\begin{Verbatim}[frame=single]
\end{Verbatim}
\end{minipage}\end{center}
\noindent Example 3: 
\begin{center}\begin{minipage}{15cm}\begin{Verbatim}[frame=single]
\end{Verbatim}
\end{minipage}\end{center}
\noindent Example 4: 
\begin{center}\begin{minipage}{15cm}\begin{Verbatim}[frame=single]
\end{Verbatim}
\end{minipage}\end{center}
\noindent Example 5: 
\begin{center}\begin{minipage}{15cm}\begin{Verbatim}[frame=single]
\end{Verbatim}
\end{minipage}\end{center}
See also: \textbf{$==$} (\ref{labequal}), \textbf{$>$} (\ref{labgt}), \textbf{$>=$} (\ref{labge}), \textbf{$<=$} (\ref{lable}), \textbf{$<$} (\ref{lablt}), \textbf{!} (\ref{labnot}), \textbf{$\&\&$} (\ref{laband}), \textbf{$||$} (\ref{labor}), \textbf{error} (\ref{laberror}), \textbf{prec} (\ref{labprec})

\subsection{nop}
\label{labnop}
\noindent Name: \textbf{nop}\\
no operation\\
\noindent Usage: 
\begin{center}
\textbf{nop} : \textsf{void} $\rightarrow$ \textsf{void}\\
\textbf{nop}() : \textsf{void} $\rightarrow$ \textsf{void}\\
\textbf{nop}(\emph{n}) : \textsf{integer} $\rightarrow$ \textsf{void}\\
\end{center}
\noindent Description: \begin{itemize}

\item The command \\textbf{nop} does nothing. This means it is an explicit parse\n   element in the \\sollya language that finally does not produce any\n   result or side-effect.\n
\item The command \\textbf{nop} may take an optional positive integer argument \\emph{n}. The argument controls how much (useless) integer additions \\sollya performs while doing nothing. \n   With this behaviour, \\textbf{nop} can be used for calibration of timing tests.\n
\item The keyword \\textbf{nop} is implicit in some procedure\n   definitions. Procedures without imperative body get parsed as if they\n   had an imperative body containing one \\textbf{nop} statement.\n\end{itemize}
\noindent Example 1: 
\begin{center}\begin{minipage}{15cm}\begin{Verbatim}[frame=single]
\end{Verbatim}
\end{minipage}\end{center}
\noindent Example 2: 
\begin{center}\begin{minipage}{15cm}\begin{Verbatim}[frame=single]
\end{Verbatim}
\end{minipage}\end{center}
\noindent Example 3: 
\begin{center}\begin{minipage}{15cm}\begin{Verbatim}[frame=single]
\end{Verbatim}
\end{minipage}\end{center}
See also: \textbf{proc} (\ref{labproc})

\subsection{!}
\label{labnot}
\noindent Name: \textbf{!}\\
boolean NOT operator\\
\noindent Usage: 
\begin{center}
\textbf{!} \emph{expr} : \textsf{boolean} $\rightarrow$ \textsf{boolean}\\
\end{center}
Parameters: 
\begin{itemize}
\item \emph{expr} represents a boolean expression
\end{itemize}
\noindent Description: \begin{itemize}

\item \\textbf{!} evaluates to the boolean NOT of the boolean expression\n   \\emph{expr}. \\textbf{!} \\emph{expr} evaluates to true iff \\emph{expr} does not evaluate\n   to true.\n\end{itemize}
\noindent Example 1: 
\begin{center}\begin{minipage}{15cm}\begin{Verbatim}[frame=single]
\end{Verbatim}
\end{minipage}\end{center}
\noindent Example 2: 
\begin{center}\begin{minipage}{15cm}\begin{Verbatim}[frame=single]
\end{Verbatim}
\end{minipage}\end{center}
See also: \textbf{$\&\&$} (\ref{laband}), \textbf{$||$} (\ref{labor})

\subsection{numberroots}
\label{labnumberroots}
\noindent Name: \textbf{numberroots}\\
Computes the number of roots of a polynomial in a given range\\
\noindent Usage: 
\begin{center}
\textbf{numberroots}(\emph{p}, \emph{I}) : (\textsf{function}, \textsf{range}) $\rightarrow$ \textsf{integer}\\
\end{center}
Parameters: 
\begin{itemize}
\item \emph{p} is the polynomials to be analyzed.
\item \emph{I} is the interval over which the polynomial is to be analyzed.
\end{itemize}
\noindent Description: \begin{itemize}

\item Nothing.
\end{itemize}
\noindent Example 1: 
\begin{center}\begin{minipage}{15cm}\begin{Verbatim}[frame=single]
> numberroots(0.25*x^3 - x^2 + 1,[1;7]);
2
> findzeros(0.25*x^3 - x^2 + 1,[1;7]);
[|[1.193359375;1.194091796875], [3.709228515625;3.7099609375]|]
\end{Verbatim}
\end{minipage}\end{center}
See also: \textbf{dirtyfindzeros} (\ref{labdirtyfindzeros}), \textbf{findzeros} (\ref{labfindzeros}), \textbf{autodiff} (\ref{labautodiff}), \textbf{taylorform} (\ref{labtaylorform})

\subsection{numerator}
\label{labnumerator}
\noindent Name: \textbf{numerator}\\
gives the numerator of an expression\\
\noindent Usage: 
\begin{center}
\textbf{numerator}(\emph{expr}) : \textsf{function} $\rightarrow$ \textsf{function}\\
\end{center}
Parameters: 
\begin{itemize}
\item \emph{expr} represents an expression
\end{itemize}
\noindent Description: \begin{itemize}

\item If \\emph{expr} represents a fraction \\emph{expr1}/\\emph{expr2}, \\textbf{numerator}(\\emph{expr})\n   returns the numerator of this fraction, i.e. \\emph{expr1}.\n    \n   If \\emph{expr} represents something else, \\textbf{numerator}(\\emph{expr}) \n   returns the expression itself, i.e. \\emph{expr}.\n    \n   Note that for all expressions \\emph{expr}, \\textbf{numerator}(\\emph{expr}) \\textbf{/} \\textbf{denominator}(\\emph{expr})\n   is equal to \\emph{expr}.\n\end{itemize}
\noindent Example 1: 
\begin{center}\begin{minipage}{15cm}\begin{Verbatim}[frame=single]
\end{Verbatim}
\end{minipage}\end{center}
\noindent Example 2: 
\begin{center}\begin{minipage}{15cm}\begin{Verbatim}[frame=single]
\end{Verbatim}
\end{minipage}\end{center}
\noindent Example 3: 
\begin{center}\begin{minipage}{15cm}\begin{Verbatim}[frame=single]
\end{Verbatim}
\end{minipage}\end{center}
\noindent Example 4: 
\begin{center}\begin{minipage}{15cm}\begin{Verbatim}[frame=single]
\end{Verbatim}
\end{minipage}\end{center}
See also: \textbf{denominator} (\ref{labdenominator})

\subsection{off}
\label{laboff}
\noindent Name: \textbf{off}\\
special value for certain global variables.\\
\noindent Description: \begin{itemize}

\item \\textbf{off} is a special value used to deactivate certain functionnalities\n   of \\sollya (namely \\textbf{canonical}, \\textbf{timing}, \\textbf{fullparentheses}, \\textbf{midpointmode} and \\textbf{rationalmode}).\n
\item As any value it can be affected to a variable and stored in lists.\n\end{itemize}
\noindent Example 1: 
\begin{center}\begin{minipage}{15cm}\begin{Verbatim}[frame=single]
\end{Verbatim}
\end{minipage}\end{center}
See also: \textbf{on} (\ref{labon}), \textbf{canonical} (\ref{labcanonical}), \textbf{timing} (\ref{labtiming}), \textbf{fullparentheses} (\ref{labfullparentheses}), \textbf{midpointmode} (\ref{labmidpointmode}), \textbf{rationalmode} (\ref{labrationalmode})

\subsection{on}
\label{labon}
\noindent Name: \textbf{on}\\
special value for certain global variables.\\
\noindent Description: \begin{itemize}

\item \\textbf{on} is a special value used to activate certain functionnalities of \\sollya\n   (namely \\textbf{canonical}, \\textbf{timing}, \\textbf{fullparentheses}, \\textbf{midpointmode} and \\textbf{rationalmode}).\n
\item As any value it can be affected to a variable and stored in lists.\n\end{itemize}
\noindent Example 1: 
\begin{center}\begin{minipage}{15cm}\begin{Verbatim}[frame=single]
\end{Verbatim}
\end{minipage}\end{center}
See also: \textbf{off} (\ref{laboff}), \textbf{canonical} (\ref{labcanonical}), \textbf{timing} (\ref{labtiming}), \textbf{fullparentheses} (\ref{labfullparentheses}), \textbf{midpointmode} (\ref{labmidpointmode}), \textbf{rationalmode} (\ref{labrationalmode})

\subsection{ or }
\noindent Name: \textbf{$||$}\\
boolean OR operator\\

\noindent Usage: 
\begin{center}
\emph{expr1} \textbf{$||$} \emph{expr2} : (\textsf{boolean}, \textsf{boolean}) $\rightarrow$ \textsf{boolean}\\
\end{center}
Parameters: 
\emph{expr1} and \emph{expr2} represent boolean expressions\\

\noindent Description: \begin{itemize}

\item \textbf{$||$} evaluates to the boolean OR of the two
   boolean expressions \emph{expr1} and \emph{expr2}. \textbf{$||$} evaluates to 
   true iff at least one of \emph{expr1} or \emph{expr2} evaluate to true.
\end{itemize}
\noindent Example 1: 
\begin{center}\begin{minipage}{14.8cm}\begin{Verbatim}[frame=single]
   > false || false;
   false
\end{Verbatim}
\end{minipage}\end{center}
\noindent Example 2: 
\begin{center}\begin{minipage}{14.8cm}\begin{Verbatim}[frame=single]
   > (1 == exp(0)) || (0 == log(1));
   true
\end{Verbatim}
\end{minipage}\end{center}
See also: \textbf{$\&\&$}, \textbf{!}

\subsection{ parse }
\noindent Name: \textbf{parse}\\
parses an expression contained in a string\\

\noindent Usage: 
\begin{center}
\textbf{parse}(\emph{string}) : \textsf{string} $\rightarrow$ \textsf{function} | \textsf{error}\\
\end{center}
Parameters: 
\emph{string} represents a character sequence\\

\noindent Description: \begin{itemize}

\item \textbf{parse}(\emph{string}) parses the character sequence \emph{string} containing
   an expression built on constants and base functions.
   If the character sequence does not contain a well-defined expression,
   a warning is displayed indicating a syntax error and \textbf{parse} returns
   a \textbf{error} of type \textsf{error}.
\end{itemize}
\noindent Example 1: 
\begin{center}\begin{minipage}{14.8cm}\begin{Verbatim}[frame=single]
   > parse("exp(x)");
   exp(x)
\end{Verbatim}
\end{minipage}\end{center}
\noindent Example 2: 
\begin{center}\begin{minipage}{14.8cm}\begin{Verbatim}[frame=single]
   > verbosity = 1!;
   > parse("5 + + 3");
   Warning: syntax error, unexpected PLUSTOKEN. Will try to continue parsing (expecting ";"). May leak memory.
   Warning: the string "5 + + 3" could not be parsed by the miniparser.
   Warning: at least one of the given expressions or a subexpression is not correctly typed
   or its evaluation has failed because of some error on a side-effect.
   error
\end{Verbatim}
\end{minipage}\end{center}
See also: \textbf{execute}, \textbf{readfile}

\subsection{perturb}
\label{labperturb}
\noindent Name: \textbf{perturb}\\
\phantom{aaa}indicates random perturbation of sampling points for \textbf{externalplot}\\[0.2cm]
\noindent Library names:\\
\verb|   sollya_obj_t sollya_lib_perturb()|\\
\verb|   int sollya_lib_is_perturb(sollya_obj_t)|\\[0.2cm]
\noindent Usage: 
\begin{center}
\textbf{perturb} : \textsf{perturb}\\
\end{center}
\noindent Description: \begin{itemize}

\item The use of \textbf{perturb} in the command \textbf{externalplot} enables the addition
   of some random noise around each sampling point in \textbf{externalplot}.
    
   See \textbf{externalplot} for details.
\end{itemize}
\noindent Example 1: 
\begin{center}\begin{minipage}{15cm}\begin{Verbatim}[frame=single]
> bashexecute("gcc -fPIC -c externalplotexample.c");
> bashexecute("gcc -shared -o externalplotexample externalplotexample.o -lgmp -l
mpfr");
> externalplot("./externalplotexample",relative,exp(x),[-1/2;1/2],12,perturb);
\end{Verbatim}
\end{minipage}\end{center}
See also: \textbf{externalplot} (\ref{labexternalplot}), \textbf{absolute} (\ref{lababsolute}), \textbf{relative} (\ref{labrelative}), \textbf{bashexecute} (\ref{labbashexecute})

\subsection{pi}
\label{labpi}
\noindent Name: \textbf{pi}\\
the constant $\pi$.\\

\noindent Description: \begin{itemize}

\item \textbf{pi} is the constant $\pi$, defined as half the period of sine and cosine.

\item In Sollya, \textbf{pi} is considered as a 0-ary function. This way, the constant 
   is not evaluated at the time of its definition but at the time of its use. For 
   instance, when you define a constant or a function relating to $\pi$, the current
   precision at the time of the definition does not matter. What is important is 
   the current precision when you evaluate the function or the constant value.

\item Remark that when you define an interval, the bounds are first evaluated and 
   then the interval is defined. In this case, \textbf{pi} will be evaluated as any 
   other constant value at the definition time of the interval, thus using the 
   current precision at this time.
\end{itemize}
\noindent Example 1: 
\begin{center}\begin{minipage}{15cm}\begin{Verbatim}[frame=single]
> verbosity=1!; prec=12!;
> a = 2*pi;
> a;
Warning: rounding has happened. The value displayed is a faithful rounding of th
e true result.
0.62832e1
> prec=20!;
> a;
Warning: rounding has happened. The value displayed is a faithful rounding of th
e true result.
0.62831879e1
\end{Verbatim}
\end{minipage}\end{center}
\noindent Example 2: 
\begin{center}\begin{minipage}{15cm}\begin{Verbatim}[frame=single]
> prec=12!;
> d = [pi; 5];
> d;
[0.31406e1;5]
> prec=20!;
> d;
[0.31406e1;5]
\end{Verbatim}
\end{minipage}\end{center}
See also: \textbf{cos} (\ref{labcos}), \textbf{sin} (\ref{labsin})

\subsection{plot}
\label{labplot}
\noindent Name: \textbf{plot}\\
plots one or several functions\\
\noindent Usage: 
\begin{center}
\textbf{plot}(\emph{f1}, ... ,\emph{fn}, \emph{I}) : (\textsf{function}, ... ,\textsf{function}, \textsf{range}) $\rightarrow$ \textsf{void}\\
\textbf{plot}(\emph{f1}, ... ,\emph{fn}, \emph{I}, \textbf{file}, \emph{name}) : (\textsf{function}, ... ,\textsf{function}, \textsf{range}, \textbf{file}, \textsf{string}) $\rightarrow$ \textsf{void}\\
\textbf{plot}(\emph{f1}, ... ,\emph{fn}, \emph{I}, \textbf{postscript}, \emph{name}) : (\textsf{function}, ... ,\textsf{function}, \textsf{range}, \textbf{postscript}, \textsf{string}) $\rightarrow$ \textsf{void}\\
\textbf{plot}(\emph{f1}, ... ,\emph{fn}, \emph{I}, \textbf{postscriptfile}, \emph{name}) : (\textsf{function}, ... ,\textsf{function}, \textsf{range}, \textbf{postscriptfile}, \textsf{string}) $\rightarrow$ \textsf{void}\\
\textbf{plot}(\emph{L}, \emph{I}) : (\textsf{list}, \textsf{range}) $\rightarrow$ \textsf{void}\\
\textbf{plot}(\emph{L}, \emph{I}, \textbf{file}, \emph{name}) : (\textsf{list}, \textsf{range}, \textbf{file}, \textsf{string}) $\rightarrow$ \textsf{void}\\
\textbf{plot}(\emph{L}, \emph{I}, \textbf{postscript}, \emph{name}) : (\textsf{list}, \textsf{range}, \textbf{postscript}, \textsf{string}) $\rightarrow$ \textsf{void}\\
\textbf{plot}(\emph{L}, \emph{I}, \textbf{postscriptfile}, \emph{name}) : (\textsf{list}, \textsf{range}, \textbf{postscriptfile}, \textsf{string}) $\rightarrow$ \textsf{void}\\
\end{center}
Parameters: 
\begin{itemize}
\item \emph{f1}, ..., \emph{fn} are functions to be plotted.
\item \emph{L} is a list of functions to be plotted.
\item \emph{I} is the interval where the functions have to be plotted.
\item \emph{name} is a string representing the name of a file.
\end{itemize}
\noindent Description: \begin{itemize}

\item This command plots one or several functions \\emph{f1}, ... ,\\emph{fn} on an interval \\emph{I}.\n   Functions can be either given as parameters of \\textbf{plot} or as a list \\emph{L}\n   which elements are functions.\n   The functions are drawn on the same plot with different colors.\n
\item If \\emph{L} contains an element that is not a function (or a constant), an error\n   occurs.\n
\item \\textbf{plot} relies on the value of global variable \\textbf{points}. Let $n$ be the \n   value of this variable. The algorithm is the following: each function is \n   evaluated at $n$ evenly distributed points in \\emph{I}. At each point, the \n   computed value is a faithful rounding of the exact value with a sufficiently\n   high precision. Each point is finally plotted.\n   This should avoid numerical artefacts such as critical cancellations.\n
\item You can save the function plot either as a data file or as a postscript file.\n
\item If you use argument \\textbf{file} with a string \\emph{name}, \\sollya will save a data file\n   called name.dat and a gnuplot directives file called name.p. Invoking gnuplot\n   on name.p will plot the data stored in name.dat.\n
\item If you use argument \\textbf{postscript} with a string \\emph{name}, \\sollya will save a \n   postscript file called name.eps representing your plot.\n
\item If you use argument \\textbf{postscriptfile} with a string \\emph{name}, \\sollya will \n   produce the corresponding name.dat, name.p and name.eps.\n
\item This command uses gnuplot to produce the final plot.\n   If your terminal is not graphic (typically if you use \\sollya through \n   ssh without -X)\n   gnuplot should be able to detect that and produce an ASCII-art version on the\n   standard output. If it is not the case, you can either store the plot in a\n   postscript file to view it locally, or use \\textbf{asciiplot} command.\n
\item If every function is constant, \\textbf{plot} will not plot them but just display\n   their value.\n
\item If the interval is reduced to a single point, \\textbf{plot} will just display the\n   value of the functions at this point.\n\end{itemize}
\noindent Example 1: 
\begin{center}\begin{minipage}{15cm}\begin{Verbatim}[frame=single]
\end{Verbatim}
\end{minipage}\end{center}
\noindent Example 2: 
\begin{center}\begin{minipage}{15cm}\begin{Verbatim}[frame=single]
\end{Verbatim}
\end{minipage}\end{center}
\noindent Example 3: 
\begin{center}\begin{minipage}{15cm}\begin{Verbatim}[frame=single]
\end{Verbatim}
\end{minipage}\end{center}
\noindent Example 4: 
\begin{center}\begin{minipage}{15cm}\begin{Verbatim}[frame=single]
\end{Verbatim}
\end{minipage}\end{center}
See also: \textbf{externalplot} (\ref{labexternalplot}), \textbf{asciiplot} (\ref{labasciiplot}), \textbf{file} (\ref{labfile}), \textbf{postscript} (\ref{labpostscript}), \textbf{postscriptfile} (\ref{labpostscriptfile}), \textbf{points} (\ref{labpoints})

\subsection{$+$}
\label{labplus}
\noindent Name: \textbf{$+$}\\
addition function\\
\noindent Usage: 
\begin{center}
\emph{function1} \textbf{$+$} \emph{function2} : (\textsf{function}, \textsf{function}) $\rightarrow$ \textsf{function}\\
\end{center}
Parameters: 
\begin{itemize}
\item \emph{function1} and \emph{function2} represent functions
\end{itemize}
\noindent Description: \begin{itemize}

\item \textbf{$+$} represents the addition (function) on reals. 
   The expression \emph{function1} \textbf{$+$} \emph{function2} stands for
   the function composed of the addition function and the two
   functions \emph{function1} and \emph{function2}.
\end{itemize}
\noindent Example 1: 
\begin{center}\begin{minipage}{15cm}\begin{Verbatim}[frame=single]
> 1 + 2;
3
\end{Verbatim}
\end{minipage}\end{center}
\noindent Example 2: 
\begin{center}\begin{minipage}{15cm}\begin{Verbatim}[frame=single]
> x + 2;
2 + x
\end{Verbatim}
\end{minipage}\end{center}
\noindent Example 3: 
\begin{center}\begin{minipage}{15cm}\begin{Verbatim}[frame=single]
> x + x;
x * 2
\end{Verbatim}
\end{minipage}\end{center}
\noindent Example 4: 
\begin{center}\begin{minipage}{15cm}\begin{Verbatim}[frame=single]
> diff(sin(x) + exp(x));
cos(x) + exp(x)
\end{Verbatim}
\end{minipage}\end{center}
See also: \textbf{$-$} (\ref{labminus}), \textbf{$*$} (\ref{labmult}), \textbf{/} (\ref{labdivide}), \textbf{$\mathbf{\hat{~}}$} (\ref{labpower})

\subsection{points}
\label{labpoints}
\noindent Name: \textbf{points}\\
controls the number of points chosen by Sollya in certain commands.\\

\noindent Description: \begin{itemize}

\item \textbf{points} is a global variable. Its value represents the number of points
   used in numerical algorithms of Sollya (namely \textbf{dirtyinfnorm},
   \textbf{dirtyintegral}, \textbf{dirtyfindzeros}, \textbf{plot}).
\end{itemize}
\noindent Example 1: 
\begin{center}\begin{minipage}{15cm}\begin{Verbatim}[frame=single]
> f=x^2*sin(1/x);
> points=10;
The number of points has been set to 10.
> dirtyfindzeros(f, [0;1]);
[|0, 0.318309886183790671537767526745028724068919291480918|]
> points=100;
The number of points has been set to 100.
> dirtyfindzeros(f, [0;1]);
[|0, 0.24485375860291590118289809749617594159147637806224e-1, 0.3536776513153229
6837529725193892080452102143497879e-1, 0.454728408833986673625382181064326748669
884702115589e-1, 0.53051647697298445256294587790838120678153215246819e-1, 0.6366
1977236758134307553505349005744813783858296183e-1, 0.774999999999999999999999999
99999999999999999999134e-1, 0.10610329539459689051258917558167624135630643049363
8, 0.159154943091895335768883763372514362034459645740459, 0.31830988618379067153
7767526745028724068919291480918|]
\end{Verbatim}
\end{minipage}\end{center}
See also: \textbf{dirtyinfnorm} (\ref{labdirtyinfnorm}), \textbf{dirtyintegral} (\ref{labdirtyintegral}), \textbf{dirtyfindzeros} (\ref{labdirtyfindzeros}), \textbf{plot} (\ref{labplot})

\subsection{postscriptfile}
\label{labpostscriptfile}
\noindent Name: \textbf{postscriptfile}\\
special value for commands \textbf{plot} and \textbf{externalplot}\\
\noindent Description: \begin{itemize}

\item \\textbf{postscriptfile} is a special value used in commands \\textbf{plot} and \\textbf{externalplot} to save\n   the result of the command in a data file and a postscript file.\n
\item As any value it can be affected to a variable and stored in lists.\n\end{itemize}
\noindent Example 1: 
\begin{center}\begin{minipage}{15cm}\begin{Verbatim}[frame=single]
\end{Verbatim}
\end{minipage}\end{center}
See also: \textbf{externalplot} (\ref{labexternalplot}), \textbf{plot} (\ref{labplot}), \textbf{file} (\ref{labfile}), \textbf{postscript} (\ref{labpostscript})

\subsection{postscript}
\label{labpostscript}
\noindent Name: \textbf{postscript}\\
\phantom{aaa}special value for commands \textbf{plot} and \textbf{externalplot}\\[0.2cm]
\noindent Library names:\\
\verb|   sollya_obj_t sollya_lib_postscript()|\\
\verb|   int sollya_lib_is_postscript(sollya_obj_t)|\\[0.2cm]
\noindent Description: \begin{itemize}

\item \textbf{postscript} is a special value used in commands \textbf{plot} and \textbf{externalplot} to save
   the result of the command in a postscript file.

\item As any value it can be affected to a variable and stored in lists.
\end{itemize}
\noindent Example 1: 
\begin{center}\begin{minipage}{15cm}\begin{Verbatim}[frame=single]
> savemode=postscript;
> name="plotSinCos";
> plot(sin(x),0,cos(x),[-Pi,Pi],savemode, name);
\end{Verbatim}
\end{minipage}\end{center}
See also: \textbf{externalplot} (\ref{labexternalplot}), \textbf{plot} (\ref{labplot}), \textbf{file} (\ref{labfile}), \textbf{postscriptfile} (\ref{labpostscriptfile})

\subsection{powers}
\label{labpowers}
\noindent Name: \textbf{powers}\\
\phantom{aaa}special value for global state \textbf{display}\\[0.2cm]
\noindent Library names:\\
\verb|   sollya_obj_t sollya_lib_powers()|\\
\verb|   int sollya_lib_is_powers(sollya_obj_t)|\\[0.2cm]
\noindent Description: \begin{itemize}

\item \textbf{powers} is a special value used for the global state \textbf{display}.  If
   the global state \textbf{display} is equal to \textbf{powers}, all data will be
   output in dyadic notation with numbers displayed in a Maple and
   PARI/GP compatible format.
    
   As any value it can be affected to a variable and stored in lists.
\end{itemize}
See also: \textbf{decimal} (\ref{labdecimal}), \textbf{dyadic} (\ref{labdyadic}), \textbf{hexadecimal} (\ref{labhexadecimal}), \textbf{binary} (\ref{labbinary}), \textbf{display} (\ref{labdisplay})

\subsection{power}
\label{labpower}
\noindent Name: \textbf{\^}\\
power function\\
\noindent Usage: 
\begin{center}
\emph{function1} \textbf{\^} \emph{function2} : (\textsf{function}, \textsf{function}) $\rightarrow$ \textsf{function}
\end{center}
Parameters: 
\begin{itemize}
\item \emph{function1} and \emph{function2} represent functions
\end{itemize}
\noindent Description: \begin{itemize}

\item \textbf{\^} represents the power (function) on reals. 
   The expression \emph{function1} \textbf{\^} \emph{function2} stands for
   the function composed of the power function and the two
   functions \emph{function1} and \emph{function2}, where \emph{function1} is
   the base and \emph{function2} the exponent.
   If \emph{function2} is a constant integer, \textbf{\^} is defined
   on negative values of \emph{function1}. Otherwise \textbf{\^}
   is defined as $e^{y \cdot \ln x}$.

\item Note that whenever several \textbf{\^} are composed, the priority goes
   to the last \textbf{\^}. This corresponds to the natural way of
   thinking when a tower of powers is written on a paper.
   Thus, \verb|2^3^5| is read as $2^{3^5}$ and is interpreted as $2^{(3^5)}$.
\end{itemize}
\noindent Example 1: 
\begin{center}\begin{minipage}{15cm}\begin{Verbatim}[frame=single]
> 5 ^ 2;
25
\end{Verbatim}
\end{minipage}\end{center}
\noindent Example 2: 
\begin{center}\begin{minipage}{15cm}\begin{Verbatim}[frame=single]
> x ^ 2;
x^2
\end{Verbatim}
\end{minipage}\end{center}
\noindent Example 3: 
\begin{center}\begin{minipage}{15cm}\begin{Verbatim}[frame=single]
> 3 ^ (-5);
4.1152263374485596707818930041152263374485596707818e-3
\end{Verbatim}
\end{minipage}\end{center}
\noindent Example 4: 
\begin{center}\begin{minipage}{15cm}\begin{Verbatim}[frame=single]
> (-3) ^ (-2.5);
@NaN@
\end{Verbatim}
\end{minipage}\end{center}
\noindent Example 5: 
\begin{center}\begin{minipage}{15cm}\begin{Verbatim}[frame=single]
> diff(sin(x) ^ exp(x));
sin(x)^exp(x) * ((cos(x) * exp(x)) / sin(x) + exp(x) * log(sin(x)))
\end{Verbatim}
\end{minipage}\end{center}
\noindent Example 6: 
\begin{center}\begin{minipage}{15cm}\begin{Verbatim}[frame=single]
> 2^3^5;
1.4134776518227074636666380005943348126619871175005e73
> (2^3)^5;
32768
> 2^(3^5);
1.4134776518227074636666380005943348126619871175005e73
\end{Verbatim}
\end{minipage}\end{center}
See also: \textbf{$+$} (\ref{labplus}), \textbf{$-$} (\ref{labminus}), \textbf{$*$} (\ref{labmult}), \textbf{/} (\ref{labdivide})

\subsection{precision}
\label{labprecision}
\noindent Name: \textbf{precision}\\
\phantom{aaa}returns the precision necessary to represent a number.\\[0.2cm]
\noindent Library name:\\
\verb|   sollya_obj_t sollya_lib_precision(sollya_obj_t)|\\[0.2cm]
\noindent Usage: 
\begin{center}
\textbf{precision}(\emph{x}) : \textsf{constant} $\rightarrow$ \textsf{integer}\\
\end{center}
Parameters: 
\begin{itemize}
\item \emph{x} is a dyadic number.
\end{itemize}
\noindent Description: \begin{itemize}

\item \textbf{precision}(x) is by definition $\vert x \vert$ if x equals 0, NaN, or Inf.

\item If \emph{x} is not zero, it can be uniquely written as $x = m \cdot 2^e$ where
   $m$ is an odd integer and $e$ is an integer. \textbf{precision}(x) returns the number
   of bits necessary to write $m$ in binary (i.e. $1+ \lfloor \log_2(m) \rfloor$).
\end{itemize}
\noindent Example 1: 
\begin{center}\begin{minipage}{15cm}\begin{Verbatim}[frame=single]
> a=round(Pi,20,RN);
> precision(a);
19
> m=mantissa(a);
> 1+floor(log2(m));
19
\end{Verbatim}
\end{minipage}\end{center}
\noindent Example 2: 
\begin{center}\begin{minipage}{15cm}\begin{Verbatim}[frame=single]
> a=255;
> precision(a);
8
> m=mantissa(a);
> 1+floor(log2(m));
8
\end{Verbatim}
\end{minipage}\end{center}
\noindent Example 3: 
\begin{center}\begin{minipage}{15cm}\begin{Verbatim}[frame=single]
> a=256;
> precision(a);
1
> m=mantissa(a);
> 1+floor(log2(m));
1
\end{Verbatim}
\end{minipage}\end{center}
See also: \textbf{mantissa} (\ref{labmantissa}), \textbf{exponent} (\ref{labexponent}), \textbf{round} (\ref{labround})

\subsection{prec}
\label{labprec}
\noindent Name: \textbf{prec}\\
controls the precision used in numerical computations.\\
\noindent Description: \begin{itemize}

\item \\textbf{prec} is a global variable. Its value represents the precision of the \n   floating-point format used in numerical computations.\n
\item Many commands try to adapt their working precision in order to have \n   approximately $n$ correct bits in output, where $n$ is the value of \\textbf{prec}.\n\end{itemize}
\noindent Example 1: 
\begin{center}\begin{minipage}{15cm}\begin{Verbatim}[frame=single]
\end{Verbatim}
\end{minipage}\end{center}

\subsection{.:}
\label{labprepend}
\noindent Name: \textbf{.:}\\
\phantom{aaa}add an element at the beginning of a list.\\[0.2cm]
\noindent Library name:\\
\verb|   sollya_obj_t sollya_lib_prepend(sollya_obj_t, sollya_obj_t)|\\[0.2cm]
\noindent Usage: 
\begin{center}
\emph{x}\textbf{.:}\emph{L} : (\textsf{any type}, \textsf{list}) $\rightarrow$ \textsf{list}\\
\end{center}
Parameters: 
\begin{itemize}
\item \emph{x} is an object of any type.
\item \emph{L} is a list (possibly empty).
\end{itemize}
\noindent Description: \begin{itemize}

\item \textbf{.:} adds the element \emph{x} at the beginning of the list \emph{L}.

\item Note that since \emph{x} may be of any type, it can be in particular a list.
\end{itemize}
\noindent Example 1: 
\begin{center}\begin{minipage}{15cm}\begin{Verbatim}[frame=single]
> 1.:[|2,3,4|];
[|1, 2, 3, 4|]
\end{Verbatim}
\end{minipage}\end{center}
\noindent Example 2: 
\begin{center}\begin{minipage}{15cm}\begin{Verbatim}[frame=single]
> [|1,2,3|].:[|4,5,6|];
[|[|1, 2, 3|], 4, 5, 6|]
\end{Verbatim}
\end{minipage}\end{center}
\noindent Example 3: 
\begin{center}\begin{minipage}{15cm}\begin{Verbatim}[frame=single]
> 1.:[||];
[|1|]
\end{Verbatim}
\end{minipage}\end{center}
See also: \textbf{:.} (\ref{labappend}), \textbf{@} (\ref{labconcat})

\input{printdouble}
\subsection{printexpansion}
\label{labprintexpansion}
\noindent Name: \textbf{printexpansion}\\
prints a polynomial in Horner form with its coefficients written as a expansions of double precision numbers\\
\noindent Usage: 
\begin{center}
\textbf{printexpansion}(\emph{polynomial}) : \textsf{function} $\rightarrow$ \textsf{void}\\
\end{center}
Parameters: 
\begin{itemize}
\item \emph{polynomial} represents the polynomial to be printed
\end{itemize}
\noindent Description: \begin{itemize}

\item The command \\textbf{printexpansion} prints the polynomial \\emph{polynomial} in Horner form\n   writing its coefficients as expansions of double precision\n   numbers. The double precision numbers themselves are displayed in\n   hexadecimal memory notation (see \\textbf{printhexa}). \n    \n   If some of the coefficients of the polynomial \\emph{polynomial} are not\n   floating-point constants but constant expressions, they are evaluated\n   to floating-point constants using the global precision \\textbf{prec}.  If a\n   rounding occurs in this evaluation, a warning is displayed.\n    \n   If the exponent range of double precision is not sufficient to display\n   all the mantissa bits of a coefficient, the coefficient is displayed\n   rounded and a warning is displayed.\n    \n   If the argument \\emph{polynomial} does not a polynomial, nothing but a\n   warning or a newline is displayed. Constants can be displayed using\n   \\textbf{printexpansion} since they are polynomials of degree $0$.\n\end{itemize}
\noindent Example 1: 
\begin{center}\begin{minipage}{15cm}\begin{Verbatim}[frame=single]
\end{Verbatim}
\end{minipage}\end{center}
\noindent Example 2: 
\begin{center}\begin{minipage}{15cm}\begin{Verbatim}[frame=single]
\end{Verbatim}
\end{minipage}\end{center}
\noindent Example 3: 
\begin{center}\begin{minipage}{15cm}\begin{Verbatim}[frame=single]
\end{Verbatim}
\end{minipage}\end{center}
See also: \textbf{printhexa} (\ref{labprinthexa}), \textbf{horner} (\ref{labhorner}), \textbf{print} (\ref{labprint}), \textbf{prec} (\ref{labprec}), \textbf{remez} (\ref{labremez}), \textbf{taylor} (\ref{labtaylor}), \textbf{roundcoefficients} (\ref{labroundcoefficients})

\input{printsingle}
\subsection{printxml}
\label{labprintxml}
\noindent Name: \textbf{printxml}\\
prints an expression as an MathML-Content-Tree\\
\noindent Usage: 
\begin{center}
\textbf{printxml}(\emph{expr}) : \textsf{function} $\rightarrow$ \textsf{void}\\
\textbf{printxml}(\emph{expr}) $>$ \emph{filename} : (\textsf{function}, \textsf{string}) $\rightarrow$ \textsf{void}\\
\textbf{printxml}(\emph{expr}) $>$ $>$ \emph{filename} : (\textsf{function}, \textsf{string}) $\rightarrow$ \textsf{void}\\
\end{center}
Parameters: 
\begin{itemize}
\item \emph{expr} represents a functional expression
\item \emph{filename} represents a character sequence indicating a file name
\end{itemize}
\noindent Description: \begin{itemize}

\item \\textbf{printxml}(\\emph{expr}) prints the functional expression \\emph{expr} as a tree of\n   MathML Content Definition Markups. This XML tree can be re-read in\n   external tools or by usage of the \\textbf{readxml} command.\n    \n   If a second argument \\emph{filename} is given after a single $>$, the\n   MathML tree is not output on the standard output of \\sollya but if in\n   the file \\emph{filename} that get newly created or overwritten. If a double\n   $>$ $>$ is given, the output will be appended to the file \\emph{filename}.\n\end{itemize}
\noindent Example 1: 
\begin{center}\begin{minipage}{15cm}\begin{Verbatim}[frame=single]
\end{Verbatim}
\end{minipage}\end{center}
\noindent Example 2: 
\begin{center}\begin{minipage}{15cm}\begin{Verbatim}[frame=single]
\end{Verbatim}
\end{minipage}\end{center}
\noindent Example 3: 
\begin{center}\begin{minipage}{15cm}\begin{Verbatim}[frame=single]
\end{Verbatim}
\end{minipage}\end{center}
See also: \textbf{readxml} (\ref{labreadxml}), \textbf{print} (\ref{labprint}), \textbf{write} (\ref{labwrite})

\subsection{print}
\label{labprint}
\noindent Name: \textbf{print}\\
prints an expression\\
\noindent Usage: 
\begin{center}
\textbf{print}(\emph{expr1},...,\emph{exprn}) : (\textsf{any type},..., \textsf{any type}) $\rightarrow$ \textsf{void}\\
\textbf{print}(\emph{expr1},...,\emph{exprn}) $>$ \emph{filename} : (\textsf{any type},..., \textsf{any type}, \textsf{string}) $\rightarrow$ \textsf{void}\\
\textbf{print}(\emph{expr1},...,\emph{exprn}) $>>$ \emph{filename} : (\textsf{any type},...,\textsf{any type}, \textsf{string}) $\rightarrow$ \textsf{void}\\
\end{center}
Parameters: 
\begin{itemize}
\item \emph{expr} represents an expression
\item \emph{filename} represents a character sequence indicating a file name
\end{itemize}
\noindent Description: \begin{itemize}

\item \\textbf{print}(\\emph{expr1},...,\\emph{exprn}) prints the expressions \\emph{expr1} through\n   \\emph{exprn} separated by spaces and followed by a newline.\n    \n   If a second argument \\emph{filename} is given after a single  "$>$", the\n   displaying is not output on the standard output of \\sollya but if in\n   the file \\emph{filename} that get newly created or overwritten. If a double\n    "$>>$" is given, the output will be appended to the file \\emph{filename}.\n    \n   The global variables \\textbf{display}, \\textbf{midpointmode} and \\textbf{fullparentheses} have\n   some influence on the formatting of the output (see \\textbf{display},\n   \\textbf{midpointmode} and \\textbf{fullparentheses}).\n    \n   Remark that if one of the expressions \\emph{expri} given in argument is of\n   type \\textsf{string}, the character sequence \\emph{expri} evaluates to is\n   displayed. However, if \\emph{expri} is of type \\textsf{list} and this list\n   contains a variable of type \\textsf{string}, the expression for the list\n   is displayed, i.e.  all character sequences get displayed surrounded\n   by double quotes ("). Nevertheless, escape sequences used upon defining\n   character sequences are interpreted immediately.\n\end{itemize}
\noindent Example 1: 
\begin{center}\begin{minipage}{15cm}\begin{Verbatim}[frame=single]
\end{Verbatim}
\end{minipage}\end{center}
\noindent Example 2: 
\begin{center}\begin{minipage}{15cm}\begin{Verbatim}[frame=single]
\end{Verbatim}
\end{minipage}\end{center}
\noindent Example 3: 
\begin{center}\begin{minipage}{15cm}\begin{Verbatim}[frame=single]
\end{Verbatim}
\end{minipage}\end{center}
\noindent Example 4: 
\begin{center}\begin{minipage}{15cm}\begin{Verbatim}[frame=single]
\end{Verbatim}
\end{minipage}\end{center}
\noindent Example 5: 
\begin{center}\begin{minipage}{15cm}\begin{Verbatim}[frame=single]
\end{Verbatim}
\end{minipage}\end{center}
See also: \textbf{write} (\ref{labwrite}), \textbf{printexpansion} (\ref{labprintexpansion}), \textbf{printhexa} (\ref{labprinthexa}), \textbf{printfloat} (\ref{labprintfloat}), \textbf{printxml} (\ref{labprintxml}), \textbf{readfile} (\ref{labreadfile}), \textbf{autosimplify} (\ref{labautosimplify}), \textbf{display} (\ref{labdisplay}), \textbf{midpointmode} (\ref{labmidpointmode}), \textbf{fullparentheses} (\ref{labfullparentheses}), \textbf{evaluate} (\ref{labevaluate})

\subsection{ procedure }
\noindent Name: \textbf{procedure}\\
defines and assigns a Sollya procedure\\

\noindent Usage: 
\begin{center}
\textbf{procedure} \emph{identifier}(\emph{formal parameter1}, \emph{formal parameter2},..., \emph{formal parameter n}) \textbf{begin} \emph{procedure body} \textbf{end} : \textsf{void} $\rightarrow$ \textsf{void}\\
\textbf{procedure} \emph{identifier}(\emph{formal parameter1}, \emph{formal parameter2},..., \emph{formal parameter n}) \textbf{begin} \emph{procedure body} \textbf{return} \emph{expression}; \textbf{end} : \textsf{any type} $\rightarrow$ \textsf{void}\\
\end{center}
Parameters: 
\begin{itemize}
\item \emph{identifier} represents the name of the procedure to be defined and assigned
\item \emph{formal parameter1}, \emph{formal parameter2} through \emph{formal parameter n} represent identifiers used as formal parameters
\item \emph{procedure body} represents the imperative statements in the body of the procedure
\item \emph{expression} represents the expression \textbf{procedure} shall evaluate to
\end{itemize}
\noindent Description: \begin{itemize}

\item The \textbf{procedure} keyword allows for defining and assigning procedures in
   the Sollya language. It is an abbreviation to a procedure definition
   using \textbf{proc} with the same formal parameters, procedure body and
   return-expression followed by an assignment of the procedure (object)
   to the identifier \emph{identifier}. In particular, all rules concerning
   local variables declared using the \textbf{var} keyword apply for \textbf{procedure}.
\end{itemize}
\noindent Example 1: 
\begin{center}\begin{minipage}{15cm}\begin{Verbatim}[frame=single]
> procedure succ(n) { return n + 1; };
> succ(5);
6
> 3 + succ(0);
4
> succ;
proc(n)
begin
nop;
return (n) + (1);
end
\end{Verbatim}
\end{minipage}\end{center}
See also: \textbf{proc}, \textbf{var}

\subsection{proc}
\label{labproc}
\noindent Name: \textbf{proc}\\
defines a \sollya procedure\\
\noindent Usage: 
\begin{center}
\textbf{proc}(\emph{formal parameter1}, \emph{formal parameter2},..., \emph{formal parameter n}) \textbf{begin} \emph{procedure body} \textbf{end} : \textsf{void} $\rightarrow$ \textsf{procedure}\\
\textbf{proc}(\emph{formal parameter1}, \emph{formal parameter2},..., \emph{formal parameter n}) \textbf{begin} \emph{procedure body} \textbf{return} \emph{expression}; \textbf{end} : \textsf{void} $\rightarrow$ \textsf{procedure}\\
\textbf{proc}(\emph{formal list parameter} = ...) \textbf{begin} \emph{procedure body} \textbf{end} : \textsf{void} $\rightarrow$ \textsf{procedure}\\
\textbf{proc}(\emph{formal list parameter} = ...) \textbf{begin} \emph{procedure body} \textbf{return} \emph{expression}; \textbf{end} : \textsf{void} $\rightarrow$ \textsf{procedure}\\
\end{center}
Parameters: 
\begin{itemize}
\item \emph{formal parameter1}, \emph{formal parameter2} through \emph{formal parameter n} represent identifiers used as formal parameters
\item \emph{formal list parameter} represents an identifier used as a formal parameter for the list of an arbitrary number of parameters
\item \emph{procedure body} represents the imperative statements in the body of the procedure
\item \emph{expression} represents the expression \textbf{proc} shall evaluate to
\end{itemize}
\noindent Description: \begin{itemize}

\item The \\textbf{proc} keyword allows for defining procedures in the \\sollya\n   language. These procedures are common \\sollya objects that can be\n   applied to actual parameters after definition. Upon such an\n   application, the \\sollya interpreter applies the actual parameters to\n   the formal parameters \\emph{formal parameter1} through \\emph{formal parameter n}\n   (resp. builds up the list of arguments and applies it to the list\n   \\emph{formal list parameter}) and executes the \\emph{procedure body}. The\n   procedure applied to actual parameters evaluates then to the\n   expression \\emph{expression} in the \\textbf{return} statement after the <procedure\n   body> or to \\textbf{void}, if no return statement is given (i.e. a \\textbf{return}\n   \\textbf{void} statement is implicitly given).\n
\item \\sollya procedures defined by \\textbf{proc} have no name. They can be bound\n   to an identifier by assigning the procedure object a \\textbf{proc}\n   expression produces to an identifier. However, it is possible to use\n   procedures without giving them any name. For instance, \\sollya\n   procedures, i.e. procedure objects, can be elements of lists. They can\n   even be given as an argument to other internal \\sollya procedures. See\n   also \\textbf{procedure} on this subject.\n
\item Upon definition of a \\sollya procedure using \\textbf{proc}, no type check\n   is performed. More precisely, the statements in \\emph{procedure body} are\n   merely parsed but not interpreted upon procedure definition with\n   \\textbf{proc}. Type checks are performed once the procedure is applied to\n   actual parameters or to \\textbf{void}. At this time, if the procedure was\n   defined using several different formal parameters \\emph{formal parameter 1}\n   through \\emph{formal parameter n}, it is checked whether the number of\n   actual parameters corresponds to the number of formal parameters. If\n   the procedure was defined using the syntax for a procedure with an\n   arbitrary number of parameters by giving a \\emph{formal list parameter},\n   the number of actual arguments is not checked but only a list <formal\n   list parameter> of appropriate length is built up. Type checks are\n   further performed upon execution of each statement in \\emph{procedure body}\n   and upon evaluation of the expression \\emph{expression} to be returned.\n    \n   Procedures defined by \\textbf{proc} containing a \\textbf{quit} or \\textbf{restart} command\n   cannot be executed (i.e. applied). Upon application of a procedure,\n   the \\sollya interpreter checks beforehand for such a statement. If one\n   is found, the application of the procedure to its arguments evaluates\n   to \\textbf{error}. A warning is displayed. Remark that in contrast to other\n   type or semantic correctness checks, this check is really performed\n   before interpreting any other statement in the body of the procedure.\n
\item Through the \\textbf{var} keyword it is possible to declare local\n   variables and thus to have full support of recursive procedures. This\n   means a procedure defined using \\textbf{proc} may contain in its \\emph{procedure body} \n   an application of itself to some actual parameters: it suffices\n   to assign the procedure (object) to an identifier with an appropriate\n   name.\n
\item \\sollya procedures defined using \\textbf{proc} may return other\n   procedures. Further \\emph{procedure body} may contain assignments of\n   locally defined procedure objects to identifiers. See \\textbf{var} for the\n   particular behaviour of local and global variables.\n
\item The expression \\emph{expression} returned by a procedure is evaluated with\n   regard to \\sollya commands, procedures and external\n   procedures. Simplification may be performed.  However, an application\n   of a procedure defined by \\textbf{proc} to actual parameters evaluates to the\n   expression \\emph{expression} that may contain the free global variable or\n   that may be composed.\n\end{itemize}
\noindent Example 1: 
\begin{center}\begin{minipage}{15cm}\begin{Verbatim}[frame=single]
\end{Verbatim}
\end{minipage}\end{center}
\noindent Example 2: 
\begin{center}\begin{minipage}{15cm}\begin{Verbatim}[frame=single]
\end{Verbatim}
\end{minipage}\end{center}
\noindent Example 3: 
\begin{center}\begin{minipage}{15cm}\begin{Verbatim}[frame=single]
\end{Verbatim}
\end{minipage}\end{center}
\noindent Example 4: 
\begin{center}\begin{minipage}{15cm}\begin{Verbatim}[frame=single]
\end{Verbatim}
\end{minipage}\end{center}
\noindent Example 5: 
\begin{center}\begin{minipage}{15cm}\begin{Verbatim}[frame=single]
\end{Verbatim}
\end{minipage}\end{center}
\noindent Example 6: 
\begin{center}\begin{minipage}{15cm}\begin{Verbatim}[frame=single]
\end{Verbatim}
\end{minipage}\end{center}
\noindent Example 7: 
\begin{center}\begin{minipage}{15cm}\begin{Verbatim}[frame=single]
\end{Verbatim}
\end{minipage}\end{center}
\noindent Example 8: 
\begin{center}\begin{minipage}{15cm}\begin{Verbatim}[frame=single]
\end{Verbatim}
\end{minipage}\end{center}
\noindent Example 9: 
\begin{center}\begin{minipage}{15cm}\begin{Verbatim}[frame=single]
\end{Verbatim}
\end{minipage}\end{center}
\noindent Example 10: 
\begin{center}\begin{minipage}{15cm}\begin{Verbatim}[frame=single]
\end{Verbatim}
\end{minipage}\end{center}
See also: \textbf{return} (\ref{labreturn}), \textbf{externalproc} (\ref{labexternalproc}), \textbf{void} (\ref{labvoid}), \textbf{quit} (\ref{labquit}), \textbf{restart} (\ref{labrestart}), \textbf{var} (\ref{labvar}), \textbf{@} (\ref{labconcat})

\subsection{ quit }
\noindent Name: \textbf{quit}\\
quits Sollya\\

\noindent Usage: 
\begin{center}
\textbf{quit} : \textsf{void} $\rightarrow$ \textsf{void}\\
\end{center}
\noindent Description: \begin{itemize}

\item The command \textbf{quit}, when executed abandons the execution of a Sollya
   script and leaves the Sollya interpreter unless the \textbf{quit} command 
   is executed in a Sollya script read into a main Sollya script by
   \textbf{execute} or $\#$include.
   Upon exiting the Sollya interpreter, all state is thrown away, all
   memory is deallocated, all bound libraries are unbound and the
   temporary files produced by \textbf{plot} and \textbf{externalplot} are deleted.
   If the \textbf{quit} command does not lead to the abandon of the Sollya
   interpreter, a warning is displayed.
\end{itemize}
\noindent Example 1: 
\begin{center}\begin{minipage}{14.8cm}\begin{Verbatim}[frame=single]
   > quit;
\end{Verbatim}
\end{minipage}\end{center}
See also: \textbf{restart}, \textbf{execute}, \textbf{plot}, \textbf{externalplot}

\subsection{range}
\label{labrange}
\noindent Name: \textbf{range}\\
keyword representing a \textsf{range} type \\
\noindent Usage: 
\begin{center}
\textbf{range} : \textsf{type type}\\
\end{center}
\noindent Description: \begin{itemize}

\item \\textbf{range} represents the \\textsf{range} type for declarations\n   of external procedures by means of \\textbf{externalproc}.\n    \n   Remark that in contrast to other indicators, type indicators like\n   \\textbf{range} cannot be handled outside the \\textbf{externalproc} context.  In\n   particular, they cannot be assigned to variables.\n\end{itemize}
See also: \textbf{externalproc} (\ref{labexternalproc}), \textbf{boolean} (\ref{labboolean}), \textbf{constant} (\ref{labconstant}), \textbf{function} (\ref{labfunction}), \textbf{integer} (\ref{labinteger}), \textbf{list of} (\ref{lablistof}), \textbf{string} (\ref{labstring})

\subsection{rationalapprox}
\label{labrationalapprox}
\noindent Name: \textbf{rationalapprox}\\
returns a fraction close to a given number.\\
\noindent Usage: 
\begin{center}
\textbf{rationalapprox}(\emph{x},\emph{n}) : (\textsf{constant}, \textsf{integer}) $\rightarrow$ \textsf{function}\\
\end{center}
Parameters: 
\begin{itemize}
\item \emph{x} is a number to approximate.
\item \emph{n} is a integer (representing a format).
\end{itemize}
\noindent Description: \begin{itemize}

\item \\textbf{rationalapprox}(\\emph{x},\\emph{n}) returns a constant function of the form $a/b$ where $a$ and $b$ are\n   integers. The value $a/b$ is an approximation of \\emph{x}. The quality of this \n   approximation is determined by the parameter \\emph{n} that indicates the number of\n   correct bits that $a/b$ should have.\n
\item The command is not safe in the sense that it is not ensured that the error \n   between $a/b$ and \\emph{x} is less than $2^{-n}$.\n
\item The following algorithm is used: \\emph{x} is first rounded downwards and upwards to\n   a format of \\emph{n} bits, thus obtaining an interval $[x_l,\\,x_u]$. This interval is then\n   developped into a continued fraction as far as the representation is the same\n   for every elements of $[x_l,\\,x_u]$. The corresponding fraction is returned.\n
\item Since rational numbers are not a primitive object of \\sollya, the fraction is\n   returned as a constant function. This can be quite amazing, because \\sollya\n   immediately simplifies a constant function by evaluating it when the constant\n   has to be displayed.\n   To avoid this, you can use \\textbf{print} (that displays the expression representing\n   the constant and not the constant itself) or the commands \\textbf{numerator} \n   and \\textbf{denominator}.\n\end{itemize}
\noindent Example 1: 
\begin{center}\begin{minipage}{15cm}\begin{Verbatim}[frame=single]
\end{Verbatim}
\end{minipage}\end{center}
\noindent Example 2: 
\begin{center}\begin{minipage}{15cm}\begin{Verbatim}[frame=single]
\end{Verbatim}
\end{minipage}\end{center}
See also: \textbf{print} (\ref{labprint}), \textbf{numerator} (\ref{labnumerator}), \textbf{denominator} (\ref{labdenominator})

\subsection{rationalmode}
\label{labrationalmode}
\noindent Name: \textbf{rationalmode}\\
global variable controlling if rational arithmetic is used or not.\\
\noindent Usage: 
\begin{center}
\textbf{rationalmode} = \emph{activation value} : \textsf{on$|$off} $\rightarrow$ \textsf{void}\\
\textbf{rationalmode} = \emph{activation value} ! : \textsf{on$|$off} $\rightarrow$ \textsf{void}\\
\textbf{rationalmode} : \textsf{on$|$off}\\
\end{center}
Parameters: 
\begin{itemize}
\item \emph{activation value} controls if rational arithmetic should be used or not
\end{itemize}
\noindent Description: \begin{itemize}

\item \\textbf{rationalmode} is a global variable. When its value is \\textbf{off}, which is the default,\n   \\sollya will not use rational arithmetic to simplify expressions. All computations,\n   including the evaluation of constant expressions given on the \\sollya prompt,\n   will be performed using floating-point and interval arithmetic. Constant expressions\n   will be approximated by floating-point numbers, which are in most cases faithful \n   roundings of the expressions, when shown at the prompt. \n
\item When the value of the global variable \\textbf{rationalmode} is \\textbf{on}, \\sollya will use \n   rational arithmetic when simplifying expressions. Constant expressions, given \n   at the \\sollya prompt, will be simplified to rational numbers and displayed \n   as such when they are in the set of the rational numbers. Otherwise, flaoting-point\n   and interval arithmetic will be used to compute a floating-point approximation,\n   which is in most cases a faithful rounding of the constant expression.\n\end{itemize}
\noindent Example 1: 
\begin{center}\begin{minipage}{15cm}\begin{Verbatim}[frame=single]
\end{Verbatim}
\end{minipage}\end{center}
\noindent Example 2: 
\begin{center}\begin{minipage}{15cm}\begin{Verbatim}[frame=single]
\end{Verbatim}
\end{minipage}\end{center}
See also: \textbf{on} (\ref{labon}), \textbf{off} (\ref{laboff}), \textbf{numerator} (\ref{labnumerator}), \textbf{denominator} (\ref{labdenominator}), \textbf{simplifysafe} (\ref{labsimplifysafe})

\subsection{RD}
\label{labrd}
\noindent Name: \textbf{RD}\\
\phantom{aaa}constant representing rounding-downwards mode.\\[0.2cm]
\noindent Library names:\\
\verb|   sollya_obj_t sollya_lib_round_down()|\\
\verb|   int sollya_lib_is_round_down(sollya_obj_t)|\\[0.2cm]
\noindent Description: \begin{itemize}

\item \textbf{RD} is used in command \textbf{round} to specify that the value $x$ must be rounded
   to the greatest floating-point number $y$ such that $y \le x$.
\end{itemize}
\noindent Example 1: 
\begin{center}\begin{minipage}{15cm}\begin{Verbatim}[frame=single]
> display=binary!;
> round(Pi,20,RD);
1.1001001000011111101_2 * 2^(1)
\end{Verbatim}
\end{minipage}\end{center}
See also: \textbf{RZ} (\ref{labrz}), \textbf{RU} (\ref{labru}), \textbf{RN} (\ref{labrn}), \textbf{round} (\ref{labround}), \textbf{floor} (\ref{labfloor})

\subsection{readfile}
\label{labreadfile}
\noindent Name: \textbf{readfile}\\
reads the content of a file into a string variable\\
\noindent Usage: 
\begin{center}
\textbf{readfile}(\emph{filename}) : \textsf{string} $\rightarrow$ \textsf{string}\\
\end{center}
Parameters: 
\begin{itemize}
\item \emph{filename} represents a character sequence indicating a file name
\end{itemize}
\noindent Description: \begin{itemize}

\item \\textbf{readfile} opens the file indicated by \\emph{filename}, reads it and puts its\n   contents in a character sequence of type \\textsf{string} that is returned.\n    \n   If the file indicated by \\emph{filename} cannot be opened for reading, a\n   warning is displayed and \\textbf{readfile} evaluates to an \\textbf{error} variable of\n   type \\textsf{error}.\n\end{itemize}
\noindent Example 1: 
\begin{center}\begin{minipage}{15cm}\begin{Verbatim}[frame=single]
\end{Verbatim}
\end{minipage}\end{center}
\noindent Example 2: 
\begin{center}\begin{minipage}{15cm}\begin{Verbatim}[frame=single]
\end{Verbatim}
\end{minipage}\end{center}
See also: \textbf{parse} (\ref{labparse}), \textbf{execute} (\ref{labexecute}), \textbf{write} (\ref{labwrite}), \textbf{print} (\ref{labprint})

\subsection{readxml}
\label{labreadxml}
\noindent Name: \textbf{readxml}\\
reads an expression written as a MathML-Content-Tree in a file\\
\noindent Usage: 
\begin{center}
\textbf{readxml}(\emph{filename}) : \textsf{string} $\rightarrow$ \textsf{function} $|$ \textsf{error}\\
\end{center}
Parameters: 
\begin{itemize}
\item \emph{filename} represents a character sequence indicating a file name
\end{itemize}
\noindent Description: \begin{itemize}

\item \\textbf{readxml}(\\emph{filename}) reads the first occurrence of a lambda\n   application with one bounded variable on applications of the supported\n   basic functions in file \\emph{filename} and returns it as a \\sollya\n   functional expression.\n    \n   If the file \\emph{filename} does not contain a valid MathML-Content tree,\n   \\textbf{readxml} tries to find an "annotation encoding" markup of type\n   "sollya/text". If this annotation contains a character sequence\n   that can be parsed by \\textbf{parse}, \\textbf{readxml} returns that expression.  Otherwise\n   \\textbf{readxml} displays a warning and returns an \\textbf{error} variable of type\n   \\textsf{error}.\n\end{itemize}
\noindent Example 1: 
\begin{center}\begin{minipage}{15cm}\begin{Verbatim}[frame=single]
\end{Verbatim}
\end{minipage}\end{center}
See also: \textbf{printxml} (\ref{labprintxml}), \textbf{readfile} (\ref{labreadfile}), \textbf{parse} (\ref{labparse})

\subsection{relative}
\label{labrelative}
\noindent Name: \textbf{relative}\\
indicates a relative error for 	extbf{externalplot} or 	extbf{fpminimax}\\
\noindent Usage: 
\begin{center}
\textbf{relative} : \textsf{absolute$|$relative}
\end{center}
\noindent Description: \begin{itemize}

\item The use of \textbf{relative} in the command \textbf{externalplot} indicates that during
   plotting in \textbf{externalplot} a relative error is to be considered.
    
   See \textbf{externalplot} for details.
   Used with \textbf{fpminimax}, \textbf{relative} indicates that \textbf{fpminimax} must try to minimize
   the relative error.
    
   See \textbf{fpminimax} for details.
\end{itemize}
\noindent Example 1: 
\begin{center}\begin{minipage}{15cm}\begin{Verbatim}[frame=single]
> bashexecute("gcc -fPIC -c externalplotexample.c");
> bashexecute("gcc -shared -o externalplotexample externalplotexample.o -lgmp -l
mpfr");
> externalplot("./externalplotexample",absolute,exp(x),[-1/2;1/2],12,perturb);
\end{Verbatim}
\end{minipage}\end{center}
See also: \textbf{externalplot} (\ref{labexternalplot}), \textbf{fpminimax} (\ref{labfpminimax}), \textbf{absolute} (\ref{lababsolute}), \textbf{bashexecute} (\ref{labbashexecute})

\subsection{remez}
\label{labremez}
\noindent Name: \textbf{remez}\\
computes the minimax of a function on an interval.\\

\noindent Usage: 
\begin{center}
\textbf{remez}(\emph{f}, \emph{n}, \emph{range}, \emph{w}, \emph{quality}) : (\textsf{function}, \textsf{integer}, \textsf{range}, \textsf{function}, \textsf{constant}) $\rightarrow$ \textsf{function}\\
\textbf{remez}(\emph{f}, \emph{L}, \emph{range}, \emph{w}, \emph{quality}) : (\textsf{function}, \textsf{list}, \textsf{range}, \textsf{function}, \textsf{constant}) $\rightarrow$ \textsf{function}\\
\end{center}
Parameters: 
\begin{itemize}
\item \emph{f} is the function to be approximated
\item \emph{n} is the degree of the polynomial that must approximate \emph{f}
\item \emph{L} is a list of monomials that can be used to represent the polynomial that must approximate \emph{f}
\item \emph{range} is the interval where the function must be approximated
\item \emph{w} (optional) is a weight function. Default is 1.
\item \emph{quality} (optional) is a parameter that controls the quality of the returned polynomial \emph{p}, with respect to the exact minimax $p^\star$. Default is 1e-5.
\end{itemize}
\noindent Description: \begin{itemize}

\item \textbf{remez} computes an approximation of the function $f$ with respect to 
   the weight function $w$ on the interval \emph{range}. More precisely, it 
   searches a polynomial $p$ such that $\|pw-f\|_{\infty}$ is 
   (almost minimal) among all polynomials $p$ of a certain form. The norm is
   the infinite norm, e.g. $\|g\|_{\infty} = \max \{|g(x)|, x \in \mathrm{range}\}.$

\item If $w=1$ (the default case), it consists in searching the best 
   polynomial approximation of $f$ with respect to the absolute error.
   If $f=1$ and $w$ is of the form $1/g$, it consists in 
   searching the best polynomial approximation of $g$ with respect to the 
   relative error.

\item If $n$ is given, the polynomial $p$ is searched among the 
   polynomials with degree not greater than $n$.
   If \emph{L} is given, the polynomial $p$ is searched as a linear combination 
   of monomials $X^k$ where $k$ belongs to \emph{L}.
   \emph{L} may contain ellipses but cannot be end-elliptic.

\item The polynomial is obtained by a convergent iteration called Remez' algorithm. 
   The algorithm computes a sequence $p_1,\dots ,p_k,\dots$ 
   such that $e_k = \|p_k w-f\|_{\infty}$ converges towards 
   the optimal value $e$. The algorithm is stopped when the relative error 
   between $e_k$ and $e$ is less than \emph{quality}.

\item Note: the algorithm may not converge in certain cases. Moreover, it may 
   converge towards a polynomial that is not optimal. These cases correspond to 
   the cases when Haar's condition is not fulfilled.
   See [Cheney - Approximation theory] for details.
\end{itemize}
\noindent Example 1: 
\begin{center}\begin{minipage}{15cm}\begin{Verbatim}[frame=single]
> p = remez(exp(x),5,[0;1]);
> degree(p);
5
> dirtyinfnorm(p-exp(x),[0;1]);
1.12956983951521595837351332280615326522151556200193e-6
\end{Verbatim}
\end{minipage}\end{center}
\noindent Example 2: 
\begin{center}\begin{minipage}{15cm}\begin{Verbatim}[frame=single]
> p = remez(1,[|0,2,4,6,8|],[0,Pi/4],1/cos(x));
> canonical=on!;
> p;
0.99999999994393732256022951179607681090962982860718 + (-0.499999995715568615230
392624316246303890025319140407) * x^2 + 4.16666132334739499557284125992959476146
37617431071e-2 * x^4 + (-1.3886529147154306519506582591203419117431481149344e-3)
 * x^6 + 2.43726791779385830928547029399292453807061848172366e-5 * x^8
\end{Verbatim}
\end{minipage}\end{center}
\noindent Example 3: 
\begin{center}\begin{minipage}{15cm}\begin{Verbatim}[frame=single]
> p1 = remez(exp(x),5,[0;1],default,1e-5);
> p2 = remez(exp(x),5,[0;1],default,1e-10);
> p3 = remez(exp(x),5,[0;1],default,1e-15);
> dirtyinfnorm(p1-exp(x),[0;1]);
1.12956983951521595837351332280615326522151556200193e-6
> dirtyinfnorm(p2-exp(x),[0;1]);
1.12956980227478684205752230763956235955266452555335e-6
> dirtyinfnorm(p3-exp(x),[0;1]);
1.12956980227478684205752230763956235955266452555335e-6
\end{Verbatim}
\end{minipage}\end{center}
See also: \textbf{dirtyinfnorm} (\ref{labdirtyinfnorm}), \textbf{infnorm} (\ref{labinfnorm})

\subsection{rename}
\label{labrename}
\noindent Name: \textbf{rename}\\
\phantom{aaa}rename the free variable.\\[0.2cm]
\noindent Library name:\\
\verb|   void sollya_lib_name_free_variable(const char *)|\\[0.2cm]
\noindent Usage: 
\begin{center}
\textbf{rename}(\emph{ident1},\emph{ident2}) : \textsf{void}\\
\end{center}
Parameters: 
\begin{itemize}
\item \emph{ident1} is the current name of the free variable.
\item \emph{ident2} is a fresh name.
\end{itemize}
\noindent Description: \begin{itemize}

\item \textbf{rename} permits a change of the name of the free variable. \sollya can handle
   only one free variable at a time. The first time in a session that an
   unbound name is used in a context where it can be interpreted as a free
   variable, the name is used to represent the free variable of \sollya. In the
   following, this name can be changed using \textbf{rename}.

\item Be careful: if \emph{ident2} has been set before, its value will be lost. Use
   the command \textbf{isbound} to know if \emph{ident2} is already used or not.

\item If \emph{ident1} is not the current name of the free variable, an error occurs.

\item If \textbf{rename} is used at a time when the name of the free variable has not been
   defined, \emph{ident1} is just ignored and the name of the free variable is set
   to \emph{ident2}.

\item It is always possible to use the special keyword \verb|_x_| to denote the free
   variable. Hence \emph{ident1} can be \verb|_x_|.
\end{itemize}
\noindent Example 1: 
\begin{center}\begin{minipage}{15cm}\begin{Verbatim}[frame=single]
> f=sin(x);
> f;
sin(x)
> rename(x,y);
> f;
sin(y)
\end{Verbatim}
\end{minipage}\end{center}
\noindent Example 2: 
\begin{center}\begin{minipage}{15cm}\begin{Verbatim}[frame=single]
> a=1;
> f=sin(x);
> rename(x,a);
> a;
a
> f;
sin(a)
\end{Verbatim}
\end{minipage}\end{center}
\noindent Example 3: 
\begin{center}\begin{minipage}{15cm}\begin{Verbatim}[frame=single]
> verbosity=1!;
> f=sin(x);
> rename(y, z);
Warning: the current free variable is named "x" and not "y". Can only rename the
 free variable.
The last command will have no effect.
> rename(_x_, z);
Information: the free variable has been renamed from "x" to "z".
\end{Verbatim}
\end{minipage}\end{center}
\noindent Example 4: 
\begin{center}\begin{minipage}{15cm}\begin{Verbatim}[frame=single]
> verbosity=1!;
> rename(x,y);
Information: the free variable has been named "y".
> isbound(x);
false
> isbound(y);
true
\end{Verbatim}
\end{minipage}\end{center}
See also: \textbf{isbound} (\ref{labisbound})

\subsection{restart}
\label{labrestart}
\noindent Name: \textbf{restart}\\
brings \sollya back to its initial state\\
\noindent Usage: 
\begin{center}
\textbf{restart} : \textsf{void} $\rightarrow$ \textsf{void}\\
\end{center}
\noindent Description: \begin{itemize}

\item The command \\textbf{restart} brings \\sollya back to its initial state.  All\n   current state is abandoned, all libraries unbound and all memory freed.\n    \n   The \\textbf{restart} command has no effect when executed inside a \\sollya\n   script read into a main \\sollya script using \\textbf{execute}. It is executed\n   in a \\sollya script included by a $\\#$include macro.\n    \n   Using the \\textbf{restart} command in nested elements of imperative\n   programming like for or while loops is possible. Since in most cases\n   abandoning the current state of \\sollya means altering a loop\n   invariant, warnings for the impossibility of continuing a loop may\n   follow unless the state is rebuilt.\n\end{itemize}
\noindent Example 1: 
\begin{center}\begin{minipage}{15cm}\begin{Verbatim}[frame=single]
\end{Verbatim}
\end{minipage}\end{center}
\noindent Example 2: 
\begin{center}\begin{minipage}{15cm}\begin{Verbatim}[frame=single]
\end{Verbatim}
\end{minipage}\end{center}
\noindent Example 3: 
\begin{center}\begin{minipage}{15cm}\begin{Verbatim}[frame=single]
\end{Verbatim}
\end{minipage}\end{center}
See also: \textbf{quit} (\ref{labquit}), \textbf{execute} (\ref{labexecute})

\subsection{return}
\label{labreturn}
\noindent Name: \textbf{return}\\
indicates an expression to be returned in a procedure\\
\noindent Usage: 
\begin{center}
\textbf{return} \emph{expression} : \textsf{void}\\
\end{center}
Parameters: 
\begin{itemize}
\item \emph{expression} represents the expression to be returned
\end{itemize}
\noindent Description: \begin{itemize}

\item The keyword \\textbf{return} allows for returning the (evaluated) expression\n   \\emph{expression} at the end of a begin-end-block ({}-block) used as a\n   \\sollya procedure body. See \\textbf{proc} for further details concerning\n   \\sollya procedure definitions.\n     \n   Statements for returning expressions using \\textbf{return} are only possible\n    at the end of a begin-end-block used as a \\sollya procedure\n    body. Only one \\textbf{return} statement can be given per begin-end-block.\n
\item If at the end of a procedure definition using \\textbf{proc} no \\textbf{return}\n   statement is given, a \\textbf{return} \\textbf{void} statement is implicitely\n   added. Procedures, i.e. procedure objects, when printed out in \\sollya\n   defined with an implicit \\textbf{return} \\textbf{void} statement are displayed with\n   this statement explicitly given.\n\end{itemize}
\noindent Example 1: 
\begin{center}\begin{minipage}{15cm}\begin{Verbatim}[frame=single]
\end{Verbatim}
\end{minipage}\end{center}
\noindent Example 2: 
\begin{center}\begin{minipage}{15cm}\begin{Verbatim}[frame=single]
\end{Verbatim}
\end{minipage}\end{center}
See also: \textbf{proc} (\ref{labproc}), \textbf{void} (\ref{labvoid})

\subsection{ revert }
\noindent Name: \textbf{revert}\\
reverts a list.\\

\noindent Usage: 
\begin{center}
\textbf{revert}(\emph{L}) : \textsf{list} $\rightarrow$ \textsf{list}\\
\end{center}
Parameters: 
\emph{L} is a list.\\

\noindent Description: \begin{itemize}

\item \textbf{revert}(\emph{L}) returns the same list, but with its elements in reverse order.

\item If \emph{L} is an end-elliptic list, \textbf{revert} will fail with an error.
\end{itemize}
\noindent Example 1: 
\begin{center}\begin{minipage}{14.8cm}\begin{Verbatim}[frame=single]
   > revert([| |]);
   [| |]
\end{Verbatim}
\end{minipage}\end{center}
\noindent Example 2: 
\begin{center}\begin{minipage}{14.8cm}\begin{Verbatim}[frame=single]
   > revert([|2,3,5,2,1,4|]);
   [|4, 1, 2, 5, 3, 2|]
\end{Verbatim}
\end{minipage}\end{center}

\subsection{RN}
\label{labrn}
\noindent Name: \textbf{RN}\\
constant representing rounding-to-nearest mode.\\
\noindent Description: \begin{itemize}

\item \\textbf{RN} is used in command \\textbf{round} to specify that the value must be rounded\n   to the nearest representable floating-point number.\n\end{itemize}
\noindent Example 1: 
\begin{center}\begin{minipage}{15cm}\begin{Verbatim}[frame=single]
\end{Verbatim}
\end{minipage}\end{center}
See also: \textbf{RD} (\ref{labrd}), \textbf{RU} (\ref{labru}), \textbf{RZ} (\ref{labrz}), \textbf{round} (\ref{labround})

\subsection{roundcoefficients}
\label{labroundcoefficients}
\noindent Name: \textbf{roundcoefficients}\\
rounds the coefficients of a polynomial to classical formats.\\
\noindent Usage: 
\begin{center}
\textbf{roundcoefficients}(\emph{p},\emph{L}) : (\textsf{function}, \textsf{list}) $\rightarrow$ \textsf{function}\\
\end{center}
Parameters: 
\begin{itemize}
\item \emph{p} is a function. Usually a polynomial.
\item \emph{L} is a list of formats.
\end{itemize}
\noindent Description: \begin{itemize}

\item If \\emph{p} is a polynomial and \\emph{L} a list of floating-point formats, \n   \\textbf{roundcoefficients}(\\emph{p},\\emph{L}) rounds each coefficient of \\emph{p} to the corresponding format\n   in \\emph{L}.\n
\item If \\emph{p} is not a polynomial, \\textbf{roundcoefficients} does not do anything.\n
\item If \\emph{L} contains other elements than \\textbf{D}, \\textbf{double}, \\textbf{DD}, \\textbf{doubledouble}, \\textbf{TD} and\n   \\textbf{tripledouble}, an error occurs.\n
\item The coefficients in \\emph{p} corresponding to $X^i$ is rounded to the \n   format L[i]. If \\emph{L} does not contain enough elements\n   (e.g. if \\textbf{length}(L) $<$ \\textbf{degree}(p)+1), a warning is displayed. However, the\n   coefficients corresponding to an element of \\emph{L} are rounded. The trailing \n   coefficients (that do not have a corresponding element in \\emph{L}) are kept with\n   their own precision.\n   If \\emph{L} contains too much elements, the trailing useless elements are ignored.\n   In particular \\emph{L} may be end-elliptic in which case \\textbf{roundcoefficients} has the \n   natural behavior.\n\end{itemize}
\noindent Example 1: 
\begin{center}\begin{minipage}{15cm}\begin{Verbatim}[frame=single]
\end{Verbatim}
\end{minipage}\end{center}
\noindent Example 2: 
\begin{center}\begin{minipage}{15cm}\begin{Verbatim}[frame=single]
\end{Verbatim}
\end{minipage}\end{center}
\noindent Example 3: 
\begin{center}\begin{minipage}{15cm}\begin{Verbatim}[frame=single]
\end{Verbatim}
\end{minipage}\end{center}
See also: \textbf{single} (\ref{labsingle}), \textbf{double} (\ref{labdouble}), \textbf{doubledouble} (\ref{labdoubledouble}), \textbf{tripledouble} (\ref{labtripledouble})

\subsection{roundcorrectly}
\label{labroundcorrectly}
\noindent Name: \textbf{roundcorrectly}\\
rounds an approximation range correctly to some precision\\
\noindent Usage: 
\begin{center}
\textbf{roundcorrectly}(\emph{range}) : \textsf{range} $\rightarrow$ \textsf{constant}\\
\end{center}
Parameters: 
\begin{itemize}
\item \emph{range} represents a range in which an exact value lies
\end{itemize}
\noindent Description: \begin{itemize}

\item Let \\emph{range} be a range of values, determined by some approximation\n   process, safely bounding an unknown value $v$. The command\n   \\textbf{roundcorrectly}(\\emph{range}) determines a precision such that for this precision,\n   rounding to the nearest any value in \\emph{range} yields to the same\n   result, i.e. to the correct rounding of $v$.\n    \n   If no such precision exists, a warning is displayed and \\textbf{roundcorrectly}\n   evaluates to NaN.\n\end{itemize}
\noindent Example 1: 
\begin{center}\begin{minipage}{15cm}\begin{Verbatim}[frame=single]
\end{Verbatim}
\end{minipage}\end{center}
\noindent Example 2: 
\begin{center}\begin{minipage}{15cm}\begin{Verbatim}[frame=single]
\end{Verbatim}
\end{minipage}\end{center}
See also: \textbf{round} (\ref{labround})

\subsection{roundingwarnings}
\label{labroundingwarnings}
\noindent Name: \textbf{roundingwarnings}\\
global variable controlling whether or not a warning is displayed when roundings occur.\\
\noindent Usage: 
\begin{center}
\textbf{roundingwarnings} = \emph{activation value} : \textsf{on$|$off} $\rightarrow$ \textsf{void}\\
\textbf{roundingwarnings} = \emph{activation value} ! : \textsf{on$|$off} $\rightarrow$ \textsf{void}\\
\textbf{roundingwarnings} : \textsf{on$|$off}\\
\end{center}
Parameters: 
\begin{itemize}
\item \emph{activation value} controls if warnings should be shown or not
\end{itemize}
\noindent Description: \begin{itemize}

\item \\textbf{roundingwarnings} is a global variable. When its value is \\textbf{on}, warnings are\n   emitted in appropriate verbosity modes (see \\textbf{verbosity}) when roundings\n   occur.  When its value is \\textbf{off}, these warnings are suppressed.\n
\item This mode depends on a verbosity of at least 1. See\n   \\textbf{verbosity} for more details.\n
\item Default is \\textbf{on} when the standard input is a terminal and\n   \\textbf{off} when \\sollya input is read from a file.\n\end{itemize}
\noindent Example 1: 
\begin{center}\begin{minipage}{15cm}\begin{Verbatim}[frame=single]
\end{Verbatim}
\end{minipage}\end{center}
See also: \textbf{on} (\ref{labon}), \textbf{off} (\ref{laboff}), \textbf{verbosity} (\ref{labverbosity}), \textbf{midpointmode} (\ref{labmidpointmode}), \textbf{rationalmode} (\ref{labrationalmode})

\subsection{round}
\label{labround}
\noindent Name: \textbf{round}\\
rounds a number to a floating-point format.\\
\noindent Usage: 
\begin{center}
\textbf{round}(\emph{x},\emph{n},\emph{mode}) : (\textsf{constant}, \textsf{integer}, \textbf{RD} $|$ \textbf{RU} $|$ \textbf{RN} $|$ \textbf{RZ}) $\rightarrow$ \textsf{constant}\\
\textbf{round}(\emph{x},\emph{format},\emph{mode}) : (\textsf{constant}, \textsf{D$|$double$|$DD$|$doubledouble$|$DE$|$doubleextended$|$TD$|$tripledouble}, \textbf{RD} $|$ \textbf{RU} $|$ \textbf{RN} $|$ \textbf{RZ}) $\rightarrow$ \textsf{constant}\\
\end{center}
Parameters: 
\begin{itemize}
\item \emph{x} is a constant to be rounded.
\item \emph{n} is the precision of the target format.
\item \emph{format} is the name of a supported floating-point format.
\item \emph{mode} is the desired rounding mode.
\end{itemize}
\noindent Description: \begin{itemize}

\item If used with an integer parameter \emph{n}, \textbf{round}(\emph{x},\emph{n},\emph{mode}) rounds \emph{x} to a floating-point number with 
   precision \emph{n}, according to rounding-mode \emph{mode}. 

\item If used with a format parameter \emph{format}, \textbf{round}(\emph{x},\emph{format},\emph{mode}) rounds \emph{x} to a floating-point number in the 
   floating-point format \emph{format}, according to rounding-mode \emph{mode}. 

\item Subnormal numbers are not handled are handled only if a \emph{format} parameter is given
   that is different from \textbf{doubleextended}. The range of possible exponents is the 
   range used for all numbers represented in \sollya (e.g. basically the range 
   used in the library MPFR). 
\end{itemize}
\noindent Example 1: 
\begin{center}\begin{minipage}{15cm}\begin{Verbatim}[frame=single]
> display=binary!;
> round(Pi,20,RN);
1.100100100001111111_2 * 2^(1)
\end{Verbatim}
\end{minipage}\end{center}
\noindent Example 2: 
\begin{center}\begin{minipage}{15cm}\begin{Verbatim}[frame=single]
> printhexa(round(exp(17),53,RU));
0x417709348c0ea4f9
> printhexa(D(exp(17)));
0x417709348c0ea4f9
\end{Verbatim}
\end{minipage}\end{center}
\noindent Example 3: 
\begin{center}\begin{minipage}{15cm}\begin{Verbatim}[frame=single]
> display=binary!;
> a=2^(-1100);
> round(a,53,RN);
1_2 * 2^(-1100)
> round(a,D,RN);
0
> double(a);
0
\end{Verbatim}
\end{minipage}\end{center}
See also: \textbf{RN} (\ref{labrn}), \textbf{RD} (\ref{labrd}), \textbf{RU} (\ref{labru}), \textbf{RZ} (\ref{labrz}), \textbf{double} (\ref{labdouble}), \textbf{doubleextended} (\ref{labdoubleextended}), \textbf{doubledouble} (\ref{labdoubledouble}), \textbf{tripledouble} (\ref{labtripledouble}), \textbf{roundcoefficients} (\ref{labroundcoefficients}), \textbf{roundcorrectly} (\ref{labroundcorrectly}), \textbf{printhexa} (\ref{labprinthexa})

\subsection{RU}
\label{labru}
\noindent Name: \textbf{RU}\\
constant representing rounding-upwards mode.\\
\noindent Description: \begin{itemize}

\item \\textbf{RU} is used in command \\textbf{round} to specify that the value $x$ must be rounded\n   to the smallest floating-point number $y$ such that $x \\le y$.\n\end{itemize}
\noindent Example 1: 
\begin{center}\begin{minipage}{15cm}\begin{Verbatim}[frame=single]
\end{Verbatim}
\end{minipage}\end{center}
See also: \textbf{RZ} (\ref{labrz}), \textbf{RD} (\ref{labrd}), \textbf{RN} (\ref{labrn}), \textbf{round} (\ref{labround})

\subsection{RZ}
\label{labrz}
\noindent Name: \textbf{RZ}\\
constant representing rounding-to-zero mode.\\
\noindent Description: \begin{itemize}

\item \\textbf{RZ} is used in command \\textbf{round} to specify that the value must be rounded\n   to the closest floating-point number towards zero. It just consists in \n   truncating the value to the desired format.\n\end{itemize}
\noindent Example 1: 
\begin{center}\begin{minipage}{15cm}\begin{Verbatim}[frame=single]
\end{Verbatim}
\end{minipage}\end{center}
See also: \textbf{RD} (\ref{labrd}), \textbf{RU} (\ref{labru}), \textbf{RN} (\ref{labrn}), \textbf{round} (\ref{labround})

\subsection{searchgal}
\label{labsearchgal}
\noindent Name: \textbf{searchgal}\\
searches for a preimage of a function such that the rounding the image yields an error smaller than a constant\\
\noindent Usage: 
\begin{center}
\textbf{searchgal}(\emph{function}, \emph{start}, \emph{preimage precision}, \emph{steps}, \emph{format}, \emph{error bound}) : (\textsf{function}, \textsf{constant}, \textsf{integer}, \textsf{integer}, \textsf{D$|$double$|$DD$|$doubledouble$|$DE$|$doubleextended$|$TD$|$tripledouble}, \textsf{constant}) $\rightarrow$ \textsf{list}\\
\textbf{searchgal}(\emph{list of functions}, \emph{start}, \emph{preimage precision}, \emph{steps}, \emph{list of format}, \emph{list of error bounds}) : (\textsf{list}, \textsf{constant}, \textsf{integer}, \textsf{integer}, \textsf{list}, \textsf{list}) $\rightarrow$ \textsf{list}\\
\end{center}
Parameters: 
\begin{itemize}
\item \emph{function} represents the function to be considered
\item \emph{start} represents a value around which the search is to be performed
\item \emph{preimage precision} represents the precision (discretization) for the eligible preimage values
\item \emph{steps} represents the binary logarithm ($\log_2$) of the number of search steps to be performed
\item \emph{format} represents the format the image of the function is to be rounded to
\item \emph{error bound} represents a upper bound on the relative rounding error when rounding the image
\item \emph{list of functions} represents the functions to be considered
\item \emph{list of formats} represents the respective formats the images of the functions are to be rounded to
\item \emph{list of error bounds} represents a upper bound on the relative rounding error when rounding the image
\end{itemize}
\noindent Description: \begin{itemize}

\item The command \\textbf{searchgal} searches for a preimage $z$ of function\n   \\emph{function} or a list of functions \\emph{list of functions} such that\n   $z$ is a floating-point number with \\emph{preimage precision}\n   significant mantissa bits and the image $y$ of the function,\n   respectively each image $y_i$ of the functions, rounds to\n   format \\emph{format} respectively to the corresponding format in \\emph{list of format} \n   with a relative rounding error less than \\emph{error bound}\n   respectively the corresponding value in \\emph{list of error bounds}. During\n   this search, at most $2^{steps}$ attempts are made. The search\n   starts with a preimage value equal to \\emph{start}. This value is then\n   increased and decreased by $1$ ulp in precision \\emph{preimage precision} \n   until a value is found or the step limit is reached.\n    \n   If the search finds an appropriate preimage $z$, \\textbf{searchgal}\n   evaluates to a list containing this value. Otherwise, \\textbf{searchgal}\n   evaluates to an empty list.\n\end{itemize}
\noindent Example 1: 
\begin{center}\begin{minipage}{15cm}\begin{Verbatim}[frame=single]
\end{Verbatim}
\end{minipage}\end{center}
\noindent Example 2: 
\begin{center}\begin{minipage}{15cm}\begin{Verbatim}[frame=single]
\end{Verbatim}
\end{minipage}\end{center}
\noindent Example 3: 
\begin{center}\begin{minipage}{15cm}\begin{Verbatim}[frame=single]
\end{Verbatim}
\end{minipage}\end{center}
See also: \textbf{round} (\ref{labround}), \textbf{double} (\ref{labdouble}), \textbf{doubledouble} (\ref{labdoubledouble}), \textbf{tripledouble} (\ref{labtripledouble}), \textbf{evaluate} (\ref{labevaluate}), \textbf{worstcase} (\ref{labworstcase})

\subsection{simplifysafe}
\label{labsimplifysafe}
\noindent Name: \textbf{simplifysafe}\\
simplifies an expression representing a function\\
\noindent Usage: 
\begin{center}
\textbf{simplifysafe}(\emph{function}) : \textsf{function} $\rightarrow$ \textsf{function}\\
\end{center}
Parameters: 
\begin{itemize}
\item \emph{function} represents the expression to be simplified
\end{itemize}
\noindent Description: \begin{itemize}

\item The command \\textbf{simplifysafe} simplifies the expression given in argument\n   representing the function \\emph{function}.  The command \\textbf{simplifysafe} does not\n   endanger the safety of computations even in \\sollya's floating-point\n   environment: the function returned is mathematically equal to the\n   function \\emph{function}. \n    \n   Remark that the simplification provided by \\textbf{simplifysafe} is not perfect:\n   they may exist simpler equivalent expressions for expressions returned\n   by \\textbf{simplifysafe}.\n\end{itemize}
\noindent Example 1: 
\begin{center}\begin{minipage}{15cm}\begin{Verbatim}[frame=single]
\end{Verbatim}
\end{minipage}\end{center}
\noindent Example 2: 
\begin{center}\begin{minipage}{15cm}\begin{Verbatim}[frame=single]
\end{Verbatim}
\end{minipage}\end{center}
\noindent Example 3: 
\begin{center}\begin{minipage}{15cm}\begin{Verbatim}[frame=single]
\end{Verbatim}
\end{minipage}\end{center}
See also: \textbf{simplify} (\ref{labsimplify}), \textbf{autosimplify} (\ref{labautosimplify}), \textbf{rationalmode} (\ref{labrationalmode})

\subsection{simplify}
\label{labsimplify}
\noindent Name: \textbf{simplify}\\
simplifies an expression representing a function\\
\noindent Usage: 
\begin{center}
\textbf{simplify}(\emph{function}) : \textsf{function} $\rightarrow$ \textsf{function}\\
\end{center}
Parameters: 
\begin{itemize}
\item \emph{function} represents the expression to be simplified
\end{itemize}
\noindent Description: \begin{itemize}

\item The command \\textbf{simplify} simplifies constant subexpressions of the\n   expression given in argument representing the function\n   \\emph{function}. Those constant subexpressions are evaluated using\n   floating-point arithmetic with the global precision \\textbf{prec}.\n\end{itemize}
\noindent Example 1: 
\begin{center}\begin{minipage}{15cm}\begin{Verbatim}[frame=single]
\end{Verbatim}
\end{minipage}\end{center}
\noindent Example 2: 
\begin{center}\begin{minipage}{15cm}\begin{Verbatim}[frame=single]
\end{Verbatim}
\end{minipage}\end{center}
See also: \textbf{simplifysafe} (\ref{labsimplifysafe}), \textbf{autosimplify} (\ref{labautosimplify}), \textbf{prec} (\ref{labprec}), \textbf{evaluate} (\ref{labevaluate})

\subsection{single}
\label{labsingle}
\noindent Names: \textbf{single}, \textbf{SG}\\
rounding to the nearest IEEE 754 single (binary32).\\
\noindent Description: \begin{itemize}

\item \\textbf{single} is both a function and a constant.\n
\item As a function, it rounds its argument to the nearest IEEE 754 single precision (i.e. IEEE754-2008 binary32) number.\n   Subnormal numbers are supported as well as standard numbers: it is the real\n   rounding described in the standard.\n
\item As a constant, it symbolizes the single precision format. It is used in \n   contexts when a precision format is necessary, e.g. in the commands \n   \\textbf{round} and \\textbf{roundcoefficients}. In is not supported for \\textbf{implementpoly}.\n   See the corresponding help pages for examples.\n\end{itemize}
\noindent Example 1: 
\begin{center}\begin{minipage}{15cm}\begin{Verbatim}[frame=single]
\end{Verbatim}
\end{minipage}\end{center}
See also: \textbf{double} (\ref{labdouble}), \textbf{doubleextended} (\ref{labdoubleextended}), \textbf{doubledouble} (\ref{labdoubledouble}), \textbf{tripledouble} (\ref{labtripledouble}), \textbf{roundcoefficients} (\ref{labroundcoefficients}), \textbf{implementpoly} (\ref{labimplementpoly}), \textbf{round} (\ref{labround}), \textbf{printfloat} (\ref{labprintfloat})

\subsection{sinh}
\label{labsinh}
\noindent Name: \textbf{sinh}\\
\phantom{aaa}the hyperbolic sine function.\\[0.2cm]
\noindent Library names:\\
\verb|   sollya_obj_t sollya_lib_sinh(sollya_obj_t)|\\
\verb|   sollya_obj_t sollya_lib_build_function_sinh(sollya_obj_t)|\\
\verb|   #define SOLLYA_SINH(x) sollya_lib_build_function_sinh(x)|\\[0.2cm]
\noindent Description: \begin{itemize}

\item \textbf{sinh} is the usual hyperbolic sine function: $\sinh(x) = \frac{e^x - e^{-x}}{2}$.

\item It is defined for every real number $x$.
\end{itemize}
See also: \textbf{asinh} (\ref{labasinh}), \textbf{cosh} (\ref{labcosh}), \textbf{tanh} (\ref{labtanh})

\subsection{sin}
\label{labsin}
\noindent Name: \textbf{sin}\\
\phantom{aaa}the sine function.\\[0.2cm]
\noindent Library names:\\
\verb|   sollya_obj_t sollya_lib_sin(sollya_obj_t)|\\
\verb|   sollya_obj_t sollya_lib_build_function_sin(sollya_obj_t)|\\
\verb|   #define SOLLYA_SIN(x) sollya_lib_build_function_sin(x)|\\[0.2cm]
\noindent Description: \begin{itemize}

\item \textbf{sin} is the usual sine function.

\item It is defined for every real number $x$.
\end{itemize}
See also: \textbf{asin} (\ref{labasin}), \textbf{cos} (\ref{labcos}), \textbf{tan} (\ref{labtan})

\subsection{sort}
\label{labsort}
\noindent Name: \textbf{sort}\\
sorts a list of real numbers.\\
\noindent Usage: 
\begin{center}
\textbf{sort}(\emph{L}) : \textsf{list} $\rightarrow$ \textsf{list}\\
\end{center}
Parameters: 
\begin{itemize}
\item \emph{L} is a list.
\end{itemize}
\noindent Description: \begin{itemize}

\item If \\emph{L} contains only constant values, \\textbf{sort}(\\emph{L}) returns the same list, but\n   sorted in increasing order.\n
\item If \\emph{L} contains at least one element that is not a constant, the command fails \n   with a type error.\n
\item If \\emph{L} is an end-elliptic list, \\textbf{sort} will fail with an error.\n\end{itemize}
\noindent Example 1: 
\begin{center}\begin{minipage}{15cm}\begin{Verbatim}[frame=single]
\end{Verbatim}
\end{minipage}\end{center}

\subsection{sqrt}
\label{labsqrt}
\noindent Name: \textbf{sqrt}\\
\phantom{aaa}square root.\\[0.2cm]
\noindent Library names:\\
\verb|   sollya_obj_t sollya_lib_sqrt(sollya_obj_t)|\\
\verb|   sollya_obj_t sollya_lib_build_function_sqrt(sollya_obj_t)|\\
\verb|   #define SOLLYA_SQRT(x) sollya_lib_build_function_sqrt(x)|\\[0.2cm]
\noindent Description: \begin{itemize}

\item \textbf{sqrt} is the square root, e.g. the inverse of the function square: $\sqrt{y}$
   is the unique positive $x$ such that $x^2=y$.

\item It is defined only for $x$ in $[0;+\infty]$.
\end{itemize}

\subsection{ string }
\noindent Name: \textbf{string}\\
keyword representing a \textsf{string} type \\

\noindent Usage: 
\begin{center}
\textbf{string} : \textsf{type type}\\
\end{center}
\noindent Description: \begin{itemize}

\item \textbf{string} represents the \textsf{string} type for declarations
   of external procedures by means of \textbf{externalproc}.
   Remark that in contrast to other indicators, type indicators like
   \textbf{string} cannot be handled outside the \textbf{externalproc} context.  In
   particular, they cannot be assigned to variables.
\end{itemize}
See also: \textbf{externalproc}, \textbf{boolean}, \textbf{constant}, \textbf{function}, \textbf{integer}, \textbf{list of}, \textbf{range}

\subsection{subpoly}
\label{labsubpoly}
\noindent Name: \textbf{subpoly}\\
\phantom{aaa}restricts the monomial basis of a polynomial to a list of monomials\\[0.2cm]
\noindent Library name:\\
\verb|   sollya_obj_t sollya_lib_subpoly(sollya_obj_t, sollya_obj_t)|\\[0.2cm]
\noindent Usage: 
\begin{center}
\textbf{subpoly}(\emph{polynomial}, \emph{list}) : (\textsf{function}, \textsf{list}) $\rightarrow$ \textsf{function}\\
\end{center}
Parameters: 
\begin{itemize}
\item \emph{polynomial} represents the polynomial the coefficients are taken from
\item \emph{list} represents the list of monomials to be taken
\end{itemize}
\noindent Description: \begin{itemize}

\item \textbf{subpoly} extracts the coefficients of a polynomial \emph{polynomial} and builds up a
   new polynomial out of those coefficients associated to monomial degrees figuring in
   the list \emph{list}. 
    
   If \emph{polynomial} represents a function that is not a polynomial, subpoly returns 0.
    
   If \emph{list} is a list that is end-elliptic, let be $j$ the last value explicitly specified
   in the list. All coefficients of the polynomial associated to monomials greater or
   equal to $j$ are taken.
\end{itemize}
\noindent Example 1: 
\begin{center}\begin{minipage}{15cm}\begin{Verbatim}[frame=single]
> p = taylor(exp(x),5,0);
> s = subpoly(p,[|1,3,5|]);
> print(p);
1 + x * (1 + x * (0.5 + x * (1 / 6 + x * (1 / 24 + x / 120))))
> print(s);
x * (1 + x^2 * (1 / 6 + x^2 / 120))
\end{Verbatim}
\end{minipage}\end{center}
\noindent Example 2: 
\begin{center}\begin{minipage}{15cm}\begin{Verbatim}[frame=single]
> p = remez(atan(x),10,[-1,1]);
> subpoly(p,[|1,3,5...|]);
x * (0.99986632941452949026018468446163586361700915018231 + x^2 * (-0.3303047850
2455936362667794059988443130926433421739 + x^2 * (0.1801592931781875646289423703
7824735129130095574422 + x * (2.284558411542478828511250156535857664242985696307
19e-9 + x * (-8.5156349064111377895500552996061844977507560037484e-2 + x * (-2.7
1756340962775019916818769239340943524383018921799e-9 + x * (2.084511343071147293
73239910549169872454686955894998e-2 + x * 1.108898611811290576571996643868266300
81793400489512e-9)))))))
\end{Verbatim}
\end{minipage}\end{center}
\noindent Example 3: 
\begin{center}\begin{minipage}{15cm}\begin{Verbatim}[frame=single]
> subpoly(exp(x),[|1,2,3|]);
0
\end{Verbatim}
\end{minipage}\end{center}
See also: \textbf{roundcoefficients} (\ref{labroundcoefficients}), \textbf{taylor} (\ref{labtaylor}), \textbf{remez} (\ref{labremez}), \textbf{fpminimax} (\ref{labfpminimax}), \textbf{implementpoly} (\ref{labimplementpoly})

\subsection{ substitute }
\noindent Name: \textbf{substitute}\\
replace the occurences of the free variable in an expression.\\

\noindent Usage: 
\begin{center}
\textbf{substitute}(\emph{f},\emph{g}) : (\textsf{function}, \textsf{function}) $\rightarrow$ \textsf{function}\\
\textbf{substitute}(\emph{f},\emph{t}) : (\textsf{function}, \textsf{constant}) $\rightarrow$ \textsf{constant}\\
\end{center}
Parameters: 
\emph{f} is a function.\\
\emph{g} is a function.\\
\emph{t} is a real number.\\

\noindent Description: \begin{itemize}

\item \textbf{substitute}(\emph{f}, \emph{g}) produces the function $(f \circ g) : x \mapsto f(g(x))$.

\item \textbf{substitute}(\emph{f}, \emph{t}) is the constant $f(t)$. Note that the constant is
   represented by its expression until it has been evaluated (exactly the same
   way as if you type the expression \emph{f} replacing instances of the free variable 
   by \emph{t}).

\item If \emph{f} is stored in a variable \emph{F}. It is absolutely equivalent to 
   writing \emph{F(g)} or \emph{F(t)}.
\end{itemize}
\noindent Example 1: 
\begin{center}\begin{minipage}{14.8cm}\begin{Verbatim}[frame=single]
   > f=sin(x);
   > g=cos(x);
   > substitute(f,g);
   sin(cos(x))
   > f(g);
   sin(cos(x))
\end{Verbatim}
\end{minipage}\end{center}
\noindent Example 2: 
\begin{center}\begin{minipage}{14.8cm}\begin{Verbatim}[frame=single]
   > a=1;
   > f=sin(x);
   > substitute(f,a);
   0.841470984807896506652502321630298999622563060798373
   > f(a);
   0.841470984807896506652502321630298999622563060798373
\end{Verbatim}
\end{minipage}\end{center}

\input{sup}
\subsection{tail}
\label{labtail}
\noindent Name: \textbf{tail}\\
gives the tail of a list.\\
\noindent Usage: 
\begin{center}
\textbf{tail}(\emph{L}) : \textsf{list} $\rightarrow$ \textsf{list}\\
\end{center}
Parameters: 
\begin{itemize}
\item \emph{L} is a list.
\end{itemize}
\noindent Description: \begin{itemize}

\item \\textbf{tail}(\\emph{L}) returns the list \\emph{L} without its first element.\n
\item If \\emph{L} is empty, the command will fail with an error.\n
\item \\textbf{tail} can also be used with end-elliptic lists. In this case, the result of\n   \\textbf{tail} is also an end-elliptic list.\n\end{itemize}
\noindent Example 1: 
\begin{center}\begin{minipage}{15cm}\begin{Verbatim}[frame=single]
\end{Verbatim}
\end{minipage}\end{center}
See also: \textbf{head} (\ref{labhead})

\subsection{ tanh }
\noindent Name: \textbf{tanh}\\
the hyperbolic tangent function.\\

\noindent Description: \begin{itemize}

\item \textbf{tanh} is the hyperbolic tangent function, defined by $\tanh(x) = sinh(x)/cosh(x)$.

\item It is defined for every real number x.
\end{itemize}
See also: \textbf{atanh}, \textbf{cosh}, \textbf{sinh}

\subsection{ tan }
\noindent Name: \textbf{tan}\\
the tangent function.\\

\noindent Description: \begin{itemize}

\item \textbf{tan} is the tangent function, defined by $\tan(x) = sin(x)/cos(x)$.

\item It is defined for every real number x that is not of the form $n\pi + \pi/2$ where $n$ is an integer.
\end{itemize}
See also: \textbf{atan}, \textbf{cos}, \textbf{sin}

\subsection{taylorform}
\label{labtaylorform}
\noindent Name: \textbf{taylorform}\\
computes a rigorous polynomial approximation (polynomial, interval error bound) for a function, based on Taylor expansions\\
\noindent Usage: 
\begin{center}
\textbf{taylorform}(\emph{f}, \emph{n}, \emph{$x_0$} \emph{I}, \emph{errorType}) : (\textsf{function}, \textsf{integer}, \textsf{constant}, \textsf{range}, \textsf{absolute$|$relative}) $\rightarrow$ \textsf{list}\\
\textbf{taylorform}(\emph{f}, \emph{n}, \emph{$x_0$} \emph{I}, \emph{errorType}) : (\textsf{function}, \textsf{integer}, \textsf{range}, \textsf{range}, \textsf{absolute$|$relative}) $\rightarrow$ \textsf{list}\\
\end{center}
Parameters: 
\begin{itemize}
\item \emph{f} is the function to be approximated
\item \emph{n} is the order of the Taylor form, meaning $\emph{n}-1$ is the degree of the polynomial that must approximate \emph{f}
\item \emph{$x_0$} is the point (it can be a real number or an interval) where the Taylor exansion of the function is to be considered
\item \emph{I} is the interval over which the function is to be approximated
\item \emph{errorType} is the type of error to be considered. See the detailed description below.
\end{itemize}
\noindent Description: \begin{itemize}

\item \textbf{taylorform} computes an approximation polynomial and an interval error bound for function $f$. More precisely, it 
   returns a list $L = \left[p, \textrm{coeffErrors}, \Delta \right]$ where:
   \begin{itemize}
   \item $p$ is an approximation polynomial of degree $n-1$ which is roughly speaking a numerical Taylor expansion of $f$ at the point $x_0$.
   \item coeffsErrors is a list of $n$ intervals. Each interval coeffsErrors[$i$] contains an enclosure of all the errors accumulated when computing the $i$-th coefficient of $p$.
   \item $\Delta$ is an interval that provides a bound for the approximation error between $p$ and $f$. Its significance depends on the \emph{errorType} considered.
   \end{itemize}

\item Please note that $x_0$ can be an interval. In general, it is meant to be a small interval approximating a non representable value. For instance, if one desires to compute a Taylor approximation at point $\pi$, it is possible to set $x_0$ to the (almost) point-interval $[\pi]$. It is also possible to use a large interval for $x_0$, though it is not obvious to give an intuitive sense to the result of \textbf{taylorform} in that case.

\item More formally, the mathematical property ensured by the algorithm may be stated as follows. For all $xi_0$ in $x_0$, there exist (small) values $\varepsilon_i \in \textrm{coeffsErrors}[i]$ such that:
   \\
   If \emph{errorType} is \textbf{absolute}, $\forall x \in I, \exists \delta \in \Delta,\, f(x)-p(x-xi_0) = \sum\limits_{i=0}^{n-1} \varepsilon_i\, (x-xi_0)^i + \delta$.
   \\
   If \emph{errorType} is \textbf{relative}, $\forall x \in I, \exists \delta \in \Delta,\, f(x)-p(x-xi_0) = \sum\limits_{i=0}^{n-1} \varepsilon_i\, (x-xi_0)^i + \delta\,(x-xi_0)^n$.

\item The polynomial $p$ and the bound  $\Delta$ are obtained using Taylor Models principles.

\item Note: The relative case is especially useful when functions with removable singularities are considered. In such a case, this routine is able to compute a finite remainder bound, provided that the expansion point given is the problematic removable singularity point.

\item Note: the algorithm does not guarantee that by increasing the degree of the approximation, the remainder bound will become smaller. Moreover, it may 
   even become larger due to the dependecy phenomenon present with interval arithmetic. In order to reduce this phenomenon, a possible solution is to split the definition domain $I$ into several smaller intervals. 
\end{itemize}
\noindent Example 1: 
\begin{center}\begin{minipage}{15cm}\begin{Verbatim}[frame=single]
> TL=taylorform(exp(x), 10, 0, [-1,1], absolute);
> p=TL[0];
> Delta=TL[2];
> errors=TL[1];
> p; Delta;
1 + x * (1 + x * (0.5 + x * (0.1666666666666666666666666666666666666666666666666
7 + x * (4.1666666666666666666666666666666666666666666666667e-2 + x * (8.3333333
333333333333333333333333333333333333333333e-3 + x * (1.3888888888888888888888888
8888888888888888888888889e-3 + x * (1.984126984126984126984126984126984126984126
98412698e-4 + x * (2.4801587301587301587301587301587301587301587301587e-5 + x * 
(2.75573192239858906525573192239858906525573192239859e-6 + x * 2.755731922398589
0652557319223985890652557319223986e-7)))))))))
[-2.31142719641187619441242534182684745832539555102969e-8;2.73126607556424744202
06278018039434042553645532164e-8]
\end{Verbatim}
\end{minipage}\end{center}
\noindent Example 2: 
\begin{center}\begin{minipage}{15cm}\begin{Verbatim}[frame=single]
> TL=taylorform(sin(x)/x, 10, 0, [-1,1], relative);
> p=TL[0];
> Delta=TL[2];
> errors=TL[1];
> p; Delta;
1 + x^2 * (-0.16666666666666666666666666666666666666666666666667 + x^2 * (8.3333
333333333333333333333333333333333333333333333e-3 + x^2 * (-1.9841269841269841269
8412698412698412698412698412698e-4 + x^2 * (2.7557319223985890652557319223985890
6525573192239859e-6 + x^2 * (-2.505210838544171877505210838544171877505210838544
19e-8)))))
[-1.6135797443886066084999806203254010793747502812764e-10;1.61357974438860660849
99806203254010793747502812764e-10]
\end{Verbatim}
\end{minipage}\end{center}

\subsection{taylorrecursions}
\label{labtaylorrecursions}
\noindent Name: \textbf{taylorrecursions}\\
controls the number of recursion steps when applying Taylor's rule.\\
\noindent Usage: 
\begin{center}
\textbf{taylorrecursions} = \emph{n} : \textsf{integer} $\rightarrow$ \textsf{void}\\
\textbf{taylorrecursions} = \emph{n} ! : \textsf{integer} $\rightarrow$ \textsf{void}\\
\textbf{taylorrecursions} : \textsf{integer}\\
\end{center}
Parameters: 
\begin{itemize}
\item \emph{n} represents the number of recursions
\end{itemize}
\noindent Description: \begin{itemize}

\item \\textbf{taylorrecursions} is a global variable. Its value represents the number of steps\n   of recursion that are used when applying Taylor's rule. This rule is applied\n   by the interval evaluator present in the core of \\sollya (and particularly\n   visible in commands like \\textbf{infnorm}).\n
\item To improve the quality of an interval evaluation of a function $f$, in \n   particular when there are problems of decorrelation), the evaluator of \\sollya\n   uses Taylor's rule:  $f([a,b]) \\subseteq f(m) + [a-m,\\,b-m] \\cdot f'([a,\\,b])$ where $m=\\frac{a+b}{2}$.\n   This rule can be applied recursively.\n   The number of step in this recursion process is controlled by \\textbf{taylorrecursions}.\n
\item Setting \\textbf{taylorrecursions} to 0 makes \\sollya use this rule only once;\n   setting it to 1 makes \\sollya use the rule twice, and so on.\n   In particular: the rule is always applied at least once.\n\end{itemize}
\noindent Example 1: 
\begin{center}\begin{minipage}{15cm}\begin{Verbatim}[frame=single]
\end{Verbatim}
\end{minipage}\end{center}

\subsection{taylor}
\label{labtaylor}
\noindent Name: \textbf{taylor}\\
computes a Taylor expansion of a function in a point\\
\noindent Usage: 
\begin{center}
\textbf{taylor}(\emph{function}, \emph{degree}, \emph{point}) : (\textsf{function}, \textsf{integer}, \textsf{constant}) $\rightarrow$ \textsf{function}\\
\end{center}
Parameters: 
\begin{itemize}
\item \emph{function} represents the function to be expanded
\item \emph{degree} represents the degree of the expansion to be delivered
\item \emph{point} represents the point in which the function is to be developped
\end{itemize}
\noindent Description: \begin{itemize}

\item The command \\textbf{taylor} returns an expression that is a Taylor expansion\n   of function \\emph{function} in point \\emph{point} having the degree \\emph{degree}.\n    \n   Let $f$ be the function \\emph{function}, $t$ be the point \\emph{point} and\n   $n$ be the degree \\emph{degree}. Then, \\textbf{taylor}(\\emph{function},\\emph{degree},\\emph{point}) \n   evaluates to an expression mathematically equal to \n   $$\\sum\\limits_{i=0}^n \\frac{f^{(i)}\\left(t\\right)}{i!} \\left(x - t \\right)^i$$\n    \n   Remark that \\textbf{taylor} evaluates to $0$ if the degree \\emph{degree} is negative.\n\end{itemize}
\noindent Example 1: 
\begin{center}\begin{minipage}{15cm}\begin{Verbatim}[frame=single]
\end{Verbatim}
\end{minipage}\end{center}
\noindent Example 2: 
\begin{center}\begin{minipage}{15cm}\begin{Verbatim}[frame=single]
\end{Verbatim}
\end{minipage}\end{center}
\noindent Example 3: 
\begin{center}\begin{minipage}{15cm}\begin{Verbatim}[frame=single]
\end{Verbatim}
\end{minipage}\end{center}
See also: \textbf{remez} (\ref{labremez})

\subsection{time}
\label{labtime}
\noindent Name: \textbf{time}\\
\phantom{aaa}procedure for timing \sollya code.\\[0.2cm]
\noindent Usage: 
\begin{center}
\textbf{time}(\emph{code}) : \textsf{code} $\rightarrow$ \textsf{constant}\\
\end{center}
Parameters: 
\begin{itemize}
\item \emph{code} is the code to be timed.
\end{itemize}
\noindent Description: \begin{itemize}

\item \textbf{time} permits timing a \sollya instruction, resp. a begin-end block
   of \sollya instructions. The timing value, measured in seconds, is returned
   as a \sollya constant (and not merely displayed as for \textbf{timing}). This 
   permits performing computations of the timing measurement value inside \sollya.

\item The extended \textbf{nop} command permits executing a defined number of
   useless instructions. Taking the ratio of the time needed to execute a
   certain \sollya instruction and the time for executing a \textbf{nop}
   therefore gives a way to abstract from the speed of a particular 
   machine when evaluating an algorithm's performance.
\end{itemize}
\noindent Example 1: 
\begin{center}\begin{minipage}{15cm}\begin{Verbatim}[frame=single]
> t = time(p=remez(sin(x),10,[-1;1]));
> write(t,"s were spent computing p = ",p,"\n");
0.214060999999999999974381603706774512829724699258804s were spent computing p = 
9.0486898749977990986908851357759191711354777014602e-17 * x^10 + 2.6876259511512
3596299959320959141640012683406736586e-6 * x^9 + -2.4247978492521313349073232289
246205727856268698001e-16 * x^8 + -1.9834486302096592970124560650358646122613093
7598776e-4 * x^7 + 2.2748214757753544349162426281857910162575492126267e-16 * x^6
 + 8.3333037186560980567697821420813799547276481409702e-3 * x^5 + -8.57471519897
20669741706961303549531312110511218869e-17 * x^4 + -0.16666666138601323707621656
6493953847771564552744173 * x^3 + 1.05699558969863875841493332282097022580493449
058156e-17 * x^2 + 0.99999999973628365676559825181776417246038944720794 * x + (-
3.12065309566018830243163208536426045628106466008778e-19)
\end{Verbatim}
\end{minipage}\end{center}
\noindent Example 2: 
\begin{center}\begin{minipage}{15cm}\begin{Verbatim}[frame=single]
> write(time({ p=remez(sin(x),10,[-1;1]); write("The error is 2^(", log2(dirtyin
fnorm(p-sin(x),[-1;1])), ")\n"); }), " s were spent\n");
The error is 2^(log2(2.39601979446524486606649528289933482070294808074097e-11))
0.367651999999999999960663410458749922327115200459957 s were spent
\end{Verbatim}
\end{minipage}\end{center}
\noindent Example 3: 
\begin{center}\begin{minipage}{15cm}\begin{Verbatim}[frame=single]
> t = time(bashexecute("sleep 10"));
> write(~(t-10),"s of execution overhead.\n");
2.57999999999999896083124895085347816348075866699219e-3s of execution overhead.
\end{Verbatim}
\end{minipage}\end{center}
\noindent Example 4: 
\begin{center}\begin{minipage}{15cm}\begin{Verbatim}[frame=single]
> ratio := time(p=remez(sin(x),10,[-1;1]))/time(nop(10));
> write("This ratio = ", ratio, " should somehow be independent of the type of m
achine.\n");
This ratio = 6.1924937350661460466685145865939088285876372485621 should somehow 
be independent of the type of machine.
\end{Verbatim}
\end{minipage}\end{center}
See also: \textbf{timing} (\ref{labtiming}), \textbf{nop} (\ref{labnop})

\subsection{timing}
\label{labtiming}
\noindent Name: \textbf{timing}\\
global variable controlling timing measures in \sollya.\\
\noindent Usage: 
\begin{center}
\textbf{timing} = \emph{activation value} : \textsf{on$|$off} $\rightarrow$ \textsf{void}\\
\textbf{timing} = \emph{activation value} ! : \textsf{on$|$off} $\rightarrow$ \textsf{void}\\
\textbf{timing} : \textsf{on$|$off}\\
\end{center}
Parameters: 
\begin{itemize}
\item \emph{activation value} controls if timing should be performed or not
\end{itemize}
\noindent Description: \begin{itemize}

\item \\textbf{timing} is a global variable. When its value is \\textbf{on}, the time spent in each \n   command is measured and displayed (for \\textbf{verbosity} levels higher than 1).\n\end{itemize}
\noindent Example 1: 
\begin{center}\begin{minipage}{15cm}\begin{Verbatim}[frame=single]
\end{Verbatim}
\end{minipage}\end{center}
See also: \textbf{on} (\ref{labon}), \textbf{off} (\ref{laboff}), \textbf{time} (\ref{labtime})

\subsection{ tripledouble }
\noindent Names: \textbf{tripledouble}, \textbf{TD}\\
represents a number as the sum of three IEEE doubles.\\

\noindent Description: \begin{itemize}

\item \textbf{tripledouble} is both a function and a constant.

\item As a function, it rounds its argument to the nearest number that can be written
   as the sum of three double precision numbers.

\item The algorithm used to compute \textbf{tripledouble}(x) is the following: let xh = \textbf{double}(x)
   and let xl = \textbf{doubledouble}(x-xh). Return the number xh+xl. Note that if the
   current precision is not sufficient to represent exactly xh+xl, a rounding will
   occur and the result of \textbf{tripledouble}(x) will be useless.

\item As a constant, it symbolizes the triple-double precision format. It is used in 
   contexts when a precision format is necessary, e.g. in the commands 
   \textbf{roundcoefficients} and \textbf{implementpoly}.
   See the corresponding help pages for examples.
\end{itemize}
\noindent Example 1: 
\begin{center}\begin{minipage}{14.8cm}\begin{Verbatim}[frame=single]
   > verbosity=1!;
   > a = 1+ 2^(-55)+2^(-115);
   > TD(a);
   0.100000000000000002775557561562891353466491600711095598e1
   > prec=110!;
   > TD(a);
   Warning: double rounding occurred on invoking the triple-double rounding operator.
   Try to increase the working precision.
   0.10000000000000000277555756156289135106e1
\end{Verbatim}
\end{minipage}\end{center}
See also: \textbf{double}, \textbf{doubleextended}, \textbf{doubledouble}, \textbf{roundcoefficients}, \textbf{implementpoly}

\subsection{ true }
\noindent Name: \textbf{true}\\
the boolean value representing the truth.\\

\noindent Description: \begin{itemize}

\item \textbf{true} is the usual boolean value.
\end{itemize}
\noindent Example 1: 
\begin{center}\begin{minipage}{14.8cm}\begin{Verbatim}[frame=single]
   > true && false;
   false
   > 2>1;
   true
\end{Verbatim}
\end{minipage}\end{center}
See also: \textbf{false}, \textbf{$\&\&$}, \textbf{$||$}

\subsection{var}
\label{labvar}
\noindent Name: \textbf{var}\\
declaration of a local variable in a scope\\
\noindent Usage: 
\begin{center}
\textbf{var} \emph{identifier1}, \emph{identifier2},... , \emph{identifiern} : \textsf{void}\\
\end{center}
Parameters: 
\begin{itemize}
\item \emph{identifier1}, \emph{identifier2},... , \emph{identifiern} represent variable identifiers
\end{itemize}
\noindent Description: \begin{itemize}

\item The keyword \\textbf{var} allows for the declaration of local variables\n   \\emph{identifier1} through \\emph{identifiern} in a begin-end-block ($\\lbrace \\rbrace$-block).\n   Once declared as a local variable, an identifier will shadow\n   identifiers declared in higher scopes and undeclared identifiers\n   available at top-level.\n    \n   Variable declarations using \\textbf{var} are only possible in the\n   beginning of a begin-end-block. Several \\textbf{var} statements can be\n   given. Once another statement is given in a begin-end-block, no more\n   \\textbf{var} statements can be given.\n    \n   Variables declared by \\textbf{var} statements are dereferenced as \\textbf{error}\n   until they are assigned a value. \n\end{itemize}
\noindent Example 1: 
\begin{center}\begin{minipage}{15cm}\begin{Verbatim}[frame=single]
\end{Verbatim}
\end{minipage}\end{center}
See also: \textbf{error} (\ref{laberror})

\subsection{verbosity}
\label{labverbosity}
\noindent Name: \textbf{verbosity}\\
global variable controlling the quantity of information displayed by commands.\\

\noindent Description: \begin{itemize}

\item \textbf{verbosity} accepts any integer value. At level 0, commands do not display anything
   on standard out. Note that very critical information may however be displayed on
   standard err.

\item Default level is 1. It displays important informations such as warnings when 
   roundings happen.

\item For higher levels more informations are displayed depending on the command.
\end{itemize}
\noindent Example 1: 
\begin{center}\begin{minipage}{15cm}\begin{Verbatim}[frame=single]
> verbosity=0!;
> 1.2+"toto";
error
> verbosity=1!;
> 1.2+"toto";
Warning: Rounding occurred when converting the constant "1.2" to floating-point 
with 165 bits.
If safe computation is needed, try to increase the precision.
Warning: at least one of the given expressions or a subexpression is not correct
ly typed
or its evaluation has failed because of some error on a side-effect.
error
> verbosity=2!;
> 1.2+"toto";
Warning: Rounding occurred when converting the constant "1.2" to floating-point 
with 165 bits.
If safe computation is needed, try to increase the precision.
Warning: at least one of the given expressions or a subexpression is not correct
ly typed
or its evaluation has failed because of some error on a side-effect.
Information: the expression or a partial evaluation of it has been the following
:
(0.119999999999999999999999999999999999999999999999999e1) + ("toto")
error
\end{Verbatim}
\end{minipage}\end{center}
See also: \textbf{roundingwarnings} (\ref{labroundingwarnings})

\subsection{void}
\label{labvoid}
\noindent Name: \textbf{void}\\
the functional result of a side-effect or empty argument resp. the corresponding type\\
\noindent Usage: 
\begin{center}
\textbf{void} : \textsf{void} $|$ \textsf{type type}\\
\end{center}
\noindent Description: \begin{itemize}

\item The variable \\textbf{void} represents the functional result of a\n   side-effect or an empty argument.  It is used only in combination with\n   the applications of procedures or identifiers bound through\n   \\textbf{externalproc} to external procedures.\n    \n   The \\textbf{void} result produced by a procedure or an external procedure\n   is not printed at the prompt. However, it is possible to print it out\n   in a print statement or in complex data types such as lists.\n    \n   The \\textbf{void} argument is implicit when giving no argument to a\n   procedure or an external procedure when applied. It can nevertheless be given\n   explicitly.  For example, suppose that foo is a procedure or an\n   external procedure with a void argument. Then foo() and foo(void) are\n   correct calls to foo. Here, a distinction must be made for procedures having an\n   arbitrary number of arguments. In this case, an implicit \\textbf{void}\n   as the only parameter to a call of such a procedure gets converted into \n   an empty list of arguments, an explicit \\textbf{void} gets passed as-is in the\n   formal list of parameters the procedure receives.\n
\item \\textbf{void} is used also as a type identifier for\n   \\textbf{externalproc}. Typically, an external procedure taking \\textbf{void} as an\n   argument or returning \\textbf{void} is bound with a signature \\textbf{void} $->$\n   some type or some type $->$ \\textbf{void}. See \\textbf{externalproc} for more\n   details.\n\end{itemize}
\noindent Example 1: 
\begin{center}\begin{minipage}{15cm}\begin{Verbatim}[frame=single]
\end{Verbatim}
\end{minipage}\end{center}
\noindent Example 2: 
\begin{center}\begin{minipage}{15cm}\begin{Verbatim}[frame=single]
\end{Verbatim}
\end{minipage}\end{center}
\noindent Example 3: 
\begin{center}\begin{minipage}{15cm}\begin{Verbatim}[frame=single]
\end{Verbatim}
\end{minipage}\end{center}
\noindent Example 4: 
\begin{center}\begin{minipage}{15cm}\begin{Verbatim}[frame=single]
\end{Verbatim}
\end{minipage}\end{center}
See also: \textbf{error} (\ref{laberror}), \textbf{proc} (\ref{labproc}), \textbf{externalproc} (\ref{labexternalproc})

\subsection{worstcase}
\label{labworstcase}
\noindent Name: \textbf{worstcase}\\
searches for hard-to-round cases of a function\\
\noindent Usage: 
\begin{center}
\textbf{worstcase}(\emph{function}, \emph{preimage precision}, \emph{preimage exponent range}, \emph{image precision}, \emph{error bound}) : (\textsf{function}, \textsf{integer}, \textsf{range}, \textsf{integer}, \textsf{constant}) $\rightarrow$ \textsf{void}\\
\textbf{worstcase}(\emph{function}, \emph{preimage precision}, \emph{preimage exponent range}, \emph{image precision}, \emph{error bound}, \emph{filename}) : (\textsf{function}, \textsf{integer}, \textsf{range}, \textsf{integer}, \textsf{constant}, \textsf{string}) $\rightarrow$ \textsf{void}\\
\end{center}
Parameters: 
\begin{itemize}
\item \emph{function} represents the function to be considered
\item \emph{preimage precision} represents the precision of the preimages
\item \emph{preimage exponent range} represents the exponents in the preimage format
\item \emph{image precision} represents the precision of the format the images are to be rounded to
\item \emph{error bound} represents the upper bound for the search w.r.t. the relative rounding error
\item \emph{filename} represents a character sequence containing a filename
\end{itemize}
\noindent Description: \begin{itemize}

\item The \\textbf{worstcase} command is deprecated. It searches for hard-to-round cases of\n   a function. The command \\textbf{searchgal} has a comparable functionality.\n\end{itemize}
\noindent Example 1: 
\begin{center}\begin{minipage}{15cm}\begin{Verbatim}[frame=single]
\end{Verbatim}
\end{minipage}\end{center}
See also: \textbf{round} (\ref{labround}), \textbf{searchgal} (\ref{labsearchgal}), \textbf{evaluate} (\ref{labevaluate})

\subsection{write}
\label{labwrite}
\noindent Name: \textbf{write}\\
prints an expression without separators\\
\noindent Usage: 
\begin{center}
\textbf{write}(\emph{expr1},...,\emph{exprn}) : (\textsf{any type},..., \textsf{any type}) $\rightarrow$ \textsf{void}\\
\textbf{write}(\emph{expr1},...,\emph{exprn}) $>$ \emph{filename} : (\textsf{any type},..., \textsf{any type}, \textsf{string}) $\rightarrow$ \textsf{void}\\
\textbf{write}(\emph{expr1},...,\emph{exprn}) $>>$ \emph{filename} : (\textsf{any type},...,\textsf{any type}, \textsf{string}) $\rightarrow$ \textsf{void}\\
\end{center}
Parameters: 
\begin{itemize}
\item \emph{expr} represents an expression
\item \emph{filename} represents a character sequence indicating a file name
\end{itemize}
\noindent Description: \begin{itemize}

\item \\textbf{write}(\\emph{expr1},...,\\emph{exprn}) prints the expressions \\emph{expr1} through\n   \\emph{exprn}. The character sequences corresponding to the expressions are\n   concatenated without any separator. No newline is displayed at the\n   end.  In contrast to \\textbf{print}, \\textbf{write} expects the user to give all\n   separators and newlines explicitly.\n    \n   If a second argument \\emph{filename} is given after a single "$>$", the\n   displaying is not output on the standard output of \\sollya but if in\n   the file \\emph{filename} that get newly created or overwritten. If a double\n    "$>>$" is given, the output will be appended to the file \\emph{filename}.\n    \n   The global variables \\textbf{display}, \\textbf{midpointmode} and \\textbf{fullparentheses} have\n   some influence on the formatting of the output (see \\textbf{display},\n   \\textbf{midpointmode} and \\textbf{fullparentheses}).\n    \n   Remark that if one of the expressions \\emph{expri} given in argument is of\n   type \\textsf{string}, the character sequence \\emph{expri} evaluates to is\n   displayed. However, if \\emph{expri} is of type \\textsf{list} and this list\n   contains a variable of type \\textsf{string}, the expression for the list\n   is displayed, i.e.  all character sequences get displayed surrounded\n   by quotes ("). Nevertheless, escape sequences used upon defining\n   character sequences are interpreted immediately.\n\end{itemize}
\noindent Example 1: 
\begin{center}\begin{minipage}{15cm}\begin{Verbatim}[frame=single]
\end{Verbatim}
\end{minipage}\end{center}
\noindent Example 2: 
\begin{center}\begin{minipage}{15cm}\begin{Verbatim}[frame=single]
\end{Verbatim}
\end{minipage}\end{center}
\noindent Example 3: 
\begin{center}\begin{minipage}{15cm}\begin{Verbatim}[frame=single]
\end{Verbatim}
\end{minipage}\end{center}
\noindent Example 4: 
\begin{center}\begin{minipage}{15cm}\begin{Verbatim}[frame=single]
\end{Verbatim}
\end{minipage}\end{center}
See also: \textbf{print} (\ref{labprint}), \textbf{printexpansion} (\ref{labprintexpansion}), \textbf{printhexa} (\ref{labprinthexa}), \textbf{printfloat} (\ref{labprintfloat}), \textbf{printxml} (\ref{labprintxml}), \textbf{readfile} (\ref{labreadfile}), \textbf{autosimplify} (\ref{labautosimplify}), \textbf{display} (\ref{labdisplay}), \textbf{midpointmode} (\ref{labmidpointmode}), \textbf{fullparentheses} (\ref{labfullparentheses}), \textbf{evaluate} (\ref{labevaluate})



\section{Grammar of the \sollya language}

\begin{tabular}{lcl}
program & $\rightarrow$ & statement \\
 & $|$ & program statement \\
 & & \\
statement & $\rightarrow$ & command \textbf{;} \\
 & & \\
command & $\rightarrow$ & simplecommand \\
 & $|$ & \textbf{\{} commandlist \textbf{\}} \\
 & $|$ & \textbf{\{} variabledeclarationlist commandlist \textbf{\}} \\
 & $|$ & \textbf{\{} variabledeclarationlist \textbf{\}} \\
 & $|$ & \textbf{\{} \textbf{\}} \\
 & $|$ & \textbf{if} ifcommand \\
 & $|$ & \textbf{while} thing \textbf{do} command \\
 & $|$ & \textbf{for} forcommand \\
 & & \\
ifcommand & $\rightarrow$ & thing \textbf{then} command \\
 & $|$ & thing \textbf{then} command \textbf{else} command \\
 & & \\
forcommand & $\rightarrow$ & identifier \textbf{from} thing \textbf{to} thing \textbf{do} command \\
 & $|$ & identifier \textbf{from} thing \textbf{to} thing \textbf{by} thing \textbf{do} command \\
 & $|$ & identifier \textbf{in} thing \textbf{do} command \\
 & & \\
commandlist & $\rightarrow$ & command \textbf{;} \\
 & $|$ & command \textbf{;} commandlist \\
 & & \\
variabledeclarationlist & $\rightarrow$ & variabledeclaration \textbf{;} \\
 & $|$ & variabledeclaration \textbf{;} variabledeclarationlist \\
 & & \\
variabledeclaration & $\rightarrow$ & \textbf{var} identifierlist \\
 & & \\
identifierlist & $\rightarrow$ & identifier \\
 & $|$ & identifier \textbf{,} identifierlist \\
 & & \\
procbody & $\rightarrow$ & \textbf{(} \textbf{)} \textbf{\{} commandlist \textbf{\}} \\
 & $|$ & \textbf{(} \textbf{)} \textbf{\{} variabledeclarationlist commandlist \textbf{\}} \\
 & $|$ & \textbf{(} \textbf{)} \textbf{\{} variabledeclarationlist \textbf{\}} \\
 & $|$ & \textbf{(} \textbf{)} \textbf{\{} \textbf{\}} \\
 & $|$ & \textbf{(} \textbf{)} \textbf{\{} commandlist \textbf{return} thing \textbf{;} \textbf{\}} \\
 & $|$ & \textbf{(} \textbf{)} \textbf{\{} variabledeclarationlist commandlist \textbf{return} thing \textbf{;} \textbf{\}} \\
 & $|$ & \textbf{(} \textbf{)} \textbf{\{} variabledeclarationlist \textbf{return} thing \textbf{;} \textbf{\}} \\
 & $|$ & \textbf{(} \textbf{)} \textbf{\{} \textbf{return} thing \textbf{;} \textbf{\}} \\
 & $|$ & \textbf{(} identifierlist \textbf{)} \textbf{\{} commandlist \textbf{\}} \\
 & $|$ & \textbf{(} identifierlist \textbf{)} \textbf{\{} variabledeclarationlist commandlist \textbf{\}} \\
 & $|$ & \textbf{(} identifierlist \textbf{)} \textbf{\{} variabledeclarationlist \textbf{\}} \\
 & $|$ & \textbf{(} identifierlist \textbf{)} \textbf{\{} \textbf{\}} \\
 & $|$ & \textbf{(} identifierlist \textbf{)} \textbf{\{} commandlist \textbf{return} thing \textbf{;} \textbf{\}} \\
 & $|$ & \textbf{(} identifierlist \textbf{)} \textbf{\{} variabledeclarationlist commandlist \textbf{return} thing \textbf{;} \textbf{\}} \\
 & $|$ & \textbf{(} identifierlist \textbf{)} \textbf{\{} variabledeclarationlist \textbf{return} thing \textbf{;} \textbf{\}} \\
 & $|$ & \textbf{(} identifierlist \textbf{)} \textbf{\{} \textbf{return} thing \textbf{;} \textbf{\}} \\
\end{tabular}\\
\begin{tabular}{lcl} 
simplecommand & $\rightarrow$ & \textbf{quit} \\
 & $|$ & \textbf{restart} \\
 & $|$ & \textbf{nop} \\
 & $|$ & \textbf{print} \textbf{(} thinglist \textbf{)} \\
 & $|$ & \textbf{print} \textbf{(} thinglist \textbf{)} \textbf{$>$} thing \\
 & $|$ & \textbf{print} \textbf{(} thinglist \textbf{)} \textbf{$>$} \textbf{$>$} thing \\
 & $|$ & \textbf{plot} \textbf{(} thing \textbf{,} thinglist \textbf{)} \\
 & $|$ & \textbf{printhexa} \textbf{(} thing \textbf{)} \\
 & $|$ & \textbf{printfloat} \textbf{(} thing \textbf{)} \\
 & $|$ & \textbf{printbinary} \textbf{(} thing \textbf{)} \\
 & $|$ & \textbf{printexpansion} \textbf{(} thing \textbf{)} \\
 & $|$ & \textbf{bashexecute} \textbf{(} thing \textbf{)} \\
 & $|$ & \textbf{externalplot} \textbf{(} thing \textbf{,} thing \textbf{,} thing \textbf{,} thing \textbf{,} thinglist \textbf{)} \\
 & $|$ & \textbf{write} \textbf{(} thinglist \textbf{)} \\
 & $|$ & \textbf{write} \textbf{(} thinglist \textbf{)} \textbf{$>$} thing \\
 & $|$ & \textbf{write} \textbf{(} thinglist \textbf{)} \textbf{$>$} \textbf{$>$} thing \\
 & $|$ & \textbf{asciiplot} \textbf{(} thing \textbf{,} thing \textbf{)} \\
 & $|$ & \textbf{printxml} \textbf{(} thing \textbf{)} \\
 & $|$ & \textbf{execute} \textbf{(} thing \textbf{)} \\
 & $|$ & \textbf{printxml} \textbf{(} thing \textbf{)} \textbf{$>$} thing \\
 & $|$ & \textbf{printxml} \textbf{(} thing \textbf{)} \textbf{$>$} \textbf{$>$} thing \\
 & $|$ & \textbf{worstcase} \textbf{(} thing \textbf{,} thing \textbf{,} thing \textbf{,} thing \textbf{,} thinglist \textbf{)} \\
 & $|$ & \textbf{rename} \textbf{(} identifier \textbf{,} identifier \textbf{)} \\
 & $|$ & \textbf{externalproc} \textbf{(} identifier \textbf{,} thing \textbf{,} externalproctypelist \textbf{-} \textbf{$>$} \\
 & & extendedexternalproctype \textbf{)} \\
 & $|$ & assignment \\
 & $|$ & thinglist \\
 & $|$ & \textbf{procedure} identifier procbody \\
 & & \\
assignment & $\rightarrow$ & stateassignment \\
 & $|$ & stillstateassignment \textbf{!} \\
 & $|$ & simpleassignment \\
 & $|$ & simpleassignment \textbf{!} \\
 & & \\
simpleassignment & $\rightarrow$ & identifier \textbf{=} thing \\
 & $|$ & identifier \textbf{:=} thing \\
 & $|$ & identifier \textbf{=} \textbf{library} \textbf{(} thing \textbf{)} \\
 & $|$ & indexing \textbf{=} thing \\
 & $|$ & indexing \textbf{:=} thing \\
 & & \\
stateassignment & $\rightarrow$ & \textbf{prec} \textbf{=} thing \\
 & $|$ & \textbf{points} \textbf{=} thing \\
 & $|$ & \textbf{diam} \textbf{=} thing \\
 & $|$ & \textbf{display} \textbf{=} thing \\
 & $|$ & \textbf{verbosity} \textbf{=} thing \\
 & $|$ & \textbf{canonical} \textbf{=} thing \\
 & $|$ & \textbf{autosimplify} \textbf{=} thing \\
 & $|$ & \textbf{taylorrecursions} \textbf{=} thing \\
 & $|$ & \textbf{timing} \textbf{=} thing \\
 & $|$ & \textbf{fullparentheses} \textbf{=} thing \\
 & $|$ & \textbf{midpointmode} \textbf{=} thing \\
 & $|$ & \textbf{hopitalrecursions} \textbf{=} thing \\
 \end{tabular}\\
\begin{tabular}{lcl}
stillstateassignment & $\rightarrow$ & \textbf{prec} \textbf{=} thing \\
 & $|$ & \textbf{points} \textbf{=} thing \\
 & $|$ & \textbf{diam} \textbf{=} thing \\
 & $|$ & \textbf{display} \textbf{=} thing \\
 & $|$ & \textbf{verbosity} \textbf{=} thing \\
 & $|$ & \textbf{canonical} \textbf{=} thing \\
 & $|$ & \textbf{autosimplify} \textbf{=} thing \\
 & $|$ & \textbf{taylorrecursions} \textbf{=} thing \\
 & $|$ & \textbf{timing} \textbf{=} thing \\
 & $|$ & \textbf{fullparentheses} \textbf{=} thing \\
 & $|$ & \textbf{midpointmode} \textbf{=} thing \\
 & $|$ & \textbf{hopitalrecursions} \textbf{=} thing \\
 & & \\
thinglist & $\rightarrow$ & thing \\
 & $|$ & thing \textbf{,} thinglist \\
 & & \\
thing & $\rightarrow$ & megaterm \\
 & $|$ & thing \textbf{\&\&} megaterm \\
 & $|$ & thing \textbf{$|$$|$} megaterm \\
 & $|$ & \textbf{!} megaterm \\
 & & \\
indexing & $\rightarrow$ & basicthing \textbf{[} thing \textbf{]} \\
 & & \\
megaterm & $\rightarrow$ & hyperterm \\
 & $|$ & megaterm \textbf{==} hyperterm \\
 & $|$ & megaterm \textbf{$<$} hyperterm \\
 & $|$ & megaterm \textbf{$>$} hyperterm \\
 & $|$ & megaterm \textbf{$<$} \textbf{=} hyperterm \\
 & $|$ & megaterm \textbf{$>$} \textbf{=} hyperterm \\
 & $|$ & megaterm \textbf{!=} hyperterm \\
 & & \\
hyperterm & $\rightarrow$ & term \\
 & $|$ & hyperterm \textbf{+} term \\
 & $|$ & hyperterm \textbf{-} term \\
 & $|$ & hyperterm \textbf{@} term \\
 & $|$ & hyperterm \textbf{::} term \\
 & $|$ & hyperterm \textbf{.:} term \\
 & $|$ & hyperterm \textbf{:.} term \\
 & & \\
term & $\rightarrow$ & subterm \\
 & $|$ & \textbf{-} subterm \\
 & $|$ & term \textbf{*} subterm \\
 & $|$ & term \textbf{/} subterm \\
 & & \\
subterm & $\rightarrow$ & basicthing \\
 & $|$ & subterm \textbf{\^} basicthing \\
\end{tabular}\\
\begin{tabular}{lcl}
basicthing & $\rightarrow$ & \textbf{on} \\
 & $|$ & \textbf{off} \\
 & $|$ & \textbf{dyadic} \\
 & $|$ & \textbf{powers} \\
 & $|$ & \textbf{binary} \\
 & $|$ & \textbf{hexadecimal} \\
 & $|$ & \textbf{file} \\
 & $|$ & \textbf{postscript} \\
 & $|$ & \textbf{postscriptfile} \\
 & $|$ & \textbf{perturb} \\
 & $|$ & \textbf{RD} \\
 & $|$ & \textbf{RU} \\
 & $|$ & \textbf{RZ} \\
 & $|$ & \textbf{RN} \\
 & $|$ & \textbf{honorcoeffprec} \\
 & $|$ & \textbf{true} \\
 & $|$ & \textbf{void} \\
 & $|$ & \textbf{false} \\
 & $|$ & \textbf{default} \\
 & $|$ & \textbf{decimal} \\
 & $|$ & \textbf{absolute} \\
 & $|$ & \textbf{relative} \\
 & $|$ & \textbf{error} \\
 & $|$ & \textbf{double} \\
 & $|$ & \textbf{doubleextended} \\
 & $|$ & \textbf{doubledouble} \\
 & $|$ & \textbf{tripledouble} \\
 & $|$ & \textbf{string} \\
 & $|$ & constant \\
 & $|$ & identifier \\
 & $|$ & \textbf{isbound} \textbf{(} identifier \textbf{)} \\
 & $|$ & identifier \textbf{(} thinglist \textbf{)} \\
 & $|$ & identifier \textbf{(} \textbf{)} \\
 & $|$ & list \\
 & $|$ & range \\
 & $|$ & debound \\
 & $|$ & headfunction \\
 & $|$ & \textbf{(} thing \textbf{)} \\
 & $|$ & statedereference \\
 & $|$ & indexing \\
 & $|$ & \textbf{(} thing \textbf{)} \textbf{(} thinglist \textbf{)} \\
 & $|$ & \textbf{proc} procbody \\
 & & \\
constant & $\rightarrow$ & constant \\
 & $|$ & dyadicconstant \\
 & $|$ & hexconstant \\
 & $|$ & hexadecimalconstant \\
 & $|$ & binaryconstant \\
 & $|$ & \textbf{pi} \\
 & & \\
list & $\rightarrow$ & \textbf{[} \textbf{$|$} \textbf{$|$} \textbf{]} \\
 & $|$ & \textbf{[} \textbf{$|$$|$} \textbf{]} \\
 & $|$ & \textbf{[} \textbf{$|$} simplelist \textbf{$|$} \textbf{]} \\
 & $|$ & \textbf{[} \textbf{$|$} simplelist \textbf{...} \textbf{$|$} \textbf{]} \\
\end{tabular}\\
\begin{tabular}{lcl}
simplelist & $\rightarrow$ & thing \\
 & $|$ & thing \textbf{,} simplelist \\
 & $|$ & thing \textbf{,} \textbf{...} \textbf{,} simplelist \\
 & & \\
range & $\rightarrow$ & \textbf{[} thing \textbf{,} thing \textbf{]} \\
 & $|$ & \textbf{[} thing \textbf{;} thing \textbf{]} \\
 & & \\
debound & $\rightarrow$ & \textbf{*$<$} thing \textbf{$>$*} \\
 & $|$ & \textbf{*$<$} thing \textbf{$>$.} \\
 & $|$ & \textbf{*$<$} thing \textbf{$>$\_} \\
 & $|$ & \textbf{sup} \textbf{(} thing \textbf{)} \\
 & $|$ & \textbf{mid} \textbf{(} thing \textbf{)} \\
 & $|$ & \textbf{inf} \textbf{(} thing \textbf{)} \\
 & & \\
headfunction & $\rightarrow$ & \textbf{diff} \textbf{(} thing \textbf{)} \\
 & $|$ & \textbf{simplify} \textbf{(} thing \textbf{)} \\
 & $|$ & \textbf{remez} \textbf{(} thing \textbf{,} thing \textbf{,} thinglist \textbf{)} \\
 & $|$ & \textbf{horner} \textbf{(} thing \textbf{)} \\
 & $|$ & \textbf{canonical} \textbf{(} thing \textbf{)} \\
 & $|$ & \textbf{expand} \textbf{(} thing \textbf{)} \\
 & $|$ & \textbf{simplifysafe} \textbf{(} thing \textbf{)} \\
 & $|$ & \textbf{taylor} \textbf{(} thing \textbf{,} thing \textbf{,} thing \textbf{)} \\
 & $|$ & \textbf{degree} \textbf{(} thing \textbf{)} \\
 & $|$ & \textbf{numerator} \textbf{(} thing \textbf{)} \\
 & $|$ & \textbf{denominator} \textbf{(} thing \textbf{)} \\
 & $|$ & \textbf{substitute} \textbf{(} thing \textbf{,} thing \textbf{)} \\
 & $|$ & \textbf{coeff} \textbf{(} thing \textbf{,} thing \textbf{)} \\
 & $|$ & \textbf{subpoly} \textbf{(} thing \textbf{,} thing \textbf{)} \\
 & $|$ & \textbf{roundcoefficients} \textbf{(} thing \textbf{,} thing \textbf{)} \\
 & $|$ & \textbf{rationalapprox} \textbf{(} thing \textbf{,} thing \textbf{)} \\
 & $|$ & \textbf{accurateinfnorm} \textbf{(} thing \textbf{,} thing \textbf{,} thinglist \textbf{)} \\
 & $|$ & \textbf{roundtoformat} \textbf{(} thing \textbf{,} thing \textbf{,} thing \textbf{)} \\
 & $|$ & \textbf{evaluate} \textbf{(} thing \textbf{,} thing \textbf{)} \\
 & $|$ & \textbf{parse} \textbf{(} thing \textbf{)} \\
 & $|$ & \textbf{readxml} \textbf{(} thing \textbf{)} \\
 & $|$ & \textbf{infnorm} \textbf{(} thing \textbf{,} thinglist \textbf{)} \\
 & $|$ & \textbf{findzeros} \textbf{(} thing \textbf{,} thing \textbf{)} \\
 & $|$ & \textbf{fpfindzeros} \textbf{(} thing \textbf{,} thing \textbf{)} \\
 & $|$ & \textbf{dirtyinfnorm} \textbf{(} thing \textbf{,} thing \textbf{)} \\
 & $|$ & \textbf{integral} \textbf{(} thing \textbf{,} thing \textbf{)} \\
 & $|$ & \textbf{dirtyintegral} \textbf{(} thing \textbf{,} thing \textbf{)} \\
 & $|$ & \textbf{implementpoly} \textbf{(} thing \textbf{,} thing \textbf{,} thing \textbf{,} thing \textbf{,} thing \textbf{,} thinglist \textbf{)} \\
 & $|$ & \textbf{checkinfnorm} \textbf{(} thing \textbf{,} thing \textbf{,} thing \textbf{)} \\
 & $|$ & \textbf{zerodenominators} \textbf{(} thing \textbf{,} thing \textbf{)} \\
 & $|$ & \textbf{isevaluable} \textbf{(} thing \textbf{,} thing \textbf{)} \\
 & $|$ & \textbf{searchgal} \textbf{(} thinglist \textbf{)} \\
 & $|$ & \textbf{guessdegree} \textbf{(} thing \textbf{,} thing \textbf{,} thinglist \textbf{)} \\
 & $|$ & \textbf{dirtyfindzeros} \textbf{(} thing \textbf{,} thing \textbf{)} \\
 & $|$ & \textbf{head} \textbf{(} thing \textbf{)} \\
 & $|$ & \textbf{roundcorrectly} \textbf{(} thing \textbf{)} \\
 & $|$ & \textbf{readfile} \textbf{(} thing \textbf{)} \\
 & $|$ & \textbf{revert} \textbf{(} thing \textbf{)} \\
 & $|$ & \textbf{sort} \textbf{(} thing \textbf{)} \\
 & $|$ & \textbf{mantissa} \textbf{(} thing \textbf{)} \\
 & $|$ & \textbf{exponent} \textbf{(} thing \textbf{)} \\
\end{tabular}\\
\begin{tabular}{lcl}
 & $|$ & \textbf{precision} \textbf{(} thing \textbf{)} \\
 & $|$ & \textbf{tail} \textbf{(} thing \textbf{)} \\
 & $|$ & \textbf{sqrt} \textbf{(} thing \textbf{)} \\
 & $|$ & \textbf{exp} \textbf{(} thing \textbf{)} \\
 & $|$ & \textbf{log} \textbf{(} thing \textbf{)} \\
 & $|$ & \textbf{log2} \textbf{(} thing \textbf{)} \\
 & $|$ & \textbf{log10} \textbf{(} thing \textbf{)} \\
 & $|$ & \textbf{sin} \textbf{(} thing \textbf{)} \\
 & $|$ & \textbf{cos} \textbf{(} thing \textbf{)} \\
 & $|$ & \textbf{tan} \textbf{(} thing \textbf{)} \\
 & $|$ & \textbf{asin} \textbf{(} thing \textbf{)} \\
 & $|$ & \textbf{acos} \textbf{(} thing \textbf{)} \\
 & $|$ & \textbf{atan} \textbf{(} thing \textbf{)} \\
 & $|$ & \textbf{sinh} \textbf{(} thing \textbf{)} \\
 & $|$ & \textbf{cosh} \textbf{(} thing \textbf{)} \\
 & $|$ & \textbf{tanh} \textbf{(} thing \textbf{)} \\
 & $|$ & \textbf{asinh} \textbf{(} thing \textbf{)} \\
 & $|$ & \textbf{acosh} \textbf{(} thing \textbf{)} \\
 & $|$ & \textbf{atanh} \textbf{(} thing \textbf{)} \\
 & $|$ & \textbf{abs} \textbf{(} thing \textbf{)} \\
 & $|$ & \textbf{erf} \textbf{(} thing \textbf{)} \\
 & $|$ & \textbf{erfc} \textbf{(} thing \textbf{)} \\
 & $|$ & \textbf{log1p} \textbf{(} thing \textbf{)} \\
 & $|$ & \textbf{expm1} \textbf{(} thing \textbf{)} \\
 & $|$ & \textbf{double} \textbf{(} thing \textbf{)} \\
 & $|$ & \textbf{doubledouble} \textbf{(} thing \textbf{)} \\
 & $|$ & \textbf{tripledouble} \textbf{(} thing \textbf{)} \\
 & $|$ & \textbf{doubleextended} \textbf{(} thing \textbf{)} \\
 & $|$ & \textbf{ceil} \textbf{(} thing \textbf{)} \\
 & $|$ & \textbf{floor} \textbf{(} thing \textbf{)} \\
 & $|$ & \textbf{length} \textbf{(} thing \textbf{)} \\
 & & \\
statedereference & $\rightarrow$ & \textbf{prec} \textbf{=} \textbf{?} \\
 & $|$ & \textbf{points} \textbf{=} \textbf{?} \\
 & $|$ & \textbf{diam} \textbf{=} \textbf{?} \\
 & $|$ & \textbf{display} \textbf{=} \textbf{?} \\
 & $|$ & \textbf{verbosity} \textbf{=} \textbf{?} \\
 & $|$ & \textbf{canonical} \textbf{=} \textbf{?} \\
 & $|$ & \textbf{autosimplify} \textbf{=} \textbf{?} \\
 & $|$ & \textbf{taylorrecursions} \textbf{=} \textbf{?} \\
 & $|$ & \textbf{timing} \textbf{=} \textbf{?} \\
 & $|$ & \textbf{fullparentheses} \textbf{=} \textbf{?} \\
 & $|$ & \textbf{midpointmode} \textbf{=} \textbf{?} \\
 & $|$ & \textbf{hopitalrecursions} \textbf{=} \textbf{?} \\
\end{tabular} \\
\begin{tabular}{lcl}
externalproctype & $\rightarrow$ & \textbf{constant} \\
 & $|$ & \textbf{function} \\
 & $|$ & \textbf{range} \\
 & $|$ & \textbf{integer} \\
 & $|$ & \textbf{string} \\
 & $|$ & \textbf{boolean} \\
 & $|$ & \textbf{list} \textbf{of} \textbf{constant} \\
 & $|$ & \textbf{list} \textbf{of} \textbf{function} \\
 & $|$ & \textbf{list} \textbf{of} \textbf{range} \\
 & $|$ & \textbf{list} \textbf{of} \textbf{integer} \\
 & $|$ & \textbf{list} \textbf{of} \textbf{string} \\
 & $|$ & \textbf{list} \textbf{of} \textbf{boolean} \\
 & & \\
extendedexternalproctype & $\rightarrow$ & \textbf{void} \\
 & $|$ & externalproctype \\
 & & \\
externalproctypesimplelist & $\rightarrow$ & externalproctype \\
 & $|$ & externalproctype \textbf{,} externalproctypesimplelist \\
 & & \\
externalproctypelist & $\rightarrow$ & extendedexternalproctype \\
 & $|$ & \textbf{(} externalproctypesimplelist \textbf{)} \\
 & & \\
\end{tabular}



\end{document}