\documentclass[a4paper]{article}
 
\usepackage[english]{babel}
\usepackage[naturalnames]{hyperref}
\usepackage{fullpage}
\usepackage{xspace}
\usepackage{amssymb}
\usepackage{fancyvrb}

\newcommand{\com}[1]{\texttt{#1}}
\newcommand{\key}[1]{\texttt{#1}}
\newcommand{\sollya}{\texttt{Sollya}\xspace}
\newcommand{\rlwrap}{\texttt{rlwrap}\xspace}

\newcommand{\csi}{\xi} % Workaround before dealing with commands of the form
                       % \xsomething in .shlp files.

\newcommand{\code}[1]{
\begin{center}
\begin{tabular}{|p{14.8cm}|}
\hline
#1
\hline
\end{tabular}
\end{center}
}

\newcommand{\ligne}[1]{\texttt{#1}\\}

\title{Users' manual for the \sollya tool \\ {\large Release 4.0-alpha}}

\author{Sylvain Chevillard \\ \small{\url{sylvain.chevillard@ens-lyon.org}} \and Christoph Lauter \\ \small{\url{christoph.lauter@ens-lyon.org}} \and Mioara Jolde\c{s} \\ \small{\url{mioara.joldes@ens-lyon.fr}} }

\date{\today}


\begin{document}

\maketitle

\section*{License}

The \sollya tool is Copyright \copyright~2006-2012 by \vspace{2mm} \\
\indent Laboratoire de
l'Informatique du Parall\'elisme - UMR CNRS - ENS Lyon - UCB Lyon 1 -
INRIA 5668, Lyon, France, \vspace{2mm} \\
\indent LORIA (CNRS, INPL, INRIA, UHP, U-Nancy 2), Nancy, France, \vspace{2mm} \\
\indent Laboratoire d'Informatique de Paris 6, \'Equipe PEQUAN,
UPMC Universit\'e Paris 06 - CNRS - UMR 7606 - LIP6, Paris, France,\vspace{2mm} \\ and by \vspace{2mm} \\
\indent INRIA Sophia-Antipolis M\'editerran\'ee, APICS Team, Sophia-Antipolis, France. \vspace{2mm} \\
All rights reserved.\\[0.2cm]

\noindent The \sollya tool is open software. It is distributed and can be used,
modified and redistributed under the terms of the CeCILL-C licence
available at \url{http://www.cecill.info/} and reproduced in the
\texttt{COPYING} file of the distribution. The distribution contains
parts of other libraries as a support for but not integral part of
\sollya. These libraries are reigned by the GNU Lesser General Public
License that is available at \url{http://www.gnu.org/licenses/} and
reproduced in the \texttt{COPYING} file of the distribution.\\[0.2cm]

\noindent This software (\sollya) is distributed WITHOUT ANY WARRANTY; without even the 
implied warranty of MERCHANTABILITY or FITNESS FOR A PARTICULAR PURPOSE.

\tableofcontents

\section{Compilation and installation of \sollya}

\sollya comes in two flavors:
\begin{itemize}
\item Either as an interactive tool. This is achieved by running the \sollya executable file.
\item Or as a C library that provides all the features of the tool within the C programming language.
\end{itemize}

The installation of the tool and the library follow the same steps, desribed below. The present documentation focuses more on the interactive tool. As a matter of fact, the library works exactly the same way as the tool, so it is necessary to know a little about the tool in order to correctly use the library. The reader who is only interested in the library should at least read the following Sections~\ref{sec:introduction}, \ref{sec:general_principles} and \ref{sec:data_types}. A documentation specifically describing the library usage is available in Appendix~\ref{Libsollya} at the end of the present documentation.


\subsection{Compilation dependencies}

The \sollya distribution can be compiled and installed using the usual
\texttt{./configure}, \texttt{make}, \texttt{make install}
procedure. Besides a \texttt{C} and a \texttt{C++} compiler, \sollya needs the following
software libraries and tools to be installed. The \texttt{./configure}
script checks for the installation of the libraries. However \sollya
will build without error if some of its external tools are not
installed. In this case an error will be displayed at runtime.
\begin{itemize}
\item \texttt{GMP}
\item \texttt{MPFR}
\item \texttt{MPFI}
\item \texttt{fplll}
\item \texttt{libxml2}
\item \texttt{gnuplot} (external tool)
\end{itemize}
The use of the external tool \texttt{rlwrap} is highly recommended but
not required to use the \sollya interactive tool. Use the \texttt{-A}
option of \texttt{rlwrap} for correctly displayed ANSI X3.64/ ISO/IEC
6429 colored prompts (see below).

\subsection{\sollya command line options}

\sollya can read input on standard input or in a file whose name is given 
as an argument when \sollya is invoked. The tool will always produce its 
output on standard output, unless specificly instructed by a particular
\sollya command that writes to a file.
The following lines are valid invocations of \sollya, assuming that 
\texttt{bash} is used as a shell:
\begin{center}\begin{minipage}{15cm}\begin{Verbatim}[frame=single]
~/% sollya
...
~/% sollya myfile.sollya
...
~/% sollya < myfile.sollya
\end{Verbatim}
\end{minipage}\end{center}
If a file given as an input does not exist, an error message is displayed.

All configurations of the internal state of the tool are done by
commands given on the \sollya prompt or in \sollya
scripts. Nevertheless, some command line options are supported; they
work at a very basic I/O-level and can therefore not be implemented as
commands.

The following options are supported when calling \sollya:
\begin{itemize}
\item \texttt{--donotmodifystacksize}: When invoked, \sollya trys to increase
the stack size that is available to a user process to the maximum size
supported by the kernel. On some systems, the correspondent \texttt{ioctl} 
does not work properly. Use the option to prevent \sollya from changing the 
stack size.
\item \texttt{--flush}: When this option is given, \sollya will flush
all its input and output buffers after parsing and executing each
command resp. sequence of commands. This option is needed when pipes
are used to communicate with \sollya from another program.
\item \texttt{--help}: Prints help on the usage of the tool and quits.
\item \texttt{--nocolor}: \sollya supports coloring of the output
  using ANSI X3.64/ ISO/IEC 6429 escape sequences. Coloring is
  deactivated when \sollya is connected on standard input to a file
  that is not a terminal. This option forces the deactivation of ANSI
  coloring. This might be necessary on very old grey-scale terminals or when
  encountering problems with old versions of \texttt{rlwrap}. 
\item \texttt{--noprompt}: \sollya prints a prompt symbol when
  connected on standard input to a pseudo-file that is a terminal. The
  option deactivates the prompt.
\item \texttt{--oldautoprint}: The behaviour of an undocumented
feature for displaying values has changed in \sollya from version 1.1
to version 2.0. The old feature is deprecated. If you wish to use it
nevertheless, use this deprecated option.
\item \texttt{--oldrlwrapcompatible}: This option is deprecated. It
  makes \sollya emit a non ANSI X3.64 compliant coloring escape
  sequence for making it compatible with versions of \texttt{rlwrap}
  that do not support the \texttt{-A} option. The option is considered
  a hack since it is known to garble the output of the tool under
  some particular circumstances.
\item \texttt{--warninfile[append] <file>}: Normally, \sollya emits
  warning and information messages together with all other displayed
  information on either standard output or standard error. This option
  allows all warning and information messages to get redirected to a
  file. The filename to be used must be given after the option.  When
  \texttt{--warninfile} is used, the existing content (if any) of the
  file is first removed before writing to the file. With
  \texttt{--warninfileappend}, the messages are appended to an
  existing file. Even if coloring is used for the displaying all other
  \sollya output, no coloring sequences are ever written to the
  file. Let us emphasize on the fact that any file of a unixoid system
  can be used for output, for instance also a named pipe. This allows
  for error messaging to be performed on a separate terminal. The use
  of this option is mutually exclusive with the
  \texttt{--warnonstderr} option.
\item \texttt{--warnonstderr}: Normally, \sollya prints warning and
  information messages on standard output, using a warning color 
  when coloring is activated. When this option is given, \sollya will 
  output all warning and information messages on standard error. Coloring
  will be used even on standard error, when activated. The use of 
  this option is mutually exclusive with the \texttt{--warninfile[append]} 
  option.
\end{itemize}


\section{Introduction}
\label{sec:introduction}
\sollya is an interactive tool for handling numerical functions and working with arbitrary precision. It can evaluate functions accurately, compute polynomial approximations of functions, automatically implement polynomials for use in math libraries, plot functions, compute infinity norms, etc. \sollya is also a full-featured script programming language with support for procedures~etc.

Let us begin this manual with an example. \sollya does not allow command line edition; since this may quickly become uncomfortable, we highly suggest to use the \rlwrap tool with \sollya:

\begin{center}\begin{minipage}{15cm}\begin{Verbatim}[frame=single]
~/% rlwrap -A sollya
>
\end{Verbatim}
\end{minipage}\end{center}

\sollya manipulates only functions in one variable. The first time that an unbound variable is used, this name is fixed. It will be used to refer to the free variable. For instance, try

\begin{center}\begin{minipage}{15cm}\begin{Verbatim}[frame=single]
> f = sin(x)/x;
> g = cos(y)-1;
Warning: the identifier "y" is neither assigned to, nor bound to a library funct
ion nor equal to the current free variable.
Will interpret "y" as "x".
> g;
cos(x) - 1
\end{Verbatim}
\end{minipage}\end{center}


Now, the name $x$ can only be used to refer to the free variable:

\begin{center}\begin{minipage}{15cm}\begin{Verbatim}[frame=single]
> x = 3;
Warning: the identifier "x" is already bound to the free variable, to a library 
function or to an external procedure.
The command will have no effect.
Warning: the last assignment will have no effect.
\end{Verbatim}
\end{minipage}\end{center}


If you really want to unbind $x$, you can use the \com{rename} command and change the name of the free variable:

\begin{center}\begin{minipage}{15cm}\begin{Verbatim}[frame=single]
> rename(x,y);
Information: the free variable has been renamed from "x" to "y".
> g;
cos(y) - 1
> x=3;
Warning: syntax error, unexpected IDENTIFIERTOKEN, expecting SEMICOLONTOKEN.
The last symbol read has been "x".
Will skip input until next semicolon after the unexpected token. May leak memory
.
> x;
Warning: the identifier "x" is neither assigned to, nor bound to a library funct
ion nor equal to the current free variable.
Will interpret "x" as "y".
y
\end{Verbatim}
\end{minipage}\end{center}


\sollya has a reserved keyword that can always be used to refer to the free variable. This keyword is \verb|_x_|. This is particularly useful in contexts when the name of the variable is not known: typically when refering to the free variable in a pattern matching or inside a procedure.

\begin{center}\begin{minipage}{15cm}\begin{Verbatim}[frame=single]
> f == sin(_x_)/_x_;
true
\end{Verbatim}
\end{minipage}\end{center}


As you have seen, you can name functions and easily work with them. The basic thing to do with a function is to evaluate it at some point:

\begin{center}\begin{minipage}{15cm}\begin{Verbatim}[frame=single]
> f(-2);
Warning: rounding has happened. The value displayed is a faithful rounding of th
e true result.
0.45464871341284084769800993295587242135112748572394
> evaluate(f,-2);
0.45464871341284084769800993295587242135112748572394
\end{Verbatim}
\end{minipage}\end{center}


The printed value is generally a faithful rounding of the exact value at the working precision (i.e. one of the two floating-point numbers enclosing the exact value). Internally \sollya represents numbers as floating-point numbers in arbitrary precision with radix~$2$: the fact that a faithful rounding is performed in binary does not imply much on the exactness of the digits displayed in decimal. The working precision is controlled by the global variable \com{prec}:

\begin{center}\begin{minipage}{15cm}\begin{Verbatim}[frame=single]
> prec;
165
> prec=200;
The precision has been set to 200 bits.
> prec;
200
> f(-2);
Warning: rounding has happened. The value displayed is a faithful rounding to 20
0 bits of the true result.
0.4546487134128408476980099329558724213511274857239451341894865
\end{Verbatim}
\end{minipage}\end{center}


Sometimes a faithful rounding cannot easily be computed. In such a case, a value is printed that was obtained using floating-point approximations without control on the final accuracy:

\begin{center}\begin{minipage}{15cm}\begin{Verbatim}[frame=single]
> log2(5)/log2(17) - log(5)/log(17);
Warning: rounding may have happened.
If there is rounding, the displayed value is *NOT* guaranteed to be a faithful r
ounding of the true result.
0
\end{Verbatim}
\end{minipage}\end{center}


The philosophy of \sollya is: \emph{Whenever something is not exact, print a warning}. This explains the warnings in the previous examples. If the result can be shown to be exact, there is no warning:

\begin{center}\begin{minipage}{15cm}\begin{Verbatim}[frame=single]
> sin(0);
0
\end{Verbatim}
\end{minipage}\end{center}


Let us finish this Section with a small complete example that shows a bit of what can be done with~\sollya:

% Warning: this file must be manually corrected: delete the extra ">"
\begin{center}\begin{minipage}{15cm}\begin{Verbatim}[frame=single]
> restart;
The tool has been restarted.
> prec=50;
The precision has been set to 50 bits.
> f=cos(2*exp(x));
> d=[-1/8;1/8];
> p=remez(f,2,d);
> derivativeZeros = dirtyfindzeros(diff(p-f),d);
> derivativeZeros = inf(d).:derivativeZeros:.sup(d);
> max=0;
> for t in derivativeZeros do {
     r = evaluate(abs(p-f), t);
     if r > max then { max=r; argmax=t; };
  };
> print("The infinite norm of", p-f, "is", max, "and is reached at", argmax);
The infinite norm of (-0.416265572875373) + x * ((-1.798067209218835) + x * (-3.
89710727747639e-2)) - cos(2 * exp(x)) is 8.630659443624325e-4 and is reached at 
-5.801672331417684e-2
\end{Verbatim}
\end{minipage}\end{center}


In this example, we define a function $f$, an interval $d$ and we compute the best degree-2 polynomial approximation of $f$ on $d$ with respect to the infinity norm. In other words, $\max_{x \in d} \{|p(x)-f(x)|\}$ is minimal amongst polynomials with degree not greater than $2$. Then, we compute the list of the zeros of the derivative of $p-f$ and add the bounds of $d$ to this list. Finally, we evaluate $|p-f|$ for each point in the list and store the maximum and the point where it is reached. We conclude by printing the result in a formatted way.

Let us mention as a sidenote that you do not really need to use such a script for computing an infinity norm; as we will see, the command \com{dirtyinfnorm} does this for you.

\section{General principles}\label{sec:general_principles}
The first purpose of \sollya is to help people using numerical functions and numerical algorithms in a safe way. It is first designed to be used interactively but it can also be used in scripts\footnote{Remark: some of the behaviors of \sollya slightly change when it is used in scripts. For example, no prompt is printed.}.

One of the particularities of \sollya is to work with multi-precision arithmetic (it uses the \texttt{MPFR} library). For safety purposes, \sollya knows how to use interval arithmetic. It uses interval arithmetic to produce tight and safe results with the precision required by the user.

The general philosophy of \sollya is: \emph{When you can perform a computation exactly and sufficiently quickly, do it; when you cannot, do not, unless you have been explicitly asked for.}

The precision of the tool is set by the global variable \key{prec}. In general, the variable \key{prec} determines the precision of the outputs of commands: more precisely, the command will internally determine how much precision should be used during the computations in order to ensure that the output is a faithfully rounded result with \key{prec} bits.

For decidability and efficiency reasons, this general principle cannot be applied every time, so be careful. Moreover certain commands are known to be unsafe: they give in general excellent results and give almost \key{prec} correct bits in output for everyday examples. However they are merely based on heuristics and should not be used when the result must be safe. See the documentation of each command to know precisely how confident you can be with their result.

A second principle (that comes together with the first one) is the following one: \emph{When a computation leads to inexact results, inform the user with a warning}. This can be quite irritating in some circumstances: in particular if you are using \sollya within other scripts. The global variable \key{verbosity} lets you change the level of verbosity of \sollya. When the variable is set to $0$, \sollya becomes completely silent on standard output and prints only very important messages on standard error. Increase \key{verbosity} if you want more information about what \sollya is doing. Please keep in mind that when you affect a value to a global variable, a message is always printed even if \com{verbosity} is set to $0$. In order to silently affect a global variable, use~\texttt{!}:

\begin{center}\begin{minipage}{15cm}\begin{Verbatim}[frame=single]
> prec=30;
The precision has been set to 30 bits.
> prec=30!;
>  
\end{Verbatim}
\end{minipage}\end{center}


For conviviality reasons, values are displayed in decimal by default. This lets a normal human being understand the numbers they manipulate. But since constants are internally represented in binary, this causes permanent conversions that are sources of roundings. Thus you are loosing in accuracy and \sollya is always complaining about inexact results. If you just want to store or communicate your results (to another tools for instance) you can use bit-exact representations available in \sollya. The global variable \key{display} defines the way constants are displayed. Here is an example of the five available modes:

\begin{center}\begin{minipage}{15cm}\begin{Verbatim}[frame=single]
> prec=30!;
> a = 17.25;
> display=decimal;
Display mode is decimal numbers.
> a;
1.725e1
> display=binary;
Display mode is binary numbers.
> a;
1.000101_2 * 2^(4)
> display=powers;
Display mode is dyadic numbers in integer-power-of-2 notation.
> a;
69 * 2^(-2)
> display=dyadic;
Display mode is dyadic numbers.
> a;
69b-2
> display=hexadecimal;
Display mode is hexadecimal numbers.
> a;
0x1.14p4
\end{Verbatim}
\end{minipage}\end{center}


Please keep in mind that it is possible to maintain the general verbosity level at
some higher setting while deactivating all warnings on roundings. This
feature is controlled using the \key{roundingwarnings} global
variable. It may be set to \key{on} or \key{off}. By default, the
warnings are activated (\key{roundingwarnings = on}) when \sollya is
connected on standard input to a pseudo-file that represents a
terminal. They are deactivated when \sollya is connected on standard
input to a real file. See \ref{labroundingwarnings} for further details; the behavior is
illustrated with examples there.

As always, the symbol \texttt{e} means $\times 10^\square $. The same way the symbol \texttt{b} means  $\times 2^\square $. The symbol \texttt{p} means $\times 16^\square$ and is used only with the \texttt{0x} prefix. The prefix \texttt{0x} indicates that the digits of the following number until 
a symbol \texttt{p} or white-space are hexadecimal. The suffix \texttt{\_2} indicates to \sollya that the previous number has been written in binary. \sollya can parse these notations even if you are not in the corresponding \key{display} mode, so you can always use them.

You can also use memory-dump hexadecimal notation frequently used to represent IEEE 754 \texttt{double} and \texttt{single} precision numbers. Since this notation does not allow for exactly representing numbers with arbitrary precision, there is no corresponding \key{display} mode. However, the commands \com{printdouble} respectively \com{printsingle} round the value to the nearest \texttt{double} respectively \texttt{single}. The number is then printed in hexadecimal as the integer number corresponding to the memory representation of the IEEE 754 \texttt{double} or \texttt{single} number:

\begin{center}\begin{minipage}{15cm}\begin{Verbatim}[frame=single]
> printhexa(a);
0x4031400000000000
> printfloat(a);
0x418a0000
\end{Verbatim}
\end{minipage}\end{center}


\sollya can parse these memory-dump hexadecimal notation back in any
\key{display} mode. The difference of this memory-dump
notation with the hexadecimal notation (as defined above) is made by
the presence or absence of a \texttt{p} indicator.

\section{Variables}\label{variables}

As already explained, \sollya can manipulate variate functional
expressions in one variable. These expressions contain a unique free variable the name
of which is fixed by its first usage in an expression that is not a
left-hand-side of an assignment. This global and unique free variable is 
a variable in the mathematical sense of the term. 

\sollya also provides variables in the sense programming languages
give to the term.  These variables, which must be different in their
name from the global free variable, may be global or declared and
attached to a block of statements, i.e. a begin-end-block. These
programming language variables may hold any object of the \sollya
language, as for example functional expressions, strings, intervals,
constant values, procedures, external functions and procedures, etc.

Global variables need not to be declared. They start existing,
i.e. can be correctly used in expressions that are not left-hand-sides
of assignments, when they are assigned a value in an assignment. Since
they are global, this kind of variables is recommended only for small
\sollya scripts.  Larger scripts with code reuse should use
declared variables in order to avoid name clashes for example in loop
variables.

Declared variables are attached to a begin-end-block. The block
structure builds scopes for declared variables. Declared variables in
inner scopes shadow (global and declared) variables of outer
scopes. The global free variable, i.e. the mathematical variable for
variate functional expressions in one variable, cannot be shadowed. Variables are
declared using the \key{var} keyword. See section \ref{labvar} for details
on its usage and semantic.

The following code examples illustrate the use of variables.


\begin{center}\begin{minipage}{15cm}\begin{Verbatim}[frame=single]
> f = exp(x);
> f;
exp(x)
> a = "Hello world";
> a;
Hello world
> b = 5;
> f(b);
Warning: rounding has happened. The value displayed is a faithful rounding of th
e true result.
1.48413159102576603421115580040552279623487667593878e2
> {var b; b = 4; f(b); };
Warning: rounding has happened. The value displayed is a faithful rounding of th
e true result.
5.45981500331442390781102612028608784027907370386137e1
> {var x; x = 3; };
Warning: the identifier "x" is already bound to the current free variable.
It cannot be declared as a local variable. The declaration of "x" will have no e
ffect.
Warning: the identifier "x" is already bound to the free variable, to a library 
function, library constant or to an external procedure.
The command will have no effect.
Warning: the last assignment will have no effect.
> {var a, b; a=5; b=3; {var a; var b; b = true; a = 1; a; b;}; a; b; };
1
true
5
3
> a;
Hello world
\end{Verbatim}
\end{minipage}\end{center}


Let us state that a variable identifier, just as every identifier in
\sollya, contains at least one character, starts with a ASCII letter
and continues with ASCII letters or numerical digits.



\section{Data types}\label{sec:data_types}
\sollya has a (very) basic system of types. If you try to perform an illicit operation (such as adding a number and a string, for instance), you will get a typing error. Let us see the available data types.

\subsection{Booleans}
There are two special values \key{true} and \key{false}. Boolean expressions can be constructed using the boolean connectors \key{\&\&} (and), \key{||} (or), \key{!} (not), and comparisons.

The comparison operators \key{<}, \key{<=}, \key{>} and \key{>=} can only be used between two numbers or constant expressions.

The comparison operators \key{==} and \key{!=} are polymorphic. You can use them to compare any two objects, like two strings, two intervals, etc. As a matter of fact, polymorphism is allowed on both sides: it is possible to compare objects of different type. Such objects of different type, as they can never be syntactically equal, will always compare unequal (see exception for \key{error}, section \ref{laberror}) and never equal. It is important to remember that testing the equality between two functions will return \key{true} if and only if the expression trees representing the two functions are exactly the same. See \ref{laberror} for an exception concerning the special object \key{error}. Example:

\begin{center}\begin{minipage}{15cm}\begin{Verbatim}[frame=single]
> 1+x==1+x;
true
\end{Verbatim}
\end{minipage}\end{center}


\subsection{Numbers} \label{sec:numbers}
\sollya represents numbers as binary multi-precision floating-point values. For integer values and values in dyadic, binary, hexadecimal or memory dump notation, it 
automatically uses a precision needed for representing the value exactly (unless this behaviour is overridden using the syntax given below). Additionally, automatic precision adaption takes place for all 
integer values (even in decimal notation) written without the exponent sign \texttt{e} or with the exponent sign \texttt{e} and an exponent sufficiently 
small that they are less than $10^{999}$. Otherwise the values are represented with the current precision \com{prec}. When a number must be rounded, it is rounded to the precision \com{prec} before the expression get evaluated:

\begin{center}\begin{minipage}{15cm}\begin{Verbatim}[frame=single]
> prec=12!;
> 4097.1;
Warning: Rounding occurred when converting the constant "4097.1" to floating-poi
nt with 12 bits.
If safe computation is needed, try to increase the precision.
4098
> 4098.1;
Warning: Rounding occurred when converting the constant "4098.1" to floating-poi
nt with 12 bits.
If safe computation is needed, try to increase the precision.
4098
> 4097.1+1;
Warning: Rounding occurred when converting the constant "4097.1" to floating-poi
nt with 12 bits.
If safe computation is needed, try to increase the precision.
4099
\end{Verbatim}
\end{minipage}\end{center}


As a matter of fact, each variable has its own precision that corresponds to its intrinsic precision or, if it cannot be represented, to the value of \com{prec} when the variable was set. Thus you can work with variables having a precision higher than the current precision.

The same way, if you define a function that refers to some constant, this constant is stored in the function with the current precision and will keep this value in the future, even if \com{prec} becomes smaller.

If you define a function that refers to some variable, the precision of the variable is kept, independently of the current precision:

\begin{center}\begin{minipage}{15cm}\begin{Verbatim}[frame=single]
> prec = 50!;
> a = 4097.1;
Warning: Rounding occurred when converting the constant "4097.1" to floating-poi
nt with 50 bits.
If safe computation is needed, try to increase the precision.
> prec = 12!;
> f = x + a;
> g = x + 4097.1;
Warning: Rounding occurred when converting the constant "4097.1" to floating-poi
nt with 12 bits.
If safe computation is needed, try to increase the precision.
> prec = 120;
The precision has been set to 120 bits.
> f;
4.097099999999998544808477163314819335e3 + x
> g;
4098 + x
\end{Verbatim}
\end{minipage}\end{center}


In some rare cases, it is necessary to read in decimal constants with
a particular precision being used in the conversion to the binary
floating-point format, which \sollya uses. Setting \key{prec} to that
precision may prove to be an insufficient means for doing so, for
example when several different precisions have to be used in one
expression. For these rare cases, \sollya provides the following
syntax: decimal constants may be written {\tt
  \%}\emph{precision}{\tt\%}\emph{constant}, where \emph{precision} is
a constant integer, written in decimal, and \emph{constant} is the
decimal constant. \sollya will convert the constant \emph{constant}
with precision \emph{precision}, regardless of the global variable
\key{prec} and regardless if \emph{constant} is an integer or would
otherwise be representable.

\begin{center}\begin{minipage}{15cm}\begin{Verbatim}[frame=single]
> prec = 24;
The precision has been set to 24 bits.
> a = 0.1;
Warning: Rounding occurred when converting the constant "0.1" to floating-point 
with 24 bits.
If safe computation is needed, try to increase the precision.
> b = 33554432;
> prec = 64;
The precision has been set to 64 bits.
> display = binary;
Display mode is binary numbers.
> a;
1.10011001100110011001101_2 * 2^(-4)
> 0.1;
Warning: Rounding occurred when converting the constant "0.1" to floating-point 
with 64 bits.
If safe computation is needed, try to increase the precision.
1.100110011001100110011001100110011001100110011001100110011001101_2 * 2^(-4)
> %24%0.1;
> c = 33554432;
> b;
> c;
> %24%33554432;
> 
> 
\end{Verbatim}
\end{minipage}\end{center}


\sollya is an environment that uses floating-point arithmetic. The
IEEE 754-2008 standard on floating-point arithmetic does not only
define floating-point numbers that represent real numbers but also
floating-point data representing infinities and Not-a-Numbers (NaNs).
\sollya also supports infinities and NaNs in the spirit of the IEEE
754-2008 standard without taking the standard's choices literally. 

\begin{itemize}
\item Signed infinities are available through the \sollya objects
\texttt{infty, -infty, @Inf@} and \texttt{-@Inf@}.
\item Not-a-Numbers are supported through the \sollya objects
\texttt{NaN} and \texttt{@NaN@}. \sollya does not have support for NaN
payloads, signaling or quiet NaNs or signs of NaNs. Signaling NaNs
are supported on input for single and double precision memory
notation (see section \ref{sec:general_principles}). However, they
immediately get converted to plain \sollya NaNs.
\end{itemize}

The evaluation of an expression involving a NaN or the evaluation of a
function at a point being NaN always results in a NaN. 

Infinities are considered to be the limits of expressions tending to
infinity. They are supported as bounds of intervals in some
cases. However, particular commands might prohibit their use even
though there might be a mathematical meaning attached to such
expressions. For example, while \sollya will evaluate expressions such
as $\lim\limits_{x \rightarrow -\infty} e^x$, expressed e.g. through
\texttt{evaluate(exp(x),[-infty;0])}, it will not accept to compute
the (finite) value of
$$\int\limits_{-\infty}^0 e^x \,\mbox{d}x.$$

The following examples give an idea of what can be done with \sollya
infinities and NaNs. Here is what can be done with infinities:
\begin{center}\begin{minipage}{15cm}\begin{Verbatim}[frame=single]
> f = exp(x) + 5;
> f(-infty);
5
> evaluate(f,[-infty;infty]);
[5;@Inf@]
> f(infty);
Warning: the given expression is undefined or numerically unstable.
@NaN@
> [-infty;5] * [3;4];
[-@Inf@;20]
> -infty < 5;
true
> log(0);
Warning: the given expression is undefined or numerically unstable.
@NaN@
> [log(0);17];
Warning: the given expression is not a constant but an expression to evaluate an
d
a faithful evaluation is not possible.
Will use a plain floating-point evaluation, which might yield a completely wrong
 value.
Warning: inclusion property is satisfied but the diameter may be greater than th
e least possible.
[-@Inf@;17]
> 
\end{Verbatim}
\end{minipage}\end{center}

And the following example illustrates NaN behavior.
\begin{center}\begin{minipage}{15cm}\begin{Verbatim}[frame=single]
> 3/0;
Warning: the given expression is undefined or numerically unstable.
@NaN@
> (-3)/0;
Warning: the given expression is undefined or numerically unstable.
@NaN@
> infty/infty;
Warning: the given expression is undefined or numerically unstable.
@NaN@
> infty + infty;
Warning: the given expression is undefined or numerically unstable.
@Inf@
> infty - infty;
Warning: the given expression is undefined or numerically unstable.
@NaN@
> f = exp(x) + 5;
> f(NaN);
@NaN@
> NaN == 5;
false
> NaN == NaN;
false
> NaN != NaN;
false
> X = "Vive la Republique!";
> !(X == X);
false
> X = 5;
> !(X == X);
false
> X = NaN;
> !(X == X);
true
> 
\end{Verbatim}
\end{minipage}\end{center}


\subsection{Rational numbers and rational arithmetic}\label{sec:rationalmode}

The \sollya tool is mainly based on floating-point arithmetic:
wherever possible, floating-point algorithms, including algorithms
using interval arithmetic, are used to produce approximate but safe
results. For some particular cases, floating-point arithmetic is not
sufficient: some algorithms just require natural and rational numbers
to be handled exactly. More importantly, for these applications, it is
required that rational numbers be displayed as such.

\sollya implements a particular mode that offers a lightweight support
for rational arithmetic. When needed, it can be enabled by assigning
\com{on} to the global variable \com{rationalmode}. It is disabled by
assigning \com{off}; the default is \com{off}.

When the mode for rational arithmetic is enabled, \sollya's behavior
will change as follows:
\begin{itemize}
\item When a constant expression is given at the \sollya prompt,
  \sollya will first try to simplify the expression to a rational
  number. If such an evaluation to a rational number is possible,
  \sollya will display that number as an integer or a fraction of two
  integers. Only if \sollya is not able to simplify the constant
  expression to a rational number, it will launch the default behavior
  of evaluating constant expressions to floating-point numbers that
  are generally faithful roundings of the expressions.
\item When the global mode \com{autosimplify} is \com{on}, which is
  the default, \sollya will additionally use rational arithmetic while
  trying to simplify expressions given in argument of commands. 
\end{itemize}

Even when \com{rationalmode} is \com{on}, \sollya will not be able to
exhibit integer ratios between transcendental quantities. For example,
\sollya will not display $\frac{1}{6}$ for $\arcsin\left(
\frac{1}{2} \right) / \pi$ but $0.16666\dots$. \sollya's evaluator
for rational arithmetic is only able to simplify rational expressions
based on addition, subtraction, multiplication, division, negation,
perfect squares (for square root) and integer powers.

The following example illustrates what can and what cannot be done
with \sollya's mode for rational arithmetic: 

\begin{center}\begin{minipage}{15cm}\begin{Verbatim}[frame=single]
> 1/3 - 1/7;
Warning: rounding has happened. The value displayed is a faithful rounding of th
e true result.
0.19047619047619047619047619047619047619047619047619
> rationalmode = on;
Rational mode has been activated.
> 1/3 - 1/7;
4 / 21
> (2 + 1/7)^2 + (6/7)^2 + 2 * (2 + 1/7) * 6/7;
9
> rationalmode = off;
Rational mode has been deactivated.
> (2 + 1/7)^2 + (6/7)^2 + 2 * (2 + 1/7) * 6/7;
Warning: rounding has happened. The value displayed is a faithful rounding of th
e true result.
9
> rationalmode = on;
Rational mode has been activated.
> asin(1)/pi;
Warning: rounding has happened. The value displayed is a faithful rounding of th
e true result.
0.5
> sin(1/6 * pi);
Warning: rounding has happened. The value displayed is a faithful rounding of th
e true result.
0.5
> exp(1/7 - 3/21) / 7;
1 / 7
> rationalmode = off;
Rational mode has been deactivated.
> exp(1/7 - 3/21) / 7;
Warning: rounding has happened. The value displayed is a faithful rounding of th
e true result.
0.142857142857142857142857142857142857142857142857145
> print(1/7 - 3/21);
1 / 7 - 3 / 21
> rationalmode = on;
Rational mode has been activated.
> print(1/7 - 3/21);
0
\end{Verbatim}
\end{minipage}\end{center}


\subsection{Intervals and interval arithmetic}

\sollya can manipulate intervals that are closed subsets of the real
numbers. Several ways of defining intervals exist in \sollya. There is the
most common way where intervals are composed of two numbers or
constant expressions representing the lower and the upper bound. These
values are separated either by commas or semi-colons. Interval bound 
evaluation is performed in a way that ensures the inclusion property:
all points in the original, unevaluated interval will be contained in
the interval with its bounds evaluated to floating-point numbers. 

\begin{center}\begin{minipage}{15cm}\begin{Verbatim}[frame=single]
> d=[1;2];
> d2=[1,1+1];
> d==d2;
true
> prec=12!;
> 8095.1;
Warning: Rounding occurred when converting the constant "8095.1" to floating-poi
nt with 12 bits.
If safe computation is needed, try to increase the precision.
8096
> [8095.1; 8096.1];
Warning: Rounding occurred when converting the constant "8095.1" to floating-poi
nt with 12 bits.
If safe computation is needed, try to increase the precision.
Warning: Rounding occurred when converting the constant "8096.1" to floating-poi
nt with 12 bits.
If safe computation is needed, try to increase the precision.
[8094;8098]
\end{Verbatim}
\end{minipage}\end{center}


\sollya has a mode for printing intervals that are that thin that
their bounds have a number of decimal digits in common when
printed. That mode is called \com{midpointmode}; see below for an
introduction and section \ref{labmidpointmode} for details. As \sollya
must be able to parse back its own output, a syntax is provided to
input intervals in midpoint~mode. However, please pay attention to the fact that the
notation used in midpoint~mode generally increases the width of
intervals: hence when an interval is displayed in midpoint~mode and
read again, the resulting interval may be wider than the original
interval.

\begin{center}\begin{minipage}{15cm}\begin{Verbatim}[frame=single]
> midpointmode = on!;
> [1.725e4;1.75e4];
0.17~2/5~e5
> 1.7~25/5~e4;
0.17~2/5~e5
> midpointmode = off!;
> 1.7~25/5~e4;
[17250;17500]
\end{Verbatim}
\end{minipage}\end{center}


In some cases, intervals become infinitely thin in theory, in which
case one tends to think of point intervals even if their
floating-point representation is not infinitely thin. \sollya provides
a very covenient way for input of such point intervals. Instead of
writing \texttt{[a;a]}, it is possible to just write
\texttt{[a]}. \sollya will expand the notation while making sure that
the inclusion property is satisfied:

\begin{center}\begin{minipage}{15cm}\begin{Verbatim}[frame=single]
> [3];
[3;3]
> [1/7];
Warning: the given expression is not a constant but an expression to evaluate. A
 faithful evaluation will be used.
[0.14285713;0.14285716]
> [exp(8)];
Warning: the given expression is not a constant but an expression to evaluate. A
 faithful evaluation will be used.
[2.980957e3;2.9809589e3]
\end{Verbatim}
\end{minipage}\end{center}


When the mode \com{midpointmode} is set to \com{on} (see
\ref{labmidpointmode}), \sollya will display intervals that are
provably reduced to one point in this extended interval syntax. It
will use \com{midpointmode} syntax for intervals that are sufficiently
thin but not reduced to one point (see section \ref{labmidpointmode}
for details):

\begin{center}\begin{minipage}{15cm}\begin{Verbatim}[frame=single]
> midpointmode = off;
Midpoint mode has been deactivated.
> [17;17];
[17;17]
> [exp(pi);exp(pi)];
Warning: the given expression is not a constant but an expression to evaluate. A
 faithful evaluation to 170 bits will be used.
[2.31406926327792690057290863679485473802661062426e1;2.3140692632779269005729086
3679485473802661062426008e1]
> midpointmode = on;
Midpoint mode has been activated.
> [17;17];
[17]
> [exp(pi);exp(pi)];
Warning: the given expression is not a constant but an expression to evaluate. A
 faithful evaluation to 170 bits will be used.
0.231406926327792690057290863679485473802661062426~0/1~e2
> 
\end{Verbatim}
\end{minipage}\end{center}


\sollya intervals are internally represented with floating-point
numbers as bounds; rational numbers are not supported here. If bounds
are defined by constant expressions, these are evaluated to
floating-point numbers using the current precision. Numbers or
variables containing numbers keep their precision for the interval
bounds.

Constant expressions get evaluated to floating-point values
immediately; this includes $\pi$ and rational numbers, even when
\com{rationalmode} is \com{on} (see section \ref{sec:rationalmode} for
this mode).

\begin{center}\begin{minipage}{15cm}\begin{Verbatim}[frame=single]
> prec = 300!;
> a = 4097.1;
Warning: Rounding occurred when converting the constant "4097.1" to floating-poi
nt with 300 bits.
If safe computation is needed, try to increase the precision.
> prec = 12!;
> d = [4097.1; a];
Warning: Rounding occurred when converting the constant "4097.1" to floating-poi
nt with 12 bits.
If safe computation is needed, try to increase the precision.
Warning: the bounds of the given range are in wrong order. Will reverse them.
> prec = 300!;
> d;
[4.0971e3;4098]
> prec = 30!;
> [-pi;pi];
[-3.141592659;3.141592659]
\end{Verbatim}
\end{minipage}\end{center}


You can get the upper-bound (respectively the lower-bound) of an interval with the command \com{sup} (respectively \com{inf}). The middle of the interval can be computed with the command \com{mid}. Let us also mention that these commands can also be used on numbers (in that case, the number is interpreted as an interval containing only one single point. In that case the commands \com{inf}, \com{mid} and \com{sup} are just the identity):

\begin{center}\begin{minipage}{15cm}\begin{Verbatim}[frame=single]
> d=[1;3];
> inf(d);
1
> mid(d);
2
> sup(4);
4
\end{Verbatim}
\end{minipage}\end{center}


Let us mention that the \com{mid} operator never provokes a
rounding. It is rewritten as an unevaluated expression in terms of
\com{inf} and \com{sup}.

\sollya permits intervals to also have non-real bounds, such as
infinities or NaNs. When evaluating certain expressions, in particular
given as interval bounds, \sollya will itself generate intervals
containing infinities or NaNs. When evaluation yields an interval with
a NaN bound, the given expression is most likely undefined or
numerically unstable. Such results should not be trusted; a warning is
displayed.

While computations on intervals with bounds being NaN will always
fail, \sollya will try to interpret infinities in the common way as
limits. However, this is not guaranteed to work, even if it is
guaranteed that no unsafe results will be produced. See also section
\ref{sec:numbers} for more detail on infinities in \sollya. The behavior of interval arithmetic on intervals containing infinities or NaNs is subject to debate; moreover, there is no complete consensus on what should be the result of the evaluation of a function $f$ over an interval $I$ containing points where $f$ is not defined. \sollya has its own philosophy regarding these questions. This philosophy is explained in Appendix~\ref{IntervalArithmeticPhilopshy} at the end of this document.

\begin{center}\begin{minipage}{15cm}\begin{Verbatim}[frame=single]
> evaluate(exp(x),[-infty;0]);
[0;1]
> dirtyinfnorm(exp(x),[-infty;0]);
Warning: a bound of the interval is infinite or NaN.
This command cannot handle such intervals.
@NaN@
> 
> f = log(x);
> [f(0); f(1)];
Warning: the given expression is not a constant but an expression to evaluate an
d
a faithful evaluation is not possible.
Will use a plain floating-point evaluation, which might yield a completely wrong
 value.
Warning: inclusion property is satisfied but the diameter may be greater than th
e least possible.
[-@Inf@;0]
> 
\end{Verbatim}
\end{minipage}\end{center}


\sollya internally uses interval arithmetic extensively to provide
safe answers. In order to provide for algorithms written in the
\sollya language being able to use interval arithmetic, \sollya offers
native support of interval arithmetic. Intervals can be added,
subtracted, multiplied, divided, raised to powers, for short, given in
argument to any \sollya function. The tool will apply the rules of
interval arithmetic in order to compute output intervals that safely
encompass the hull of the image of the function on the given interval:

\begin{center}\begin{minipage}{15cm}\begin{Verbatim}[frame=single]
> [1;2] + [3;4];
[4;6]
> [1;2] * [3;4];
[3;8]
> sqrt([9;25]);
[3;5]
> exp(sin([10;100]));
[0.36787942;2.7182819]
\end{Verbatim}
\end{minipage}\end{center}


When such expressions involving intervals are given, \sollya will
follow the rules of interval arithmetic in precision \com{prec} for
immediately evaluating them to interval enclosures. While \sollya's
evaluator always guarantees the inclusion property, it also applies
some optimisations in some cases in order to make the image interval
as thin as possible. For example, \sollya will use a Taylor expansion
based evaluation if a composed function, call it $f$, is applied to an
interval. In other words, in this case \sollya will behave as if the
\com{evaluate} command (see section \ref{labevaluate}) were implicitly
used. In most cases, the result will be different from the one obtained
by replacing all occurences of the free variable of a function by the 
interval the function is to be evaluated on:

\begin{center}\begin{minipage}{15cm}\begin{Verbatim}[frame=single]
> f = x - sin(x);
> [-1b-10;1b-10] - sin([-1b-10;1b-10]);
[-1.95312484477957829894e-3;1.95312484477957829894e-3]
> f([-1b-10;1b-10]);
[-1.55220421701117626897e-10;1.55220421701117626897e-10]
> evaluate(f,[-1b-10;1b-10]);
[-1.55220421701117626897e-10;1.55220421701117626897e-10]
\end{Verbatim}
\end{minipage}\end{center}


\subsection{Functions}
\sollya knows only about functions with one single variable. The first time in a session that an unbound name is used (without being assigned) it determines the name used to refer to the free variable.

The basic functions available in \sollya are the following:
\begin{itemize}
\item \com{+}, \com{-}, \com{*}, \com{/}, \com{\^{}}
\item \com{sqrt}
\item \com{abs}
\item \com{sin}, \com{cos}, \com{tan}, \com{sinh}, \com{cosh}, \com{tanh}
\item \com{asin}, \com{acos}, \com{atan}, \com{asinh}, \com{acosh}, \com{atanh}
\item \com{exp}, \com{expm1} (defined as $\mathrm{expm1}(x) = \exp(x)-1$)
\item \com{log} (natural logarithm), \com{log2} (binary logarithm), \com{log10} (decimal logarithm), \com{log1p} (defined as $\mathrm{log1p}(x) = \log(1+x)$)
\item \com{erf}, \com{erfc}
\item \com{halfprecision}, \com{single}, \com{double}, \com{doubleextended}, \com{doubledouble}, \com{quad}, \com{tripledouble} (see sections \ref{labhalfprecision}, \ref{labsingle}, \ref{labdouble}, \ref{labdoubleextended}, \ref{labdoubledouble}, \ref{labquad} and \ref{labtripledouble})
\item \com{HP}, \com{SG}, \com{D}, \com{DE}, \com{DD}, \com{QD}, \com{TD} (see sections \ref{labhalfprecision}, \ref{labsingle}, \ref{labdouble}, \ref{labdoubleextended}, \ref{labdoubledouble}, \ref{labquad} and \ref{labtripledouble})
\item \com{floor}, \com{ceil}, \com{nearestint}.
\end{itemize}

The constant $\pi$ is available through the keyword \key{pi} as a $0$-ary function: 

\begin{center}\begin{minipage}{15cm}\begin{Verbatim}[frame=single]
> display=binary!;
> prec=12!;
> a=pi;
> a;
Warning: rounding has happened. The value displayed is a faithful rounding to 12
 bits of the true result.
1.10010010001_2 * 2^(1)
> prec=30!;
> a;
Warning: rounding has happened. The value displayed is a faithful rounding to 30
 bits of the true result.
1.10010010000111111011010101001_2 * 2^(1)
\end{Verbatim}
\end{minipage}\end{center}


The reader may wish to see Sections \ref{lablibrary} and \ref{labfunction} for ways of dynamically adding other base functions to Sollya.

\subsection{Strings}
Anything written between quotes is interpreted as a string. The infix operator \com{@} concatenates two strings. To get the length of a string, use the \com{length} function. You can access the $i$-th character of a string using brackets (see the example below). There is no character type in \sollya: the $i$-th character of a string is returned as a string itself.

\begin{center}\begin{minipage}{15cm}\begin{Verbatim}[frame=single]
> s1 = "Hello "; s2 = "World!";
> s = s1@s2;
> length(s);
12
> s[0];
H
> s[11];
!
\end{Verbatim}
\end{minipage}\end{center}


Strings may contain the following escape sequences:
\texttt{$\backslash\backslash$}, \texttt{$\backslash$\"},
\texttt{$\backslash$?}, \texttt{$\backslash$\'},
\texttt{$\backslash$n}, \texttt{$\backslash$t},
\texttt{$\backslash$a}, \texttt{$\backslash$b},
\texttt{$\backslash$f}, \texttt{$\backslash$r},
\texttt{$\backslash$v}, \texttt{$\backslash$x}[hexadecimal number] and
\texttt{$\backslash$}[octal number]. Refer to the C99 standard for their
meaning.

\subsection{Particular values}
\sollya knows about some particular values. These values do not really have a type. They can be stored in variables and in lists. A (possibly not exhaustive) list of such values is the following one:

\begin{itemize}
\item \com{on}, \com{off} (see sections \ref{labon} and \ref{laboff})
\item \com{dyadic}, \com{powers}, \com{binary}, \com{decimal}, \com{hexadecimal} (see sections \ref{labdyadic}, \ref{labpowers}, \ref{labbinary}, \ref{labdecimal} and \ref{labhexadecimal})
\item \com{file}, \com{postscript}, \com{postscriptfile} (see sections \ref{labfile}, \ref{labpostscript} and \ref{labpostscriptfile})
\item \com{RU}, \com{RD}, \com{RN}, \com{RZ} (see sections \ref{labru}, \ref{labrd}, \ref{labrn} and \ref{labrz})
\item \com{absolute}, \com{relative} (see sections \ref{lababsolute} and \ref{labrelative})
\item \com{floating}, \com{fixed} (see sections \ref{labfloating} and \ref{labfixed})
\item \com{halfprecision}, \com{single}, \com{double}, \com{doubleextended}, \com{doubledouble}, \com{quad}, \com{tripledouble} (see sections \ref{labhalfprecision}, \ref{labsingle}, \ref{labdouble}, \ref{labdoubleextended}, \ref{labdoubledouble}, \ref{labquad} and \ref{labtripledouble})
\item \com{HP}, \com{SG}, \com{D}, \com{DE}, \com{DD}, \com{QD}, \com{TD} (see sections \ref{labhalfprecision}, \ref{labsingle}, \ref{labdouble}, \ref{labdoubleextended}, \ref{labdoubledouble}, \ref{labquad} and \ref{labtripledouble})
\item \com{perturb} (see section \ref{labperturb})
\item \com{honorcoeffprec} (see section \ref{labhonorcoeffprec})
\item \com{default} (see section \ref{labdefault})
\item \com{error} (see section \ref{laberror})
\item \com{void} (see section \ref{labvoid})
\end{itemize}

\subsection{Lists}
Objects can be grouped into lists. A list can contain elements with different types. As for strings, you can concatenate two lists with \com{@}. The function \com{length} also gives the length of a list.

You can prepend an element to a list using \com{.:} and you can append an element to a list using \com{:.}\\ The following example illustrates some features:

\begin{center}\begin{minipage}{15cm}\begin{Verbatim}[frame=single]
> l = [| "foo" |];
> l = l:.1;
> l = "bar".:l;
> l;
[|"bar", "foo", 1|]
> l[1];
foo
> l@l;
[|"bar", "foo", 1, "bar", "foo", 1|]
\end{Verbatim}
\end{minipage}\end{center}


Lists can be considered arrays and elements of lists can be
referenced using brackets. Possible indices start at $0$. The
following example illustrates this point:

\begin{center}\begin{minipage}{15cm}\begin{Verbatim}[frame=single]
> L = [|1,2,3,4,5|];
> L;
[|1, 2, 3, 4, 5|]
> L[3];
4
\end{Verbatim}
\end{minipage}\end{center}


Please be aware of the fact that the complexity for accessing an
element of the list using indices is~$\mathcal{O}(n)$, where $n$ is the length of the whole list.

Lists may contain ellipses indicated by \texttt{,...,} between
elements that are constant and evaluate to integers that are
incrementally ordered. \sollya translates such ellipses to the full
list upon evaluation. The use of ellipses between elements that are not
constants is not allowed. This feature is provided for ease of
programming; remark that the complexity for expanding such lists is
high. For illustration, see the following example:

\begin{center}\begin{minipage}{15cm}\begin{Verbatim}[frame=single]
> [|1,...,5|];
[|1, 2, 3, 4, 5|]
> [|-5,...,5|];
[|-5, -4, -3, -2, -1, 0, 1, 2, 3, 4, 5|]
> [|3,...,1|];
Warning: at least one of the given expressions or a subexpression is not correct
ly typed
or its evaluation has failed because of some error on a side-effect.
error
> [|true,...,false|];
Warning: at least one of the given expressions or a subexpression is not correct
ly typed
or its evaluation has failed because of some error on a side-effect.
error
\end{Verbatim}
\end{minipage}\end{center}


Lists may be continued to infinity by means of the \texttt{...}
indicator after the last element given. At least one element must
explicitly be given. If the last element given is a constant
expression that evaluates to an integer, the list is considered as
continued to infinity by all integers greater than that last
element. If the last element is another object, the list is considered
as continued to infinity by re-duplicating this last element. Let us remark
that bracket notation is supported for such end-elliptic lists even
for implicitly given elements. However, evaluation complexity is
high. Combinations of ellipses inside a list and in its end are
possible. The usage of lists described here is best illustrated by the
following examples:

\begin{center}\begin{minipage}{15cm}\begin{Verbatim}[frame=single]
> L = [|1,2,true,3...|];
> L;
[|1, 2, true, 3...|]
> L[2];
true
> L[3];
3
> L[4];
4
> L[1200];
1200
> L = [|1,...,5,true...|];
> L;
[|1, 2, 3, 4, 5, true...|]
> L[1200];
true
\end{Verbatim}
\end{minipage}\end{center}


\subsection{Structures}

In a similar way as in lists, \sollya allows data to be grouped in --
untyped -- structures. A structure forms an object to which other
objects can be added as elements and identified by their names. The
elements of a structure can be retrieved under their name and used as
usual. The following sequence shows that point:

\begin{center}\begin{minipage}{15cm}\begin{Verbatim}[frame=single]
> s.a = 17;
> s.b = exp(x);
> s.a;
17
> s.b;
exp(x)
> s.b(1);
Warning: rounding has happened. The value displayed is a faithful rounding of th
e true result.
2.71828182845904523536028747135266249775724709369998
> s.d.a = [-1;1];
> s.d.b = sin(x);
> inf(s.d.a);
-1
> diff(s.d.b);
cos(x)
\end{Verbatim}
\end{minipage}\end{center}


Structures can also be defined literally using the syntax illustrated
in the next example. They will also be printed in that syntax.

\begin{center}\begin{minipage}{15cm}\begin{Verbatim}[frame=single]
> a = { .f = exp(x), .dom = [-1;1] };
> a;
{ .f = exp(x), .dom = [-1;1] }
> a.f;
exp(x)
> a.dom;
[-1;1]
> b.f = sin(x);
> b.dom = [-1b-5;1b-5];
> b;
{ .dom = [-3.125e-2;3.125e-2], .f = sin(x) }
> { .f = asin(x), .dom = [-1;1] }.f(1);
Warning: rounding has happened. The value displayed is a faithful rounding of th
e true result.
1.57079632679489661923132169163975144209858469968754
\end{Verbatim}
\end{minipage}\end{center}


If the variable \texttt{a} is bound to an existing structure, it is possible to use the ``dot notation'' \texttt{a.b} to assign the value of the field \texttt{b} of the structure \texttt{a}. This works even if \texttt{b} is not yet a field of \texttt{a}: in this case a new field is created inside the structure \texttt{a}. 

Besides, the dot notation can be used even when \texttt{a} is unassigned. In this case a new structure is created with a field \texttt{b}, and this structure is bound to \texttt{a}. However, the dot notation cannot be used if \texttt{a} is already bound to something that is not a structure.

These principles apply recursively: for instance, if \texttt{a} is a structure that contains only one field \texttt{d}, the command \texttt{a.b.c = 3} creates a new field named \texttt{b} inside the structure \texttt{a}; this field itself is a structure containing the field \texttt{c}. The command \texttt{a.d.c = 3} is allowed if \texttt{a.d} is already a structure, but forbidden otherwise (e.g. if \texttt{a.d} was equal to \texttt{sin(x)}). This is summed up in the following example.

\begin{center}\begin{minipage}{15cm}\begin{Verbatim}[frame=single]
> restart;
The tool has been restarted.
> a.f = exp(x);
> a.dom = [-1;1];
> a.info.text = "My akrnoximation problem";
> a;
{ .info = { .text = "My akrnoximation problem" }, .dom = [-1;1], .f = exp(x) }
> 
> a.info.text = "My approximation problem";
> a;
{ .info = { .text = "My approximation problem" }, .dom = [-1;1], .f = exp(x) }
> 
> b = exp(x);
> b.a = 5;
Warning: cannot modify an element of something that is not a structure.
Warning: the last assignment will have no effect.
> b;
exp(x)
> 
> a.dom.a = -1;
Warning: cannot modify an element of something that is not a structure.
Warning: the last assignment will have no effect.
> a;
{ .info = { .text = "My approximation problem" }, .dom = [-1;1], .f = exp(x) }
\end{Verbatim}
\end{minipage}\end{center}


When printed, the elements of a structure are not sorted in any
manner. They get printed in an arbitrary order that just maintains the
order given in the definition of literate structures. That said, when
compared, two structures compare equal iff they contain the same
number of identifiers, with the same names and iff the elements of
corresponding names all compare equal. This means the order does
not matter in comparisons and otherwise does only for printing.

The following example illustrates this matter:

\begin{center}\begin{minipage}{15cm}\begin{Verbatim}[frame=single]
> a = { .f = exp(x), .a = -1, .b = 1 };
> a;
{ .f = exp(x), .a = -1, .b = 1 }
> a.info = "My function";
> a;
{ .info = "My function", .f = exp(x), .a = -1, .b = 1 }
> 
> b = { .a = -1, .f = exp(x), .info = "My function", .b = 1 };
> b;
{ .a = -1, .f = exp(x), .info = "My function", .b = 1 }
> 
> a == b;
true
> 
> b.info = "My other function";
> a == b;
false
> 
> b.info = "My function";
> a == b;
true
> b.something = true;
> a == b;
false
\end{Verbatim}
\end{minipage}\end{center}


\section{Iterative language elements: assignments, conditional statements and loops}

\subsection{Blocks}

Statements in \sollya can be grouped in blocks, so-called
begin-end-blocks.  This can be done using the key tokens \key{$\lbrace$} and
\key{$\rbrace$}. Blocks declared this way are considered to be one single
statement. As already explained in section \ref{variables}, using
begin-end-blocks also opens the possibility of declaring variables
through the keyword \key{var}. 

\subsection{Assignments}

\sollya has two different assignment operators, \texttt{=} and
\texttt{:=}. The assignment operator \texttt{=} assigns its
right-hand-object ``as is'', i.e. without evaluating functional
expressions. For instance, \texttt{i = i + 1;} will dereferentiate the
identifier \texttt{i} with some content, notate it $y$, build up the
expression (function) $y + 1$ and assign this expression back to
\texttt{i}. In the example, if \texttt{i} stood for the value $1000$,
the statement \texttt{i = i + 1;} would assign ``$1000 + 1$'' -- and not
``$1001$'' -- to \texttt{i}. The assignment operator \texttt{:=} evaluates
constant functional expressions before assigning them. On other
expressions it behaves like \texttt{=}. Still in the example, the
statement \texttt{i := i + 1;} really assigns $1001$ to \texttt{i}.

Both \sollya assignment operators support indexing of lists or strings
elements using brackets on the left-hand-side of the assignment
operator. The indexed element of the list or string gets replaced by
the right-hand-side of the assignment operator.  When indexing strings
this way, that right-hand side must evaluate to a string of length
$1$. End-elliptic lists are supported with their usual semantic for
this kind of assignment.  When referencing and assigning a value in
the implicit part of the end-elliptic list, the list gets expanded to
the corresponding length.

The following examples well illustrate the behavior of assignment
statements:

\begin{center}\begin{minipage}{15cm}\begin{Verbatim}[frame=single]
> autosimplify = off;
Automatic pure tree simplification has been deactivated.
> i = 1000;
> i = i + 1;
> print(i);
1000 + 1
> i := i + 1;
> print(i);
1002
> L = [|1,...,5|];
> print(L);
[|1, 2, 3, 4, 5|]
> L[3] = L[3] + 1;
> L[4] := L[4] + 1;
> print(L);
[|1, 2, 3, 4 + 1, 6|]
> L[5] = true;
> L;
[|1, 2, 3, 5, 6, true|]
> s = "Hello world";
> s;
Hello world
> s[1] = "a";
> s;
Hallo world
> s[2] = "foo";
Warning: the string to be assigned is not of length 1.
This command will have no effect.
> L = [|true,1,...,5,9...|];
> L;
[|true, 1, 2, 3, 4, 5, 9...|]
> L[13] = "Hello";
> L;
[|true, 1, 2, 3, 4, 5, 9, 10, 11, 12, 13, 14, 15, "Hello"...|]
\end{Verbatim}
\end{minipage}\end{center}



The indexing of lists on left-hand sides of assignments is reduced to
the first order. Multiple indexing of lists of lists on assignment is
not supported for complexity reasons. Multiple indexing is possible in
right-hand sides.

\begin{center}\begin{minipage}{15cm}\begin{Verbatim}[frame=single]
> L = [| 1, 2, [|"a", "b", [|true, false|] |] |];
> L[2][2][1];
false
> L[2][2][1] = true;
Warning: the first element is not an identifier.
This command will have no effect.
> L[2][2] = "c";
Warning: the first element is not an identifier.
This command will have no effect.
> L[2] = 3;
> L;
[|1, 2, 3|]
\end{Verbatim}
\end{minipage}\end{center}


\subsection{Conditional statements}

\sollya supports conditional statements expressed with the keywords
\key{if}, \key{then} and optionally \key{else}. Let us mention that only
conditional statements are supported and not conditional expressions. 

The following examples illustrate both syntax and semantic of
conditional statements in \sollya. Concerning syntax, be aware that there must not be any semicolon
before the \key{else} keyword.

\begin{center}\begin{minipage}{15cm}\begin{Verbatim}[frame=single]
> a = 3;
> b = 4;
> if (a == b) then print("Hello world");
> b = 3;
> if (a == b) then print("Hello world");
Hello world
> if (a == b) then print("You are telling the truth") else print("Liar!");
You are telling the truth
\end{Verbatim}
\end{minipage}\end{center}


\subsection{Loops}

\sollya supports three kinds of loops. General \emph{while-condition}
loops can be expressed using the keywords \key{while} and
\key{do}. One has to be aware of the fact that the condition test is
executed always before the loop, there is no \emph{do-until-condition}
loop. Consider the following examples for both syntax and semantic:

\begin{center}\begin{minipage}{15cm}\begin{Verbatim}[frame=single]
> verbosity = 0!;
> prec = 30!;
> i = 5;
> while (expm1(i) > 0) do { expm1(i); i := i - 1; };
1.474131591e2
5.359815e1
1.908553692e1
6.3890561
1.718281827
> print(i);
0
\end{Verbatim}
\end{minipage}\end{center}


The second kind of loops are loops on a variable ranging from a
numerical start value and a end value. These kind of loops can be
expressed using the keywords \key{for}, \key{from}, \key{to}, \key{do}
and optionally \key{by}. The \key{by} statement indicates the width of
the steps on the variable from the start value to the end value. Once
again, syntax and semantic are best explained with an example:

\begin{center}\begin{minipage}{15cm}\begin{Verbatim}[frame=single]
> for i from 1 to 5 do print ("Hello world",i);
Hello world 1
Hello world 2
Hello world 3
Hello world 4
Hello world 5
> for i from 2 to 1 by -0.5 do print("Hello world",i);
Hello world 2
Hello world 1.5
Hello world 1
\end{Verbatim}
\end{minipage}\end{center}


The third kind of loops are loops on a variable ranging on values
contained in a list. In order to ensure the termination of the loop,
that list must not be end-elliptic. The loop is expressed using the
keywords \key{for}, \key{in} and \key{do} as in the following
examples:

\begin{center}\begin{minipage}{15cm}\begin{Verbatim}[frame=single]
> L = [|true, false, 1,...,4, "Hello", exp(x)|];
> for i in L do i;
true
false
1
2
3
4
Hello
exp(x)
\end{Verbatim}
\end{minipage}\end{center}


For both types of \key{for} loops, assigning the loop variable is
allowed and possible. When the loop terminates, the loop variable will
contain the value that made the loop condition fail. Consider the
following examples:

\begin{center}\begin{minipage}{15cm}\begin{Verbatim}[frame=single]
> for i from 1 to 5 do { if (i == 3) then i = 4 else i; };
1
2
5
> i;
6
\end{Verbatim}
\end{minipage}\end{center}


\section{Functional language elements: procedures and pattern matching}

\subsection{Procedures}
\sollya has some elements of functional languages. In order to
avoid confusion with mathematical functions, the associated
programming objects are called \emph{procedures} in \sollya.

\sollya procedures are common objects that can be, for example,
assigned to variables or stored in lists. Procedures are declared by
the \key{proc} keyword; see section \ref{labproc} for details. The
returned procedure object must then be assigned to a variable. It can
hence be applied to arguments with common application syntax. The
\key{procedure} keyword provides an abbreviation for declaring and
assigning a procedure; see section \ref{labprocedure} for details.

\sollya procedures can return objects using the \key{return} keyword
at the end of the begin-end-block of the procedure. Section
\ref{labreturn} gives details on the usage of \key{return}. Procedures
further can take any type of object in argument, in particular also
other procedures that are then applied to arguments. Procedures can
be declared inside other procedures.

Common \sollya procedures are declared with a certain number of formal
parameters. When the procedure is applied to actual parameters, a
check is performed if the right number of actual parameters is
given. Then the actual parameters are applied to the formal
parameters. In some cases, it is required that the number of
parameters of a procedure be variable. \sollya provides support for
the case with procedures with an arbitrary number of actual arguments.
When the procedure is called, those actual arguments are gathered in a
list which is applied to the only formal list parameter of a procedure
with an arbitrary number of arguments. See section \ref{labprocedure}
for the exact syntax and details; an example is given just below.

Let us remark that declaring a procedure does not involve any evaluation or
other interpretation of the procedure body. In particular, this means
that constants are evaluated to floating-point values inside \sollya
when the procedure is applied to actual parameters and the global
precision valid at this moment.

\sollya procedures are well illustrated with the following examples:

\begin{center}\begin{minipage}{15cm}\begin{Verbatim}[frame=single]
> succ = proc(n) { return n + 1; };
> succ(5);
6
> 3 + succ(0);
4
> succ;
proc(n)
begin
nop;
return (n) + (1);
end
> add = proc(m,n) { var res; res := m + n; return res; };
> add(5,6);
11
> hey = proc() { print("Hello world."); };
> hey();
Hello world.
> print(hey());
Hello world.
void
> hey;
proc()
begin
print("Hello world.");
return void;
end
> fac = proc(n) { var res; if (n == 0) then res := 1 else res := n * fac(n - 1);
 return res; };
> fac(5);
120
> fac(11);
39916800
> fac;
proc(n)
begin
var res;
if (n) == (0) then
res := 1
else
res := (n) * (fac((n) - (1)));
return res;
end
\end{Verbatim}
\end{minipage}\end{center}


\begin{center}\begin{minipage}{15cm}\begin{Verbatim}[frame=single]
> add = proc(m,n) { var res; res := m + n; return res; };
> add(5,6);
11
> hey = proc() { print("Hello world."); };
> hey();
Hello world.
> print(hey());
Hello world.
void
> hey;
proc()
{
print("Hello world.");
return void;
}
\end{Verbatim}
\end{minipage}\end{center}


\begin{center}\begin{minipage}{15cm}\begin{Verbatim}[frame=single]
> fac = proc(n) { var res; if (n == 0) then res := 1 else res := n * fac(n - 1);
 return res; };
> fac(5);
120
> fac(11);
39916800
> fac;
proc(n)
{
var res;
if (n) == (0) then
res := 1
else
res := (n) * (fac((n) - (1)));
return res;
}
\end{Verbatim}
\end{minipage}\end{center}


\begin{center}\begin{minipage}{15cm}\begin{Verbatim}[frame=single]
> sumall = proc(args = ...) { var i, acc; acc = 0; for i in args do acc = acc + 
i; return acc; };
> sumall;
proc(args = ...)
begin
var i, acc;
acc = 0;
for i in args do
acc = (acc) + (i);
return acc;
end
> sumall();
0
> sumall(1);
1
> sumall(1,5);
6
> sumall(1,5,9);
15
> sumall @ [|1,5,9,4,8|];
27
> 
\end{Verbatim}
\end{minipage}\end{center}


Let us note that, when writing a procedure, one does not know what will
be the name of the free variable at run-time. This is typically the context when one
should use the special keyword \verb|_x_|:

\begin{center}\begin{minipage}{15cm}\begin{Verbatim}[frame=single]
> ChebPolynomials = proc(n) {
    var i, res;
    if (n<0) then res = [||]
    else if (n==0) then res = [|1|]
    else {
       res = [|1, _x_|];
       for i from 2 to n do res[i] = horner(2*_x_*res[i-1]-res[i-2]);
    };
    return res;
  };
> 
> f = sin(x);
> T = ChebPolynomials(4);
Warning: a SIGSEGV signal has been handled.
Warning: the last command could not be executed. May leak memory.
Warning: releasing the variable frame stack.
  canonical = on!;
> for i from 0 to 4 do T[i];
Warning: the identifier "T" is neither assigned to, nor bound to a library funct
ion nor external procedure, nor equal to the current free variable.
Will interpret "T" as "x".
Warning: at least one of the given expressions or a subexpression is not correct
ly typed
or its evaluation has failed because of some error on a side-effect.
error
Warning: the identifier "T" is neither assigned to, nor bound to a library funct
ion nor external procedure, nor equal to the current free variable.
Will interpret "T" as "x".
Warning: at least one of the given expressions or a subexpression is not correct
ly typed
or its evaluation has failed because of some error on a side-effect.
error
Warning: the identifier "T" is neither assigned to, nor bound to a library funct
ion nor external procedure, nor equal to the current free variable.
Will interpret "T" as "x".
Warning: at least one of the given expressions or a subexpression is not correct
ly typed
or its evaluation has failed because of some error on a side-effect.
error
Warning: the identifier "T" is neither assigned to, nor bound to a library funct
ion nor external procedure, nor equal to the current free variable.
Will interpret "T" as "x".
Warning: at least one of the given expressions or a subexpression is not correct
ly typed
or its evaluation has failed because of some error on a side-effect.
error
Warning: the identifier "T" is neither assigned to, nor bound to a library funct
ion nor external procedure, nor equal to the current free variable.
Will interpret "T" as "x".
Warning: at least one of the given expressions or a subexpression is not correct
ly typed
or its evaluation has failed because of some error on a side-effect.
error
\end{Verbatim}
\end{minipage}\end{center}


\sollya also supports external procedures, i.e. procedures written in
\texttt{C} (or some other language) and dynamically bound to \sollya
identifiers. See \ref{labexternalproc} for details.

\subsection{Pattern matching}

Starting with version 3.0, \sollya supports matching expressions with
expression patterns. This feature is important for an extended
functional programming style. Further, and most importantly, it allows
expression trees to be recursively decomposed using native constructs
of the \sollya language. This means no help from external procedures
or other compiled-language mechanisms is needed here anymore.

Basically, pattern matching supports relies on one \sollya construct: 
\begin{center}
\begin{minipage}{0.8\textwidth}
\key{match {\it expr} with \\
{\it pattern1} : ({\it return-expr1}) \\
{\it pattern2} : ({\it return-expr2}) \\
\dots \\
{\it patternN} : ({\it return-exprN}) }
\end{minipage}
\end{center} 
That construct has the following semantic: try to match the
expression {\it expr} with the patterns {\it pattern1} through {\it
  patternN}, proceeding in natural order. If a pattern
{\it patternI} is found that matches, evaluate the whole \key{match
  \dots~with} construct to the return expression {\it return-exprI}
associated with the matching pattern {\it patternI}. If no matching
pattern is found, display an error warning and return \key{error}. Note that the parentheses around the expressions {\it return-exprI} are mandatory.

Matching a pattern means the following: 
\begin{itemize}
  \item If a pattern does not contain any programming-language-level
    variables (different from the free mathematical variable), it
    matches expressions that are syntactically equal to itself. For
    instance, the pattern \key{exp(sin(3 * x))} will match the
    expression \key{exp(sin(3 * x))}, but it does not match \key{exp(sin(x * 3))} because the expressions are not syntactically equal.
  \item If a pattern does contain variables, it matches an expression
    {\it expr} if these variables can be bound to subexpressions of
    {\it expr} such that once the pattern is evaluated with that
    variable binding, it becomes syntactically equal to the expression
    {\it expr}. For instance, the pattern \key{exp(sin(a * x))} will
    match the expression \key{exp(sin(3 * x))} as it is possible to
    bind \key{a} to \key{3} such that \key{exp(sin(a~*~x))} evaluates
    to \key{exp(sin(3~*~x))}.
\end{itemize}

If a pattern {\it patternI} with variables is matched in a \key{match
  \dots~with} construct, the variables in the pattern stay bound
during the evaluation of the corresponding return expression {\it
  return-exprI}. This allows subexpressions to be extracted from
expressions and/or recursively handled as needed.

The following examples illustrate the basic principles of pattern
matching in \sollya. One can remark that it is useful to use the
keyword \verb|_x_| when one wants to be sure to refer to the free
variable in a pattern matching:

\begin{center}\begin{minipage}{15cm}\begin{Verbatim}[frame=single]
> match exp(x) with 
      exp(x)      : (1) 
      sin(x)      : (2)
      default     : (3);
1
> 
> match sin(x) with 
      exp(x)      : (1) 
      sin(x)      : (2)
      default     : (3);
2
> 
> match exp(sin(x)) with
      exp(x)      : ("Exponential of x")
      exp(sin(x)) : ("Exponential of sine of x")
      default     : ("Something else");
Exponential of sine of x
> 
> match exp(sin(x)) with
      exp(x)      : ("Exponential of x")
      exp(a)      : ("Exponential of " @ a)
      default     : ("Something else");
Exponential of sin(x)
> 
> 
> procedure differentiate(f) {
      return match f with 
          g + h   : (differentiate(g) + differentiate(h))
          g * h   : (differentiate(g) * h + differentiate(h) * g)
          g / h   : ((differentiate(g) * h - differentiate(h) * g) / (h^2))
          exp(_x_)  : (exp(_x_))
          sin(_x_)  : (cos(_x_))
          cos(_x_)  : (-sin(_x_))
          g(h)    : ((differentiate(g))(h) * differentiate(h))
          _x_       : (1)
          h(_x_)    : (NaN)
          c       : (0);
  };
> 
> rename(x,y);
Information: the free variable has been renamed from "x" to "y".
> differentiate(exp(sin(y + y)));
exp(sin(y * 2)) * cos(y * 2) * 2
> diff(exp(sin(y + y)));
exp(sin(y * 2)) * cos(y * 2) * 2
> 
\end{Verbatim}
\end{minipage}\end{center}


As \sollya is not a purely functional language, the \key{match
  \dots~with} construct can also be used in a more imperative style,
which makes it become closer to constructs like \key{switch} in {\tt
  C} or {\tt Perl}. In lieu of a simple return expression, a whole
block of imperative statements can be given. The expression to be
returned by that block is indicated in the end of the block, using
the \key{return} keyword. That syntax is illustrated in the next
example:

\begin{center}\begin{minipage}{15cm}\begin{Verbatim}[frame=single]
> match exp(sin(x)) with
      exp(a)  : { 
                   write("Exponential of ", a, "\n");
                   return a;
                }
      sin(x)  : {
                   var foo;
                   foo = 17;
                   write("Sine of x\n");
                   return foo;
                }
      default : {
                   write("Something else\n");
                   bashexecute("LANG=C date");
                   return true;
                };
Exponential of sin(x)
sin(x)
> 
> match sin(x) with
      exp(a)  : { 
                   write("Exponential of ", a, "\n");
                   return a;
                }
      sin(x)  : {
                   var foo;
                   foo = 17;
                   write("Sine of x\n");
                   return foo;
                }
      default : {
                   write("Something else\n");
                   bashexecute("LANG=C date");
                   return true;
                };
Sine of x
17
> 
> match acos(17 * pi * x) with
      exp(a)  : { 
                   write("Exponential of ", a, "\n");
                   return a;
                }
      sin(x)  : {
                   var foo;
                   foo = 17;
                   write("Sine of x\n");
                   return foo;
                }
      default : {
                   write("Something else\n");
                   bashexecute("LANG=C date");
                   return true;
                };
Something else
Sun May  1 12:36:50 CEST 2011
true
\end{Verbatim}
\end{minipage}\end{center}


In the case when no return statement is indicated for a
statement-block in a \key{match \dots~with} construct, the construct
evaluates to the special value \key{void} if that pattern matches.

In order to well understand pattern matching in \sollya, it is
important to realize the meaning of variables in patterns. This
meaning is different from the one usually found for variables. In a
pattern, variables are never evaluated to whatever they might have set
before the pattern is executed. In contrast, all variables in patterns
are new, free variables that will freshly be bound to subexpressions
of the matching expression. If a variable of the same name already
exists, it will be shadowed during the evaluation of the statement
block and the return expression corresponding to the matching
expression. This type of semantic implies that patterns can never be
computed at run-time, they must always be hard-coded
beforehand. However this is necessary to make pattern matching
context-free.

As a matter of course, all variables figuring in the expression {\it
  expr} to be matched are evaluated before pattern matching is
attempted. In fact, {\it expr} is a usual \sollya expression, not a
pattern.

In \sollya, the use of variables in patterns does not need to be
linear. This means the same variable might appear twice or more in a
pattern. Such a pattern will only match an expression if it contains
the same subexpression, associated with the variable, in all places
indicated by the variable in the pattern.

The following examples illustrate the use of variables in patterns in
detail:

\begin{center}\begin{minipage}{15cm}\begin{Verbatim}[frame=single]
> a = 5;
> b = 6;
> match exp(x + 3) with 
        exp(a + b) : {
                        print("Exponential");
                        print("a = ", a);
                        print("b = ", b);
                     }
        sin(x)     : {
                        print("Sine of x");
                     };
Exponential
a =  x
b =  3
> print("a = ", a, ", b = ", b);
a =  5 , b =  6
> 
> a = 5;
> b = 6;
> match exp(x + 3) with 
        exp(a + b) : {
                        var a, c;
                        a = 17;
                        c = "Hallo";
                        print("Exponential");
                        print("a = ", a);
                        print("b = ", b);
                        print("c = ", c);
                     }
        sin(x)     : {
                        print("Sine of x");
                     };
Exponential
a =  17
b =  3
c =  Hallo
> print("a = ", a, ", b = ", b);
a =  5 , b =  6
\end{Verbatim}
\end{minipage}\end{center}


\begin{center}\begin{minipage}{15cm}\begin{Verbatim}[frame=single]
> match exp(sin(x)) + sin(x) with
        exp(a) + a : {
                        print("Winner");
                        print("a = ", a);
                     }
        default    : {
                        print("Loser");
                     };
Winner
a =  sin(x)
> 
> match exp(sin(x)) + sin(3 * x) with
        exp(a) + a : {
                        print("Winner");
                        print("a = ", a);
                     }
        default    : {
                        print("Loser");
                     };
Loser
> 
> f = exp(x);
> match f with
        sin(x) : (1)
        cos(x) : (2)
        exp(x) : (3)
        default : (4);
3
\end{Verbatim}
\end{minipage}\end{center}


Pattern matching is meant to be a means to decompose expressions
structurally. For this reason and in an analogous way to variables, no
evaluation is performed at all on (sub-)expressions that form constant
functions. As a consequence, patterns match constant expressions
only if they are structurally identical. For example $5 + 1$ only
matches $5 + 1$ and not $1 + 5$, $3 + 3$ nor $6$.

This general rule on constant expressions admits one exception.
Intervals in \sollya can be defined using constant expressions as
bounds. These bounds are immediately evaluated to floating-point
constants, though. In order to permit pattern matching on intervals,
constant expressions given as bounds of intervals that form patterns
are evaluated before pattern matching. However, in order not conflict
with the rules of no evaluation of variables, these constant
expressions as bounds of intervals in patterns must not contain free
variables.

\begin{center}\begin{minipage}{15cm}\begin{Verbatim}[frame=single]
> match 5 + 1 with 
      1 + 5 : ("One plus five")
      6     : ("Six")
      5 + 1 : ("Five plus one");
Five plus one
> 
> match 6 with 
      1 + 5 : ("One plus five")
      6     : ("Six")
      5 + 1 : ("Five plus one");
Six
>   
> match 1 + 5 with 
      1 + 5 : ("One plus five")
      6     : ("Six")
      5 + 1 : ("Five plus one");
One plus five
> 
> match [1; 5 + 1] with
      [1; 1 + 5] : ("Interval from one to one plus five")
      [1; 6]     : ("Interval from one to six")
      [1; 5 + 1] : ("Interval from one to five plus one");
Interval from one to one plus five
> 
> match [1; 6] with
      [1; 1 + 5] : ("Interval from one to one plus five")
      [1; 6]     : ("Interval from one to six")
      [1; 5 + 1] : ("Interval from one to five plus one");
Interval from one to one plus five
> 
\end{Verbatim}
\end{minipage}\end{center}


The \sollya keyword \key{default} has a special meaning in patterns.
It acts like a wild-card, matching any (sub-)expression, as long as
the whole expression stays correctly typed. Upon matching with
\key{default}, no variable gets bound. This feature is illustrated in
the next example:

\begin{center}\begin{minipage}{15cm}\begin{Verbatim}[frame=single]
> match exp(x) with 
      sin(x)    : ("Sine of x")
      atan(x^2) : ("Arctangent of square of x")
      default   : ("Something else")
      exp(x)    : ("Exponential of x");
Something else
> 
> match atan(x^2) with 
      sin(x)          : ("Sine of x")
      atan(default^2) : ("Arctangent of the square of something")
      default         : ("Something else");
Arctangent of the square of something
> 
> match atan(exp(x)^2) with 
      sin(x)          : ("Sine of x")
      atan(default^2) : ("Arctangent of the square of something")
      default         : ("Something else");
Arctangent of the square of something
> 
> match exp("Hello world") with 
      exp(default)    : ("A miracle has happened")
      default         : ("Something else");
Warning: at least one of the given expressions or a subexpression is not correct
ly typed
or its evaluation has failed because of some error on a side-effect.
error
\end{Verbatim}
\end{minipage}\end{center}


In \sollya, pattern matching is possible on the following \sollya
types and operations defined on them:
\begin{itemize}
\item Expressions that define univariate functions, as explained above,
\item Intervals with one, two or no bound defined in the pattern by a variable,
\item Character sequences, literate or defined using the \key{@} operator, possibly with a variable on one of the sides of the \key{@} operator,
\item Lists, literate, literate with variables or defined using the \key{.:}, \key{:.} and \key{@} operators, possibly with a variable on one of the sides of the \key{@} operator or one or two variables for \key{.:} and \key{:.},
\item Structures, literate or literate with variables, and
\item All other \sollya objects, matchable with themselves (\key{DE} matches \key{DE}, \key{on} matches \key{on}, \key{perturb} matches \key{perturb} etc.)
\end{itemize}

\begin{center}\begin{minipage}{15cm}\begin{Verbatim}[frame=single]
> procedure detector(obj) {
      match obj with 
          exp(a * x)            : { "Exponential of ", a, " times x"; }
          [ a; 17 ]             : { "An interval from ", a, " to 17"; }
          [| |]                 : { "Empty list"; }
          [| a, b, 2, exp(c) |] : { "A list of ", a, ", ", b, ", 2 and ",
                                    "exponential of ", c; }
          a @ [| 2, 3 |]        : { "Concatenation of the list ", a, " and ",
                                    "the list of 2 and 3"; }
          a .: [| 9 ... |]      : { a, " prepended to all integers >= 9"; }
          "Hello" @ w           : { "Hello concatenated with ", w; }
          { .a = sin(b); 
            .b = [c;d] }        : { "A structure containing as .a the ",
                                    "sine of ", b,
                                    " and as .b the range from ", c, 
                                    " to ", d; }
          perturb               : { "The special object perturb"; }
          default               : { "Something else"; };
  };
> 
> detector(exp(5 * x));
Exponential of 5 times x
> detector([3.25;17]);
An interval from 3.25 to 17
> detector([||]);
Empty list
> detector([| sin(x), nearestint(x), 2, exp(5 * atan(x)) |]);
A list of sin(x), nearestint(x), 2 and exponential of 5 * atan(x)
> detector([| sin(x), cos(5 * x), "foo", 2, 3 |]);
Concatenation of the list [|sin(x), cos(x * 5), "foo"|] and the list of 2 and 3
> detector([| DE, 9... |]);
doubleextended prepended to all integers >= 9
> detector("Hello world");
Hello concatenated with  world
> detector({ .a = sin(x); .c = "Hello"; .b = [9;10] });
A structure containing as .a the sine of x and as .b the range from 9 to 10
> detector(perturb);
The special object perturb
> detector([13;19]);
Something else
\end{Verbatim}
\end{minipage}\end{center}


Concerning intervals, please pay attention to the fact that expressions involving 
intervals are immediately evaluated and that structural pattern matching on functions
on intervals is not possible. This point is illustrated in the next example:

\begin{center}\begin{minipage}{15cm}\begin{Verbatim}[frame=single]
> match exp([1;2]) with 
        [a;b]              : {
                                a,", ",b;
                             }
        default            : {
                                "Something else";
                             };
2.71828182845904523536028747135266249775724709369989, 7.389056098930650227230427
4605750078131803155705518
> 
> match exp([1;2]) with 
        exp([a;b])         : {
                                a,", ", b;
                             }
        default            : {
                                "Something else";
                             };
Warning: at least one of the given expressions or a subexpression is not correct
ly typed
or its evaluation has failed because of some error on a side-effect.
error
> 
> match exp([1;2]) with 
    exp(a)  : {
                "Exponential of ", a;
              }
    default : {
                "Something else";
              };
Something else
\end{Verbatim}
\end{minipage}\end{center}


With respect to pattern matching on lists or character sequences
defined using the \key{@} operator, the following is to be mentionned:
\begin{itemize}
\item Patterns like \key{a @ b} are not allowed as they would need to
  perform an ambiguous cut of the list or character sequence to be
  matched. This restriction is maintained even if the variables (here
  \key{a} and \key{b}) are constrained by other occurrences in the
  pattern (for example in a list) which would make the cut
  unambiguous.
\item Recursive use of the \key{@} operator (even mixed with the
  operators \key{.:} and \key{:.}) is possible under the condition
  that there must not exist any other parenthezation of the term in
  concatenations (\key{@}) such that the rule of one single variable
  for \key{@} above gets violated. For instance, \key{( [| 1 |] @ a) @
    (b @ [| 4 |])} is not possible as it can be re-parenthesized \key{
    [| 1 |] @ (a @ b) @ [| 4 |]}, which exhibits the ambiguous case.
\end{itemize}
These points are illustrated in this example:

\begin{center}\begin{minipage}{15cm}\begin{Verbatim}[frame=single]
> match [| exp(sin(x)), sin(x), 4, DE(x), 9... |] with
        exp(a) .: (a .: (([||] :. 4) @ (b @ [| 13... |]))) : 
                           { "a = ", a, ", b = ", b; };
a = sin(x), b = [|doubleextended(x), 9, 10, 11, 12|]
> 
> match [| 1, 2, 3, 4, D... |] with 
        a @ [| 4, D...|] : (a);
[|1, 2, 3|]
> 
> match [| 1, 2, 3, 4, D... |] with 
        a @ [| D...|] : (a);
[|1, 2, 3, 4|]
> 
> match [| 1, 2, 3, 4... |] with 
        a @ [| 3...|] : (a);
[|1, 2|]
> 
> match [| 1, 2, 3, 4... |] with 
        a @ [| 4...|] : (a);
[|1, 2, 3|]
> 
> match [| 1, 2, 3, 4... |] with 
        a @ [| 17...|] : (a);
[|1, 2, 3, 4, 5, 6, 7, 8, 9, 10, 11, 12, 13, 14, 15, 16|]
> 
> match [| 1, 2, 3, 4... |] with 
        a @ [| 17, 18, 19 |] : (a)
        default              : ("Something else");
Something else
\end{Verbatim}
\end{minipage}\end{center}


As mentionned above, pattern matching on \sollya structures is
possible. Patterns for such a match are given in a literately,
i.e. using the syntax \key{ \{ .a = {\it exprA}, .b = {\it exprB},
  {\it \dots}~\}}. A structure pattern {\it sp} will be matched by a
structure {\it s} iff that structure {\it s} contains at least all the
elements (like \key{.a}, \key{.b} etc.) of the structure pattern {\it
  sp} and iff each of the elements of the structure {\it s} matches
the pattern in the corresponding element of the structure pattern {\it
  sp}. The user should be aware of the fact that the structure to be
matched is only supposed to have at least the elements of the pattern
but that it may contain more elements is a particular \sollya
feature. For instance with pattern matching, it is hence possible to
ensure that access to particular elements will be possible in a
particular code segment. The following example is meant to clarify
this point:

\begin{center}\begin{minipage}{15cm}\begin{Verbatim}[frame=single]
> structure.f = exp(x);
> structure.dom = [1;2];
> structure.formats = [| DD, D, D, D |];
> match structure with 
       { .f = sin(x);
         .dom = [a;b]
       }                    : { "Sine, ",a,", ",b; }
       { .f = exp(c);
         .dom = [a;b];
         .point = default 
       }                    : { "Exponential, ",a, ", ", b, ", ", c; }
       { .f = exp(x);
         .dom = [a;b]
       }                    : { "Exponential, ",a, ", ", b; }
       default              : { "Something else"; };
Exponential, 1, 2
> 
> structure.f = sin(x);
> match structure with 
       { .f = sin(x);
         .dom = [a;b]
       }                    : { "Sine, ",a,", ",b; }
       { .f = exp(c);
         .dom = [a;b];
         .point = default 
       }                    : { "Exponential, ",a, ", ", b, ", ", c; }
       { .f = exp(x);
         .dom = [a;b]
       }                    : { "Exponential, ",a, ", ", b; }
       default              : { "Something else"; };
Sine, 1, 2
> 
> structure.f = exp(x + 2);
> structure.point = 23;
> match structure with 
       { .f = sin(x);
         .dom = [a;b]
       }                    : { "Sine, ",a,", ",b; }
       { .f = exp(c);
         .dom = [a;b];
         .point = default 
       }                    : { "Exponential, ",a, ", ", b, ", ", c; }
       { .f = exp(x);
         .dom = [a;b]
       }                    : { "Exponential, ",a, ", ", b; }
       default              : { "Something else"; };
Exponential, 1, 2, 2 + x
\end{Verbatim}
\end{minipage}\end{center}


\section{Commands and functions}
\label{commandsAndFunctions}
\subsection{ abs }
\noindent Name: \textbf{abs}\\
the absolute value.\\

\noindent Description: \begin{itemize}

\item \textbf{abs} is the absolute value function. \textbf{abs}(x)=x if x>0 and -x otherwise.
\end{itemize}

\subsection{ absolute }
\noindent Name: \textbf{perturb}\\
indicates an absolute error for \textbf{externalplot}\\

\noindent Usage: 
\begin{center}
\textbf{perturb} : \textsf{absolute$|$relative}\\
\end{center}
\noindent Description: \begin{itemize}

\item The use of \textbf{perturb} in the command \textbf{externalplot} indicates that during
   plotting in \textbf{externalplot} an absolute error is to be considered.
   See \textbf{externalplot} for details.
\end{itemize}
\noindent Example 1: 
\begin{center}\begin{minipage}{15cm}\begin{Verbatim}[frame=single]
> bashexecute("gcc -fPIC -c externalplotexample.c");
> bashexecute("gcc -shared -o externalplotexample externalplotexample.o -lgmp -l
mpfr");
> externalplot("./externalplotexample",absolute,exp(x),[-1/2;1/2],12,perturb);
\end{Verbatim}
\end{minipage}\end{center}
See also: \textbf{externalplot}, \textbf{relative}, \textbf{bashexecute}

\subsection{ accurateinfnorm }
\noindent Name: \textbf{accurateinfnorm}\\
computes a faithful rounding of the infinite norm of a function \\

\noindent Usage: 
\begin{center}
\textbf{accurateinfnorm}(\emph{function},\emph{range},\emph{constant}) : (\textsf{function}, \textsf{range}, \textsf{constant}) $\rightarrow$ \textsf{constant}\\
\textbf{accurateinfnorm}(\emph{function},\emph{range},\emph{constant},\emph{exclusion range 1},...,\emph{exclusion range n}) : (\textsf{function}, \textsf{range}, \textsf{constant}, \textsf{range}, ..., \textsf{range}) $\rightarrow$ \textsf{constant}\\
\end{center}
Parameters: 
\emph{function} represents the function whose infinite norm is to be computed\\
\emph{range} represents the infinite norm is to be considered on\\
\emph{constant} represents the number of bits in the significant of the result\\
\emph{exclusion range 1} through \emph{exclusion range n} represent ranges to be excluded \\

\noindent Description: \begin{itemize}

\item The command \textbf{accurateinfnorm} computes an upper bound to the infinite norm of
   function \emph{function} in \emph{range}. This upper bound is the least
   floating-point number greater than the value of the infinite norm that
   lies in the set of dyadic floating point numbers having \emph{constant}
   significant mantissa bits. This means the value \textbf{accurateinfnorm} evaluates to
   is at the time an upper bound and a faithful rounding to \emph{constant}
   bits of the infinite norm of function \emph{function} on range \emph{range}.
   If given, the fourth and further arguments of the command \textbf{accurateinfnorm},
   \emph{exclusion range 1} through \emph{exclusion range n} the infinite norm of
   the function \emph{function} is not to be considered on.
\end{itemize}
\noindent Example 1: 
\begin{center}\begin{minipage}{14.8cm}\begin{Verbatim}[frame=single]
   > p = remez(exp(x), 5, [-1;1]);
   > accurateinfnorm(p - exp(x), [-1;1], 20);
   0.45205641072243e-4
   > accurateinfnorm(p - exp(x), [-1;1], 30);
   0.45205621802324458e-4
   > accurateinfnorm(p - exp(x), [-1;1], 40);
   0.45205621769406345578e-4
\end{Verbatim}
\end{minipage}\end{center}
\noindent Example 2: 
\begin{center}\begin{minipage}{14.8cm}\begin{Verbatim}[frame=single]
   > p = remez(exp(x), 5, [-1;1]);
   > midpointmode = on!;
   > infnorm(p - exp(x), [-1;1]);
   0.45205~5/7~e-4
   > accurateinfnorm(p - exp(x), [-1;1], 40);
   0.45205621769406345578e-4
\end{Verbatim}
\end{minipage}\end{center}
See also: \textbf{infnorm}, \textbf{dirtyinfnorm}, \textbf{checkinfnorm}, \textbf{remez}, \textbf{diam}

\subsection{acos}
\label{labacos}
\noindent Name: \textbf{acos}\\
the arccosine function.\\

\noindent Description: \begin{itemize}

\item \textbf{acos} is the inverse of the function \textbf{cos}: \textbf{acos}(y) is the unique number 
   $x \in [0; \pi]$ such that \textbf{cos}(x)=y.

\item It is defined only for $y \in [-1;1]$.
\end{itemize}
See also: \textbf{cos} (\ref{labcos})

\subsection{acosh}
\label{labacosh}
\noindent Name: \textbf{acosh}\\
the arg-hyperbolic cosine function.\\
\noindent Description: \begin{itemize}

\item \textbf{acosh} is the inverse of the function \textbf{cosh}: \textbf{acosh}(y) is the unique number 
   $x \in [0; +\infty]$ such that \textbf{cosh}(x)=y.

\item It is defined only for $y \in [0;+\infty]$.
\end{itemize}
See also: \textbf{cosh} (\ref{labcosh})

\subsection{$\&\&$}
\label{laband}
\noindent Name: \textbf{$\&\&$}\\
boolean AND operator\\
\noindent Usage: 
\begin{center}
\emph{expr1} \textbf{$\&\&$} \emph{expr2} : (\textsf{boolean}, \textsf{boolean}) $\rightarrow$ \textsf{boolean}\\
\end{center}
Parameters: 
\begin{itemize}
\item \emph{expr1} and \emph{expr2} represent boolean expressions
\end{itemize}
\noindent Description: \begin{itemize}

\item \textbf{$\&\&$} evaluates to the boolean AND of the two
   boolean expressions \emph{expr1} and \emph{expr2}. \textbf{$\&\&$} evaluates to 
   true iff both \emph{expr1} and \emph{expr2} evaluate to true.
\end{itemize}
\noindent Example 1: 
\begin{center}\begin{minipage}{15cm}\begin{Verbatim}[frame=single]
> true && false;
false
\end{Verbatim}
\end{minipage}\end{center}
\noindent Example 2: 
\begin{center}\begin{minipage}{15cm}\begin{Verbatim}[frame=single]
> (1 == exp(0)) && (0 == log(1));
true
\end{Verbatim}
\end{minipage}\end{center}
See also: \textbf{$||$} (\ref{labor}), \textbf{!} (\ref{labnot})

\subsection{append}
\label{labappend}
\noindent Name: \textbf{:.}\\
add an element at the end of a list.\\

\noindent Usage: 
\begin{center}
\emph{L}\textbf{:.}\emph{x} : (\textsf{list}, \textsf{any type}) $\rightarrow$ \textsf{list}\\
\end{center}
Parameters: 
\begin{itemize}
\item \emph{L} is a list (possibly empty).
\item \emph{x} is an object of any type.
\end{itemize}
\noindent Description: \begin{itemize}

\item \textbf{:.} adds the element \emph{x} at the end of the list \emph{L}.

\item Note that since \emph{x} may be of any type, it can be in particular a list.
\end{itemize}
\noindent Example 1: 
\begin{center}\begin{minipage}{15cm}\begin{Verbatim}[frame=single]
> [|2,3,4|]:.5;
[|2, 3, 4, 5|]
\end{Verbatim}
\end{minipage}\end{center}
\noindent Example 2: 
\begin{center}\begin{minipage}{15cm}\begin{Verbatim}[frame=single]
> [|1,2,3|]:.[|4,5,6|];
[|1, 2, 3, [|4, 5, 6|]|]
\end{Verbatim}
\end{minipage}\end{center}
\noindent Example 3: 
\begin{center}\begin{minipage}{15cm}\begin{Verbatim}[frame=single]
> [||]:.1;
[|1|]
\end{Verbatim}
\end{minipage}\end{center}
See also: \textbf{.:} (\ref{labprepend}), \textbf{@} (\ref{labconcat})

\subsection{approx}
\label{labapprox}
\noindent Name: \textbf{$\sim$}\\
floating-point evaluation of a constant expression\\

\noindent Usage: 
\begin{center}
\textbf{$\sim$} \emph{expression} : \textsf{function} $\rightarrow$ \textsf{constant}\\
\textbf{$\sim$} \emph{something} : \textsf{any type} $\rightarrow$ \textsf{any type}\\
\end{center}
Parameters: 
\begin{itemize}
\item \emph{expression} stands for an expression that is a constant
\item \emph{something} stands for some language element that is not a constant expression
\end{itemize}
\noindent Description: \begin{itemize}

\item \textbf{$\sim$} \emph{expression} evaluates the \emph{expression} that is a constant
   term to a floating-point constant. The evaluation may involve a
   rounding. If \emph{expression} is not a constant, the evaluated constant is
   a faithful rounding of \emph{expression} with \textbf{precision} bits, unless the
   \emph{expression} is exactly $0$ as a result of cancellation. In the
   latter case, a floating-point approximation of some (unknown) accuracy
   is returned.

\item \textbf{$\sim$} does not do anything on all language elements that are not a
   constant expression.  In other words, it behaves like the identity
   function on any type that is not a constant expression. It can hence
   be used in any place where one wants to be sure that expressions are
   simplified using floating-point computations to constants of a known
   precision, regardless of the type of actual language elements.

\item \textbf{$\sim$} \textbf{error} evaluates to error and provokes a warning.

\item \textbf{$\sim$} is a prefix operator not requiring parentheses. Its
   precedence is the same as for the unary $+$ and $-$
   operators. It cannot be repeatedly used without brackets.
\end{itemize}
\noindent Example 1: 
\begin{center}\begin{minipage}{15cm}\begin{Verbatim}[frame=single]
> print(exp(5));
exp(5)
> print(~ exp(5));
1.48413159102576603421115580040552279623487667593878e2
\end{Verbatim}
\end{minipage}\end{center}
\noindent Example 2: 
\begin{center}\begin{minipage}{15cm}\begin{Verbatim}[frame=single]
> autosimplify = off!;
\end{Verbatim}
\end{minipage}\end{center}
\noindent Example 3: 
\begin{center}\begin{minipage}{15cm}\begin{Verbatim}[frame=single]
> print(~sin(5 * pi));
-4.3878064621853914052425209013193794551397356335691e-12715
\end{Verbatim}
\end{minipage}\end{center}
\noindent Example 4: 
\begin{center}\begin{minipage}{15cm}\begin{Verbatim}[frame=single]
> print(~exp(x));
exp(x)
> print(~ "Hello");
Hello
\end{Verbatim}
\end{minipage}\end{center}
\noindent Example 5: 
\begin{center}\begin{minipage}{15cm}\begin{Verbatim}[frame=single]
> print(~exp(x*5*Pi));
exp((pi) * 5 * x)
> print(exp(x* ~(5*Pi)));
exp(x * 1.57079632679489661923132169163975144209858469968757e1)
\end{Verbatim}
\end{minipage}\end{center}
\noindent Example 6: 
\begin{center}\begin{minipage}{15cm}\begin{Verbatim}[frame=single]
> print(~exp(5)*x);
1.48413159102576603421115580040552279623487667593878e2 * x
> print( (~exp(5))*x);
1.48413159102576603421115580040552279623487667593878e2 * x
> print(~(exp(5)*x));
exp(5) * x
\end{Verbatim}
\end{minipage}\end{center}
See also: \textbf{evaluate} (\ref{labevaluate}), \textbf{prec} (\ref{labprec}), \textbf{error} (\ref{laberror})

\subsection{asciiplot}
\label{labasciiplot}
\noindent Name: \textbf{asciiplot}\\
plots a function in a range using ASCII characters\\
\noindent Usage: 
\begin{center}
\textbf{asciiplot}(\emph{function}, \emph{range}) : (\textsf{function}, \textsf{range}) $\rightarrow$ \textsf{void}
\end{center}
Parameters: 
\begin{itemize}
\item \emph{function} represents a function to be plotted
\item \emph{range} represents a range the function is to be plotted in 
\end{itemize}
\noindent Description: \begin{itemize}

\item \textbf{asciiplot} plots the function \emph{function} in range \emph{range} using ASCII
   characters.  On systems that provide the necessary 
   \texttt{TIOCGWINSZ ioctl}, \sollya determines the size of the
   terminal for the plot size if connected to a terminal. If it is not
   connected to a terminal or if the test is not possible, the plot is of
   fixed size $77\times25$ characters.  The function is
   evaluated on a number of points equal to the number of columns
   available. Its value is rounded to the next integer in the range of
   lines available. A letter \texttt{x} is written at this place. If zero is in
   the hull of the image domain of the function, a x-axis is
   displayed. If zero is in range, an y-axis is displayed.  If the
   function is constant or if the range is reduced to one point, the
   function is evaluated to a constant and the constant is displayed
   instead of a plot.
\end{itemize}
\noindent Example 1: 
\begin{center}\begin{minipage}{15cm}\begin{Verbatim}[frame=single]
> asciiplot(exp(x),[1;2]);
                                                                           x
                                                                         xx 
                                                                      xxx   
                                                                    xx      
                                                                  xx        
                                                               xxx          
                                                             xx             
                                                          xxx               
                                                        xx                  
                                                     xxx                    
                                                  xxx                       
                                               xxx                          
                                            xxx                             
                                         xxx                                
                                      xxx                                   
                                  xxxx                                      
                              xxxx                                          
                           xxx                                              
                       xxxx                                                 
                  xxxxx                                                     
             xxxxx                                                          
         xxxx                                                               
    xxxxx                                                                   
xxxx                                                                        
\end{Verbatim}
\end{minipage}\end{center}
\noindent Example 2: 
\begin{center}\begin{minipage}{15cm}\begin{Verbatim}[frame=single]
> asciiplot(expm1(x),[-1;2]);
                         |                                                 x
                         |                                                x 
                         |                                               x  
                         |                                              x   
                         |                                             x    
                         |                                           xx     
                         |                                          x       
                         |                                         x        
                         |                                       xx         
                         |                                     xx           
                         |                                   xx             
                         |                                 xx               
                         |                                x                 
                         |                             xxx                  
                         |                           xx                     
                         |                        xxx                       
                         |                     xxx                          
                         |                 xxxx                             
                         |             xxxx                                 
                         |         xxxx                                     
                         |   xxxxxx                                         
---------------------xxxxxxxx-----------------------------------------------
         xxxxxxxxxxxx    |                                                  
xxxxxxxxx                |                                                  
\end{Verbatim}
\end{minipage}\end{center}
\noindent Example 3: 
\begin{center}\begin{minipage}{15cm}\begin{Verbatim}[frame=single]
> asciiplot(5,[-1;1]);
5
\end{Verbatim}
\end{minipage}\end{center}
\noindent Example 4: 
\begin{center}\begin{minipage}{15cm}\begin{Verbatim}[frame=single]
> asciiplot(exp(x),[1;1]);
2.71828182845904523536028747135266249775724709369998
\end{Verbatim}
\end{minipage}\end{center}
See also: \textbf{plot} (\ref{labplot})

\subsection{asin}
\label{labasin}
\noindent Name: \textbf{asin}\\
\phantom{aaa}the arcsine function.\\[0.2cm]
\noindent Library names:\\
\verb|   sollya_obj_t sollya_lib_asin(sollya_obj_t)|\\
\verb|   sollya_obj_t sollya_lib_build_function_asin(sollya_obj_t)|\\
\verb|   #define SOLLYA_ASIN(x) sollya_lib_build_function_asin(x)|\\[0.2cm]
\noindent Description: \begin{itemize}

\item \textbf{asin} is the inverse of the function \textbf{sin}: \textbf{asin}($y$) is the unique number 
   $x \in [-\pi/2; \pi/2]$ such that \textbf{sin}($x$)=$y$.

\item It is defined only for $y \in [-1;1]$.
\end{itemize}
See also: \textbf{sin} (\ref{labsin})

\subsection{asinh}
\label{labasinh}
\noindent Name: \textbf{asinh}\\
the arg-hyperbolic sine function.\\
\noindent Description: \begin{itemize}

\item \textbf{asinh} is the inverse of the function \textbf{sinh}: \textbf{asinh}($y$) is the unique number 
   $x \in [-\infty; +\infty]$ such that \textbf{sinh}($x$)=$y$.

\item It is defined for every real number y.
\end{itemize}
See also: \textbf{sinh} (\ref{labsinh})

\subsection{atan}
\label{labatan}
\noindent Name: \textbf{atan}\\
\phantom{aaa}the arctangent function.\\[0.2cm]
\noindent Library names:\\
\verb|   sollya_obj_t sollya_lib_atan(sollya_obj_t)|\\
\verb|   sollya_obj_t sollya_lib_build_function_atan(sollya_obj_t)|\\
\verb|   #define SOLLYA_ATAN(x) sollya_lib_build_function_atan(x)|\\[0.2cm]
\noindent Description: \begin{itemize}

\item \textbf{atan} is the inverse of the function \textbf{tan}: \textbf{atan}($y$) is the unique number 
   $x \in [-\pi/2; +\pi/2]$ such that \textbf{tan}($x$)=$y$.

\item It is defined for every real number y.
\end{itemize}
See also: \textbf{tan} (\ref{labtan})

\subsection{atanh}
\label{labatanh}
\noindent Name: \textbf{atanh}\\
the hyperbolic arctangent function.\\
\noindent Description: \begin{itemize}

\item \textbf{atanh} is the inverse of the function \textbf{tanh}: \textbf{atanh}(y) is the unique number 
   $x \in [-\infty; +\infty]$ such that \textbf{tanh}(x)=y.

\item It is defined only for $y \in [-1; 1]$.
\end{itemize}
See also: \textbf{tanh} (\ref{labtanh})

\subsection{autodiff}
\label{labautodiff}
\noindent Name: \textbf{autodiff}\\
nothing\\
\noindent Usage: 
\begin{center}
\textbf{autodiff}(\emph{f}, \emph{n}, \emph{I}) : (\textsf{function}, \textsf{integer}, \textsf{range}) $\rightarrow$ \textsf{list}\\
\end{center}
Parameters: 
\begin{itemize}
\item \emph{f} is the function to be approximated.
\item \emph{n} is the order of differentiation.
\item \emph{I} is the interval over which the function is differentiated.
\end{itemize}
\noindent Description: \begin{itemize}

\item Nothing.
\end{itemize}
\noindent Example 1: 
\begin{center}\begin{minipage}{15cm}\begin{Verbatim}[frame=single]
> L = autodiff(exp(x), 5, 0);
\end{Verbatim}
\end{minipage}\end{center}
See also: \textbf{diff} (\ref{labdiff})

\subsection{autosimplify}
\label{labautosimplify}
\noindent Name: \textbf{autosimplify}\\
\phantom{aaa}activates, deactivates or inspects the value of the automatic simplification state variable\\[0.2cm]
\noindent Library names:\\
\verb|   void sollya_lib_set_autosimplify_and_print(sollya_obj_t)|\\
\verb|   void sollya_lib_set_autosimplify(sollya_obj_t)|\\
\verb|   sollya_obj_t sollya_lib_get_autosimplify()|\\[0.2cm]
\noindent Usage: 
\begin{center}
\textbf{autosimplify} = \emph{activation value} : \textsf{on$|$off} $\rightarrow$ \textsf{void}\\
\textbf{autosimplify} = \emph{activation value} ! : \textsf{on$|$off} $\rightarrow$ \textsf{void}\\
\textbf{autosimplify} : \textsf{on$|$off}\\
\end{center}
Parameters: 
\begin{itemize}
\item \emph{activation value} represents \textbf{on} or \textbf{off}, i.e. activation or deactivation
\end{itemize}
\noindent Description: \begin{itemize}

\item An assignment \textbf{autosimplify} = \emph{activation value}, where \emph{activation value}
   is one of \textbf{on} or \textbf{off}, activates respectively deactivates the
   automatic safe simplification of expressions of functions generated by
   the evaluation of commands or in argument of other commands.
    
   \sollya commands like \textbf{remez}, \textbf{taylor} or \textbf{rationalapprox} sometimes
   produce expressions that can be simplified. Constant subexpressions
   can be evaluated to dyadic floating-point numbers, monomials with
   coefficients $0$ can be eliminated. Further, expressions
   indicated by the user perform better in many commands when simplified
   before being passed in argument to a command. When the automatic
   simplification of expressions is activated, \sollya automatically
   performs a safe (not value changing) simplification process on such
   expressions.
    
   The automatic generation of subexpressions can be annoying, in
   particular if it takes too much time for not enough benefit. Further the
   user might want to inspect the structure of the expression tree
   returned by a command. In this case, the automatic simplification
   should be deactivated.
    
   If the assignment \textbf{autosimplify} = \emph{activation value} is followed by an
   exclamation mark, no message indicating the new state is
   displayed. Otherwise the user is informed of the new state of the
   global mode by an indication.
\end{itemize}
\noindent Example 1: 
\begin{center}\begin{minipage}{15cm}\begin{Verbatim}[frame=single]
> autosimplify = on !;
> print(x - x);
0
> autosimplify = off ;
Automatic pure tree simplification has been deactivated.
> print(x - x);
x - x
\end{Verbatim}
\end{minipage}\end{center}
\noindent Example 2: 
\begin{center}\begin{minipage}{15cm}\begin{Verbatim}[frame=single]
> autosimplify = on !; 
> print(rationalapprox(sin(pi/5.9),7));
33 / 65
> autosimplify = off !; 
> print(rationalapprox(sin(pi/5.9),7));
33 / 65
\end{Verbatim}
\end{minipage}\end{center}
See also: \textbf{print} (\ref{labprint}), \textbf{prec} (\ref{labprec}), \textbf{points} (\ref{labpoints}), \textbf{diam} (\ref{labdiam}), \textbf{display} (\ref{labdisplay}), \textbf{verbosity} (\ref{labverbosity}), \textbf{canonical} (\ref{labcanonical}), \textbf{taylorrecursions} (\ref{labtaylorrecursions}), \textbf{timing} (\ref{labtiming}), \textbf{fullparentheses} (\ref{labfullparentheses}), \textbf{midpointmode} (\ref{labmidpointmode}), \textbf{hopitalrecursions} (\ref{labhopitalrecursions}), \textbf{remez} (\ref{labremez}), \textbf{rationalapprox} (\ref{labrationalapprox}), \textbf{taylor} (\ref{labtaylor})

\subsection{bashevaluate}
\label{labbashevaluate}
\noindent Name: \textbf{bashevaluate}\\
\phantom{aaa}executes a shell command and returns its output as a string\\[0.2cm]
\noindent Library names:\\
\verb|   sollya_obj_t sollya_lib_bashevaluate(sollya_obj_t, ...)|\\
\verb|   sollya_obj_t sollya_lib_v_bashevaluate(sollya_obj_t, va_list)|\\[0.2cm]
\noindent Usage: 
\begin{center}
\textbf{bashevaluate}(\emph{command}) : \textsf{string} $\rightarrow$ \textsf{string}\\
\textbf{bashevaluate}(\emph{command},\emph{input}) : (\textsf{string}, \textsf{string}) $\rightarrow$ \textsf{string}\\
\end{center}
Parameters: 
\begin{itemize}
\item \emph{command} is a command to be interpreted by the shell.
\item \emph{input} is an optional character sequence to be fed to the command.
\end{itemize}
\noindent Description: \begin{itemize}

\item \textbf{bashevaluate}(\emph{command}) will execute the shell command \emph{command} in a shell.
   All output on the command's standard output is collected and returned 
   as a character sequence.

\item If an additional argument \emph{input} is given in a call to
   \textbf{bashevaluate}(\emph{command},\emph{input}), this character sequence is written to the
   standard input of the command \emph{command} that gets executed.

\item All characters output by \emph{command} are included in the character
   sequence to which \textbf{bashevaluate} evaluates but two exceptions. Every NULL
   character ($`\backslash 0$') in the output is replaced with
   `?' as \sollya is unable to handle character sequences containing that
   character. Additionally, if the output ends in a newline character
   ($`\backslash$n'), this character is stripped off. Other
   newline characters which are not at the end of the output are left as
   such.
\end{itemize}
\noindent Example 1: 
\begin{center}\begin{minipage}{15cm}\begin{Verbatim}[frame=single]
> bashevaluate("LANG=C date");
Fri Jan 11 13:58:35 CET 2013
\end{Verbatim}
\end{minipage}\end{center}
\noindent Example 2: 
\begin{center}\begin{minipage}{15cm}\begin{Verbatim}[frame=single]
> [| bashevaluate("echo Hello") |];
[|"Hello"|]
\end{Verbatim}
\end{minipage}\end{center}
\noindent Example 3: 
\begin{center}\begin{minipage}{15cm}\begin{Verbatim}[frame=single]
> a = bashevaluate("sed -e 's/a/e/g;'", "Hallo");
> a;
Hello
\end{Verbatim}
\end{minipage}\end{center}
See also: \textbf{bashexecute} (\ref{labbashexecute})

\subsection{bashexecute}
\label{labbashexecute}
\noindent Name: \textbf{bashexecute}\\
\phantom{aaa}executes a shell command.\\[0.2cm]
\noindent Library name:\\
\verb|   void sollya_lib_bashexecute(sollya_obj_t)|\\[0.2cm]
\noindent Usage: 
\begin{center}
\textbf{bashexecute}(\emph{command}) : \textsf{string} $\rightarrow$ \textsf{void}\\
\end{center}
Parameters: 
\begin{itemize}
\item \emph{command} is a command to be interpreted by the shell.
\end{itemize}
\noindent Description: \begin{itemize}

\item \textbf{bashexecute}(\emph{command}) lets the shell interpret \emph{command}. It is useful to execute
   some external code within \sollya.

\item \textbf{bashexecute} does not return anything. It just executes its argument. However, if
   \emph{command} produces an output in a file, this result can be imported in \sollya
   with help of commands like \textbf{execute}, \textbf{readfile} and \textbf{parse}.
\end{itemize}
\noindent Example 1: 
\begin{center}\begin{minipage}{15cm}\begin{Verbatim}[frame=single]
> bashexecute("LANG=C date");
Thu May 23 16:40:51 CEST 2013
\end{Verbatim}
\end{minipage}\end{center}
See also: \textbf{execute} (\ref{labexecute}), \textbf{readfile} (\ref{labreadfile}), \textbf{parse} (\ref{labparse}), \textbf{bashevaluate} (\ref{labbashevaluate})

\subsection{binary}
\label{labbinary}
\noindent Name: \textbf{hexadecimal}\\
special value for global state \textbf{display}\\
\noindent Description: \begin{itemize}

\item \textbf{hexadecimal} is a special value used for the global state \textbf{display}.  If
   the global state \textbf{display} is equal to \textbf{hexadecimal}, all data will be
   output in binary notation.
    
   As any value it can be affected to a variable and stored in lists.
\end{itemize}
See also: \textbf{decimal} (\ref{labdecimal}), \textbf{dyadic} (\ref{labdyadic}), \textbf{powers} (\ref{labpowers}), \textbf{hexadecimal} (\ref{labhexadecimal})

\subsection{boolean}
\label{labboolean}
\noindent Name: \textbf{boolean}\\
\phantom{aaa}keyword representing a \textsf{boolean} type \\[0.2cm]
\noindent Usage: 
\begin{center}
\textbf{boolean} : \textsf{type type}\\
\end{center}
\noindent Description: \begin{itemize}

\item \textbf{boolean} represents the \textsf{boolean} type for declarations
   of external procedures by means of \textbf{externalproc}.
    
   Remark that in contrast to other indicators, type indicators like
   \textbf{boolean} cannot be handled outside the \textbf{externalproc} context.  In
   particular, they cannot be assigned to variables.
\end{itemize}
See also: \textbf{externalproc} (\ref{labexternalproc}), \textbf{constant} (\ref{labconstant}), \textbf{function} (\ref{labfunction}), \textbf{integer} (\ref{labinteger}), \textbf{list of} (\ref{lablistof}), \textbf{range} (\ref{labrange}), \textbf{string} (\ref{labstring})

\subsection{canonical}
\label{labcanonical}
\noindent Name: \textbf{canonical}\\
\phantom{aaa}brings all polynomial subexpressions of an expression to canonical form or activates, deactivates or checks canonical form printing\\[0.2cm]
\noindent Library names:\\
\verb|   void sollya_lib_set_canonical_and_print(sollya_obj_t)|\\
\verb|   void sollya_lib_set_canonical(sollya_obj_t)|\\
\verb|   sollya_obj_t sollya_lib_canonical(sollya_obj_t)|\\
\verb|   sollya_obj_t sollya_lib_get_canonical()|\\[0.2cm]
\noindent Usage: 
\begin{center}
\textbf{canonical}(\emph{function}) : \textsf{function} $\rightarrow$ \textsf{function}\\
\textbf{canonical} = \emph{activation value} : \textsf{on$|$off} $\rightarrow$ \textsf{void}\\
\textbf{canonical} = \emph{activation value} ! : \textsf{on$|$off} $\rightarrow$ \textsf{void}\\
\end{center}
Parameters: 
\begin{itemize}
\item \emph{function} represents the expression to be rewritten in canonical form
\item \emph{activation value} represents \textbf{on} or \textbf{off}, i.e. activation or deactivation
\end{itemize}
\noindent Description: \begin{itemize}

\item The command \textbf{canonical} rewrites the expression representing the function
   \emph{function} in a way such that all polynomial subexpressions (or the
   whole expression itself, if it is a polynomial) are written in
   canonical form, i.e. as a sum of monomials in the canonical base. The
   canonical base is the base of the integer powers of the global free
   variable. The command \textbf{canonical} does not endanger the safety of
   computations even in \sollya's floating-point environment: the
   function returned is mathematically equal to the function \emph{function}.

\item An assignment \textbf{canonical} = \emph{activation value}, where \emph{activation value}
   is one of \textbf{on} or \textbf{off}, activates respectively deactivates the
   automatic printing of polynomial expressions in canonical form,
   i.e. as a sum of monomials in the canonical base. If automatic
   printing in canonical form is deactivated, automatic printing yields to
   displaying polynomial subexpressions in Horner form.
    
   If the assignment \textbf{canonical} = \emph{activation value} is followed by an
   exclamation mark, no message indicating the new state is
   displayed. Otherwise the user is informed of the new state of the
   global mode by an indication.
\end{itemize}
\noindent Example 1: 
\begin{center}\begin{minipage}{15cm}\begin{Verbatim}[frame=single]
> print(canonical(1 + x * (x + 3 * x^2)));
1 + x^2 + 3 * x^3
> print(canonical((x + 1)^7));
1 + 7 * x + 21 * x^2 + 35 * x^3 + 35 * x^4 + 21 * x^5 + 7 * x^6 + x^7
\end{Verbatim}
\end{minipage}\end{center}
\noindent Example 2: 
\begin{center}\begin{minipage}{15cm}\begin{Verbatim}[frame=single]
> print(canonical(exp((x + 1)^5) - log(asin(((x + 2) + x)^4 * (x + 1)) + x)));
exp(1 + 5 * x + 10 * x^2 + 10 * x^3 + 5 * x^4 + x^5) - log(asin(16 + 80 * x + 16
0 * x^2 + 160 * x^3 + 80 * x^4 + 16 * x^5) + x)
\end{Verbatim}
\end{minipage}\end{center}
\noindent Example 3: 
\begin{center}\begin{minipage}{15cm}\begin{Verbatim}[frame=single]
> canonical;
off
> (x + 2)^9;
512 + x * (2304 + x * (4608 + x * (5376 + x * (4032 + x * (2016 + x * (672 + x *
 (144 + x * (18 + x))))))))
> canonical = on;
Canonical automatic printing output has been activated.
> (x + 2)^9;
512 + 2304 * x + 4608 * x^2 + 5376 * x^3 + 4032 * x^4 + 2016 * x^5 + 672 * x^6 +
 144 * x^7 + 18 * x^8 + x^9
> canonical;
on
> canonical = off!;
> (x + 2)^9;
512 + x * (2304 + x * (4608 + x * (5376 + x * (4032 + x * (2016 + x * (672 + x *
 (144 + x * (18 + x))))))))
\end{Verbatim}
\end{minipage}\end{center}
See also: \textbf{horner} (\ref{labhorner}), \textbf{print} (\ref{labprint}), \textbf{autosimplify} (\ref{labautosimplify})

\subsection{ceil}
\label{labceil}
\noindent Name: \textbf{ceil}\\
the usual function ceil.\\
\noindent Description: \begin{itemize}

\item \textbf{ceil} is defined as usual: \textbf{ceil}(x) is the smallest integer y such that $y \ge x$.

\item It is defined for every real number x.
\end{itemize}
See also: \textbf{floor} (\ref{labfloor})

\subsection{checkinfnorm}
\label{labcheckinfnorm}
\noindent Name: \textbf{checkinfnorm}\\
\phantom{aaa}checks whether the infinity norm of a function is bounded by a value\\[0.2cm]
\noindent Library name:\\
\verb|   sollya_obj_t sollya_lib_checkinfnorm(sollya_obj_t, sollya_obj_t, sollya_obj_t)|\\[0.2cm]
\noindent Usage: 
\begin{center}
\textbf{checkinfnorm}(\emph{function},\emph{range},\emph{constant}) : (\textsf{function}, \textsf{range}, \textsf{constant}) $\rightarrow$ \textsf{boolean}\\
\end{center}
Parameters: 
\begin{itemize}
\item \emph{function} represents the function whose infinity norm is to be checked
\item \emph{range} represents the infinity norm is to be considered on
\item \emph{constant} represents the upper bound the infinity norm is to be checked to
\end{itemize}
\noindent Description: \begin{itemize}

\item The command \textbf{checkinfnorm} checks whether the infinity norm of the given
   function \emph{function} in the range \emph{range} can be proven (by \sollya) to
   be less than the given bound \emph{bound}. This means, if \textbf{checkinfnorm}
   evaluates to \textbf{true}, the infinity norm has been proven (by \sollya's
   interval arithmetic) to be less than the bound. If \textbf{checkinfnorm} evaluates
   to \textbf{false}, there are two possibilities: either the bound is less than
   or equal to the infinity norm of the function or the bound is greater
   than the infinity norm but \sollya could not conclude using its
   internal interval arithmetic.
    
   \textbf{checkinfnorm} is sensitive to the global variable \textbf{diam}. The smaller \textbf{diam},
   the more time \sollya will spend on the evaluation of \textbf{checkinfnorm} in
   order to prove the bound before returning \textbf{false} although the infinity
   norm is bounded by the bound. If \textbf{diam} is equal to $0$, \sollya will
   eventually spend infinite time on instances where the given bound
   \emph{bound} is less or equal to the infinity norm of the function
   \emph{function} in range \emph{range}. In contrast, with \textbf{diam} being zero,
   \textbf{checkinfnorm} evaluates to \textbf{true} iff the infinity norm of the function in
   the range is bounded by the given bound.
\end{itemize}
\noindent Example 1: 
\begin{center}\begin{minipage}{15cm}\begin{Verbatim}[frame=single]
> checkinfnorm(sin(x),[0;1.75], 1);
true
> checkinfnorm(sin(x),[0;1.75], 1/2); checkinfnorm(sin(x),[0;20/39],1/2);
false
true
\end{Verbatim}
\end{minipage}\end{center}
\noindent Example 2: 
\begin{center}\begin{minipage}{15cm}\begin{Verbatim}[frame=single]
> p = remez(exp(x), 5, [-1;1]);
> b = dirtyinfnorm(p - exp(x), [-1;1]);
> checkinfnorm(p - exp(x), [-1;1], b);
false
> b1 = round(b, 15, RU);
> checkinfnorm(p - exp(x), [-1;1], b1);
true
> b2 = round(b, 25, RU);
> checkinfnorm(p - exp(x), [-1;1], b2);
false
> diam = 1b-20!;
> checkinfnorm(p - exp(x), [-1;1], b2);
true
\end{Verbatim}
\end{minipage}\end{center}
See also: \textbf{infnorm} (\ref{labinfnorm}), \textbf{dirtyinfnorm} (\ref{labdirtyinfnorm}), \textbf{supnorm} (\ref{labsupnorm}), \textbf{accurateinfnorm} (\ref{labaccurateinfnorm}), \textbf{remez} (\ref{labremez}), \textbf{diam} (\ref{labdiam})

\subsection{coeff}
\label{labcoeff}
\noindent Name: \textbf{coeff}\\
gives the coefficient of degree n of a polynomial\\

\noindent Usage: 
\begin{center}
\textbf{coeff}(\emph{f},\emph{n}) : (\textsf{function}, \textsf{integer}) $\rightarrow$ \textsf{constant}\\
\end{center}
Parameters: 
\begin{itemize}
\item \emph{f} is a function (usually a polynomial).
\item \emph{n} is an integer
\end{itemize}
\noindent Description: \begin{itemize}

\item If \emph{f} is a polynomial, \textbf{coeff}(\emph{f}, \emph{n}) returns the coefficient of
   degree \emph{n} in \emph{f}.

\item If \emph{f} is a function that is not a polynomial, \textbf{coeff}(\emph{f}, \emph{n}) returns 0.
\end{itemize}
\noindent Example 1: 
\begin{center}\begin{minipage}{15cm}\begin{Verbatim}[frame=single]
> coeff((1+x)^5,3);
10
\end{Verbatim}
\end{minipage}\end{center}
\noindent Example 2: 
\begin{center}\begin{minipage}{15cm}\begin{Verbatim}[frame=single]
> coeff(sin(x),0);
0
\end{Verbatim}
\end{minipage}\end{center}
See also: \textbf{degree} (\ref{labdegree})

\subsection{@}
\label{labconcat}
\noindent Name: \textbf{@}\\
concatenates two lists or strings or applies a list as arguments to a procedure\\
\noindent Usage: 
\begin{center}
\emph{L1}\textbf{@}\emph{L2} : (\textsf{list}, \textsf{list}) $\rightarrow$ \textsf{list}\\
\emph{string1}\textbf{@}\emph{string2} : (\textsf{string}, \textsf{string}) $\rightarrow$ \textsf{string}\\
\emph{proc}\textbf{@}\emph{L1} : (\textsf{procedure}, \textsf{list}) $\rightarrow$ \textsf{any type}\\
\end{center}
Parameters: 
\begin{itemize}
\item \emph{L1} and \emph{L2} are two lists.
\item \emph{string1} and \emph{string2} are two strings.
\item \emph{proc} is a procedure.
\end{itemize}
\noindent Description: \begin{itemize}

\item In its first usage form, \\textbf{@} concatenates two lists or strings.\n
\item In its second usage form, \\textbf{@} applies the elements of a list as\n   arguments to a procedure. In the case when \\emph{proc} is a procedure \n   with a fixed number of arguments, a check is done if the number of\n   elements in the list corresponds to the number of formal parameters\n   of the procedure. An empty list can therefore applied only to a \n   procedure that does not take any argument. In the case of a \n   procedure with an arbitrary number of arguments, no such check is \n   performed.\n\end{itemize}
\noindent Example 1: 
\begin{center}\begin{minipage}{15cm}\begin{Verbatim}[frame=single]
\end{Verbatim}
\end{minipage}\end{center}
\noindent Example 2: 
\begin{center}\begin{minipage}{15cm}\begin{Verbatim}[frame=single]
\end{Verbatim}
\end{minipage}\end{center}
\noindent Example 3: 
\begin{center}\begin{minipage}{15cm}\begin{Verbatim}[frame=single]
\end{Verbatim}
\end{minipage}\end{center}
\noindent Example 4: 
\begin{center}\begin{minipage}{15cm}\begin{Verbatim}[frame=single]
\end{Verbatim}
\end{minipage}\end{center}
\noindent Example 5: 
\begin{center}\begin{minipage}{15cm}\begin{Verbatim}[frame=single]
\end{Verbatim}
\end{minipage}\end{center}
See also: \textbf{.:} (\ref{labprepend}), \textbf{:.} (\ref{labappend}), \textbf{procedure} (\ref{labprocedure}), \textbf{proc} (\ref{labproc})

\subsection{constant}
\label{labconstant}
\noindent Name: \textbf{constant}\\
keyword representing a \textsf{constant} type \\

\noindent Usage: 
\begin{center}
\textbf{constant} : \textsf{type type}\\
\end{center}
\noindent Description: \begin{itemize}

\item \textbf{constant} represents the \textsf{constant} type for declarations
   of external procedures by means of \textbf{externalproc}.
   Remark that in contrast to other indicators, type indicators like
   \textbf{constant} cannot be handled outside the \textbf{externalproc} context.  In
   particular, they cannot be assigned to variables.
\end{itemize}
See also: \textbf{externalproc} (\ref{labexternalproc}), \textbf{boolean} (\ref{labboolean}), \textbf{function} (\ref{labfunction}), \textbf{integer} (\ref{labinteger}), \textbf{list of} (\ref{lablistof}), \textbf{range} (\ref{labrange}), \textbf{string} (\ref{labstring})

\subsection{cos}
\label{labcos}
\noindent Name: \textbf{cos}\\
the cosine function.\\
\noindent Description: \begin{itemize}

\item \textbf{cos} is the usual cosine function.

\item It is defined for every real number $x$.
\end{itemize}
See also: \textbf{acos} (\ref{labacos}), \textbf{sin} (\ref{labsin}), \textbf{tan} (\ref{labtan})

\subsection{cosh}
\label{labcosh}
\noindent Name: \textbf{cosh}\\
the hyperbolic cosine function.\\
\noindent Description: \begin{itemize}

\item \textbf{cosh} is the usual hyperbolic function: $\cosh(x) = \frac{e^x + e^{-x}}{2}$.

\item It is defined for every real number $x$.
\end{itemize}
See also: \textbf{acosh} (\ref{labacosh}), \textbf{sinh} (\ref{labsinh}), \textbf{tanh} (\ref{labtanh}), \textbf{exp} (\ref{labexp})

\subsection{D}
\label{labd}
\noindent Name: \textbf{D}\\
\phantom{aaa}short form for \textbf{double}\\[0.2cm]
See also: \textbf{double} (\ref{labdouble})

\subsection{DD}
\label{labdd}
\noindent Name: \textbf{DD}\\
short form for \textbf{doubledouble}\\
See also: \textbf{doubledouble} (\ref{labdoubledouble})

\subsection{DE}
\label{labde}
\noindent Name: \textbf{DE}\\
short form for \textbf{doubleextended}\\
See also: \textbf{doubleextended} (\ref{labdoubleextended})

\subsection{decimal}
\label{labdecimal}
\noindent Name: \textbf{decimal}\\
special value for global state \textbf{display}\\
\noindent Description: \begin{itemize}

\item \textbf{decimal} is a special value used for the global state \textbf{display}.
   If the global state \textbf{display} is equal to \textbf{decimal}, all data will
   be output in decimal notation.
    
   As any value it can be affected to a variable and stored in lists.
\end{itemize}
See also: \textbf{dyadic} (\ref{labdyadic}), \textbf{powers} (\ref{labpowers}), \textbf{hexadecimal} (\ref{labhexadecimal}), \textbf{binary} (\ref{labbinary})

\subsection{default}
\label{labdefault}
\noindent Name: \textbf{default}\\
\phantom{aaa}default value for some commands.\\[0.2cm]
\noindent Library names:\\
\verb|   sollya_obj_t sollya_lib_default()|\\
\verb|   int sollya_lib_is_default(sollya_obj_t)|\\[0.2cm]
\noindent Description: \begin{itemize}

\item \textbf{default} is a special value and is replaced by something depending on the 
   context where it is used. It can often be used as a joker, when you want to 
   specify one of the optional parameters of a command and not the others: set 
   the value of uninteresting parameters to \textbf{default}.

\item Global variables can be reset by affecting them the special value \textbf{default}.
\end{itemize}
\noindent Example 1: 
\begin{center}\begin{minipage}{15cm}\begin{Verbatim}[frame=single]
> p = remez(exp(x),5,[0;1],default,1e-5);
> q = remez(exp(x),5,[0;1],1,1e-5);
> p==q;
true
\end{Verbatim}
\end{minipage}\end{center}
\noindent Example 2: 
\begin{center}\begin{minipage}{15cm}\begin{Verbatim}[frame=single]
> prec;
165
> prec=200;
The precision has been set to 200 bits.
\end{Verbatim}
\end{minipage}\end{center}

\subsection{degree}
\label{labdegree}
\noindent Name: \textbf{degree}\\
gives the degree of a polynomial.\\
\noindent Usage: 
\begin{center}
\textbf{degree}(\emph{f}) : \textsf{function} $\rightarrow$ \textsf{integer}\\
\end{center}
Parameters: 
\begin{itemize}
\item \emph{f} is a function (usually a polynomial).
\end{itemize}
\noindent Description: \begin{itemize}

\item If \\emph{f} is a polynomial, \\textbf{degree}(\\emph{f}) returns the degree of \\emph{f}.\n
\item Contrary to the usage, \\sollya considers that the degree of the null polynomial\n   is 0.\n
\item If \\emph{f} is a function that is not a polynomial, \\textbf{degree}(\\emph{f}) returns -1.\n\end{itemize}
\noindent Example 1: 
\begin{center}\begin{minipage}{15cm}\begin{Verbatim}[frame=single]
\end{Verbatim}
\end{minipage}\end{center}
\noindent Example 2: 
\begin{center}\begin{minipage}{15cm}\begin{Verbatim}[frame=single]
\end{Verbatim}
\end{minipage}\end{center}
See also: \textbf{coeff} (\ref{labcoeff})

\subsection{denominator}
\label{labdenominator}
\noindent Name: \textbf{denominator}\\
gives the denominator of an expression\\
\noindent Usage: 
\begin{center}
\textbf{denominator}(\emph{expr}) : \textsf{function} $\rightarrow$ \textsf{function}\\
\end{center}
Parameters: 
\begin{itemize}
\item \emph{expr} represents an expression
\end{itemize}
\noindent Description: \begin{itemize}

\item If \emph{expr} represents a fraction \emph{expr1}/\emph{expr2}, \textbf{denominator}(\emph{expr})
   returns the denominator of this fraction, i.e. \emph{expr2}.
    
   If \emph{expr} represents something else, \textbf{denominator}(\emph{expr}) 
   returns 1.
    
   Note that for all expressions \emph{expr}, \textbf{numerator}(\emph{expr}) \textbf{/} \textbf{denominator}(\emph{expr})
   is equal to \emph{expr}.
\end{itemize}
\noindent Example 1: 
\begin{center}\begin{minipage}{15cm}\begin{Verbatim}[frame=single]
> denominator(5/3);
3
\end{Verbatim}
\end{minipage}\end{center}
\noindent Example 2: 
\begin{center}\begin{minipage}{15cm}\begin{Verbatim}[frame=single]
> denominator(exp(x));
1
\end{Verbatim}
\end{minipage}\end{center}
\noindent Example 3: 
\begin{center}\begin{minipage}{15cm}\begin{Verbatim}[frame=single]
> a = 5/3;
> b = numerator(a)/denominator(a);
> print(a);
5 / 3
> print(b);
5 / 3
\end{Verbatim}
\end{minipage}\end{center}
\noindent Example 4: 
\begin{center}\begin{minipage}{15cm}\begin{Verbatim}[frame=single]
> a = exp(x/3);
> b = numerator(a)/denominator(a);
> print(a);
exp(x / 3)
> print(b);
exp(x / 3)
\end{Verbatim}
\end{minipage}\end{center}
See also: \textbf{numerator} (\ref{labnumerator})

\subsection{diam}
\label{labdiam}
\noindent Name: \textbf{diam}\\
parameter used in safe algorithms of \sollya and controlling the maximal length of the involved intervals.\\
\noindent Usage: 
\begin{center}
\textbf{diam} = \emph{width} : \textsf{constant} $\rightarrow$ \textsf{void}\\
\textbf{diam} = \emph{width} ! : \textsf{constant} $\rightarrow$ \textsf{void}\\
\textbf{diam} : \textsf{constant}\\
\end{center}
Parameters: 
\begin{itemize}
\item \emph{width} represents the maximal relative width of the intervals used
\end{itemize}
\noindent Description: \begin{itemize}

\item \textbf{diam} is a global variable. Its value represents the maximal width allowed
   for intervals involved in safe algorithms of \sollya (namely \textbf{infnorm},
   \textbf{checkinfnorm}, \textbf{accurateinfnorm}, \textbf{integral}, \textbf{findzeros}, \textbf{supnorm}).

\item More precisely, \textbf{diam} is relative to the width of the input interval of
   the command. For instance, suppose that \textbf{diam}=1e-5: if \textbf{infnorm} is called
   on interval $[0,\,1]$, the maximal width of an interval will be 1e-5. But if it
   is called on interval $[0,\,1\mathrm{e}{-3}]$, the maximal width will be 1e-8.
\end{itemize}
See also: \textbf{infnorm} (\ref{labinfnorm}), \textbf{checkinfnorm} (\ref{labcheckinfnorm}), \textbf{accurateinfnorm} (\ref{labaccurateinfnorm}), \textbf{integral} (\ref{labintegral}), \textbf{findzeros} (\ref{labfindzeros}), \textbf{diam} (\ref{labdiam})

\subsection{dieonerrormode}
\label{labdieonerrormode}
\noindent Name: \textbf{dieonerrormode}\\
\phantom{aaa}global variable controlling if \sollya is exited on an error or not.\\[0.2cm]
\noindent Library names:\\
\verb|   void sollya_lib_set_dieonerrormode_and_print(sollya_obj_t)|\\
\verb|   void sollya_lib_set_dieonerrormode(sollya_obj_t)|\\
\verb|   sollya_obj_t sollya_lib_get_dieonerrormode()|\\[0.2cm]
\noindent Usage: 
\begin{center}
\textbf{dieonerrormode} = \emph{activation value} : \textsf{on$|$off} $\rightarrow$ \textsf{void}\\
\textbf{dieonerrormode} = \emph{activation value} ! : \textsf{on$|$off} $\rightarrow$ \textsf{void}\\
\textbf{dieonerrormode} : \textsf{on$|$off}\\
\end{center}
Parameters: 
\begin{itemize}
\item \emph{activation value} controls if \sollya is exited on an error or not.
\end{itemize}
\noindent Description: \begin{itemize}

\item \textbf{dieonerrormode} is a global variable. When its value is \textbf{off}, which is the default,
   \sollya will not exit on any syntax, typing, side-effect errors. These
   errors will be caught by the tool, even if a memory might be leaked at 
   that point. On evaluation, the \textbf{error} special value will be produced.

\item When the value of the \textbf{dieonerrormode} variable is \textbf{on}, \sollya will exit
   on any syntax, typing, side-effect errors. A warning message will
   be printed in these cases at appropriate \textbf{verbosity} levels. 
\end{itemize}
\noindent Example 1: 
\begin{center}\begin{minipage}{15cm}\begin{Verbatim}[frame=single]
> verbosity = 1!;
> dieonerrormode = off;
Die-on-error mode has been deactivated.
> for i from true to false do i + "Salut";
Warning: one of the arguments of the for loop does not evaluate to a constant.
The for loop will not be executed.
> exp(17);
Warning: rounding has happened. The value displayed is a faithful rounding to 16
5 bits of the true result.
2.41549527535752982147754351803858238798675673527224e7
\end{Verbatim}
\end{minipage}\end{center}
\noindent Example 2: 
\begin{center}\begin{minipage}{15cm}\begin{Verbatim}[frame=single]
> verbosity = 1!;
> dieonerrormode = off!;
> 5 */  4;
Warning: syntax error, unexpected /.
The last symbol read has been "/".
Will skip input until next semicolon after the unexpected token. May leak memory
.
  exp(17);
Warning: rounding has happened. The value displayed is a faithful rounding to 16
5 bits of the true result.
2.41549527535752982147754351803858238798675673527224e7
\end{Verbatim}
\end{minipage}\end{center}
\noindent Example 3: 
\begin{center}\begin{minipage}{15cm}\begin{Verbatim}[frame=single]
> verbosity = 1!;
> dieonerrormode;
off
> dieonerrormode = on!;
> dieonerrormode;
on
> for i from true to false do i + "Salut";
Warning: one of the arguments of the for loop does not evaluate to a constant.
The for loop will not be executed.
Warning: some syntax, typing or side-effect error has occurred.
As the die-on-error mode is activated, the tool will be exited.
\end{Verbatim}
\end{minipage}\end{center}
\noindent Example 4: 
\begin{center}\begin{minipage}{15cm}\begin{Verbatim}[frame=single]
> verbosity = 1!;
> dieonerrormode = on!;
> 5 */  4;
Warning: syntax error, unexpected /.
The last symbol read has been "/".
Will skip input until next semicolon after the unexpected token. May leak memory
.
Warning: some syntax, typing or side-effect error has occurred.
As the die-on-error mode is activated, the tool will be exited.
\end{Verbatim}
\end{minipage}\end{center}
\noindent Example 5: 
\begin{center}\begin{minipage}{15cm}\begin{Verbatim}[frame=single]
> verbosity = 0!;
> dieonerrormode = on!;
> 5 */  4;
\end{Verbatim}
\end{minipage}\end{center}
See also: \textbf{on} (\ref{labon}), \textbf{off} (\ref{laboff}), \textbf{verbosity} (\ref{labverbosity}), \textbf{error} (\ref{laberror})

\subsection{diff}
\label{labdiff}
\noindent Name: \textbf{diff}\\
\phantom{aaa}differentiation operator\\[0.2cm]
\noindent Library name:\\
\verb|   sollya_obj_t sollya_lib_diff(sollya_obj_t)|\\[0.2cm]
\noindent Usage: 
\begin{center}
\textbf{diff}(\emph{function}) : \textsf{function} $\rightarrow$ \textsf{function}\\
\end{center}
Parameters: 
\begin{itemize}
\item \emph{function} represents a function
\end{itemize}
\noindent Description: \begin{itemize}

\item \textbf{diff}(\emph{function}) returns the symbolic derivative of the function
   \emph{function} by the global free variable.
    
   If \emph{function} represents a function symbol that is externally bound
   to some code by \textbf{library}, the derivative is performed as a symbolic
   annotation to the returned expression tree.
\end{itemize}
\noindent Example 1: 
\begin{center}\begin{minipage}{15cm}\begin{Verbatim}[frame=single]
> diff(sin(x));
cos(x)
\end{Verbatim}
\end{minipage}\end{center}
\noindent Example 2: 
\begin{center}\begin{minipage}{15cm}\begin{Verbatim}[frame=single]
> diff(x);
1
\end{Verbatim}
\end{minipage}\end{center}
\noindent Example 3: 
\begin{center}\begin{minipage}{15cm}\begin{Verbatim}[frame=single]
> diff(x^x);
x^x * (1 + log(x))
\end{Verbatim}
\end{minipage}\end{center}
See also: \textbf{library} (\ref{lablibrary}), \textbf{autodiff} (\ref{labautodiff}), \textbf{taylor} (\ref{labtaylor}), \textbf{taylorform} (\ref{labtaylorform})

\subsection{dirtyfindzeros}
\label{labdirtyfindzeros}
\noindent Name: \textbf{dirtyfindzeros}\\
gives a list of numerical values listing the zeros of a function on an interval.\\
\noindent Usage: 
\begin{center}
\textbf{dirtyfindzeros}(\emph{f},\emph{I}) : (\textsf{function}, \textsf{range}) $\rightarrow$ \textsf{list}\\
\end{center}
Parameters: 
\begin{itemize}
\item \emph{f} is a function.
\item \emph{I} is an interval.
\end{itemize}
\noindent Description: \begin{itemize}

\item \textbf{dirtyfindzeros}(\emph{f},\emph{I}) returns a list containing some zeros of \emph{f} in the
   interval \emph{I}. The values in the list are numerical approximation of the exact
   zeros. The precision of these approximations is approximately the precision
   stored in \textbf{prec}. If \emph{f} does not have two zeros very close to each other, it 
   can be expected that all zeros are listed. However, some zeros may be
   forgotten. This command should be considered as a numerical algorithm and
   should not be used if safety is critical.

\item More precisely, the algorithm relies on global variables \textbf{prec} and \textbf{points} and it performs the following steps: 
   let $n$ be the value of variable \textbf{points} and $t$ be the value
   of variable \textbf{prec}.
   \begin{itemize}
   \item Evaluate $|f|$ at $n$ evenly distributed points in the interval $I$.
     The working precision to be used is automatically chosen in order to ensure that the sign
     is correct.
   \item Whenever $f$ changes its sign for two consecutive points,
     find an approximation $x$ of its zero with precision $t$ using
     Newton's algorithm. The number of steps in Newton's iteration depends on $t$:
     the precision of the approximation is supposed to be doubled at each step.
   \item Add this value to the list.
   \end{itemize}
\end{itemize}
\noindent Example 1: 
\begin{center}\begin{minipage}{15cm}\begin{Verbatim}[frame=single]
> dirtyfindzeros(sin(x),[-5;5]);
[|-3.14159265358979323846264338327950288419716939937508, 0, 3.141592653589793238
46264338327950288419716939937508|]
\end{Verbatim}
\end{minipage}\end{center}
\noindent Example 2: 
\begin{center}\begin{minipage}{15cm}\begin{Verbatim}[frame=single]
> L1=dirtyfindzeros(x^2*sin(1/x),[0;1]);
> points=1000!;
> L2=dirtyfindzeros(x^2*sin(1/x),[0;1]);
> length(L1); length(L2);
18
25
\end{Verbatim}
\end{minipage}\end{center}
See also: \textbf{prec} (\ref{labprec}), \textbf{points} (\ref{labpoints}), \textbf{findzeros} (\ref{labfindzeros}), \textbf{dirtyinfnorm} (\ref{labdirtyinfnorm}), \textbf{numberroots} (\ref{labnumberroots})

\subsection{dirtyinfnorm}
\label{labdirtyinfnorm}
\noindent Name: \textbf{dirtyinfnorm}\\
computes a numerical approximation of the infinite norm of a function on an interval.\\

\noindent Usage: 
\begin{center}
\textbf{dirtyinfnorm}(\emph{f},\emph{I}) : (\textsf{function}, \textsf{range}) $\rightarrow$ \textsf{constant}\\
\end{center}
Parameters: 
\begin{itemize}
\item \emph{f} is a function.
\item \emph{I} is an interval.
\end{itemize}
\noindent Description: \begin{itemize}

\item \textbf{dirtyinfnorm}(\emph{f},\emph{I}) computes an approximation of the infinite norm of the 
   given function $f$ on the interval $I$, e.g. $\max_{x \in I} \{|f(x)|\}$.

\item The interval must be bound. If the interval contains one of -Inf or +Inf, the 
   result of \textbf{dirtyinfnorm} is NaN.

\item The result of this command depends on the global variables \textbf{prec} and \textbf{points}.
   Therefore, the returned result is generally a good approximation of the exact
   infinite norm, with precision \textbf{prec}. However, the result is generally 
   underestimated and should not be used when safety is critical.
   Use \textbf{infnorm} instead.

\item The following algorithm is used: let $n$ be the value of variable \textbf{points}
   and $t$ be the value of variable \textbf{prec}.
   \begin{itemize}
   \item  Evaluate $|f|$ at $n$ evenly distributed points in the
     interval $I$. The evaluation are faithful roundings of the exact
     results at precision $t$.
   \item  Whenever the derivative of $f$ changes its sign for two consecutive 
     points, find an approximation $x$ of its zero with precision $t$.
     Then compute a faithful rounding of $|f(x)|$ at precision $t$.
   \item  Return the maximum of all computed values.
   \end{itemize}
\end{itemize}
\noindent Example 1: 
\begin{center}\begin{minipage}{15cm}\begin{Verbatim}[frame=single]
> dirtyinfnorm(sin(x),[-10;10]);
1
\end{Verbatim}
\end{minipage}\end{center}
\noindent Example 2: 
\begin{center}\begin{minipage}{15cm}\begin{Verbatim}[frame=single]
> prec=15!;
> dirtyinfnorm(exp(cos(x))*sin(x),[0;5]);
1.45856
> prec=40!;
> dirtyinfnorm(exp(cos(x))*sin(x),[0;5]);
1.458528537135
> prec=100!;
> dirtyinfnorm(exp(cos(x))*sin(x),[0;5]);
1.458528537136237644438147455024
> prec=200!;
> dirtyinfnorm(exp(cos(x))*sin(x),[0;5]);
1.458528537136237644438147455023841718299214087993682374094153
\end{Verbatim}
\end{minipage}\end{center}
\noindent Example 3: 
\begin{center}\begin{minipage}{15cm}\begin{Verbatim}[frame=single]
> dirtyinfnorm(x^2, [log(0);log(1)]);
@NaN@
\end{Verbatim}
\end{minipage}\end{center}
See also: \textbf{prec} (\ref{labprec}), \textbf{points} (\ref{labpoints}), \textbf{infnorm} (\ref{labinfnorm}), \textbf{checkinfnorm} (\ref{labcheckinfnorm})

\subsection{ dirtyintegral }
\noindent Name: \textbf{dirtyintegral}\\
computes a numerical approximation of the integral of a function on an interval.\\

\noindent Usage: 
\begin{center}
\textbf{dirtyintegral}(\emph{f},\emph{I}) : (\textsf{function}, \textsf{range}) $\rightarrow$ \textsf{constant}\\
\end{center}
Parameters: 
\emph{f} is a function.\\
\emph{I} is an interval.\\

\noindent Description: \begin{itemize}

\item \textbf{dirtyintegral}(\emph{f},\emph{I}) computes an approximation of the integral of \emph{f} on \emph{I}.

\item The interval must be bound. If the interval contains one of -Inf or +Inf, the 
   result of \textbf{dirtyintegral} is NaN, even if the integral has a meaning.

\item The result of this command depends on the global variables \textbf{prec} and \textbf{points}.
   The method used is the trapezium rule applied at $n$ evenly distributed
   points in the interval, where $n$ is the value of global variable \textbf{points}.

\item This command computes a numerical approximation of the exact value of the 
   integral. It should not be used if safety is critical. In this case, use
   command \textbf{integral} instead.

\item Warning: this command is known to be currently unsatisfactory. If you really
   need to compute integrals, think of using an other tool or report a feature
   request to sylvain.chevillard@ens-lyon.fr.
\end{itemize}
\noindent Example 1: 
\begin{center}\begin{minipage}{14.8cm}\begin{Verbatim}[frame=single]
   > sin(10);
   -0.544021110889369813404747661851377281683643012916219
   > dirtyintegral(cos(x),[0;10]);
   -0.544003049051526298224480588824753820365362983562818375241
   > points=2000!;
   > dirtyintegral(cos(x),[0;10]);
   -0.544019977511583219722226973125831990359958379268927638295
\end{Verbatim}
\end{minipage}\end{center}
See also: \textbf{prec}, \textbf{points}, \textbf{integral}

\subsection{display}
\label{labdisplay}
\noindent Name: \textbf{display}\\
sets or inspects the global variable specifying number notation\\
\noindent Usage: 
\begin{center}
\textbf{display} = \emph{notation value} : \textsf{decimal$|$binary$|$dyadic$|$powers$|$hexadecimal} $\rightarrow$ \textsf{void}
\\ 
\textbf{display} = \emph{notation value} ! : \textsf{decimal$|$binary$|$dyadic$|$powers$|$hexadecimal} $\rightarrow$ \textsf{void}
\\ 
\end{center}
Parameters: 
\begin{itemize}
\item \emph{notation value} represents a variable of type \textsf{decimal$|$binary$|$dyadic$|$powers$|$hexadecimal}
\end{itemize}
\noindent Description: \begin{itemize}

\item An assignment \textbf{display} = \emph{notation value}, where \emph{notation value} is
   one of \textbf{decimal}, \textbf{dyadic}, \textbf{powers}, \textbf{binary} or \textbf{hexadecimal}, activates
   the corresponding notation for output of values in \textbf{print}, \textbf{write} or
   at the \sollya prompt.
    
   If the global notation variable \textbf{display} is \textbf{decimal}, all numbers will
   be output in scientific decimal notation.  If the global notation
   variable \textbf{display} is \textbf{dyadic}, all numbers will be output as dyadic
   numbers with Gappa notation.  If the global notation variable \textbf{display}
   is \textbf{powers}, all numbers will be output as dyadic numbers with a
   notation compatible with Maple and PARI/GP.  If the global notation
   variable \textbf{display} is \textbf{binary}, all numbers will be output in binary
   notation.  If the global notation variable \textbf{display} is \textbf{hexadecimal},
   all numbers will be output in C99/ IEEE754R notation.  All output
   notations can be parsed back by \sollya, inducing no error if the input
   and output precisions are the same (see \textbf{prec}).
    
   If the assignment \textbf{display} = \emph{notation value} is followed by an
   exclamation mark, no message indicating the new state is
   displayed. Otherwise the user is informed of the new state of the
   global mode by an indication.
\end{itemize}
\noindent Example 1: 
\begin{center}\begin{minipage}{15cm}\begin{Verbatim}[frame=single]
> display = decimal;
Display mode is decimal numbers.
> a = evaluate(sin(pi * x), 0.25);
> a;
0.70710678118654752440084436210484903928483593768847
> display = binary;
Display mode is binary numbers.
> a;
1.011010100000100111100110011001111111001110111100110010010000100010110010111110
11000100110110011011101010100101010111110100111110001110101101111011000001011101
010001_2 * 2^(-1)
> display = hexadecimal;
Display mode is hexadecimal numbers.
> a;
0xb.504f333f9de6484597d89b3754abe9f1d6f60ba88p-4
> display = dyadic;
Display mode is dyadic numbers.
> a;
33070006991101558613323983488220944360067107133265b-165
> display = powers;
Display mode is dyadic numbers in integer-power-of-2 notation.
> a;
33070006991101558613323983488220944360067107133265 * 2^(-165)
\end{Verbatim}
\end{minipage}\end{center}
See also: \textbf{print} (\ref{labprint}), \textbf{write} (\ref{labwrite}), \textbf{decimal} (\ref{labdecimal}), \textbf{dyadic} (\ref{labdyadic}), \textbf{powers} (\ref{labpowers}), \textbf{binary} (\ref{labbinary}), \textbf{hexadecimal} (\ref{labhexadecimal}), \textbf{prec} (\ref{labprec})

\subsection{divide}
\label{labdivide}
\noindent Name: \textbf{/}\\
division function\\

\noindent Usage: 
\begin{center}
\emph{function1} \textbf{/} \emph{function2} : (\textsf{function}, \textsf{function}) $\rightarrow$ \textsf{function}\\
\end{center}
Parameters: 
\begin{itemize}
\item \emph{function1} and \emph{function2} represent functions
\end{itemize}
\noindent Description: \begin{itemize}

\item \textbf{/} represents the division (function) on reals. 
   The expression \emph{function1} \textbf{/} \emph{function2} stands for
   the function composed of the division function and the two
   functions \emph{function1} and \emph{function2}, where \emph{function1} is
   the numerator and \emph{function2} the denominator.
\end{itemize}
\noindent Example 1: 
\begin{center}\begin{minipage}{15cm}\begin{Verbatim}[frame=single]
> 5 / 2;
2.5
\end{Verbatim}
\end{minipage}\end{center}
\noindent Example 2: 
\begin{center}\begin{minipage}{15cm}\begin{Verbatim}[frame=single]
> x / 2;
x * 0.5
\end{Verbatim}
\end{minipage}\end{center}
\noindent Example 3: 
\begin{center}\begin{minipage}{15cm}\begin{Verbatim}[frame=single]
> x / x;
1
\end{Verbatim}
\end{minipage}\end{center}
\noindent Example 4: 
\begin{center}\begin{minipage}{15cm}\begin{Verbatim}[frame=single]
> 3 / 0;
@NaN@
\end{Verbatim}
\end{minipage}\end{center}
\noindent Example 5: 
\begin{center}\begin{minipage}{15cm}\begin{Verbatim}[frame=single]
> diff(sin(x) / exp(x));
(exp(x) * cos(x) - sin(x) * exp(x)) / exp(x)^2
\end{Verbatim}
\end{minipage}\end{center}
See also: \textbf{$+$} (\ref{labplus}), \textbf{$-$} (\ref{labminus}), \textbf{$*$} (\ref{labmult}), \textbf{\^} (\ref{labpower})

\subsection{ double }
\noindent Names: \textbf{double}, \textbf{D}\\
rounding to the nearest IEEE double.\\

\noindent Description: \begin{itemize}

\item \textbf{double} is both a function and a constant.

\item As a function, it rounds its argument to the nearest double precision number.
   Subnormal numbers are supported as well as standard numbers: it is the real
   ounding described in the standard.

\item As a constant, it symoblizes the double precision format. It is used in 
   contexts when a precision format is necessary, e.g. in the commands 
   \textbf{roundcoefficients} and \textbf{implementpoly}.
   See the corresponding help pages for examples.
\end{itemize}
\noindent Example 1: 
\begin{center}\begin{minipage}{14.8cm}\begin{Verbatim}[frame=single]
   > display=binary!;
   > D(0.1);
   1.100110011001100110011001100110011001100110011001101_2 * 2^(-4)
   > D(4.17);
   1.000010101110000101000111101011100001010001111010111_2 * 2^(2)
   > D(1.011_2 * 2^(-1073));
   1.1_2 * 2^(-1073)
\end{Verbatim}
\end{minipage}\end{center}
See also: \textbf{doubleextended}, \textbf{doubledouble}, \textbf{tripledouble}, \textbf{roundcoefficients}, \textbf{implementpoly}

\subsection{doubledouble}
\label{labdoubledouble}
\noindent Names: \textbf{doubledouble}, \textbf{DD}\\
represents a number as the sum of two IEEE doubles.\\
\noindent Description: \begin{itemize}

\item \textbf{doubledouble} is both a function and a constant.

\item As a function, it rounds its argument to the nearest number that can be written
   as the sum of two double precision numbers.

\item The algorithm used to compute \textbf{doubledouble}($x$) is the following: let $x_h$ = \textbf{double}($x$)
   and let $x_l$ = \textbf{double}($x-x_h$). Return the number $x_h+x_l$. Note that if the current 
   precision is not sufficient to exactly represent $x_h + x_l$, a rounding will occur
   and the result of \textbf{doubledouble}($x$) will be useless.

\item As a constant, it symbolizes the double-double precision format. It is used in 
   contexts when a precision format is necessary, e.g. in the commands 
   \textbf{round}, \textbf{roundcoefficients} and \textbf{implementpoly}.
   See the corresponding help pages for examples.
\end{itemize}
\noindent Example 1: 
\begin{center}\begin{minipage}{15cm}\begin{Verbatim}[frame=single]
> verbosity=1!;
> a = 1+ 2^(-100);
> DD(a);
1.0000000000000000000000000000007888609052210118054
> prec=50!;
> DD(a);
Warning: double rounding occurred on invoking the double-double rounding operato
r.
Try to increase the working precision.
1
\end{Verbatim}
\end{minipage}\end{center}
See also: \textbf{single} (\ref{labsingle}), \textbf{double} (\ref{labdouble}), \textbf{doubleextended} (\ref{labdoubleextended}), \textbf{tripledouble} (\ref{labtripledouble}), \textbf{roundcoefficients} (\ref{labroundcoefficients}), \textbf{implementpoly} (\ref{labimplementpoly}), \textbf{round} (\ref{labround})

\subsection{doubleextended}
\label{labdoubleextended}
\noindent Names: \textbf{doubleextended}, \textbf{DE}\\
computes the nearest number with 64 bits of mantissa.\\

\noindent Description: \begin{itemize}

\item \textbf{doubleextended} is a function that computes the nearest floating-point number with
   64 bits of mantissa to a given number. Since it is a function, it can be
   composed with other functions of Sollya such as \textbf{exp}, \textbf{sin}, etc.

\item It does not handle subnormal numbers. The range of possible exponents is the 
   range used for all numbers represented in Sollya (e.g. basically the range 
   used in the library MPFR).

\item Since it is a function and not a command, its behavior is a bit different from 
   the behavior of \textbf{round}(x,64,RN) even if the result is exactly the same.
   \textbf{round}(x,64,RN) is immediately evaluated whereas \textbf{doubleextended}(x) can be composed 
   with other functions (and thus be plotted and so on).

\item Be aware that \textbf{doubleextended} cannot be used as a constant to represent a format in the
   commands \textbf{roundcoefficients} and \textbf{implementpoly} (contrary to \textbf{D}, \textbf{DD},and \textbf{TD}).
\end{itemize}
\noindent Example 1: 
\begin{center}\begin{minipage}{15cm}\begin{Verbatim}[frame=single]
> display=binary!;
> DE(0.1);
1.100110011001100110011001100110011001100110011001100110011001101_2 * 2^(-4)
> round(0.1,64,RN);
1.100110011001100110011001100110011001100110011001100110011001101_2 * 2^(-4)
\end{Verbatim}
\end{minipage}\end{center}
\noindent Example 2: 
\begin{center}\begin{minipage}{15cm}\begin{Verbatim}[frame=single]
> D(2^(-2000));
0
> DE(2^(-2000));
0.870980981621721667557619549477887229585910374270538862e-602
\end{Verbatim}
\end{minipage}\end{center}
\noindent Example 3: 
\begin{center}\begin{minipage}{15cm}\begin{Verbatim}[frame=single]
> verbosity=1!;
> f = sin(DE(x));
> f(pi);
Warning: rounding has happened. The value displayed is a faithful rounding of th
e true result.
-0.501655761266833202355732708033075701383156167025495e-19
> g = sin(round(x,64,RN));
Warning: at least one of the given expressions or a subexpression is not correct
ly typed
or its evaluation has failed because of some error on a side-effect.
\end{Verbatim}
\end{minipage}\end{center}
See also: \textbf{double} (\ref{labdouble}), \textbf{doubledouble} (\ref{labdoubledouble}), \textbf{tripledouble} (\ref{labtripledouble}), \textbf{round} (\ref{labround})

\subsection{dyadic}
\label{labdyadic}
\noindent Name: \textbf{dyadic}\\
special value for global state 	extbf{display}\\
\noindent Description: \begin{itemize}

\item \textbf{dyadic} is a special value used for the global state \textbf{display}.
   If the global state \textbf{display} is equal to \textbf{dyadic}, all data will
   be output in dyadic notation with numbers displayed in Gappa format.
    
   As any value it can be affected to a variable and stored in lists.
\end{itemize}
See also: \textbf{decimal} (\ref{labdecimal}), \textbf{powers} (\ref{labpowers}), \textbf{hexadecimal} (\ref{labhexadecimal}), \textbf{binary} (\ref{labbinary})

\subsection{ equal }
\noindent Name: \textbf{$==$}\\
equality test operator\\

\noindent Usage: 
\begin{center}
\emph{expr1} \textbf{$==$} \emph{expr2} : (\textsf{any type}, \textsf{any type}) $\rightarrow$ \textsf{boolean}\\
\end{center}
Parameters: 
\emph{expr1} and \emph{expr2} represent expressions\\

\noindent Description: \begin{itemize}

\item The operator \textbf{$==$} evaluates to true iff its operands \emph{expr1} and
   \emph{expr2} are syntactically equal and different from \textbf{error} or constant
   expressions that evaluate to the same floating-point number with the
   global precision \textbf{prec}. The user should be aware of the fact that
   because of floating-point evaluation, the operator \textbf{$==$} is not
   exactly the same as the mathematical equality.
\end{itemize}
\noindent Example 1: 
\begin{center}\begin{minipage}{14.8cm}\begin{Verbatim}[frame=single]
   > "Hello" == "Hello";
   true
   > "Hello" == "Salut";
   false
   > "Hello" == 5;
   false
   > 5 + x == 5 + x;
   true
\end{Verbatim}
\end{minipage}\end{center}
\noindent Example 2: 
\begin{center}\begin{minipage}{14.8cm}\begin{Verbatim}[frame=single]
   > 1 == exp(0);
   true
   > asin(1) * 2 == pi;
   true
   > exp(5) == log(4);
   false
\end{Verbatim}
\end{minipage}\end{center}
\noindent Example 3: 
\begin{center}\begin{minipage}{14.8cm}\begin{Verbatim}[frame=single]
   > prec = 12;
   The precision has been set to 12 bits.
   > 16384 == 16385;
   true
\end{Verbatim}
\end{minipage}\end{center}
\noindent Example 4: 
\begin{center}\begin{minipage}{14.8cm}\begin{Verbatim}[frame=single]
   > error == error;
   false
\end{Verbatim}
\end{minipage}\end{center}
See also: \textbf{!$=$}, \textbf{$>$}, \textbf{$>=$}, \textbf{$<=$}, \textbf{$<$}, \textbf{!}, \textbf{$\&\&$}, \textbf{$||$}, \textbf{error}, \textbf{prec}

\subsection{erf}
\label{laberf}
\noindent Name: \textbf{erf}\\
the error function.\\

\noindent Description: \begin{itemize}

\item \textbf{erf} is the error function defined by:
   $$\mathrm{erf}(x) = \frac{2}{\sqrt{\pi}} \int_0^x e^{-t^2} {\rm d}t.$$

\item It is defined for every real number x.
\end{itemize}
See also: \textbf{erfc} (\ref{laberfc}), \textbf{exp} (\ref{labexp})

\subsection{erfc}
\label{laberfc}
\noindent Name: \textbf{erfc}\\
the complementary error function.\\
\noindent Description: \begin{itemize}

\item \textbf{erfc} is the complementary error function defined by $\mathrm{erfc}(x) = 1 - \mathrm{erf}(x)$.

\item It is defined for every real number $x$.
\end{itemize}
See also: \textbf{erf} (\ref{laberf})

\subsection{error}
\label{laberror}
\noindent Name: \textbf{error}\\
expression representing an input that is wrongly typed or that cannot be executed\\

\noindent Usage: 
\begin{center}
\textbf{error} : \textsf{error}\\
\end{center}
\noindent Description: \begin{itemize}

\item The variable \textbf{error} represents an input during the evaluation of
   which a type or execution error has been detected or is to be
   detected. Inputs that are syntactically correct but wrongly typed
   evaluate to \textbf{error} at some stage.  Inputs that are correctly typed
   but containing commands that depend on side-effects that cannot be
   performed or inputs that are wrongly typed at meta-level (cf. \textbf{parse}),
   evaluate to \textbf{error}.
    
   Remark that in contrast to all other elements of the Sollya language,
   \textbf{error} compares neither equal nor unequal to itself. This provides a
   means of detecting syntax errors inside the Sollya language itself
   without introducing issues of two different wrongly typed input being
   equal.
\end{itemize}
\noindent Example 1: 
\begin{center}\begin{minipage}{15cm}\begin{Verbatim}[frame=single]
> print(5 + "foo");
error
\end{Verbatim}
\end{minipage}\end{center}
\noindent Example 2: 
\begin{center}\begin{minipage}{15cm}\begin{Verbatim}[frame=single]
> error;
error
\end{Verbatim}
\end{minipage}\end{center}
\noindent Example 3: 
\begin{center}\begin{minipage}{15cm}\begin{Verbatim}[frame=single]
> error == error;
false
> error != error;
false
\end{Verbatim}
\end{minipage}\end{center}
\noindent Example 4: 
\begin{center}\begin{minipage}{15cm}\begin{Verbatim}[frame=single]
> correct = 5 + 6;
> incorrect = 5 + "foo";
> (correct == error || correct != error);
true
> (incorrect == error || incorrect != error);
false
\end{Verbatim}
\end{minipage}\end{center}
See also: \textbf{void} (\ref{labvoid}), \textbf{parse} (\ref{labparse})

\subsection{evaluate}
\label{labevaluate}
\noindent Name: \textbf{evaluate}\\
evaluates a function at a constant point or in a range\\

\noindent Usage: 
\begin{center}
\textbf{evaluate}(\emph{function}, \emph{constant}) : (\textsf{function}, \textsf{constant}) $\rightarrow$ \textsf{constant} | \textsf{range}\\
\textbf{evaluate}(\emph{function}, \emph{range}) : (\textsf{function}, \textsf{range}) $\rightarrow$ \textsf{range}\\
\textbf{evaluate}(\emph{function}, \emph{function2}) : (\textsf{function}, \textsf{function}) $\rightarrow$ \textsf{function}\\
\end{center}
Parameters: 
\begin{itemize}
\item \emph{function} represents a function
\item \emph{constant} represents a constant point
\item \emph{range} represents a range
\item \emph{function2} represents a function that is not constant
\end{itemize}
\noindent Description: \begin{itemize}

\item If its second argument is a constant \emph{constant}, \textbf{evaluate} evaluates
   its first argument \emph{function} at the point indicated by
   \emph{constant}. This evaluation is performed in a way that the result is a
   faithful rounding of the real value of the \emph{function} at \emph{constant} to
   the current global precision. If such a faithful rounding is not
   possible, \textbf{evaluate} returns a range surely encompassing the real value
   of the function \emph{function} at \emph{constant}. If even interval evaluation
   is not possible because the expression is undefined or numerically
   unstable, NaN will be produced.

\item If its second argument is a range \emph{range}, \textbf{evaluate} evaluates its
   first argument \emph{function} by interval evaluation on this range
   \emph{range}. This ensures that the image domain of the function \emph{function}
   on the pre-image domain \emph{range} is surely enclosed in the returned
   range.

\item If its second argument is a function \emph{function2} that is not a
   constant, \textbf{evaluate} replaces all occurences of the free variable in
   function \emph{function} by function \emph{function2}.
\end{itemize}
\noindent Example 1: 
\begin{center}\begin{minipage}{15cm}\begin{Verbatim}[frame=single]
> print(evaluate(sin(pi * x), 2.25));
0.707106781186547524400844362104849039284835937688470741
\end{Verbatim}
\end{minipage}\end{center}
\noindent Example 2: 
\begin{center}\begin{minipage}{15cm}\begin{Verbatim}[frame=single]
> print(evaluate(sin(pi * x), 2));
[-0.172986452514381269516508615031098129542836767991679e-12714;0.759411982011879
631450695643145256617060390843900679e-12715]
\end{Verbatim}
\end{minipage}\end{center}
\noindent Example 3: 
\begin{center}\begin{minipage}{15cm}\begin{Verbatim}[frame=single]
> print(evaluate(sin(pi * x), [2, 2.25]));
[-0.514339027267725463004699891996191240734922416542101e-49;0.707106781186547524
400844362104849039284835937688663]
\end{Verbatim}
\end{minipage}\end{center}
\noindent Example 4: 
\begin{center}\begin{minipage}{15cm}\begin{Verbatim}[frame=single]
> print(evaluate(sin(pi * x), 2 + 0.25 * x));
sin((pi) * (2 + 0.25 * x))
\end{Verbatim}
\end{minipage}\end{center}
\noindent Example 5: 
\begin{center}\begin{minipage}{15cm}\begin{Verbatim}[frame=single]
> print(evaluate(sin(pi * 1/x), 0));
@NaN@
\end{Verbatim}
\end{minipage}\end{center}
See also: \textbf{isevaluable} (\ref{labisevaluable})

\subsection{ execute }
\noindent Name: \textbf{execute}\\
executes the content of a file\\

\noindent Usage: 
\begin{center}
\textbf{execute}(\emph{filename}) : \textsf{string} $\rightarrow$ \textsf{void}\\
\end{center}
Parameters: 
\emph{filename} is a string representing a file name\\

\noindent Description: \begin{itemize}

\item \textbf{execute} opens the file indicated by \emph{filename}, and executes the sequence of 
   commands it contains. This command is evaluated at execution time: this way you
   can modify the file \emph{filename} (for instance using \textbf{bashexecute}) and execute it
   just after.

\item If \emph{filename} contains a command \textbf{execute}, it will be executed recursively.

\item If \emph{filename} contains a call to \textbf{restart}, it will be neglected.

\item If \emph{filename} contains a call to \textbf{quit}, the commands following \textbf{quit}
   in \emph{filename} will be neglected.
\end{itemize}
\noindent Example 1: 
\begin{center}\begin{minipage}{14.8cm}\begin{Verbatim}[frame=single]
   > a=2;
   > a;
   2
   > print("a=1;") > "example.sollya";
   > execute("example.sollya"); 
   > a;
   1
\end{Verbatim}
\end{minipage}\end{center}
\noindent Example 2: 
\begin{center}\begin{minipage}{14.8cm}\begin{Verbatim}[frame=single]
   > verbosity=1!;
   > print("a=1; restart; a=2;") > "example.sollya";
   > execute("example.sollya"); 
   Warning: a restart command has been used in a file read into another.
   This restart command will be neglected.
   > a;
   2
\end{Verbatim}
\end{minipage}\end{center}
\noindent Example 3: 
\begin{center}\begin{minipage}{14.8cm}\begin{Verbatim}[frame=single]
   > verbosity=1!;
   > print("a=1; quit; a=2;") > "example.sollya";
   > execute("example.sollya"); 
   Warning: the execution of a file read by execute demanded stopping the interpretation but it is not stopped.
   > a;
   1
\end{Verbatim}
\end{minipage}\end{center}
See also: \textbf{parse}, \textbf{readfile}, \textbf{write}, \textbf{print}, \textbf{bashexecute}

\subsection{ exp }
\noindent Name: \textbf{exp}\\
the exponential function.\\

\noindent Description: \begin{itemize}

\item \textbf{exp} is the usual exponential function defined as the solution of the
   ordinary differential equation $y'=y$ with $y(0)=1$.

\item \textbf{exp}(x) is defined for every real number x.
\end{itemize}
See also: \textbf{exp}, \textbf{log}

\subsection{ expand }
\noindent Name: \textbf{expand}\\
expands polynomial subexpressions\\

\noindent Usage: 
\begin{center}
\textbf{expand}(\emph{function}) : \textsf{function} $\rightarrow$ \textsf{function}\\
\end{center}
Parameters: 
\begin{itemize}
\item \emph{function} represents a function
\end{itemize}
\noindent Description: \begin{itemize}

\item \textbf{expand}(\emph{function}) expands all polynomial subexpressions in function
   \emph{function} is far as possible. Factors of sums are multiplied out,
   power operators with constant positive integer exponents are replaced
   by multiplications and divisions are multiplied out, i.e. denomiators
   are brought at the most interior point of expressions.
\end{itemize}
\noindent Example 1: 
\begin{center}\begin{minipage}{15cm}\begin{Verbatim}[frame=single]
> print(expand(x^3));
x * x * x
\end{Verbatim}
\end{minipage}\end{center}
\noindent Example 2: 
\begin{center}\begin{minipage}{15cm}\begin{Verbatim}[frame=single]
> print(expand((x + 2)^3 + 2 * x));
8 + 12 * x + 6 * x * x + x * x * x + 2 * x
\end{Verbatim}
\end{minipage}\end{center}
\noindent Example 3: 
\begin{center}\begin{minipage}{15cm}\begin{Verbatim}[frame=single]
> print(expand(exp((x + (x + 3))^5)));
exp(243 + 405 * x + 270 * x * x + 90 * x * x * x + 15 * x * x * x * x + x * x * 
x * x * x + x * 405 + 108 * x * 5 * x + 54 * x * x * 5 * x + 12 * x * x * x * 5 
* x + x * x * x * x * 5 * x + x * x * 270 + 27 * x * x * x * 10 + 9 * x * x * x 
* x * 10 + x * x * x * x * x * 10 + x * x * x * 90 + 6 * x * x * x * x * 10 + x 
* x * x * x * x * 10 + x * x * x * x * 5 * x + 15 * x * x * x * x + x * x * x * 
x * x)
\end{Verbatim}
\end{minipage}\end{center}
See also: \textbf{simplify}, \textbf{simplifysafe}, \textbf{horner}

\subsection{expm1}
\label{labexpm1}
\noindent Name: \textbf{expm1}\\
translated exponential function.\\

\noindent Description: \begin{itemize}

\item \textbf{expm1} is defined by ${\rm expm1}(x) = \exp(x)-1$.

\item It is defined for every real number x.
\end{itemize}
See also: \textbf{exp} (\ref{labexp})

\subsection{ exponent }
\noindent Name: \textbf{exponent}\\
returns the scaled binary exponent of a number.\\

\noindent Usage: 
\begin{center}
\textbf{exponent}(\emph{x}) : \textsf{constant} $\rightarrow$ \textsf{integer}\\
\end{center}
Parameters: 
\emph{x} is a dyadic number.\\

\noindent Description: \begin{itemize}

\item \textbf{exponent}(x) is by definition 0 if x equals 0, NaN, or Inf.

\item If \emph{x} is not zero, it can be uniquely written as $x = m \cdot 2^e$ where
   $m$ is an odd integer and $e$ is an integer. \textbf{exponent}(x) returns $e$. 
\end{itemize}
\noindent Example 1: 
\begin{center}\begin{minipage}{14.8cm}\begin{Verbatim}[frame=single]
   > a=round(Pi,20,RN);
   > e=exponent(a);
   > e;
   -17
   > m=mantissa(a);
   > a-m*2^e;
   0
\end{Verbatim}
\end{minipage}\end{center}
See also: \textbf{mantissa}, \textbf{precision}

\subsection{ externalplot }
\noindent Name: \textbf{externalplot}\\
plots the error of an external code with regard to a function\\

\noindent Usage: 
\begin{center}
\textbf{externalplot}(\emph{filename}, \emph{mode}, \emph{function}, \emph{range}, \emph{precision}) : (\textsf{string}, \textsf{absolute|relative}, \textsf{function}, \textsf{range}, \textsf{integer}) $\rightarrow$ \textsf{void}\\
\textbf{externalplot}(\emph{filename}, \emph{mode}, \emph{function}, \emph{range}, \emph{precision}, \emph{perturb}) : (\textsf{string}, \textsf{absolute|relative}, \textsf{function}, \textsf{range}, \textsf{integer}, \textsf{perturb}) $\rightarrow$ \textsf{void}\\
\textbf{externalplot}(\emph{filename}, \emph{mode}, \emph{function}, \emph{range}, \emph{precision}, \emph{plot mode}, \emph{result filename}) : (\textsf{string}, \textsf{absolute|relative}, \textsf{function}, \textsf{range}, \textsf{integer}, \textsf{file|postscript|postscriptfile}, \textsf{string}) $\rightarrow$ \textsf{void}\\
\textbf{externalplot}(\emph{filename}, \emph{mode}, \emph{function}, \emph{range}, \emph{precision}, \emph{perturb}, \emph{plot mode}, \emph{result filename}) : (\textsf{string}, \textsf{absolute|relative}, \textsf{function}, \textsf{range}, \textsf{integer}, \textsf{perturb}, \textsf{file|postscript|postscriptfile}, \textsf{string}) $\rightarrow$ \textsf{void}\\
\end{center}
\noindent Description: \begin{itemize}

\item The command \textbf{externalplot} plots the error of an external function
   evaluation code sequence implemented in the object file named
   \emph{filename} with regard to the function \emph{function}.  If \emph{mode}
   evaluates to \emph{absolute}, the difference of both functions is
   considered as an error function; if \emph{mode} evaluates to \emph{relative},
   the difference is quotiented by the function \emph{function}. The resulting
   error function is plotted on all floating-point numbers with
   \emph{precision} significant mantissa bits in the range \emph{range}. 
   If the sixth argument of the command \textbf{externalplot} is given an evaluates to
   \textbf{perturb}, each of these floating-point numbers is perturbed by a
   random value that is uniformly distributed in $\pm1$ ulp
   around the original \emph{precision} bit floating-point variable.
   If a sixth and seventh argument, respectively a seventh and eighth
   argument in the presence of \textbf{perturb} as a sixth argument, are given
   that evaluate to a variable of type \textsf{file|postscript|postscriptfile} respectively to a
   character sequence of type \textsf{string}, \textbf{externalplot} will plot
   (additionally) to a file in the same way as the command \textbf{plot}
   does. See \textbf{plot} for details.
   The external function evaluation code given in the object file name
   \emph{filename} is supposed to define a function name \texttt{f} as
   follows (here in C syntax): \texttt{void f(mpfr\_t rop, mpfr\_ op)}. 
   This function is supposed to evaluate \texttt{op} with an accuracy corresponding
   to the precision of \texttt{rop} and assign this value to
   \texttt{rop}.
\end{itemize}
\noindent Example 1: 
\begin{center}\begin{minipage}{14.8cm}\begin{Verbatim}[frame=single]
   > bashexecute("gcc -fPIC -c externalplotexample.c");
   > bashexecute("gcc -shared -o externalplotexample externalplotexample.o -lgmp -lmpfr");
   > externalplot("./externalplotexample",relative,exp(x),[-1/2;1/2],12,perturb);
\end{Verbatim}
\end{minipage}\end{center}
See also: \textbf{plot}, \textbf{asciiplot}, \textbf{perturb}, \textbf{absolute}, \textbf{relative}, \textbf{file}, \textbf{postscript}, \textbf{postscriptfile}, \textbf{bashexecute}, \textbf{externalproc}, \textbf{library}

\subsection{externalproc}
\label{labexternalproc}
\noindent Name: \textbf{externalproc}\\
\phantom{aaa}binds an external code to a \sollya procedure\\[0.2cm]
\noindent Usage: 
\begin{center}
\textbf{externalproc}(\emph{identifier}, \emph{filename}, \emph{argumenttype} $->$ \emph{resulttype}) : (\textsf{identifier type}, \textsf{string}, \textsf{type type}, \textsf{type type}) $\rightarrow$ \textsf{void}\\
\end{center}
Parameters: 
\begin{itemize}
\item \emph{identifier} represents the identifier the code is to be bound to
\item \emph{filename} of type \textsf{string} represents the name of the object file where the code of procedure can be found
\item \emph{argumenttype} represents a definition of the types of the arguments of the \sollya procedure and the external code
\item \emph{resulttype} represents a definition of the result type of the external code
\end{itemize}
\noindent Description: \begin{itemize}

\item \textbf{externalproc} allows for binding the \sollya identifier
   \emph{identifier} to an external code.  After this binding, when \sollya
   encounters \emph{identifier} applied to a list of actual parameters, it
   will evaluate these parameters and call the external code with these
   parameters. If the external code indicated success, it will receive
   the result produced by the external code, transform it to \sollya's
   internal representation and return it.
    
   In order to allow correct evaluation and typing of the data in
   parameter and in result to be passed to and received from the external
   code, \textbf{externalproc} has a third parameter \emph{argumenttype} $->$ \emph{resulttype}.
   Both \emph{argumenttype} and \emph{resulttype} are one of \textbf{void}, \textbf{constant},
   \textbf{function}, \textbf{range}, \textbf{integer}, \textbf{string}, \textbf{boolean}, \textbf{list of} \textbf{constant}, \textbf{list of} \textbf{function}, 
   \textbf{list of} \textbf{range}, \textbf{list of} \textbf{integer}, \textbf{list of} \textbf{string}, \textbf{list of} \textbf{boolean}.
    
   If upon a usage of a procedure bound to an external procedure the type
   of the actual parameters given or its number is not correct, \sollya
   produces a type error. An external function not applied to arguments
   represents itself and prints out with its argument and result types.
    
   The external function is supposed to return an integer indicating
   success.  It returns its result depending on its \sollya result type
   as follows. Here, the external procedure is assumed to be implemented
   as a C function.
    
   If the \sollya result type is void, the C function has no pointer
   argument for the result.  If the \sollya result type is \textbf{constant}, the
   first argument of the C function is of C type \texttt{mpfr\_t *}, the result is
   returned by affecting the MPFR variable.  If the \sollya result type
   is \textbf{function}, the first argument of the C function is of C type \texttt{node **},
   the result is returned by the \texttt{node *} pointed with a new \texttt{node *}.
   If the \sollya result type is \textbf{range}, the first argument of the C
   function is of C type \texttt{mpfi\_t *}, the result is returned by affecting
   the interval-arithmetic variable.  If the \sollya result type is \textbf{integer}, the first
   argument of the C function is of C type \texttt{int *}, the result is returned
   by affecting the int variable.  If the \sollya result type is \textbf{string},
   the first argument of the C function is of C type \texttt{char **}, the result
   is returned by the \texttt{char *} pointed with a new \texttt{char *}.  If the \sollya
   result type is \textbf{boolean}, the first argument of the C function is of C
   type \texttt{int *}, the result is returned by affecting the int variable with
   a boolean value.  If the \sollya result type is \textbf{list of} type, the
   first argument of the C function is of C type \texttt{chain **}, the result is
   returned by the \texttt{chain *} pointed with a new \texttt{chain *}.  This chain
   contains for \sollya type \textbf{constant} pointers \texttt{mpfr\_t *} to new MPFR
   variables, for \sollya type \textbf{function} pointers \texttt{node *} to new nodes, for
   \sollya type \textbf{range} pointers \texttt{mpfi\_t *}  to new interval-arithmetic variables, for
   \sollya type \textbf{integer} pointers \texttt{int *} to new int variables for \sollya
   type \textbf{string} pointers \texttt{char *} to new \texttt{char *} variables and for \sollya
   type \textbf{boolean} pointers \texttt{int *} to new int variables representing boolean
   values.
    	       
   The external procedure affects its possible pointer argument if and
   only if it succeeds.  This means, if the function returns an integer
   indicating failure, it does not leak any memory to the encompassing
   environment.
    
   The external procedure receives its arguments as follows: If the
   \sollya argument type is \textbf{void}, no argument array is given.  Otherwise
   the C function receives a C \texttt{void **} argument representing an array of
   size equal to the arity of the function where each entry (of C type
   \texttt{void *}) represents a value with a C type depending on the
   corresponding \sollya type. If the \sollya type is \textbf{constant}, the C
   type the \texttt{void *} is to be casted to is \texttt{mpfr\_t *}.  If the \sollya type
   is \textbf{function}, the C type the \texttt{void *} is to be casted to is \texttt{node *}.  If
   the \sollya type is \textbf{range}, the C type the \texttt{void *} is to be casted to is
   \texttt{mpfi\_t *}.  If the \sollya type is \textbf{integer}, the C type the \texttt{void *} is to
   be casted to is \texttt{int *}.  If the \sollya type is \textbf{string}, the C type the
   \texttt{void *} is to be casted to is \texttt{char *}.  If the \sollya type is \textbf{boolean},
   the C type the \texttt{void *} is to be casted to is \texttt{int *}.  If the \sollya
   type is \textbf{list of} type, the C type the \texttt{void *} is to be casted to is
   \texttt{chain *}.  Here depending on type, the values in the chain are to be
   casted to \texttt{mpfr\_t *}  for \sollya type \textbf{constant}, \texttt{node *} for \sollya type
   \textbf{function}, \texttt{mpfi\_t *} for \sollya type \textbf{range}, \texttt{int *} for \sollya type
   \textbf{integer}, \texttt{char *} for \sollya type \textbf{string} and \texttt{int *} for \sollya type
   \textbf{boolean}.
    
   The external procedure is not supposed to alter the memory pointed by
   its array argument \texttt{void **}.
    
   In both directions (argument and result values), empty lists are
   represented by \texttt{chain * NULL} pointers.
    
   In contrast to internal procedures, externally bounded procedures can
   be considered to be objects inside \sollya that can be assigned to other
   variables, stored in list etc.
\end{itemize}
\noindent Example 1: 
\begin{center}\begin{minipage}{15cm}\begin{Verbatim}[frame=single]
> bashexecute("gcc -fPIC -Wall -c externalprocexample.c");
> bashexecute("gcc -fPIC -shared -o externalprocexample externalprocexample.o");

> externalproc(foo, "./externalprocexample", (integer, integer) -> integer);
> foo;
foo(integer, integer) -> integer
> foo(5, 6);
11
> verbosity = 1!;
> foo();
Warning: at least one of the given expressions or a subexpression is not correct
ly typed
or its evaluation has failed because of some error on a side-effect.
error
> a = foo;
> a(5,6);
11
\end{Verbatim}
\end{minipage}\end{center}
See also: \textbf{library} (\ref{lablibrary}), \textbf{libraryconstant} (\ref{lablibraryconstant}), \textbf{externalplot} (\ref{labexternalplot}), \textbf{bashexecute} (\ref{labbashexecute}), \textbf{void} (\ref{labvoid}), \textbf{constant} (\ref{labconstant}), \textbf{function} (\ref{labfunction}), \textbf{range} (\ref{labrange}), \textbf{integer} (\ref{labinteger}), \textbf{string} (\ref{labstring}), \textbf{boolean} (\ref{labboolean}), \textbf{list of} (\ref{lablistof}), \textbf{object} (\ref{labobject})

\subsection{false}
\label{labfalse}
\noindent Name: \textbf{false}\\
the boolean value representing the false.\\
\noindent Description: \begin{itemize}

\item \\textbf{false} is the usual boolean value.\n\end{itemize}
\noindent Example 1: 
\begin{center}\begin{minipage}{15cm}\begin{Verbatim}[frame=single]
\end{Verbatim}
\end{minipage}\end{center}
See also: \textbf{true} (\ref{labtrue}), \textbf{$\&\&$} (\ref{laband}), \textbf{$||$} (\ref{labor})

\subsection{file}
\label{labfile}
\noindent Name: \textbf{file}\\
special value for commands \textbf{plot} and \textbf{externalplot}\\
\noindent Description: \begin{itemize}

\item \textbf{file} is a special value used in commands \textbf{plot} and \textbf{externalplot} to save
   the result of the command in a data file.

\item As any value it can be affected to a variable and stored in lists.
\end{itemize}
\noindent Example 1: 
\begin{center}\begin{minipage}{15cm}\begin{Verbatim}[frame=single]
> savemode=file;
> name="plotSinCos";
> plot(sin(x),0,cos(x),[-Pi,Pi],savemode, name);
\end{Verbatim}
\end{minipage}\end{center}
See also: \textbf{externalplot} (\ref{labexternalplot}), \textbf{plot} (\ref{labplot}), \textbf{postscript} (\ref{labpostscript}), \textbf{postscriptfile} (\ref{labpostscriptfile})

\subsection{findzeros}
\label{labfindzeros}
\noindent Name: \textbf{findzeros}\\
gives a list of intervals containing all zeros of a function on an interval.\\

\noindent Usage: 
\begin{center}
\textbf{findzeros}(\emph{f},\emph{I}) : (\textsf{function}, \textsf{range}) $\rightarrow$ \textsf{list}\\
\end{center}
Parameters: 
\begin{itemize}
\item \emph{f} is a function.
\item \emph{I} is an interval.
\end{itemize}
\noindent Description: \begin{itemize}

\item \textbf{findzeros}(\emph{f},\emph{I}) returns a list of intervals \emph{I1}, ... ,\emph{In} such that, for 
   every zero $z$ of $f$, there exists some $k$ such that $z \in I_k$.

\item The list may contain intervals \emph{Ik} that do not contain any zero of \emph{f}.
   An interval \emph{Ik} may contain many zeros of \emph{f}.

\item This command is ment for cases when safety is critical. If you want to be sure
   not to forget any zero, use \textbf{findzeros}. However, if you just want to know 
   numerical values for the zeros of \emph{f}, \textbf{dirtyfindzeros} should be quite 
   satisfactory and a lot faster.

\item If $\delta$ denotes the value of global variable \textbf{diam}, the algorithm ensures
   that for each $k$, $|I_k| \le \delta \cdot |I|$.

\item The algorithm used is basically a bisection algorithm. It is the same algorithm
   that the one used for \textbf{infnorm}. See the help page of this command for more 
   details. In short, the behavior of the algorithm depends on global variables
   \textbf{prec}, \textbf{diam}, \textbf{taylorrecursions} and \textbf{hopitalrecursions}.
\end{itemize}
\noindent Example 1: 
\begin{center}\begin{minipage}{15cm}\begin{Verbatim}[frame=single]
> findzeros(sin(x),[-5;5]);
[|[-3.14208984375;-3.140869140625], [-1.220703125e-3;1.220703125e-3], [3.1408691
40625;3.14208984375]|]
> diam=1e-10!;
> findzeros(sin(x),[-5;5]);
[|[-3.14159265370108187198638916015625;-3.141592652536928653717041015625], [-1.1
6415321826934814453125e-9;1.16415321826934814453125e-9], [3.14159265253692865371
7041015625;3.14159265370108187198638916015625]|]
\end{Verbatim}
\end{minipage}\end{center}
See also: \textbf{dirtyfindzeros} (\ref{labdirtyfindzeros}), \textbf{infnorm} (\ref{labinfnorm}), \textbf{prec} (\ref{labprec}), \textbf{diam} (\ref{labdiam}), \textbf{taylorrecursions} (\ref{labtaylorrecursions}), \textbf{hopitalrecursions} (\ref{labhopitalrecursions})

\subsection{fixed}
\label{labfixed}
\noindent Name: \textbf{fixed}\\
indicates that fixed-point formats should be used for \textbf{fpminimax}\\
\noindent Usage: 
\begin{center}
\textbf{fixed} : \textsf{fixed$|$floating}\\
\end{center}
\noindent Description: \begin{itemize}

\item The use of \\textbf{fixed} in the command \\textbf{fpminimax} indicates that the list of\n   formats given as argument is to be considered to be a list of fixed-point\n   formats.\n   See \\textbf{fpminimax} for details.\n\end{itemize}
\noindent Example 1: 
\begin{center}\begin{minipage}{15cm}\begin{Verbatim}[frame=single]
\end{Verbatim}
\end{minipage}\end{center}
See also: \textbf{fpminimax} (\ref{labfpminimax}), \textbf{floating} (\ref{labfloating})

\subsection{floating}
\label{labfloating}
\noindent Name: \textbf{floating}\\
\phantom{aaa}indicates that floating-point formats should be used for \textbf{fpminimax}\\[0.2cm]
\noindent Library names:\\
\verb|   sollya_obj_t sollya_lib_floating()|\\
\verb|   int sollya_lib_is_floating(sollya_obj_t)|\\[0.2cm]
\noindent Usage: 
\begin{center}
\textbf{floating} : \textsf{fixed$|$floating}\\
\end{center}
\noindent Description: \begin{itemize}

\item The use of \textbf{floating} in the command \textbf{fpminimax} indicates that the list of
   formats given as argument is to be considered to be a list of floating-point
   formats.
   See \textbf{fpminimax} for details.
\end{itemize}
\noindent Example 1: 
\begin{center}\begin{minipage}{15cm}\begin{Verbatim}[frame=single]
> fpminimax(cos(x),6,[|D...|],[-1;1],floating);
0.99999974816012948686250183527590706944465637207031 + x * (5.521004406122249513
1782035802443168321913900126185e-14 + x * (-0.4999928698019768802396356477402150
630950927734375 + x * (-3.95371609372064761555136192612768146546591008227978e-13
 + x * (4.16335155285858099505347240665287245064973831176758e-2 + x * (5.2492670
395835122748014980938834327670386437070249e-13 + x * (-1.33822408807599468535953
768366653093835338950157166e-3))))))
\end{Verbatim}
\end{minipage}\end{center}
See also: \textbf{fpminimax} (\ref{labfpminimax}), \textbf{fixed} (\ref{labfixed})

\subsection{floor}
\label{labfloor}
\noindent Name: \textbf{floor}\\
the usual function floor.\\
\noindent Description: \begin{itemize}

\item \textbf{floor} is defined as usual: \textbf{floor}(x) is the greatest integer y such that $y \le x$.

\item It is defined for every real number x.
\end{itemize}
See also: \textbf{ceil} (\ref{labceil})

\subsection{fpminimax}
\label{labfpminimax}
\noindent Name: \textbf{fpminimax}\\
computes a good polynomial approximation with fixed-point or floating-point coefficients\\
\noindent Usage: 
\begin{center}
\textbf{fpminimax}(\emph{f}, \emph{n}, \emph{formats}, \emph{range}, \emph{indic1}, \emph{indic2}, \emph{indic3}, \emph{P}) : (\textsf{function}, \textsf{integer}, \textsf{list}, \textsf{range}, \textsf{absolute$|$relative} $|$ \textsf{fixed$|$floating} $|$ \textsf{function}, \textsf{absolute$|$relative} $|$ \textsf{fixed$|$floating} $|$ \textsf{function}, \textsf{absolute$|$relative} $|$ \textsf{fixed$|$floating} $|$ \textsf{function}, \textsf{function}) $\rightarrow$ \textsf{function}
\textbf{fpminimax}(\emph{f}, \emph{monomials}, \emph{formats}, \emph{range}, \emph{indic1}, \emph{indic2}, \emph{indic3}, \emph{P}) : (\textsf{function}, \textsf{list}, \textsf{list}, \textsf{range},  \textsf{absolute$|$relative} $|$ \textsf{fixed$|$floating} $|$ \textsf{function}, \textsf{absolute$|$relative} $|$ \textsf{fixed$|$floating} $|$ \textsf{function}, \textsf{absolute$|$relative} $|$ \textsf{fixed$|$floating} $|$ \textsf{function}, \textsf{function}) $\rightarrow$ \textsf{function}
\textbf{fpminimax}(\emph{f}, \emph{n}, \emph{formats}, \emph{L}, \emph{indic1}, \emph{indic2}, \emph{indic3}, \emph{P}) : (\textsf{function}, \textsf{integer}, \textsf{list}, \textsf{list},  \textsf{absolute$|$relative} $|$ \textsf{fixed$|$floating} $|$ \textsf{function}, \textsf{absolute$|$relative} $|$ \textsf{fixed$|$floating} $|$ \textsf{function}, \textsf{absolute$|$relative} $|$ \textsf{fixed$|$floating} $|$ \textsf{function}, \textsf{function}) $\rightarrow$ \textsf{function}
\textbf{fpminimax}(\emph{f}, \emph{monomials}, \emph{formats}, \emph{L}, \emph{indic1}, \emph{indic2}, \emph{indic3}, \emph{P}) : (\textsf{function}, \textsf{list}, \textsf{list}, \textsf{list},  \textsf{absolute$|$relative} $|$ \textsf{fixed$|$floating} $|$ \textsf{function}, \textsf{absolute$|$relative} $|$ \textsf{fixed$|$floating} $|$ \textsf{function}, \textsf{absolute$|$relative} $|$ \textsf{fixed$|$floating} $|$ \textsf{function}, \textsf{function}) $\rightarrow$ \textsf{function}
\end{center}
Parameters: 
\begin{itemize}
\item \emph{f} is the function to be approximated
\item \emph{n} is the degree of the polynomial that must approximate \emph{f}
\item \emph{monomials} is the list of monomials that must be used to represent the polynomial that approximates~\emph{f}
\item \emph{formats} is a list indicating the formats that the coefficients of the polynomial must have
\item \emph{range} is the interval where the function must be approximated
\item \emph{L} is a list of interpolation points used by the method
\item \emph{indic1} (optional) is one of the optional indication parameters. See the detailed description below.
\item \emph{indic2} (optional) is one of the optional indication parameters. See the detailed description below.
\item \emph{indic3} (optional) is one of the optional indication parameters. See the detailed description below.
\item \emph{P} (optional) is the minimax polynomial to be considered for solving the problem.
\end{itemize}
\noindent Description: \begin{itemize}

\item \textbf{fpminimax} uses a heuristic (but practically efficient) method to find a good
   polynomial approximation of a function \emph{f} on an interval \emph{range}. It 
   implements the method published in the article:\\
   Efficient polynomial $L^\infty$-approximations\\ 
   Nicolas Brisebarre and Sylvain Chevillard\\
   Proceedings of the 18th IEEE Symposium on Computer Arithmetic (ARITH 18)\\
   pp. 169-176

\item The basic usage of this command is \textbf{fpminimax}(\emph{f}, \emph{n}, \emph{formats}, \emph{range}).
   It computes a polynomial approximation of $f$ with degree at most $n$
   on the interval \emph{range}. \emph{formats} is a list of integers or format types 
   (such as \textbf{double}, \textbf{doubledouble}, etc.). The polynomial returned by the
   command has its coefficients that fit the formats indications. For 
   instance, if formats[0] is 35, the coefficient of degree 0 of the 
   polynomial will fit a floating-point format of 35 bits. If formats[1] 
   is D, the coefficient of degree 1 will be representable by a floating-point
   number with a precision of 53 bits (that is not necessarily an IEEE double
   precision number. See the remark below), etc.

\item The second argument may be either an integer or a list of integers
   interpreted as the list of desired monomials. For instance, the list
   $[|0,\,2,\,4,\,6|]$ indicates that the polynomial must be even and of
   degree at most 6. Giving an integer $n$ as second argument is equivalent
   as giving $[|0,\,\dots,\,n|]$.\\
   The list of format is interpreted with respect to the list of monomials. For
   instance, if the list of monomials is $[|0,\,2,\,4,\,6|]$ and the list
   of formats is $[|161,\,107,\,53,\,24|]$, the coefficients of degree 0 is 
   searched as a floating-point number with precision 161, the coefficient of 
   degree 2 is searched as a number of precision 107, and so on.

\item The list of formats may contain either integers or format types (\textbf{double},
   \textbf{doubledouble}, \textbf{tripledouble} and \textbf{doubleextended}). The list may be too big
   or even infinite. Only the first indications will be considered. For 
   instance, for a degree $n$ polynomial, $\mathrm{formats}[n+1]$ and above will
   be discarded. This lets one use elliptical indications for the last
   coefficients.

\item The floating-point coefficients considered by \textbf{fpminimax} do not have an
   exponent range. In particular, in the format list, \textbf{double} or 53 does not
   imply that the corresponding coefficient is an IEEE-754 double.

\item By default, the list of formats is interpreted as a list of floating-point
   formats. This may be changed by passing \textbf{fixed} as an optional argument (see
   below). Let us take an example: \textbf{fpminimax}(f, 2, [107, DD, 53], [0;1]).
   Here the optional argument is missing (we could have set it to \textbf{floating}).
   Thus, \textbf{fpminimax} will search for a polynomial of degree 2 with a constant 
   coefficient that is a 107 bits floating-point number, etc.\\
   Currently, \textbf{doubledouble} is just a synonym for 107 and \textbf{tripledouble} a 
   synonym for 161. This behavior may change in the future (taking into
   account the fact that some double-doubles are not representable with
   107 bits).\\
   Second example: \textbf{fpminimax}(f, 2, [25, 18, 30], [0;1], \textbf{fixed}).
   In this case, \textbf{fpminimax} will search for a polynomial of degree 2 with a
   constant coefficient of the form $m/2^{25}$ where $m$ is an
   integer. In other words, it is a fixed-point number with 25 bits after
   the point. Note that even with argument \textbf{fixed}, the formats list is 
   allowed to contain \textbf{double}, \textbf{doubledouble} or \textbf{tripledouble}. In this this
   case, it is just a synonym for 53, 107 or 161. This is deprecated and may
   change in the future.

\item The fourth argument may be a range or a list. Lists are for advanced users
   that know what they are doing. The core of the  method is a kind of
   approximated interpolation. The list given here is a list of points that
   must be considered for the interpolation. It must contain at least as 
   many points as unknown coefficients. If you give a list, it is also 
   recommended that you provide the minimax polynomial as last argument.
   If you give a range, the list of points will be automatically computed.

\item The fifth, sixth and seventh arguments are optional. By default, \textbf{fpminimax}
   will approximate $f$ optimizing the relative error, and interpreting
   the list of formats as a list of floating-point formats.\\
   This default behavior may be changed with these optional arguments. You
   may provide zero, one, two or three of the arguments and in any order.
   This lets the user indicate only the non-default arguments.\\
   The three possible arguments are: \begin{itemize}
   \item \textbf{relative} or \textbf{absolute}: the error to be optimized;
   \item \textbf{floating} or \textbf{fixed}: formats of the coefficients;
   \item a constrained part $q$.
   \end{itemize}
   The constrained part lets the user assign in advance some of the
   coefficients. For instance, for approximating $\exp(x)$, it may
   be interesting to search for a polynomial $p$ of the form
                   $$p = 1 + x + \frac{x^2}{2} + a_3 x^3 + a_4 x^4.$$
   Thus, there is a constrained part $q = 1 + x + x^2/2$ and the unknown
   polynomial should be considered in the monomial basis $[|3, 4|]$.
   Calling \textbf{fpminimax} with monomial basis $[|3,\,4|]$ and constrained
   part $q$, will return a polynomial with the right form.

\item The last argument is for advanced users. It is the minimax polynomial that
   approximates the function $f$ in the monomial basis. If it is not given
   this polynomial will be automatically computed by \textbf{fpminimax}.
   \\
   This minimax polynomial is used to compute the list of interpolation
   points required by the method. In general, you do not have to provide this
   argument. But if you want to obtain several polynomials of the same degree
   that approximate the same function on the same range, just changing the
   formats, you should probably consider computing only once the minimax
   polynomial and the list of points instead of letting \textbf{fpminimax} recompute
   them each time.
   \\
   Note that in the case when a constrained part is given, the minimax 
   polynomial must take it into account. For instance, in the previous
   example, the minimax would be obtained by the following command:
          \begin{center}\verb~P = remez(1-(1+x+x^2/2)/exp(x), [|3,4|], range, 1/exp(x));~\end{center}
   Note that the constrained part is not to be added to $P$.

\item Note that \textbf{fpminimax} internally computes a minimax polynomial (using
   the same algorithm as \textbf{remez} command). Thus \textbf{fpminimax} may encounter
   the same problems as \textbf{remez}. In particular, it may be very long 
   when Haar condition is not fulfilled. Another consequence is that
   currently \textbf{fpminimax} has to be run with a sufficiently high precision.
\end{itemize}
\noindent Example 1: 
\begin{center}\begin{minipage}{15cm}\begin{Verbatim}[frame=single]
> P = fpminimax(cos(x),6,[|DD, DD, D...|],[-1b-5;1b-5]);
> printexpansion(P);
(0x3ff0000000000000 + 0xbc09fda20235c100) + x * ((0x3b29ecd485d34781 + 0xb7c1cbc
97120359a) + x * (0xbfdfffffffffff98 + x * (0xbbfa6e0b3183cb0d + x * (0x3fa55555
55145337 + x * (0x3ca3540480618939 + x * 0xbf56c138142d8c3b)))))
\end{Verbatim}
\end{minipage}\end{center}
\noindent Example 2: 
\begin{center}\begin{minipage}{15cm}\begin{Verbatim}[frame=single]
> P = fpminimax(sin(x),6,[|32...|],[-1b-5;1b-5], fixed, absolute);
> display = powers!;
> P;
x * (1 + x^2 * (-357913941 * 2^(-31) + x^2 * (35789873 * 2^(-32))))
\end{Verbatim}
\end{minipage}\end{center}
\noindent Example 3: 
\begin{center}\begin{minipage}{15cm}\begin{Verbatim}[frame=single]
> P = fpminimax(exp(x), [|3,4|], [|D,24|], [-1/256; 1/246], 1+x+x^2/2);
> display = powers!;
> P;
1 + x * (1 + x * (1 * 2^(-1) + x * (375300225001191 * 2^(-51) + x * (5592621 * 2
^(-27)))))
\end{Verbatim}
\end{minipage}\end{center}
\noindent Example 4: 
\begin{center}\begin{minipage}{15cm}\begin{Verbatim}[frame=single]
> f = cos(exp(x));
> pstar = remez(f, 5, [-1b-7;1b-7]);
> listpoints = dirtyfindzeros(f-pstar, [-1b-7; 1b-7]);
> P1 = fpminimax(f, 5, [|DD...|], listpoints, absolute, default, default, pstar)
;
> P2 = fpminimax(f, 5, [|D...|], listpoints, absolute, default, default, pstar);

> P3 = fpminimax(f, 5, [|D, D, D, 24...|], listpoints, absolute, default, defaul
t, pstar);
> print("Error of pstar: ", dirtyinfnorm(f-pstar, [-1b-7; 1b-7]));
Error of pstar:  7.9048441305459735102879831325718745399379329453102e-16
> print("Error of P1:    ", dirtyinfnorm(f-P1, [-1b-7; 1b-7]));
Error of P1:     7.9048441305459735159848647089192667442047469404883e-16
> print("Error of P2:    ", dirtyinfnorm(f-P2, [-1b-7; 1b-7]));
Error of P2:     8.2477144579950871061147021597406077993657714575238e-16
> print("Error of P3:    ", dirtyinfnorm(f-P3, [-1b-7; 1b-7]));
Error of P3:     1.08454277156993282593701156841863009789063333951055e-15
\end{Verbatim}
\end{minipage}\end{center}
See also: \textbf{remez} (\ref{labremez}), \textbf{dirtyfindzeros} (\ref{labdirtyfindzeros}), \textbf{absolute} (\ref{lababsolute}), \textbf{relative} (\ref{labrelative}), \textbf{fixed} (\ref{labfixed}), \textbf{floating} (\ref{labfloating}), \textbf{default} (\ref{labdefault})

\subsection{fullparentheses}
\label{labfullparentheses}
\noindent Name: \textbf{fullparentheses}\\
activates, deactivates or inspects the state variable controlling output with full parenthesising\\

\noindent Usage: 
\begin{center}
\textbf{fullparentheses} = \emph{activation value} : \textsf{on$|$off} $\rightarrow$ \textsf{void}\\
\textbf{fullparentheses} = \emph{activation value} ! : \textsf{on$|$off} $\rightarrow$ \textsf{void}\\
\end{center}
Parameters: 
\begin{itemize}
\item \emph{activation value} represents \textbf{on} or \textbf{off}, i.e. activation or deactivation
\end{itemize}
\noindent Description: \begin{itemize}

\item An assignment \textbf{fullparentheses} = \emph{activation value}, where \emph{activation value}
   is one of \textbf{on} or \textbf{off}, activates respectively deactivates the output
   of expressions with full parenthesising. In full parenthesising mode,
   Sollya commands like \textbf{print}, \textbf{write} and the implicit command when an
   expression is given at the prompt will output expressions with
   parenthesising at all places where it is necessary for expressions
   containing infix operators to be parsed back with the same
   result. Otherwise parentheses around associative operators are
   omitted.
    
   If the assignment \textbf{fullparentheses} = \emph{activation value} is followed by an
   exclamation mark, no message indicating the new state is
   displayed. Otherwise the user is informed of the new state of the
   global mode by an indication.
\end{itemize}
\noindent Example 1: 
\begin{center}\begin{minipage}{15cm}\begin{Verbatim}[frame=single]
> autosimplify = off!;
> fullparentheses = off;
Full parentheses mode has been deactivated.
> print(1 + 2 + 3);
1 + 2 + 3
> fullparentheses = on;
Full parentheses mode has been activated.
> print(1 + 2 + 3);
(1 + 2) + 3
\end{Verbatim}
\end{minipage}\end{center}
See also: \textbf{print} (\ref{labprint}), \textbf{write} (\ref{labwrite}), \textbf{autosimplify} (\ref{labautosimplify})

\subsection{ function }
\noindent Name: \textbf{function}\\
keyword representing a \textsf{function} type \\

\noindent Usage: 
\begin{center}
\textbf{function} : \textsf{type type}\\
\end{center}
\noindent Description: \begin{itemize}

\item \textbf{function} represents the \textsf{function} type for declarations
   of external procedures by means of \textbf{externalproc}.
   Remark that in contrast to other indicators, type indicators like
   \textbf{function} cannot be handled outside the \textbf{externalproc} context.  In
   particular, they cannot be assigned to variables.
\end{itemize}
See also: \textbf{externalproc}, \textbf{boolean}, \textbf{constant}, \textbf{integer}, \textbf{list of}, \textbf{range}, \textbf{string}

\subsection{ge}
\label{labge}
\noindent Name: \textbf{$>=$}\\
greater-than-or-equal-to operator\\
\noindent Usage: 
\begin{center}
\emph{expr1} \textbf{$>=$} \emph{expr2} : (\textsf{constant}, \textsf{constant}) $\rightarrow$ \textsf{boolean}
\\ 
\end{center}
Parameters: 
\begin{itemize}
\item \emph{expr1} and \emph{expr2} represent constant expressions
\end{itemize}
\noindent Description: \begin{itemize}

\item The operator \textbf{$>=$} evaluates to true iff its operands \emph{expr1} and
   \emph{expr2} evaluate to two floating-point numbers $a_1$
   respectively $a_2$ with the global precision \textbf{prec} and
   $a_1$ is greater than or equal to $a_2$. The user should
   be aware of the fact that because of floating-point evaluation, the
   operator \textbf{$>=$} is not exactly the same as the mathematical
   operation \emph{greater-than-or-equal-to}.
\end{itemize}
\noindent Example 1: 
\begin{center}\begin{minipage}{15cm}\begin{Verbatim}[frame=single]
> 5 >= 4;
true
> 5 >= 5;
true
> 5 >= 6;
false
> exp(2) >= exp(1);
true
> log(1) >= exp(2);
false
\end{Verbatim}
\end{minipage}\end{center}
\noindent Example 2: 
\begin{center}\begin{minipage}{15cm}\begin{Verbatim}[frame=single]
> prec = 12;
The precision has been set to 12 bits.
> 16384.1 >= 16385.1;
true
\end{Verbatim}
\end{minipage}\end{center}
See also: \textbf{$==$} (\ref{labequal}), \textbf{!$=$} (\ref{labneq}), \textbf{$>$} (\ref{labgt}), \textbf{$<=$} (\ref{lable}), \textbf{$<$} (\ref{lablt}), \textbf{!} (\ref{labnot}), \textbf{$\&\&$} (\ref{laband}), \textbf{$||$} (\ref{labor}), \textbf{prec} (\ref{labprec})

\subsection{gt}
\label{labgt}
\noindent Name: \textbf{$>$}\\
greater-than operator\\
\noindent Usage: 
\begin{center}
\emph{expr1} \textbf{$>$} \emph{expr2} : (\textsf{constant}, \textsf{constant}) $\rightarrow$ \textsf{boolean}
\\ 
\end{center}
Parameters: 
\begin{itemize}
\item \emph{expr1} and \emph{expr2} represent constant expressions
\end{itemize}
\noindent Description: \begin{itemize}

\item The operator \textbf{$>$} evaluates to true iff its operands \emph{expr1} and
   \emph{expr2} evaluate to two floating-point numbers $a_1$
   respectively $a_2$ with the global precision \textbf{prec} and
   $a_1$ is greater than $a_2$. The user should
   be aware of the fact that because of floating-point evaluation, the
   operator \textbf{$>$} is not exactly the same as the mathematical
   operation \emph{greater-than}.
\end{itemize}
\noindent Example 1: 
\begin{center}\begin{minipage}{15cm}\begin{Verbatim}[frame=single]
> 5 > 4;
true
> 5 > 5;
false
> 5 > 6;
false
> exp(2) > exp(1);
true
> log(1) > exp(2);
false
\end{Verbatim}
\end{minipage}\end{center}
\noindent Example 2: 
\begin{center}\begin{minipage}{15cm}\begin{Verbatim}[frame=single]
> prec = 12;
The precision has been set to 12 bits.
> 16385.1 > 16384.1;
false
\end{Verbatim}
\end{minipage}\end{center}
See also: \textbf{$==$} (\ref{labequal}), \textbf{!$=$} (\ref{labneq}), \textbf{$>=$} (\ref{labge}), \textbf{$<=$} (\ref{lable}), \textbf{$<$} (\ref{lablt}), \textbf{!} (\ref{labnot}), \textbf{$\&\&$} (\ref{laband}), \textbf{$||$} (\ref{labor}), \textbf{prec} (\ref{labprec})

\subsection{ guessdegree }
\noindent Name: \textbf{guessdegree}\\
returns the minimal degree needed for a polynomial to approximate a function with a certain error on an interval.\\

\noindent Usage: 
\begin{center}
\textbf{guessdegree}(\emph{f},\emph{I},\emph{eps},\emph{w}) : (\textsf{function}, \textsf{range}, \textsf{constant}, \textsf{function}) $\rightarrow$ \textsf{range}\\
\end{center}
Parameters: 
\begin{itemize}
\item \emph{f} is the function to be approximated.
\item \emph{I} is the interval where the function must be approximated.
\item \emph{eps} is the maximal acceptable error.
\item \emph{w} (optional) is a weight function. Default is 1.
\end{itemize}
\noindent Description: \begin{itemize}

\item \textbf{guessdegree} tries to find the minimal degree needed to approximate \emph{f}
   on \emph{I} by a polynomial with an infinite error not greater than \emph{eps}.
   More precisely, it finds $n$ minimal such that there exists a
   polynomial $p$ of degree $n$ such that $\|pw-f\|_{\infty} < \mathrm{eps}$.

\item \textbf{guessdegree} returns an interval: for common cases, this interval is reduced to a 
   single number (e.g. the minimal degree). But in certain cases, \textbf{guessdegree} does
   not succeed in finding the minimal degree. In such cases the returned interval
   is of the form $[n,\,p]$ such that:
   \begin{itemize}
   \item no polynomial of degree $n-1$ gives an error less than \emph{eps}.
   \item there exists a polynomial of degree $p$ giving an error less than \emph{eps}. 
   \end{itemize}
\end{itemize}
\noindent Example 1: 
\begin{center}\begin{minipage}{15cm}\begin{Verbatim}[frame=single]
> guessdegree(exp(x),[-1;1],1e-10);
[10;10]
\end{Verbatim}
\end{minipage}\end{center}
\noindent Example 2: 
\begin{center}\begin{minipage}{15cm}\begin{Verbatim}[frame=single]
> guessdegree(1, [-1;1], 1e-8, 1/exp(x));
[8;9]
\end{Verbatim}
\end{minipage}\end{center}
See also: \textbf{dirtyinfnorm}, \textbf{remez}

\subsection{halfprecision}
\label{labhalfprecision}
\noindent Names: \textbf{halfprecision}, \textbf{HP}\\
rounding to the nearest IEEE 754 half-precision number (binary16).\\
\noindent Description: \begin{itemize}

\item \textbf{halfprecision} is both a function and a constant.

\item As a function, it rounds its argument to the nearest IEEE 754 half-precision (i.e. IEEE754-2008 binary16) number.
   Subnormal numbers are supported as well as standard numbers: it is the real
   rounding described in the standard.

\item As a constant, it symbolizes the half-precision format. It is used in 
   contexts when a precision format is necessary, e.g. in the commands 
   \textbf{round} and \textbf{roundcoefficients}. It is not supported for \textbf{implementpoly}.
   See the corresponding help pages for examples.
\end{itemize}
\noindent Example 1: 
\begin{center}\begin{minipage}{15cm}\begin{Verbatim}[frame=single]
> display=binary!;
> HP(0.1);
1.100110011_2 * 2^(-4)
> HP(4.17);
1.00001011_2 * 2^(2)
> HP(1.011_2 * 2^(-23));
1.1_2 * 2^(-23)
\end{Verbatim}
\end{minipage}\end{center}
See also: \textbf{single} (\ref{labsingle}), \textbf{double} (\ref{labdouble}), \textbf{doubleextended} (\ref{labdoubleextended}), \textbf{doubledouble} (\ref{labdoubledouble}), \textbf{quad} (\ref{labquad}), \textbf{tripledouble} (\ref{labtripledouble}), \textbf{roundcoefficients} (\ref{labroundcoefficients}), \textbf{fpminimax} (\ref{labfpminimax}), \textbf{implementpoly} (\ref{labimplementpoly}), \textbf{round} (\ref{labround}), \textbf{printsingle} (\ref{labprintsingle})

\subsection{head}
\label{labhead}
\noindent Name: \textbf{head}\\
gives the first element of a list.\\
\noindent Usage: 
\begin{center}
\textbf{head}(\emph{L}) : \textsf{list} $\rightarrow$ \textsf{any type}
\end{center}
Parameters: 
\begin{itemize}
\item \emph{L} is a list.
\end{itemize}
\noindent Description: \begin{itemize}

\item \textbf{head}(\emph{L}) returns the first element of the list \emph{L}. It is equivalent
   to L[0].

\item If \emph{L} is empty, the command will fail with an error.
\end{itemize}
\noindent Example 1: 
\begin{center}\begin{minipage}{15cm}\begin{Verbatim}[frame=single]
> head([|1,2,3|]);
1
> head([|1,2...|]);
1
\end{Verbatim}
\end{minipage}\end{center}
See also: \textbf{tail} (\ref{labtail})

\subsection{hexadecimal}
\label{labhexadecimal}
\noindent Name: \textbf{hexadecimal}\\
special value for global state 	extbf{display}\\
\noindent Description: \begin{itemize}

\item \textbf{hexadecimal} is a special value used for the global state \textbf{display}.  If
   the global state \textbf{display} is equal to \textbf{hexadecimal}, all data will be
   output in hexadecimal C99/ IEEE 754R notation.
    
   As any value it can be affected to a variable and stored in lists.
\end{itemize}
See also: \textbf{decimal} (\ref{labdecimal}), \textbf{dyadic} (\ref{labdyadic}), \textbf{powers} (\ref{labpowers}), \textbf{binary} (\ref{labbinary})

\subsection{honorcoeffprec}
\label{labhonorcoeffprec}
\noindent Name: \textbf{honorcoeffprec}\\
indicates the (forced) honoring the precision of the coefficients in \textbf{implementpoly}\\
\noindent Usage: 
\begin{center}
\textbf{honorcoeffprec} : \textsf{honorcoeffprec}\\
\end{center}
\noindent Description: \begin{itemize}

\item Used with command \\textbf{implementpoly}, \\textbf{honorcoeffprec} makes \\textbf{implementpoly} honor\n   the precision of the given polynomial. This means if a coefficient\n   needs a double-double or a triple-double to be exactly stored,\n   \\textbf{implementpoly} will allocate appropriate space and use a double-double\n   or triple-double operation even if the automatic (heuristic)\n   determination implemented in command \\textbf{implementpoly} indicates that the\n   coefficient could be stored on less precision or, respectively, the\n   operation could be performed with less precision. See \\textbf{implementpoly}\n   for details.\n\end{itemize}
\noindent Example 1: 
\begin{center}\begin{minipage}{15cm}\begin{Verbatim}[frame=single]
\end{Verbatim}
\end{minipage}\end{center}
See also: \textbf{implementpoly} (\ref{labimplementpoly}), \textbf{printexpansion} (\ref{labprintexpansion})

\subsection{hopitalrecursions}
\label{labhopitalrecursions}
\noindent Name: \textbf{hopitalrecursions}\\
controls the number of recursion steps when applying L'Hopital's rule.\\

\noindent Description: \begin{itemize}

\item \textbf{hopitalrecursions} is a global variable. Its value represents the number of steps of
   recursion that are tried when applying L'Hopital's rule. This rule is applied
   by the interval evaluator present in the core of Sollya (and particularly
   visible in commands like \textbf{infnorm}).

\item If an expression of the form $f/g$ has to be evaluated by interval 
   arithmetic on an interval $I$ and if $f$ and $g$ have a common zero
   in $I$, a direct evaluation leads to NaN.
   Sollya implements a safe heuristic to avoid this, based on L'Hopital's rule: in 
   such a case, it can be shown that $(f/g)(I) \subseteq (f'/g')(I)$. Since
   the same problem may hold for $f'/g'$, the rule is applied recursively.
   The number of step in this recursion process is controlled by \textbf{hopitalrecursions}.

\item Setting \textbf{hopitalrecursions} to 0 makes Sollya use this rule only one time ;
   setting it to 1 makes Sollya use the rule two times, and so on.
   In particular: the rule is always applied at least once, if necessary.
\end{itemize}
\noindent Example 1: 
\begin{center}\begin{minipage}{15cm}\begin{Verbatim}[frame=single]
> hopitalrecursions=0;
The number of recursions for Hopital's rule has been set to 0.
> evaluate(log(1+x)^2/x^2,[-1/2; 1]);
[-@Inf@;@Inf@]
> hopitalrecursions=1;
The number of recursions for Hopital's rule has been set to 1.
> evaluate(log(1+x)^2/x^2,[-1/2; 1]);
[-2.52258872223978123766892848583270627230200053744108;6.77258872223978123766892
84858327062723020005374411]
\end{Verbatim}
\end{minipage}\end{center}

\subsection{ horner }
\noindent Name: \textbf{horner}\\
brings all polynomial subexpressions of an expression to Horner form\\

\noindent Usage: 
\begin{center}
\textbf{horner}(\emph{function}) : \textsf{function} $\rightarrow$ \textsf{function}\\
\end{center}
Parameters: 
\emph{function} represents the expression to be rewritten in Horner form\\

\noindent Description: \begin{itemize}

\item The command \textbf{horner} rewrites the expression representing the function
   \emph{function} in a way such that all polynomial subexpressions (or the
   whole expression itself, if it is a polynomial) are written in Horner
   form.  The command \textbf{horner} does not endanger the safety of
   computations even in Sollya's floating-point environment: the
   function returned is mathematically equal to the function \emph{function}.
\end{itemize}
\noindent Example 1: 
\begin{center}\begin{minipage}{14.8cm}\begin{Verbatim}[frame=single]
   > print(horner(1 + 2 * x + 3 * x^2));
   1 + x * (2 + x * 3)
   > print(horner((x + 1)^7));
   1 + x * (7 + x * (21 + x * (35 + x * (35 + x * (21 + x * (7 + x))))))
\end{Verbatim}
\end{minipage}\end{center}
\noindent Example 2: 
\begin{center}\begin{minipage}{14.8cm}\begin{Verbatim}[frame=single]
   > print(horner(exp((x + 1)^5) - log(asin(x + x^3) + x)));
   exp(1 + x * (5 + x * (10 + x * (10 + x * (5 + x))))) - log(asin(x * (1 + x^2)) + x)
\end{Verbatim}
\end{minipage}\end{center}
See also: \textbf{canonical}, \textbf{print}

\subsection{HP}
\label{labhp}
\noindent Name: \textbf{HP}\\
\phantom{aaa}short form for \textbf{halfprecision}\\[0.2cm]
See also: \textbf{halfprecision} (\ref{labhalfprecision})

\subsection{implementconstant}
\label{labimplementconstant}
\noindent Name: \textbf{implementconstant}\\
\phantom{aaa}implements a constant in arbitrary precision\\[0.2cm]
\noindent Usage: 
\begin{center}
\textbf{implementconstant}(\emph{expr}) : \textsf{constant} $\rightarrow$ \textsf{void}\\
\textbf{implementconstant}(\emph{expr},\emph{filename}) : (\textsf{constant}, \textsf{string}) $\rightarrow$ \textsf{void}\\
\textbf{implementconstant}(\emph{expr},\emph{filename},\emph{functionname}) : (\textsf{constant}, \textsf{string}, \textsf{string}) $\rightarrow$ \textsf{void}\\
\end{center}
\noindent Description: \begin{itemize}

\item The command \textbf{implementconstant} implements the constant expression \emph{expr} in 
   arbitrary precision. More precisely, it generates the source code (written
   in C, and using MPFR) of a C function \texttt{const\_something} with the following
   signature:
   \begin{center}
   \texttt{void const\_something (mpfr\_ptr y, mp\_prec\_t prec)}
   \end{center}
   Let us denote by $c$ the exact mathematical value of the constant defined by
   the expression \emph{expr}. When called with arguments $y$ and prec (where the
   variable $y$ is supposed to be already initialized), the function
   \texttt{mpfr\_const\_something} sets the precision of $y$ to a suitable precision and
   stores in it an approximate value of $c$ such that
   $$|y-c| \le |c|\,2^{1-\mathrm{prec}}.$$

\item When no filename \emph{filename} is given or if \textbf{default} is given as
   \emph{filename}, the source code produced by \textbf{implementconstant} is printed on
   standard output. Otherwise, when \emph{filename} is given as a 
   string of characters, the source code is output to a file 
   named \emph{filename}. If that file cannot be opened and/or 
   written to, \textbf{implementconstant} fails and has no other effect.

\item When \emph{functionname} is given as an argument to \textbf{implementconstant} and
   \emph{functionname} evaluates to a string of characters, the default name
   for the C function \texttt{const\_something} is
   replaced by \emph{functionname}. When \textbf{default} is given as \emph{functionname},
   the default name is used nevertheless, as if no \emph{functionname}
   argument were given.  When choosing a character sequence for
   \emph{functionname}, the user should keep attention to the fact that
   \emph{functionname} must be a valid C identifier in order to enable
   error-free compilation of the produced code.

\item If \emph{expr} refers to a constant defined with \textbf{libraryconstant}, the produced
   code uses the external code implementing this constant. The user should
   keep in mind that it is up to them to make sure the symbol for that 
   external code can get resolved when the newly generated code is to 
   be loaded.

\item If a subexpression of \emph{expr} evaluates to $0$, \textbf{implementconstant} will most
   likely fail with an error message.

\item \textbf{implementconstant} is unable to implement constant expressions \emph{expr} that
   contain procedure-based functions, i.e. functions created from \sollya
   procedures using the \textbf{function} construct. If \emph{expr} contains such a
   procedure-based function, \textbf{implementconstant} prints a warning and fails
   silently. The reason for this lack of functionality is that the
   produced C source code, which is supposed to be compiled, would have
   to call back to the \sollya interpreter in order to evaluate the
   procedure-based function.

\item Similarily, \textbf{implementconstant} is currently unable to implement constant
   expressions \emph{expr} that contain library-based functions, i.e.
   functions dynamically bound to \sollya using the \textbf{library} construct.
   If \emph{expr} contains such a library-based function, \textbf{implementconstant} prints
   a warning and fails silently. Support for this feature is in principle
   feasible from a technical standpoint and might be added in a future
   release of \sollya.

\item Currently, non-differentiable functions such as \textbf{double}, \textbf{doubledouble},
   \textbf{tripledouble}, \textbf{single}, \textbf{halfprecision}, \textbf{quad}, \textbf{doubleextended}, 
   \textbf{floor}, \textbf{ceil}, \textbf{nearestint} are not supported by \textbf{implementconstant}. 
   If \textbf{implementconstant} encounters one of them, a warning message is displayed 
   and no code is produced. However, if \textbf{autosimplify} equals on, it is 
   possible that \sollya silently simplifies subexpressions of \emph{expr} 
   containing such functions and that \textbf{implementconstant} successfully produces 
   code for evaluating \emph{expr}.

\item While it produces an MPFR-based C source code for \emph{expr}, \textbf{implementconstant}
   takes architectural and system-dependent parameters into account.  For
   example, it checks whether literal constants figuring in \emph{expr} can be
   represented on a C \texttt{long int} type or if they must
   be stored in a different manner not to affect their accuracy. These
   tests, performed by \sollya during execution of \textbf{implementconstant}, depend
   themselves on the architecture \sollya is running on. Users should
   keep this matter in mind, especially when trying to compile source
   code on one machine whilst it has been produced on another.
\end{itemize}
\noindent Example 1: 
\begin{center}\begin{minipage}{15cm}\begin{Verbatim}[frame=single]
> implementconstant(exp(1)+log(2)/sqrt(1/10));
#include <mpfr.h>

void
const_something (mpfr_ptr y, mp_prec_t prec)
{
  /* Declarations */
  mpfr_t tmp1;
  mpfr_t tmp2;
  mpfr_t tmp3;
  mpfr_t tmp4;
  mpfr_t tmp5;
  mpfr_t tmp6;
  mpfr_t tmp7;

  /* Initializations */
  mpfr_init2 (tmp2, prec+5);
  mpfr_init2 (tmp1, prec+3);
  mpfr_init2 (tmp4, prec+8);
  mpfr_init2 (tmp3, prec+7);
  mpfr_init2 (tmp6, prec+11);
  mpfr_init2 (tmp7, prec+11);
  mpfr_init2 (tmp5, prec+11);

  /* Core */
  mpfr_set_prec (tmp2, prec+4);
  mpfr_set_ui (tmp2, 1, MPFR_RNDN);
  mpfr_set_prec (tmp1, prec+3);
  mpfr_exp (tmp1, tmp2, MPFR_RNDN);
  mpfr_set_prec (tmp4, prec+8);
  mpfr_set_ui (tmp4, 2, MPFR_RNDN);
  mpfr_set_prec (tmp3, prec+7);
  mpfr_log (tmp3, tmp4, MPFR_RNDN);
  mpfr_set_prec (tmp6, prec+11);
  mpfr_set_ui (tmp6, 1, MPFR_RNDN);
  mpfr_set_prec (tmp7, prec+11);
  mpfr_set_ui (tmp7, 10, MPFR_RNDN);
  mpfr_set_prec (tmp5, prec+11);
  mpfr_div (tmp5, tmp6, tmp7, MPFR_RNDN);
  mpfr_set_prec (tmp4, prec+7);
  mpfr_sqrt (tmp4, tmp5, MPFR_RNDN);
  mpfr_set_prec (tmp2, prec+5);
  mpfr_div (tmp2, tmp3, tmp4, MPFR_RNDN);
  mpfr_set_prec (y, prec+3);
  mpfr_add (y, tmp1, tmp2, MPFR_RNDN);

  /* Cleaning stuff */
  mpfr_clear(tmp1);
  mpfr_clear(tmp2);
  mpfr_clear(tmp3);
  mpfr_clear(tmp4);
  mpfr_clear(tmp5);
  mpfr_clear(tmp6);
  mpfr_clear(tmp7);
}
\end{Verbatim}
\end{minipage}\end{center}
\noindent Example 2: 
\begin{center}\begin{minipage}{15cm}\begin{Verbatim}[frame=single]
> implementconstant(sin(13/17),"sine_of_thirteen_seventeenth.c");
> readfile("sine_of_thirteen_seventeenth.c");
#include <mpfr.h>

void
const_something (mpfr_ptr y, mp_prec_t prec)
{
  /* Declarations */
  mpfr_t tmp1;
  mpfr_t tmp2;
  mpfr_t tmp3;

  /* Initializations */
  mpfr_init2 (tmp2, prec+6);
  mpfr_init2 (tmp3, prec+6);
  mpfr_init2 (tmp1, prec+6);

  /* Core */
  mpfr_set_prec (tmp2, prec+6);
  mpfr_set_ui (tmp2, 13, MPFR_RNDN);
  mpfr_set_prec (tmp3, prec+6);
  mpfr_set_ui (tmp3, 17, MPFR_RNDN);
  mpfr_set_prec (tmp1, prec+6);
  mpfr_div (tmp1, tmp2, tmp3, MPFR_RNDN);
  mpfr_set_prec (y, prec+2);
  mpfr_sin (y, tmp1, MPFR_RNDN);

  /* Cleaning stuff */
  mpfr_clear(tmp1);
  mpfr_clear(tmp2);
  mpfr_clear(tmp3);
}

\end{Verbatim}
\end{minipage}\end{center}
\noindent Example 3: 
\begin{center}\begin{minipage}{15cm}\begin{Verbatim}[frame=single]
> implementconstant(asin(1/3 * pi),default,"arcsin_of_one_third_pi");
#include <mpfr.h>

void
arcsin_of_one_third_pi (mpfr_ptr y, mp_prec_t prec)
{
  /* Declarations */
  mpfr_t tmp1;
  mpfr_t tmp2;
  mpfr_t tmp3;

  /* Initializations */
  mpfr_init2 (tmp2, prec+8);
  mpfr_init2 (tmp3, prec+8);
  mpfr_init2 (tmp1, prec+8);

  /* Core */
  mpfr_set_prec (tmp2, prec+8);
  mpfr_const_pi (tmp2, MPFR_RNDN);
  mpfr_set_prec (tmp3, prec+8);
  mpfr_set_ui (tmp3, 3, MPFR_RNDN);
  mpfr_set_prec (tmp1, prec+8);
  mpfr_div (tmp1, tmp2, tmp3, MPFR_RNDN);
  mpfr_set_prec (y, prec+2);
  mpfr_asin (y, tmp1, MPFR_RNDN);

  /* Cleaning stuff */
  mpfr_clear(tmp1);
  mpfr_clear(tmp2);
  mpfr_clear(tmp3);
}
\end{Verbatim}
\end{minipage}\end{center}
\noindent Example 4: 
\begin{center}\begin{minipage}{15cm}\begin{Verbatim}[frame=single]
> implementconstant(ceil(log(19 + 1/3)),"constant_code.c","magic_constant");
> readfile("constant_code.c");
#include <mpfr.h>

void
magic_constant (mpfr_ptr y, mp_prec_t prec)
{
  /* Initializations */

  /* Core */
  mpfr_set_prec (y, prec);
  mpfr_set_ui (y, 3, MPFR_RNDN);
}

\end{Verbatim}
\end{minipage}\end{center}
\noindent Example 5: 
\begin{center}\begin{minipage}{15cm}\begin{Verbatim}[frame=single]
> bashexecute("gcc -fPIC -Wall -c libraryconstantexample.c -I$HOME/.local/includ
e");
> bashexecute("gcc -shared -o libraryconstantexample libraryconstantexample.o -l
gmp -lmpfr");
> euler_gamma = libraryconstant("./libraryconstantexample");
> implementconstant(euler_gamma^(1/3));
#include <mpfr.h>

void
const_something (mpfr_ptr y, mp_prec_t prec)
{
  /* Declarations */
  mpfr_t tmp1;

  /* Initializations */
  mpfr_init2 (tmp1, prec+1);

  /* Core */
  euler_gamma (tmp1, prec+1);
  mpfr_set_prec (y, prec+2);
  mpfr_root (y, tmp1, 3, MPFR_RNDN);

  /* Cleaning stuff */
  mpfr_clear(tmp1);
}
\end{Verbatim}
\end{minipage}\end{center}
See also: \textbf{implementpoly} (\ref{labimplementpoly}), \textbf{libraryconstant} (\ref{lablibraryconstant}), \textbf{library} (\ref{lablibrary}), \textbf{function} (\ref{labfunction})

\subsection{implementpoly}
\label{labimplementpoly}
\noindent Name: \textbf{implementpoly}\\
implements a polynomial using double, double-double and triple-double arithmetic and generates a Gappa proof\\

\noindent Usage: 
\begin{center}
\textbf{implementpoly}(\emph{polynomial}, \emph{range}, \emph{error bound}, \emph{format}, \emph{functionname}, \emph{filename}) : (\textsf{function}, \textsf{range}, \textsf{constant}, \textsf{D$|$double$|$DD$|$doubledouble$|$TD$|$tripledouble}, \textsf{string}, \textsf{string}) $\rightarrow$ \textsf{function}\\
\textbf{implementpoly}(\emph{polynomial}, \emph{range}, \emph{error bound}, \emph{format}, \emph{functionname}, \emph{filename}, \emph{honor coefficient precisions}) : (\textsf{function}, \textsf{range}, \textsf{constant}, \textsf{D$|$double$|$DD$|$doubledouble$|$TD$|$tripledouble}, \textsf{string}, \textsf{string}, \textsf{honorcoeffprec}) $\rightarrow$ \textsf{function}\\
\textbf{implementpoly}(\emph{polynomial}, \emph{range}, \emph{error bound}, \emph{format}, \emph{functionname}, \emph{filename}, \emph{proof filename}) : (\textsf{function}, \textsf{range}, \textsf{constant}, \textsf{D$|$double$|$DD$|$doubledouble$|$TD$|$tripledouble}, \textsf{string}, \textsf{string}, \textsf{string}) $\rightarrow$ \textsf{function}\\
\textbf{implementpoly}(\emph{polynomial}, \emph{range}, \emph{error bound}, \emph{format}, \emph{functionname}, \emph{filename}, \emph{honor coefficient precisions}, \emph{proof filename}) : (\textsf{function}, \textsf{range}, \textsf{constant}, \textsf{D$|$double$|$DD$|$doubledouble$|$TD$|$tripledouble}, \textsf{string}, \textsf{string}, \textsf{honorcoeffprec}, \textsf{string}) $\rightarrow$ \textsf{function}\\
\end{center}
\noindent Description: \begin{itemize}

\item The command \textbf{implementpoly} implements the polynomial \emph{polynomial} in range
   \emph{range} as a function called \emph{functionname} in \texttt{C} code
   using double, double-double and triple-double arithmetic in a way that
   the rounding error (estimated at its first order) is bounded by \emph{error bound}. 
   The produced code is output in a file named \emph{filename}. The
   argument \emph{format} indicates the double, double-double or triple-double
   format of the variable in which the polynomial varies, influencing
   also in the signature of the \texttt{C} function.
    
   If a seventh or eightth argument \emph{proof filename} is given and if this
   argument evaluates to a variable of type \textsf{string}, the command
   \textbf{implementpoly} will produce a \texttt{Gappa} proof that the
   rounding error is less than the given bound. This proof will be output
   in \texttt{Gappa} syntax in a file name \emph{proof filename}.
    
   The command \textbf{implementpoly} returns the polynomial that has been
   implemented. As the command \textbf{implementpoly} tries to adapt the precision
   needed in each evaluation step to its strict minimum and as it
   renormalizes double-double and triple-double precision coefficients to
   a round-to-nearest expansion, the polynomial return may differ from
   the polynomial \emph{polynomial}. Nevertheless the difference will be small
   enough that the rounding error bound with regard to the polynomial
   \emph{polynomial} (estimated at its first order) will be less than the
   given error bound.
    
   If a seventh argument \emph{honor coefficient precisions} is given and
   evaluates to a variable \textbf{honorcoeffprec} of type \textsf{honorcoeffprec},
   \textbf{implementpoly} will honor the precision of the given polynomial
   \emph{polynomials}. This means if a coefficient needs a double-double or a
   triple-double to be exactly stored, \textbf{implementpoly} will allocate appropriate
   space and use a double-double or triple-double operation even if the
   automatic (heuristical) determination implemented in command \textbf{implementpoly}
   indicates that the coefficient could be stored on less precision or,
   respectively, the operation could be performed with less
   precision. The use of \textbf{honorcoeffprec} has advantages and
   disadvantages. If the polynomial \emph{polynomial} given has not been
   determined by a process considering directly polynomials with
   floating-point coefficients, \textbf{honorcoeffprec} should not be
   indicated. The \textbf{implementpoly} command can then determine the needed
   precision using the same error estimation as used for the
   determination of the precisions of the operations. Generally, the
   coefficients will get rounded to double, double-double and
   triple-double precision in a way that minimizes their number and
   respects the rounding error bound \emph{error bound}.  Indicating
   \textbf{honorcoeffprec} may in this case short-circuit most precision
   estimations leading to sub-optimal code. On the other hand, if the
   polynomial \emph{polynomial} has been determined with floating-point
   precisions in mind, \textbf{honorcoeffprec} should be indicated because such
   polynomials often are very sensitive in terms of error propgation with
   regard to their coefficients' values. Indicating \textbf{honorcoeffprec}
   prevents the \textbf{implementpoly} command from rounding the coefficients and
   altering by many orders of magnitude approximation error of the
   polynomial with regard to the function it approximates.
    
   The implementer behind the \textbf{implementpoly} command makes some assumptions on
   its input and verifies them. If some assumption cannot be verified,
   the implementation will not succeed and \textbf{implementpoly} will evaluate to a
   variable \textbf{error} of type \textsf{error}. The same behaviour is observed if
   some file is not writeable or some other side-effect fails, e.g. if
   the implementer runs out of memory.
    
   As error estimation is performed only on the first order, the code
   produced by the \textbf{implementpoly} command should be considered valid iff a
   \texttt{Gappa} proof has been produced and successfully run
   in \texttt{Gappa}.
\end{itemize}
\noindent Example 1: 
\begin{center}\begin{minipage}{15cm}\begin{Verbatim}[frame=single]
> implementpoly(1 - 1/6 * x^2 + 1/120 * x^4, [-1b-10;1b-10], 1b-30, D, "p","impl
ementation.c");
1 + x^2 * ((-0.166666666666666657414808128123695496469736099243164) + x^2 * 8.33
33333333333332176851016015461937058717012405395e-3)
> readfile("implementation.c");
#define p_coeff_0h 1.00000000000000000000000000000000000000000000000000000000000
000000000000000000000e+00
#define p_coeff_2h -1.6666666666666665741480812812369549646973609924316406250000
0000000000000000000000e-01
#define p_coeff_4h 8.33333333333333321768510160154619370587170124053955078125000
000000000000000000000e-03


void p(double *p_resh, double x) {
double p_x_0_pow2h;


p_x_0_pow2h = x * x;


double p_t_1_0h;
double p_t_2_0h;
double p_t_3_0h;
double p_t_4_0h;
double p_t_5_0h;
 


p_t_1_0h = p_coeff_4h;
p_t_2_0h = p_t_1_0h * p_x_0_pow2h;
p_t_3_0h = p_coeff_2h + p_t_2_0h;
p_t_4_0h = p_t_3_0h * p_x_0_pow2h;
p_t_5_0h = p_coeff_0h + p_t_4_0h;
*p_resh = p_t_5_0h;


}

\end{Verbatim}
\end{minipage}\end{center}
\noindent Example 2: 
\begin{center}\begin{minipage}{15cm}\begin{Verbatim}[frame=single]
> implementpoly(1 - 1/6 * x^2 + 1/120 * x^4, [-1b-10;1b-10], 1b-30, D, "p","impl
ementation.c","implementation.gappa");
1 + x^2 * ((-0.166666666666666657414808128123695496469736099243164) + x^2 * 8.33
33333333333332176851016015461937058717012405395e-3)
\end{Verbatim}
\end{minipage}\end{center}
\noindent Example 3: 
\begin{center}\begin{minipage}{15cm}\begin{Verbatim}[frame=single]
> verbosity = 1!;
> q = implementpoly(1 - simplify(TD(1/6)) * x^2,[-1b-10;1b-10],1b-60,DD,"p","imp
lementation.c");
Warning: at least one of the coefficients of the given polynomial has been round
ed in a way
that the target precision can be achieved at lower cost. Nevertheless, the imple
mented polynomial
is different from the given one.
> printexpansion(q);
0x3ff0000000000000 + x^2 * 0xbfc5555555555555
> r = implementpoly(1 - simplify(TD(1/6)) * x^2,[-1b-10;1b-10],1b-60,DD,"p","imp
lementation.c",honorcoeffprec);
Warning: the infered precision of the 2th coefficient of the polynomial is great
er than
the necessary precision computed for this step. This may make the automatic dete
rmination
of precisions useless.
> printexpansion(r);
0x3ff0000000000000 + x^2 * (0xbfc5555555555555 + 0xbc65555555555555 + 0xb9055555
55555555)
\end{Verbatim}
\end{minipage}\end{center}
\noindent Example 4: 
\begin{center}\begin{minipage}{15cm}\begin{Verbatim}[frame=single]
> p = 0x3ff0000000000000 + x * (0x3ff0000000000000 + x * (0x3fe0000000000000 + x
 * (0x3fc5555555555559 + x * (0x3fa55555555555bd + x * (0x3f811111111106e2 + x
 * (0x3f56c16c16bf5eb7 + x * (0x3f2a01a01a292dcd + x * (0x3efa01a0218a016a + x
 * (0x3ec71de360331aad + x * (0x3e927e42e3823bf3 + x * (0x3e5ae6b2710c2c9a + x
 * (0x3e2203730c0a7c1d + x * 0x3de5da557e0781df))))))))))));
> q = implementpoly(p,[-1/2;1/2],1b-60,D,"p","implementation.c",honorcoeffprec,"
implementation.gappa");
> if (q != p) then print("During implementation, rounding has happened.") else p
rint("Polynomial implemented as given.");    
Polynomial implemented as given.
\end{Verbatim}
\end{minipage}\end{center}
See also: \textbf{honorcoeffprec} (\ref{labhonorcoeffprec}), \textbf{roundcoefficients} (\ref{labroundcoefficients}), \textbf{double} (\ref{labdouble}), \textbf{doubledouble} (\ref{labdoubledouble}), \textbf{tripledouble} (\ref{labtripledouble}), \textbf{readfile} (\ref{labreadfile}), \textbf{printexpansion} (\ref{labprintexpansion}), \textbf{error} (\ref{laberror})

\subsection{in}
\label{labin}
\noindent Name: \textbf{in}\\
\phantom{aaa}containment test operator\\[0.2cm]
\noindent Library name:\\
\verb|   sollya_obj_t sollya_lib_cmp_in(sollya_obj_t, sollya_obj_t)|\\[0.2cm]
\noindent Usage: 
\begin{center}
\emph{expr} \textbf{in} \emph{range1} : (\textsf{constant}, \textsf{range}) $\rightarrow$ \textsf{boolean}\\
\emph{range1} \textbf{in} \emph{range2} : (\textsf{range}, \textsf{range}) $\rightarrow$ \textsf{boolean}\\
\end{center}
Parameters: 
\begin{itemize}
\item \emph{expr} represents a constant expression
\item \emph{range1} and \emph{range2} represent ranges (intervals)
\end{itemize}
\noindent Description: \begin{itemize}

\item When its first operand is a constant expression \emph{expr},
   the operator \textbf{in} evaluates to true iff the constant value
   of the expression \emph{expr} is contained in the interval \emph{range1}.

\item When both its operands are ranges (intervals), 
   the operator \textbf{in} evaluates to true iff all values
   in \emph{range1} are contained in the interval \emph{range2}.

\item \textbf{in} is also used as a keyword for loops over the different
   elements of a list.
\end{itemize}
\noindent Example 1: 
\begin{center}\begin{minipage}{15cm}\begin{Verbatim}[frame=single]
> 5 in [-4;7];
true
> 4 in [-1;1];
false
> 0 in sin([-17;17]);
true
\end{Verbatim}
\end{minipage}\end{center}
\noindent Example 2: 
\begin{center}\begin{minipage}{15cm}\begin{Verbatim}[frame=single]
> [5;7] in [2;8];
true
> [2;3] in [4;5];
false
> [2;3] in [2.5;5];
false
\end{Verbatim}
\end{minipage}\end{center}
\noindent Example 3: 
\begin{center}\begin{minipage}{15cm}\begin{Verbatim}[frame=single]
> for i in [|1,...,5|] do print(i);
1
2
3
4
5
\end{Verbatim}
\end{minipage}\end{center}
See also: \textbf{$==$} (\ref{labequal}), \textbf{!$=$} (\ref{labneq}), \textbf{$>=$} (\ref{labge}), \textbf{$>$} (\ref{labgt}), \textbf{$<=$} (\ref{lable}), \textbf{$<$} (\ref{lablt}), \textbf{!} (\ref{labnot}), \textbf{$\&\&$} (\ref{laband}), \textbf{$||$} (\ref{labor}), \textbf{prec} (\ref{labprec}), \textbf{print} (\ref{labprint})

\subsection{ inf }
\noindent Name: \textbf{inf}\\
gives the lower bound of an interval.\\

\noindent Usage: 
\begin{center}
\textbf{inf}(\emph{I}) : \textsf{range} $\rightarrow$ \textsf{constant}\\
\textbf{inf}(\emph{x}) : \textsf{constant} $\rightarrow$ \textsf{constant}\\
\end{center}
Parameters: 
\emph{I} is an interval.\\
\emph{x} is a real number.\\

\noindent Description: \begin{itemize}

\item Returns the lower bound of the interval \emph{I}. Each bound of an interval has its 
   own precision, so this command is exact, even if the current precision is too 
   small to represent the bound.

\item When called on a real number \emph{x}, \textbf{inf} considers it as an interval formed
   of a single point: [x, x]. In other words, \textbf{inf} behaves like the identity.
\end{itemize}
\noindent Example 1: 
\begin{center}\begin{minipage}{14.8cm}\begin{Verbatim}[frame=single]
   > inf([1;3]);
   1
   > inf(0);
   0
\end{Verbatim}
\end{minipage}\end{center}
\noindent Example 2: 
\begin{center}\begin{minipage}{14.8cm}\begin{Verbatim}[frame=single]
   > display=binary!;
   > I=[0.111110000011111_2; 1];
   > inf(I);
   1.11110000011111_2 * 2^(-1)
   > prec=12!;
   > inf(I);
   1.11110000011111_2 * 2^(-1)
\end{Verbatim}
\end{minipage}\end{center}
See also: \textbf{mid}, \textbf{sup}

\subsection{infnorm}
\label{labinfnorm}
\noindent Name: \textbf{infnorm}\\
computes an interval bounding the infinity norm of a function on an interval.\\
\noindent Usage: 
\begin{center}
\textbf{infnorm}(\emph{f},\emph{I},\emph{filename},\emph{Ilist}) : (\textsf{function}, \textsf{range}, \textsf{string}, \textsf{list}) $\rightarrow$ \textsf{range}
\end{center}
Parameters: 
\begin{itemize}
\item \emph{f} is a function.
\item \emph{I} is an interval.
\item \emph{filename} (optional) is the name of the file into a proof will be saved.
\item \emph{IList} (optional) is a list of intervals to be excluded.
\end{itemize}
\noindent Description: \begin{itemize}

\item \textbf{infnorm}(\emph{f},\emph{range}) computes an interval bounding the infinity norm of the 
   given function $f$ on the interval $I$, e.g. computes an interval $J$
   such that $\max_{x \in I} \{|f(x)|\} \subseteq J$.

\item If \emph{filename} is given, a proof in English will be produced (and stored in file
   called \emph{filename}) proving that  $\max_{x \in I} \{|f(x)|\} \subseteq J$.

\item If a list \emph{IList} of intervals \emph{I1}, ... ,\emph{In} is given, the infinity norm will
   be computed on $I \ (I_1 \cup \dots \cup I_n)$.

\item The function \emph{f} is assumed to be at least twice continuous on \emph{I}. More 
   generally, if \emph{f} is $\mathcal{C}^k$, global variables \textbf{hopitalrecursions} and
   \textbf{taylorrecursions} must have values not greater than $k$.  

\item If the interval is reduced to a single point, the result of \textbf{infnorm} is an 
   interval containing the exact absolute value of \emph{f} at this point.

\item If the interval is not bound, the result will be $[0,\,+\infty]$ 
   which is true but perfectly useless. \textbf{infnorm} is not meant to be used with 
   infinite intervals.

\item The result of this command depends on the global variables \textbf{prec}, \textbf{diam},
   \textbf{taylorrecursions} and \textbf{hopitalrecursions}. The contribution of each variable is 
   not easy even to analyse.
   \begin{itemize}
   \item  The algorithm uses interval arithmetic with precision \textbf{prec}. The
     precision should thus be set big enough to ensure that no critical
     cancellation will occur.
   \item  When an evaluation is performed on an interval $[a,\,b]$, if the result
     is considered being too large, the interval is split into $[a,\,\frac{a+b}{2}]$
     and $[\frac{a+b}{2},\,b]$ and so on recursively. This recursion step
     is  not performed if the $(b-a) < \delta \cdot |I|$ where $\delta$ is the value
     of variable \textbf{diam}. In other words, \textbf{diam} controls the minimum length of an
     interval during the algorithm.
   \item  To perform the evaluation of a function on an interval, Taylor's rule is
     applied, e.g. $f([a,b]) \subseteq f(m) + [a-m,\,b-m] \cdot f'([a,\,b])$
     where $m=\frac{a+b}{2}$. This rule is applied recursively $n$ times
     where $n$ is the value of variable \textbf{taylorrecursions}. Roughly speaking,
     the evaluations will avoid decorrelation up to order $n$.
   \item  When a function of the form $\frac{g}{h}$ has to be evaluated on an
     interval $[a,\,b]$ and when $g$ and $h$ vanish at a same point
     $z$ of the interval, the ratio may be defined even if the expression
     $\frac{g(z)}{h(z)}=\frac{0}{0}$ does not make any sense. In this case, L'Hopital's rule
     may be used and $\left(\frac{g}{h}\right)([a,\,b]) \subseteq \left(\frac{g'}{h'}\right)([a,\,b])$.
     Since the same can occur with the ratio $\frac{g'}{h'}$, the rule is applied
     recursively. Variable \textbf{hopitalrecursions} controls the number of 
     recursion steps.
   \end{itemize}

\item The algorithm used for this command is quite complex to be explained here. 
   Please find a complete description in the following article:\\
        S. Chevillard and C. Lauter\\
        A certified infinity norm for the implementation of elementary functions\\
        LIP Research Report number RR2007-26\\
        http://prunel.ccsd.cnrs.fr/ensl-00119810\\
\end{itemize}
\noindent Example 1: 
\begin{center}\begin{minipage}{15cm}\begin{Verbatim}[frame=single]
> infnorm(exp(x),[-2;3]);
[2.00855369231876677409285296545817178969879078385537e1;2.0085536923187667740928
5296545817178969879078385544e1]
\end{Verbatim}
\end{minipage}\end{center}
\noindent Example 2: 
\begin{center}\begin{minipage}{15cm}\begin{Verbatim}[frame=single]
> infnorm(exp(x),[-2;3],"proof.txt");
[2.00855369231876677409285296545817178969879078385537e1;2.0085536923187667740928
5296545817178969879078385544e1]
\end{Verbatim}
\end{minipage}\end{center}
\noindent Example 3: 
\begin{center}\begin{minipage}{15cm}\begin{Verbatim}[frame=single]
> infnorm(exp(x),[-2;3],[| [0;1], [2;2.5] |]);
[2.00855369231876677409285296545817178969879078385537e1;2.0085536923187667740928
5296545817178969879078385544e1]
\end{Verbatim}
\end{minipage}\end{center}
\noindent Example 4: 
\begin{center}\begin{minipage}{15cm}\begin{Verbatim}[frame=single]
> infnorm(exp(x),[-2;3],"proof.txt", [| [0;1], [2;2.5] |]);
[2.00855369231876677409285296545817178969879078385537e1;2.0085536923187667740928
5296545817178969879078385544e1]
\end{Verbatim}
\end{minipage}\end{center}
\noindent Example 5: 
\begin{center}\begin{minipage}{15cm}\begin{Verbatim}[frame=single]
> infnorm(exp(x),[1;1]);
[2.71828182845904523536028747135266249775724709369989;2.718281828459045235360287
47135266249775724709369998]
\end{Verbatim}
\end{minipage}\end{center}
\noindent Example 6: 
\begin{center}\begin{minipage}{15cm}\begin{Verbatim}[frame=single]
> infnorm(exp(x), [log(0);log(1)]);
[0;@Inf@]
\end{Verbatim}
\end{minipage}\end{center}
See also: \textbf{prec} (\ref{labprec}), \textbf{diam} (\ref{labdiam}), \textbf{hopitalrecursions} (\ref{labhopitalrecursions}), \textbf{dirtyinfnorm} (\ref{labdirtyinfnorm}), \textbf{checkinfnorm} (\ref{labcheckinfnorm})

\subsection{integer}
\label{labinteger}
\noindent Name: \textbf{integer}\\
keyword representing a machine integer type \\
\noindent Usage: 
\begin{center}
\textbf{integer} : \textsf{type type}\\
\end{center}
\noindent Description: \begin{itemize}

\item \textbf{integer} represents the machine integer type for declarations
   of external procedures \textbf{externalproc}.
    
   Remark that in contrast to other indicators, type indicators like
   \textbf{integer} cannot be handled outside the \textbf{externalproc} context.  In
   particular, they cannot be assigned to variables.
\end{itemize}
See also: \textbf{externalproc} (\ref{labexternalproc}), \textbf{boolean} (\ref{labboolean}), \textbf{constant} (\ref{labconstant}), \textbf{function} (\ref{labfunction}), \textbf{list of} (\ref{lablistof}), \textbf{range} (\ref{labrange}), \textbf{string} (\ref{labstring})

\subsection{ integral }
\noindent Name: \textbf{integral}\\
computes an interval bounding the integral of a function on an interval.\\

\noindent Usage: 
\begin{center}
\textbf{integral}(\emph{f},\emph{I}) : (\textsf{function}, \textsf{range}) $\rightarrow$ \textsf{range}\\
\end{center}
Parameters: 
\emph{f} is a function.\\
\emph{I} is an interval.\\

\noindent Description: \begin{itemize}

\item \textbf{integral}(\emph{f},\emph{I}) returns an interval $J$ such that the exact value of 
   the integral of \emph{f} on \emph{I} lies in $J$.

\item This command is safe but very unefficient. Use \textbf{dirtyintegral} if you just want
   an approximate value.

\item The result of this command depends on the global variable \textbf{diam}.
   The method used is the following: \emph{I} is cut into intervals of length not 
   greater then $\delta \cdot |I|$ where $\delta$ is the value
   of global variable \textbf{diam}.
   On each small interval \emph{J}, an evaluation of \emph{f} by interval is
   performed. The result is multiplied by the length of \emph{J}. Finally all values 
   are summed.
\end{itemize}
\noindent Example 1: 
\begin{center}\begin{minipage}{14.8cm}\begin{Verbatim}[frame=single]
   > sin(10);
   -0.544021110889369813404747661851377281683643012916219
   > integral(cos(x),[0;10]);
   [-0.547101979835796902240976371635259430756985992573329;-0.540940151300131838481505408813733707440537411917285]
   > diam=1e-5!;
   > integral(cos(x),[0;10]);
   [-0.544329156859554271018577802959369567752938763827772;-0.543713064012499695080396442219274890104258031735553]
\end{Verbatim}
\end{minipage}\end{center}
See also: \textbf{points}, \textbf{dirtyintegral}

\subsection{isbound}
\label{labisbound}
\noindent Name: \textbf{isbound}\\
indicates whether a variable is bound or not.\\
\noindent Usage: 
\begin{center}
\textbf{isbound}(\emph{ident}) : \textsf{boolean}
\end{center}
Parameters: 
\begin{itemize}
\item \emph{ident} is a name.
\end{itemize}
\noindent Description: \begin{itemize}

\item \textbf{isbound}(\emph{ident}) returns a boolean value indicating whether the name \emph{ident}
   is used or not to represent a variable. It returns true when \emph{ident} is the 
   name used to represent the global variable or if the name is currently used
   to refer to a (possibly local) variable.

\item When a variable is defined in a block and has not been defined outside, 
   \textbf{isbound} returns true when called inside the block, and false outside.
   Note that \textbf{isbound} returns true as soon as a variable has been declared with 
   \textbf{var}, even if no value is actually stored in it.

\item If \emph{ident1} is bound to a variable and if \emph{ident2} refers to the global 
   variable, the command \textbf{rename}(\emph{ident2}, \emph{ident1}) hides the value of \emph{ident1}
   which becomes the global variable. However, if the global variable is again
   renamed, \emph{ident1} gets its value back. In this case, \textbf{isbound}(\emph{ident1}) returns
   true. If \emph{ident1} was not bound before, \textbf{isbound}(\emph{ident1}) returns false after
   that \emph{ident1} has been renamed.
\end{itemize}
\noindent Example 1: 
\begin{center}\begin{minipage}{15cm}\begin{Verbatim}[frame=single]
> isbound(x);
false
> isbound(f);
false
> isbound(g);
false
> f=sin(x);
> isbound(x);
true
> isbound(f);
true
> isbound(g);
false
\end{Verbatim}
\end{minipage}\end{center}
\noindent Example 2: 
\begin{center}\begin{minipage}{15cm}\begin{Verbatim}[frame=single]
> isbound(a);
false
> { var a; isbound(a); };
true
> isbound(a);
false
\end{Verbatim}
\end{minipage}\end{center}
\noindent Example 3: 
\begin{center}\begin{minipage}{15cm}\begin{Verbatim}[frame=single]
> f=sin(x);
> isbound(x);
true
> rename(x,y);
> isbound(x);
false
\end{Verbatim}
\end{minipage}\end{center}
\noindent Example 4: 
\begin{center}\begin{minipage}{15cm}\begin{Verbatim}[frame=single]
> x=1;
> f=sin(y);
> rename(y,x);
> f;
sin(x)
> x;
x
> isbound(x);
true
> rename(x,y);
> isbound(x);
true
> x;
1
\end{Verbatim}
\end{minipage}\end{center}
See also: \textbf{rename} (\ref{labrename})

\subsection{ isevaluable }
\noindent Name: \textbf{isevaluable}\\
tests whether a function can be evaluated at a point \\

\noindent Usage: 
\begin{center}
\textbf{isevaluable}(\emph{function}, \emph{constant}) : (\textsf{function}, \textsf{constant}) $\rightarrow$ \textsf{boolean}\\
\end{center}
Parameters: 
\begin{itemize}
\item \emph{function} represents a function
\item \emph{constant} represents a constant point
\end{itemize}
\noindent Description: \begin{itemize}

\item \textbf{isevaluable} applied to function \emph{function} and a constant \emph{constant} returns
   a boolean indicating whether or not a subsequent call to \textbf{evaluate} on the
   same function \emph{function} and constant \emph{constant} will produce a numerical
   result or NaN. I.e. \textbf{isevaluable} returns false iff \textbf{evaluate} will return NaN.
\end{itemize}
\noindent Example 1: 
\begin{center}\begin{minipage}{15cm}\begin{Verbatim}[frame=single]
> isevaluable(sin(pi * 1/x), 0.75);
true
> print(evaluate(sin(pi * 1/x), 0.75));
-0.866025403784438646763723170752936183471402626905185165
\end{Verbatim}
\end{minipage}\end{center}
\noindent Example 2: 
\begin{center}\begin{minipage}{15cm}\begin{Verbatim}[frame=single]
> isevaluable(sin(pi * 1/x), 0.5);
true
> print(evaluate(sin(pi * 1/x), 0.5));
[-0.172986452514381269516508615031098129542836767991679e-12714;0.759411982011879
631450695643145256617060390843900679e-12715]
\end{Verbatim}
\end{minipage}\end{center}
\noindent Example 3: 
\begin{center}\begin{minipage}{15cm}\begin{Verbatim}[frame=single]
> isevaluable(sin(pi * 1/x), 0);
false
> print(evaluate(sin(pi * 1/x), 0));
@NaN@
\end{Verbatim}
\end{minipage}\end{center}
See also: \textbf{evaluate}

\subsection{le}
\label{lable}
\noindent Name: \textbf{$<=$}\\
less-than-or-equal-to operator\\

\noindent Usage: 
\begin{center}
\emph{expr1} \textbf{$<=$} \emph{expr2} : (\textsf{constant}, \textsf{constant}) $\rightarrow$ \textsf{boolean}\\
\end{center}
Parameters: 
\begin{itemize}
\item \emph{expr1} and \emph{expr2} represent constant expressions
\end{itemize}
\noindent Description: \begin{itemize}

\item The operator \textbf{$<=$} evaluates to true iff its operands \emph{expr1} and
   \emph{expr2} evaluate to two floating-point numbers $a_1$
   respectively $a_2$ with the global precision \textbf{prec} and
   $a_1$ is less than or equal to $a_2$. The user should
   be aware of the fact that because of floating-point evaluation, the
   operator \textbf{$<=$} is not exactly the same as the mathematical
   operation \emph{less-than-or-equal-to}.
\end{itemize}
\noindent Example 1: 
\begin{center}\begin{minipage}{15cm}\begin{Verbatim}[frame=single]
> 5 <= 4;
false
> 5 <= 5;
true
> 5 <= 6;
true
> exp(2) <= exp(1);
false
> log(1) <= exp(2);
true
\end{Verbatim}
\end{minipage}\end{center}
\noindent Example 2: 
\begin{center}\begin{minipage}{15cm}\begin{Verbatim}[frame=single]
> prec = 12;
The precision has been set to 12 bits.
> 16385 <= 16384;
true
\end{Verbatim}
\end{minipage}\end{center}
See also: \textbf{$==$} (\ref{labequal}), \textbf{!$=$} (\ref{labneq}), \textbf{$>=$} (\ref{labge}), \textbf{$>$} (\ref{labgt}), \textbf{$<$} (\ref{lablt}), \textbf{!} (\ref{labnot}), \textbf{$\&\&$} (\ref{laband}), \textbf{$||$} (\ref{labor}), \textbf{prec} (\ref{labprec})

\subsection{length}
\label{lablength}
\noindent Name: \textbf{length}\\
computes the length of a list or string.\\

\noindent Usage: 
\begin{center}
\textbf{length}(\emph{L}) : \textsf{list} $\rightarrow$ \textsf{integer}\\
\textbf{length}(\emph{s}) : \textsf{string} $\rightarrow$ \textsf{integer}\\
\end{center}
Parameters: 
\begin{itemize}
\item \emph{L} is a list.
\item \emph{s} is a string.
\end{itemize}
\noindent Description: \begin{itemize}

\item \textbf{length} returns the length of a list or a string, e.g. the number of elements
   or letters.

\item The empty list or string have length 0.
   If \emph{L} is an end-elliptic list, \textbf{length} returns +Inf.
\end{itemize}
\noindent Example 1: 
\begin{center}\begin{minipage}{15cm}\begin{Verbatim}[frame=single]
> length("Hello World!");
12
\end{Verbatim}
\end{minipage}\end{center}
\noindent Example 2: 
\begin{center}\begin{minipage}{15cm}\begin{Verbatim}[frame=single]
> length([|1,...,5|]);
5
\end{Verbatim}
\end{minipage}\end{center}
\noindent Example 3: 
\begin{center}\begin{minipage}{15cm}\begin{Verbatim}[frame=single]
> length([| |]);
0
\end{Verbatim}
\end{minipage}\end{center}
\noindent Example 4: 
\begin{center}\begin{minipage}{15cm}\begin{Verbatim}[frame=single]
> length([|1,2...|]);
@Inf@
\end{Verbatim}
\end{minipage}\end{center}

\subsection{library}
\label{lablibrary}
\noindent Name: \textbf{library}\\
binds an external mathematical function to a variable in \sollya\\
\noindent Usage: 
\begin{center}
\textbf{library}(\emph{path}) : \textsf{string} $\rightarrow$ \textsf{function}
\\ 
\end{center}
\noindent Description: \begin{itemize}

\item The command \textbf{library} lets you extends the set of mathematical
   functions known by \sollya.
   By default, \sollya knows the most common mathematical functions such
   as \textbf{exp}, \textbf{sin}, \textbf{erf}, etc. Within \sollya, these functions may be
   composed. This way, \sollya should satisfy the needs of a lot of
   users. However, for particular applications, one may want to
   manipulates other functions such as Bessel functions, or functions
   defined by an integral or even a particular solution of an ODE.

\item \textbf{library} makes it possible to let \sollya know about new functions. In
   order to let it know, you have to provide an implementation of the
   function you are interested with. This implementation is a C file containing
   a function of the form:
   \begin{verbatim} int my_ident(mpfi_t result, mpfi_t op, int n)\end{verbatim}
   The semantic of this function is the following: it is an implementation of
   the function and its derivatives in interval arithmetic.
   \verb|my_ident(result, I, n)| shall store in \verb|result| an enclosure 
   of the image set of the n-th derivative
   of the function f over \verb|I|: $f^{(n)}(I) \subseteq \mathrm{result}$.

\item The integer returned value has no meaning currently.

\item You must not provide a non trivial implementation for any \verb|n|. Most functions
   of \sollya needs a relevant implementation of $f$, $f'$ and $f''$. For higher 
   derivatives, its is not so critical and the implementation may just store 
   $[-\infty,\,+\infty]$ in result whenever $n>2$.

\item Note that you should respect somehow MPFI standards in your implementation:
   \verb|result| has its own precision and you should perform the 
   intermediate computations so that \verb|result| is as tighter as possible.

\item You can include sollya.h in your implementation and use library 
   functionnalities of \sollya for your implementation.

\item To bind your function into \sollya, you must use the same identifier as the
   function name used in your implementation file (\verb|my_ident| in the previous
   example).
\end{itemize}
\noindent Example 1: 
\begin{center}\begin{minipage}{15cm}\begin{Verbatim}[frame=single]
> bashexecute("gcc -fPIC -Wall -c libraryexample.c");
> bashexecute("gcc -shared -o libraryexample libraryexample.o -lgmp -lmpfr");
> myownlog = library("./libraryexample");
> evaluate(log(x), 2);
0.69314718055994530941723212145817656807550013436024
> evaluate(myownlog(x), 2);
0.69314718055994530941723212145817656807550013436024
\end{Verbatim}
\end{minipage}\end{center}
See also: \textbf{bashexecute} (\ref{labbashexecute}), \textbf{externalproc} (\ref{labexternalproc}), \textbf{externalplot} (\ref{labexternalplot})

\subsection{libraryconstant}
\label{lablibraryconstant}
\noindent Name: \textbf{libraryconstant}\\
binds an external mathematical constant to a variable in \sollya\\
\noindent Usage: 
\begin{center}
\textbf{libraryconstant}(\emph{path}) : \textsf{string} $\rightarrow$ \textsf{function}\\
\end{center}
\noindent Description: \begin{itemize}

\item The command \textbf{libraryconstant} lets you extend the set of mathematical
   constants known to \sollya.
   By default, the only mathematical constant known by \sollya is \textbf{pi}.
   For particular applications, one may want to
   manipulate other constants, such as Euler's gamma constant, for instance.

\item \textbf{libraryconstant} makes it possible to let \sollya know about new constants.
   In order to let it know, you have to provide an implementation of the
   constant you are interested in. This implementation is a C file containing
   a function of the form:
   \begin{verbatim} void my_ident(mpfr_t result, mp_prec_t prec)\end{verbatim}
   The semantic of this function is the following: it is an implementation of
   the constant in arbitrary precision.
   \verb|my_ident(result, prec)| shall set the
   precision of the variable result to a suitable precision (the variable is
   assumed to be already initialized) and store in result an approximate value
   of the constant with a relative error not greater than $2^{1-\mathrm{prec}}$.
   More precisely, if $c$ is the exact value of the constant, the value stored
   in result should satisfy $$|\mathrm{result}-c| \le |c|\,2^{1-\mathrm{prec}}.$$

\item You can include sollya.h in your implementation and use library 
   functionnalities of \sollya for your implementation. However, this requires to
   have compiled \sollya with \texttt{-fPIC} in order to make the \sollya executable
   code position independent and to use a system on with programs, using \texttt{dlopen}
   to open dynamic routines can dynamically open themselves.

\item To bind your constant into \sollya, you must use the same identifier as the
   function name used in your implementation file (\verb|my_ident| in the previous
   example). Once the function code has been bound to an identifier, you can use
   a simple assignment to assign the bound identifier to yet another identifier.
   This way, you may use convenient names inside \sollya even if your
   implementation environment requires you to use a less convenient name.

\item Once your constant is bound, it is considered by \sollya as an infinitely
   accurate constant (i.e. a 0-ary function, exactly like \textbf{pi}).
\end{itemize}
\noindent Example 1: 
\begin{center}\begin{minipage}{15cm}\begin{Verbatim}[frame=single]
> bashexecute("gcc -fPIC -Wall -c libraryconstantexample.c -I$HOME/.local/includ
e");
> bashexecute("gcc -shared -o libraryconstantexample libraryconstantexample.o -l
gmp -lmpfr");
> euler_gamma = libraryconstant("./libraryconstantexample");
> prec = 20!;
> euler_gamma;
0.577215
> prec = 100!;
> euler_gamma;
0.577215664901532860606512090082
> midpointmode = on;
Midpoint mode has been activated.
> [euler_gamma];
0.577215664901532860606512090~0/1~
\end{Verbatim}
\end{minipage}\end{center}
See also: \textbf{bashexecute} (\ref{labbashexecute}), \textbf{externalproc} (\ref{labexternalproc}), \textbf{externalplot} (\ref{labexternalplot}), \textbf{pi} (\ref{labpi}), \textbf{library} (\ref{lablibrary}), \textbf{evaluate} (\ref{labevaluate}), \textbf{implementconstant} (\ref{labimplementconstant})

\subsection{list of}
\label{lablistof}
\noindent Name: \textbf{list of}\\
\phantom{aaa}keyword used in combination with a type keyword\\[0.2cm]
\noindent Description: \begin{itemize}

\item \textbf{list of} is used in combination with one of the following keywords for
   indicating lists of the respective type in declarations of external
   procedures using \textbf{externalproc}: \textbf{boolean}, \textbf{constant}, \textbf{function},
   \textbf{integer}, \textbf{range} and \textbf{string}.
\end{itemize}
See also: \textbf{externalproc} (\ref{labexternalproc}), \textbf{boolean} (\ref{labboolean}), \textbf{constant} (\ref{labconstant}), \textbf{function} (\ref{labfunction}), \textbf{integer} (\ref{labinteger}), \textbf{range} (\ref{labrange}), \textbf{string} (\ref{labstring})

\subsection{log}
\label{lablog}
\noindent Name: \textbf{log}\\
natural logarithm.\\
\noindent Description: \begin{itemize}

\item \\textbf{log} is the natural logarithm defined as the inverse of the exponential\n   function: $\\log(y)$ is the unique real number $x$ such that $\\exp(x)=y$.\n
\item It is defined only for $y \\in [0; +\\infty]$.\n\end{itemize}
See also: \textbf{exp} (\ref{labexp}), \textbf{log2} (\ref{lablog2}), \textbf{log10} (\ref{lablog10})

\subsection{log10}
\label{lablog10}
\noindent Name: \textbf{log10}\\
decimal logarithm.\\
\noindent Description: \begin{itemize}

\item \\textbf{log10} is the decimal logarithm defined by: ${\\rm log10}(x) = \\log(x)/\\log(10)$.\n
\item It is defined only for $x \\in [0; +\\infty]$.\n\end{itemize}
See also: \textbf{log} (\ref{lablog}), \textbf{log2} (\ref{lablog2})

\subsection{ log1p }
\noindent Name: \textbf{log1p}\\
translated logarithm.\\

\noindent Description: \begin{itemize}

\item \textbf{log1p} is the function defined by ${\rm log1p}(x) = \log(1+x)$.

\item It is defined only for $x \in [-1; +\infty]$.
\end{itemize}
See also: \textbf{log}

\subsection{log2}
\label{lablog2}
\noindent Name: \textbf{log2}\\
\phantom{aaa}binary logarithm.\\[0.2cm]
\noindent Library names:\\
\verb|   sollya_obj_t sollya_lib_log2(sollya_obj_t)|\\
\verb|   sollya_obj_t sollya_lib_build_function_log2(sollya_obj_t)|\\
\verb|   #define SOLLYA_LOG2(x) sollya_lib_build_function_log2(x)|\\[0.2cm]
\noindent Description: \begin{itemize}

\item \textbf{log2} is the binary logarithm defined by: ${\rm log2}(x) = \log(x)/\log(2)$.

\item It is defined only for $x \in [0; +\infty]$.
\end{itemize}
See also: \textbf{log} (\ref{lablog}), \textbf{log10} (\ref{lablog10})

\subsection{lt}
\label{lablt}
\noindent Name: \textbf{$<$}\\
less-than operator\\
\noindent Usage: 
\begin{center}
\emph{expr1} \textbf{$<$} \emph{expr2} : (\textsf{constant}, \textsf{constant}) $\rightarrow$ \textsf{boolean}
\end{center}
Parameters: 
\begin{itemize}
\item \emph{expr1} and \emph{expr2} represent constant expressions
\end{itemize}
\noindent Description: \begin{itemize}

\item The operator \textbf{$<$} evaluates to true iff its operands \emph{expr1} and
   \emph{expr2} evaluate to two floating-point numbers $a_1$
   respectively $a_2$ with the global precision \textbf{prec} and
   $a_1$ is less than $a_2$. The user should
   be aware of the fact that because of floating-point evaluation, the
   operator \textbf{$<$} is not exactly the same as the mathematical
   operation \emph{less-than}.
\end{itemize}
\noindent Example 1: 
\begin{center}\begin{minipage}{15cm}\begin{Verbatim}[frame=single]
> 5 < 4;
false
> 5 < 5;
false
> 5 < 6;
true
> exp(2) < exp(1);
false
> log(1) < exp(2);
true
\end{Verbatim}
\end{minipage}\end{center}
\noindent Example 2: 
\begin{center}\begin{minipage}{15cm}\begin{Verbatim}[frame=single]
> prec = 12;
The precision has been set to 12 bits.
> 16384.1 < 16385.1;
false
\end{Verbatim}
\end{minipage}\end{center}
See also: \textbf{$==$} (\ref{labequal}), \textbf{!$=$} (\ref{labneq}), \textbf{$>=$} (\ref{labge}), \textbf{$>$} (\ref{labgt}), \textbf{$<=$} (\ref{lable}), \textbf{!} (\ref{labnot}), \textbf{$\&\&$} (\ref{laband}), \textbf{$||$} (\ref{labor}), \textbf{prec} (\ref{labprec})

\subsection{mantissa}
\label{labmantissa}
\noindent Name: \textbf{mantissa}\\
returns the integer mantissa of a number.\\
\noindent Usage: 
\begin{center}
\textbf{mantissa}(\emph{x}) : \textsf{constant} $\rightarrow$ \textsf{integer}
\\ 
\end{center}
Parameters: 
\begin{itemize}
\item \emph{x} is a dyadic number.
\end{itemize}
\noindent Description: \begin{itemize}

\item \textbf{mantissa}($x$) is by definition $x$ if $x$ equals 0, NaN, or Inf.

\item If \emph{x} is not zero, it can be uniquely written as $x = m \cdot 2^e$ where
   $m$ is an odd integer and $e$ is an integer. \textbf{mantissa}(x) returns $m$. 
\end{itemize}
\noindent Example 1: 
\begin{center}\begin{minipage}{15cm}\begin{Verbatim}[frame=single]
> a=round(Pi,20,RN);
> e=exponent(a);
> m=mantissa(a);
> m;
411775
> a-m*2^e;
0
\end{Verbatim}
\end{minipage}\end{center}
See also: \textbf{exponent} (\ref{labexponent}), \textbf{precision} (\ref{labprecision})

\subsection{max}
\label{labmax}
\noindent Name: \textbf{max}\\
determines which of given constant expressions has maximum value\\
\noindent Usage: 
\begin{center}
\textbf{max}(\emph{expr1},\emph{expr2},...,\emph{exprn}) : (\textsf{constant}, \textsf{constant}, ..., \textsf{constant}) $\rightarrow$ \textsf{constant}\\
\textbf{max}(\emph{l}) : \textsf{list} $\rightarrow$ \textsf{constant}\\
\end{center}
Parameters: 
\begin{itemize}
\item \emph{expr} are constant expressions.
\item \emph{l} is a list of constant expressions.
\end{itemize}
\noindent Description: \begin{itemize}

\item \textbf{max} determines which of a given set of constant expressions
   \emph{expr} has maximum value. To do so, \textbf{max} tries to increase the
   precision used for evaluation until it can decide the ordering or some
   maximum precision is reached. In the latter case, a warning is printed
   indicating that there might actually be another expression that has a
   greater value.

\item Even though \textbf{max} determines the maximum expression by evaluation, it 
   returns the expression that is maximum as is, i.e. as an expression
   tree that might be evaluated to any accuracy afterwards.

\item \textbf{max} can be given either an arbitrary number of constant
   expressions in argument or a list of constant expressions. The list
   however must not be end-elliptic.
\end{itemize}
\noindent Example 1: 
\begin{center}\begin{minipage}{15cm}\begin{Verbatim}[frame=single]
> max(1,2,3,exp(5),log(0.25));
1.48413159102576603421115580040552279623487667593878e2
> max(17);
17
\end{Verbatim}
\end{minipage}\end{center}
\noindent Example 2: 
\begin{center}\begin{minipage}{15cm}\begin{Verbatim}[frame=single]
> l = [|1,2,3,exp(5),log(0.25)|];
> max(l);
1.48413159102576603421115580040552279623487667593878e2
\end{Verbatim}
\end{minipage}\end{center}
\noindent Example 3: 
\begin{center}\begin{minipage}{15cm}\begin{Verbatim}[frame=single]
> print(max(exp(17),sin(62)));
exp(17)
\end{Verbatim}
\end{minipage}\end{center}
\noindent Example 4: 
\begin{center}\begin{minipage}{15cm}\begin{Verbatim}[frame=single]
> verbosity = 1!;
> print(max(17 + log2(13)/log2(9),17 + log(13)/log(9)));
Warning: minimum computation relies on floating-point result that is faithfully 
evaluated and different faithful roundings toggle the result.
17 + log2(13) / log2(9)
\end{Verbatim}
\end{minipage}\end{center}
See also: \textbf{min} (\ref{labmin}), \textbf{$==$} (\ref{labequal}), \textbf{!$=$} (\ref{labneq}), \textbf{$>=$} (\ref{labge}), \textbf{$>$} (\ref{labgt}), \textbf{$<$} (\ref{lablt}), \textbf{$<=$} (\ref{lable}), \textbf{in} (\ref{labin}), \textbf{inf} (\ref{labinf}), \textbf{sup} (\ref{labsup})

\subsection{mid}
\label{labmid}
\noindent Name: \textbf{mid}\\
gives the middle of an interval.\\
\noindent Usage: 
\begin{center}
\textbf{mid}(\emph{I}) : \textsf{range} $\rightarrow$ \textsf{constant}
\\ 
\textbf{mid}(\emph{x}) : \textsf{constant} $\rightarrow$ \textsf{constant}
\\ 
\end{center}
Parameters: 
\begin{itemize}
\item \emph{I} is an interval.
\item \emph{x} is a real number.
\end{itemize}
\noindent Description: \begin{itemize}

\item Returns the middle of the interval \emph{I}. If the middle is not exactly
   representable at the current precision, the value is returned as an
   unevaluated expression.

\item When called on a real number \emph{x}, \textbf{mid} considers it as an interval formed
   of a single point: $\left[ x, x\right]$. In other words, \textbf{mid} behaves like the identity.
\end{itemize}
\noindent Example 1: 
\begin{center}\begin{minipage}{15cm}\begin{Verbatim}[frame=single]
> mid([1;3]);
2
> mid(17);
17
\end{Verbatim}
\end{minipage}\end{center}
See also: \textbf{inf} (\ref{labinf}), \textbf{sup} (\ref{labsup})

\subsection{ midpointmode }
\noindent Name: \textbf{midpointmode}\\
global variable controlling the way intervals are displayed.\\

\noindent Description: \begin{itemize}

\item \textbf{midpointmode} is a global variable. When its value is \textbf{off}, intervals are displayed
   as usual (with the form [a;b]).
   When its value is \textbf{on}, and if a and b have the same first significant digits,
   the interval in displayed in a way that lets one immediately see the common
   digits of the two bounds.

\item This mode is supported only with \textbf{display} set to \textbf{decimal}. In other modes of 
   display, \textbf{midpointmode} value is simply ignored.
\end{itemize}
\noindent Example 1: 
\begin{center}\begin{minipage}{15cm}\begin{Verbatim}[frame=single]
> a = round(Pi,30,RD);
> b = round(Pi,30,RU);
> d = [a,b];
> d;
[0.31415926516056060791015625e1;0.31415926553308963775634765625e1]
> midpointmode=on!;
> d;
0.314159265~1/5~e1
\end{Verbatim}
\end{minipage}\end{center}
See also: \textbf{on}, \textbf{off}

\subsection{min}
\label{labmin}
\noindent Name: \textbf{min}\\
determines which of given constant expressions has minimum value\\
\noindent Usage: 
\begin{center}
\textbf{min}(\emph{expr1},\emph{expr2},...,\emph{exprn}) : (\textsf{constant}, \textsf{constant}, ..., \textsf{constant}) $\rightarrow$ \textsf{constant}\\
\textbf{min}(\emph{l}) : \textsf{list} $\rightarrow$ \textsf{constant}\\
\end{center}
Parameters: 
\begin{itemize}
\item \emph{expr} are constant expressions.
\item \emph{l} is a list of constant expressions.
\end{itemize}
\noindent Description: \begin{itemize}

\item \textbf{min} determines which of a given set of constant expressions
   \emph{expr} has minimum value. To do so, \textbf{min} tries to increase the
   precision used for evaluation until it can decide the ordering or some
   maximum precision is reached. In the latter case, a warning is printed
   indicating that there might actually be another expression that has a
   lesser value.

\item Even though \textbf{min} determines the minimum expression by evaluation, it 
   returns the expression that is minimum as is, i.e. as an expression
   tree that might be evaluated to any accuracy afterwards.

\item \textbf{min} can be given either an arbitrary number of constant
   expressions in argument or a list of constant expressions. The list
   however must not be end-elliptic.

\item Users should be aware that the behavior of \textbf{min} follows the IEEE
   754-2008 standard with respect to NaNs. In particular, \textbf{min}
   evaluates to NaN if and only if all arguments of \textbf{min} are
   NaNs. This means that NaNs may disappear during computations.
\end{itemize}
\noindent Example 1: 
\begin{center}\begin{minipage}{15cm}\begin{Verbatim}[frame=single]
> min(1,2,3,exp(5),log(0.25));
-1.3862943611198906188344642429163531361510002687205
> min(17);
17
\end{Verbatim}
\end{minipage}\end{center}
\noindent Example 2: 
\begin{center}\begin{minipage}{15cm}\begin{Verbatim}[frame=single]
> l = [|1,2,3,exp(5),log(0.25)|];
> min(l);
-1.3862943611198906188344642429163531361510002687205
\end{Verbatim}
\end{minipage}\end{center}
\noindent Example 3: 
\begin{center}\begin{minipage}{15cm}\begin{Verbatim}[frame=single]
> print(min(exp(17),sin(62)));
sin(62)
\end{Verbatim}
\end{minipage}\end{center}
\noindent Example 4: 
\begin{center}\begin{minipage}{15cm}\begin{Verbatim}[frame=single]
> verbosity = 1!;
> print(min(17 + log2(13)/log2(9),17 + log(13)/log(9)));
Warning: minimum computation relies on floating-point result that is faithfully 
evaluated and different faithful roundings toggle the result.
17 + log(13) / log(9)
\end{Verbatim}
\end{minipage}\end{center}
See also: \textbf{max} (\ref{labmax}), \textbf{$==$} (\ref{labequal}), \textbf{!$=$} (\ref{labneq}), \textbf{$>=$} (\ref{labge}), \textbf{$>$} (\ref{labgt}), \textbf{$<$} (\ref{lablt}), \textbf{$<=$} (\ref{lable}), \textbf{in} (\ref{labin}), \textbf{inf} (\ref{labinf}), \textbf{sup} (\ref{labsup})

\subsection{minus}
\label{labminus}
\noindent Name: \textbf{$-$}\\
substraction function\\

\noindent Usage: 
\begin{center}
\emph{function1} \textbf{$-$} \emph{function2} : (\textsf{function}, \textsf{function}) $\rightarrow$ \textsf{function}\\
\end{center}
Parameters: 
\begin{itemize}
\item \emph{function1} and \emph{function2} represent functions
\end{itemize}
\noindent Description: \begin{itemize}

\item \textbf{$-$} represents the substraction (function) on reals. 
   The expression \emph{function1} \textbf{$-$} \emph{function2} stands for
   the function composed of the substraction function and the two
   functions \emph{function1} and \emph{function2}, where \emph{function1} is 
   the subtrahent and \emph{function2} the substractor.
\end{itemize}
\noindent Example 1: 
\begin{center}\begin{minipage}{15cm}\begin{Verbatim}[frame=single]
> 5 - 2;
3
\end{Verbatim}
\end{minipage}\end{center}
\noindent Example 2: 
\begin{center}\begin{minipage}{15cm}\begin{Verbatim}[frame=single]
> x - 2;
-2 + x
\end{Verbatim}
\end{minipage}\end{center}
\noindent Example 3: 
\begin{center}\begin{minipage}{15cm}\begin{Verbatim}[frame=single]
> x - x;
0
\end{Verbatim}
\end{minipage}\end{center}
\noindent Example 4: 
\begin{center}\begin{minipage}{15cm}\begin{Verbatim}[frame=single]
> diff(sin(x) - exp(x));
cos(x) - exp(x)
\end{Verbatim}
\end{minipage}\end{center}
See also: \textbf{$+$} (\ref{labplus}), \textbf{$*$} (\ref{labmult}), \textbf{/} (\ref{labdivide}), \textbf{\^} (\ref{labpower})

\subsection{$*$}
\label{labmult}
\noindent Name: \textbf{$*$}\\
multiplication function\\
\noindent Usage: 
\begin{center}
\emph{function1} \textbf{$*$} \emph{function2} : (\textsf{function}, \textsf{function}) $\rightarrow$ \textsf{function}\\
\emph{interval1} \textbf{$*$} \emph{interval2} : (\textsf{range}, \textsf{range}) $\rightarrow$ \textsf{range}\\
\emph{interval1} \textbf{$*$} \emph{constant} : (\textsf{range}, \textsf{constant}) $\rightarrow$ \textsf{range}\\
\emph{interval1} \textbf{$*$} \emph{constant} : (\textsf{constant}, \textsf{range}) $\rightarrow$ \textsf{range}\\
\end{center}
Parameters: 
\begin{itemize}
\item \emph{function1} and \emph{function2} represent functions
\item \emph{interval1} and \emph{interval2} represent intervals (ranges)
\item \emph{constant} represents a constant or constant expression
\end{itemize}
\noindent Description: \begin{itemize}

\item \textbf{$*$} represents the multiplication (function) on reals. 
   The expression \emph{function1} \textbf{$*$} \emph{function2} stands for
   the function composed of the multiplication function and the two
   functions \emph{function1} and \emph{function2}.

\item \textbf{$*$} can be used for interval arithmetic on intervals
   (ranges). \textbf{$*$} will evaluate to an interval that safely
   encompasses all images of the multiplication function with arguments varying
   in the given intervals.  Any combination of intervals with intervals
   or constants (resp. constant expressions) is supported. However, it is
   not possible to represent families of functions using an interval as
   one argument and a function (varying in the free variable) as the
   other one.
\end{itemize}
\noindent Example 1: 
\begin{center}\begin{minipage}{15cm}\begin{Verbatim}[frame=single]
> 5 * 2;
10
\end{Verbatim}
\end{minipage}\end{center}
\noindent Example 2: 
\begin{center}\begin{minipage}{15cm}\begin{Verbatim}[frame=single]
> x * 2;
x * 2
\end{Verbatim}
\end{minipage}\end{center}
\noindent Example 3: 
\begin{center}\begin{minipage}{15cm}\begin{Verbatim}[frame=single]
> x * x;
x^2
\end{Verbatim}
\end{minipage}\end{center}
\noindent Example 4: 
\begin{center}\begin{minipage}{15cm}\begin{Verbatim}[frame=single]
> diff(sin(x) * exp(x));
sin(x) * exp(x) + exp(x) * cos(x)
\end{Verbatim}
\end{minipage}\end{center}
\noindent Example 5: 
\begin{center}\begin{minipage}{15cm}\begin{Verbatim}[frame=single]
> [1;2] * [3;4];
[3;8]
> [1;2] * 17;
[17;34]
> 13 * [-4;17];
[-52;221]
\end{Verbatim}
\end{minipage}\end{center}
See also: \textbf{$+$} (\ref{labplus}), \textbf{$-$} (\ref{labminus}), \textbf{/} (\ref{labdivide}), \textbf{$\mathbf{\hat{~}}$} (\ref{labpower})

\subsection{nearestint}
\label{labnearestint}
\noindent Name: \textbf{nearestint}\\
the function mapping the reals to the integers nearest to them.\\
\noindent Description: \begin{itemize}

\item \\textbf{nearestint} is defined as usual: \\textbf{nearestint}($x$) is the integer nearest to $x$, with the\n   special rule that the even integer is chosen if there exist two integers equally near to $x$.\n
\item It is defined for every real number $x$.\n\end{itemize}
See also: \textbf{ceil} (\ref{labceil}), \textbf{floor} (\ref{labfloor})

\subsection{!$=$}
\label{labneq}
\noindent Name: \textbf{!$=$}\\
\phantom{aaa}negated equality test operator\\[0.2cm]
\noindent Library name:\\
\verb|   sollya_obj_t sollya_lib_cmp_not_equal(sollya_obj_t, sollya_obj_t)|\\[0.2cm]
\noindent Usage: 
\begin{center}
\emph{expr1} \textbf{!$=$} \emph{expr2} : (\textsf{any type}, \textsf{any type}) $\rightarrow$ \textsf{boolean}\\
\end{center}
Parameters: 
\begin{itemize}
\item \emph{expr1} and \emph{expr2} represent expressions
\end{itemize}
\noindent Description: \begin{itemize}

\item The operator \textbf{!$=$} evaluates to true iff its operands \emph{expr1} and
   \emph{expr2} are syntactically unequal and both different from \textbf{error} or
   constant expressions that are not constants and that evaluate to two
   different floating-point number with the global precision \textbf{prec}. The
   user should be aware of the fact that because of floating-point
   evaluation, the operator \textbf{!$=$} is not exactly the same as the
   negation of the mathematical equality.
     
   Note that the expressions \textbf{!}(\emph{expr1} \textbf{!$=$} \emph{expr2}) and \emph{expr1}
   \textbf{$==$} \emph{expr2} do not evaluate to the same boolean value. See \textbf{error}
   for details.
\end{itemize}
\noindent Example 1: 
\begin{center}\begin{minipage}{15cm}\begin{Verbatim}[frame=single]
> "Hello" != "Hello";
false
> "Hello" != "Salut";
true
> "Hello" != 5;
true
> 5 + x != 5 + x;
false
\end{Verbatim}
\end{minipage}\end{center}
\noindent Example 2: 
\begin{center}\begin{minipage}{15cm}\begin{Verbatim}[frame=single]
> 1 != exp(0);
false
> asin(1) * 2 != pi;
false
> exp(5) != log(4);
true
\end{Verbatim}
\end{minipage}\end{center}
\noindent Example 3: 
\begin{center}\begin{minipage}{15cm}\begin{Verbatim}[frame=single]
> sin(pi/6) != 1/2 * sqrt(3);
true
\end{Verbatim}
\end{minipage}\end{center}
\noindent Example 4: 
\begin{center}\begin{minipage}{15cm}\begin{Verbatim}[frame=single]
> prec = 12;
The precision has been set to 12 bits.
> 16384.1 != 16385.1;
false
\end{Verbatim}
\end{minipage}\end{center}
\noindent Example 5: 
\begin{center}\begin{minipage}{15cm}\begin{Verbatim}[frame=single]
> error != error;
false
\end{Verbatim}
\end{minipage}\end{center}
See also: \textbf{$==$} (\ref{labequal}), \textbf{$>$} (\ref{labgt}), \textbf{$>=$} (\ref{labge}), \textbf{$<=$} (\ref{lable}), \textbf{$<$} (\ref{lablt}), \textbf{in} (\ref{labin}), \textbf{!} (\ref{labnot}), \textbf{$\&\&$} (\ref{laband}), \textbf{$||$} (\ref{labor}), \textbf{error} (\ref{laberror}), \textbf{prec} (\ref{labprec})

\subsection{nop}
\label{labnop}
\noindent Name: \textbf{nop}\\
\phantom{aaa}no operation\\[0.2cm]
\noindent Usage: 
\begin{center}
\textbf{nop} : \textsf{void} $\rightarrow$ \textsf{void}\\
\textbf{nop}() : \textsf{void} $\rightarrow$ \textsf{void}\\
\textbf{nop}(\emph{n}) : \textsf{integer} $\rightarrow$ \textsf{void}\\
\end{center}
\noindent Description: \begin{itemize}

\item The command \textbf{nop} does nothing. This means it is an explicit parse
   element in the \sollya language that finally does not produce any
   result or side-effect.

\item The command \textbf{nop} may take an optional positive integer argument \emph{n}. The argument controls how much (useless) integer additions \sollya performs while doing nothing. 
   With this behaviour, \textbf{nop} can be used for calibration of timing tests.

\item The keyword \textbf{nop} is implicit in some procedure
   definitions. Procedures without imperative body get parsed as if they
   had an imperative body containing one \textbf{nop} statement.
\end{itemize}
\noindent Example 1: 
\begin{center}\begin{minipage}{15cm}\begin{Verbatim}[frame=single]
> nop;
\end{Verbatim}
\end{minipage}\end{center}
\noindent Example 2: 
\begin{center}\begin{minipage}{15cm}\begin{Verbatim}[frame=single]
> nop(100);
\end{Verbatim}
\end{minipage}\end{center}
\noindent Example 3: 
\begin{center}\begin{minipage}{15cm}\begin{Verbatim}[frame=single]
> succ = proc(n) { return n + 1; };
> succ;
proc(n)
{
nop;
return (n) + (1);
}
> succ(5);
6
\end{Verbatim}
\end{minipage}\end{center}
See also: \textbf{proc} (\ref{labproc}), \textbf{time} (\ref{labtime})

\subsection{not}
\label{labnot}
\noindent Name: \textbf{!}\\
boolean NOT operator\\
\noindent Usage: 
\begin{center}
\textbf{!} \emph{expr} : \textsf{boolean} $\rightarrow$ \textsf{boolean}
\end{center}
Parameters: 
\begin{itemize}
\item \emph{expr} represents a boolean expression
\end{itemize}
\noindent Description: \begin{itemize}

\item \textbf{!} evaluates to the boolean NOT of the boolean expression
   \emph{expr}. \textbf{!} \emph{expr} evaluates to true iff \emph{expr} does not evaluate
   to true.
\end{itemize}
\noindent Example 1: 
\begin{center}\begin{minipage}{15cm}\begin{Verbatim}[frame=single]
> ! false;
true
\end{Verbatim}
\end{minipage}\end{center}
\noindent Example 2: 
\begin{center}\begin{minipage}{15cm}\begin{Verbatim}[frame=single]
> ! (1 == exp(0));
false
\end{Verbatim}
\end{minipage}\end{center}
See also: \textbf{$\&\&$} (\ref{laband}), \textbf{$||$} (\ref{labor})

\subsection{numberroots}
\label{labnumberroots}
\noindent Name: \textbf{numberroots}\\
Computes the number of roots of a polynomial in a given range.\\
\noindent Usage: 
\begin{center}
\textbf{numberroots}(\emph{p}, \emph{I}) : (\textsf{function}, \textsf{range}) $\rightarrow$ \textsf{integer}\\
\end{center}
Parameters: 
\begin{itemize}
\item \emph{p} is a polynomial.
\item \emph{I} is an interval.
\end{itemize}
\noindent Description: \begin{itemize}

\item \textbf{numberroots} rigorously computes the number of roots of polynomial the $p$ in
   the interval $I$. The technique used is Sturm's algorithm. The value returned
   is not just a numerical estimation of the number of roots of $p$ in $I$: it is
   the exact number of roots.

\item The command \textbf{findzeros} computes safe enclosures of all the zeros of a
   function, without forgetting any, but it is not guaranteed to separate them
   all in distinct intervals. \textbf{numberroots} is more accurate since it guarantees 
   the exact number of roots. However, it does not compute them. It may be used,
   for instance, to certify that \textbf{findzeros} did not put two distinct roots in 
   the same interval.

\item Multiple roots are counted only once.

\item The interval $I$ must be bounded. The algorithm cannot handle unbounded
   intervals. Moreover, the interval is considered as a closed interval: if one
   (or both) of the endpoints of $I$ are roots of $p$, they are counted.

\item The argument $p$ can be any expression, but if \sollya fails to prove that
   it is a polynomial an error is produced. Also, please note that if the
   coefficients of $p$ or the endpoints of $I$ are not exactly representable,
   they are first numerically evaluated, before the algorithm is used. In that
   case, the counted number of roots corresponds to the rounded polynomial on
   the rounded interval \textbf{and not} to the exact parameters given by the user.
   A warning is displayed to inform the user.
\end{itemize}
\noindent Example 1: 
\begin{center}\begin{minipage}{15cm}\begin{Verbatim}[frame=single]
> numberroots(1+x-x^2, [1,2]);
1
> findzeros(1+x-x^2, [1,2]);
[|[1.617919921875;1.6180419921875]|]
\end{Verbatim}
\end{minipage}\end{center}
\noindent Example 2: 
\begin{center}\begin{minipage}{15cm}\begin{Verbatim}[frame=single]
> numberroots((1+x)*(1-x), [-1,1]);
2
> numberroots(x^2, [-1,1]);
1
\end{Verbatim}
\end{minipage}\end{center}
\noindent Example 3: 
\begin{center}\begin{minipage}{15cm}\begin{Verbatim}[frame=single]
> verbosity = 1!;
> numberroots(x-pi, [0,4]);
Warning: the 0th coefficient of the polynomial is neither a floating point
constant nor can be evaluated without rounding to a floating point constant.
Will faithfully evaluate it with the current precision (165 bits) 
1
\end{Verbatim}
\end{minipage}\end{center}
\noindent Example 4: 
\begin{center}\begin{minipage}{15cm}\begin{Verbatim}[frame=single]
> verbosity = 1!;
> numberroots(1+x-x^2, [0, @Inf@]);
1
> numberroots(exp(x), [0, 1]);
Warning: the given function must be a polynomial in this context.
Warning: at least one of the given expressions or a subexpression is not correct
ly typed
or its evaluation has failed because of some error on a side-effect.
error
\end{Verbatim}
\end{minipage}\end{center}
See also: \textbf{dirtyfindzeros} (\ref{labdirtyfindzeros}), \textbf{findzeros} (\ref{labfindzeros})

\subsection{ numerator }
\noindent Name: \textbf{numerator}\\
gives the numerator of an expression\\

\noindent Usage: 
\begin{center}
\textbf{numerator}(\emph{expr}) : \textsf{function} $\rightarrow$ \textsf{function}\\
\end{center}
Parameters: 
\begin{itemize}
\item \emph{expr} represents an expression
\end{itemize}
\noindent Description: \begin{itemize}

\item If \emph{expr} represents a fraction \emph{expr1}/\emph{expr2}, \textbf{numerator}(\emph{expr})
   returns the numerator of this fraction, i.e. \emph{expr1}.
   If \emph{expr} represents something else, \textbf{numerator}(\emph{expr}) 
   returns the expression itself, i.e. \emph{expr}.
   Note that for all expressions \emph{expr}, \textbf{numerator}(\emph{expr}) \textbf{/} \textbf{denominator}(\emph{expr})
   is equal to \emph{expr}.
\end{itemize}
\noindent Example 1: 
\begin{center}\begin{minipage}{15cm}\begin{Verbatim}[frame=single]
> numerator(5/3);
5
\end{Verbatim}
\end{minipage}\end{center}
\noindent Example 2: 
\begin{center}\begin{minipage}{15cm}\begin{Verbatim}[frame=single]
> numerator(exp(x));
exp(x)
\end{Verbatim}
\end{minipage}\end{center}
\noindent Example 3: 
\begin{center}\begin{minipage}{15cm}\begin{Verbatim}[frame=single]
> a = 5/3;
> b = numerator(a)/denominator(a);
> print(a);
5 / 3
> print(b);
5 / 3
\end{Verbatim}
\end{minipage}\end{center}
\noindent Example 4: 
\begin{center}\begin{minipage}{15cm}\begin{Verbatim}[frame=single]
> a = exp(x/3);
> b = numerator(a)/denominator(a);
> print(a);
exp(x / 3)
> print(b);
exp(x / 3)
\end{Verbatim}
\end{minipage}\end{center}
See also: \textbf{denominator}

\subsection{off}
\label{laboff}
\noindent Name: \textbf{off}\\
special value for certain global variables.\\
\noindent Description: \begin{itemize}

\item \textbf{off} is a special value used to deactivate certain functionnalities
   of \sollya.

\item As any value it can be affected to a variable and stored in lists.
\end{itemize}
\noindent Example 1: 
\begin{center}\begin{minipage}{15cm}\begin{Verbatim}[frame=single]
> canonical=on;
Canonical automatic printing output has been activated.
> p=1+x+x^2;
> mode=off;
> p;
1 + x + x^2
> canonical=mode;
Canonical automatic printing output has been deactivated.
> p;
1 + x * (1 + x)
\end{Verbatim}
\end{minipage}\end{center}
See also: \textbf{on} (\ref{labon}), \textbf{autosimplify} (\ref{labautosimplify}), \textbf{canonical} (\ref{labcanonical}), \textbf{timing} (\ref{labtiming}), \textbf{fullparentheses} (\ref{labfullparentheses}), \textbf{midpointmode} (\ref{labmidpointmode}), \textbf{rationalmode} (\ref{labrationalmode}), \textbf{roundingwarnings} (\ref{labroundingwarnings}), \textbf{timing} (\ref{labtiming}), \textbf{dieonerrormode} (\ref{labdieonerrormode})

\subsection{ on }
\noindent Name: \textbf{on}\\
special value for certain global variables.\\

\noindent Description: \begin{itemize}

\item \textbf{on} is a special value used to activate certain functionnalities of Sollya
   (namely \textbf{canonical}, \textbf{timing}, \textbf{fullparentheses}, \textbf{midpointmode}).

\item As any value it can be affected to a variable and stored in lists.
\end{itemize}
\noindent Example 1: 
\begin{center}\begin{minipage}{15cm}\begin{Verbatim}[frame=single]
> p=1+x+x^2;
> mode=on;
> p;
1 + x * (1 + x)
> canonical=mode;
Canonical automatic printing output has been activated.
> p;
1 + x + x^2
\end{Verbatim}
\end{minipage}\end{center}
See also: \textbf{off}, \textbf{canonical}, \textbf{timing}, \textbf{fullparentheses}, \textbf{midpointmode}

\subsection{or}
\label{labor}
\noindent Name: \textbf{$||$}\\
boolean OR operator\\
\noindent Usage: 
\begin{center}
\emph{expr1} \textbf{$||$} \emph{expr2} : (\textsf{boolean}, \textsf{boolean}) $\rightarrow$ \textsf{boolean}
\\ 
\end{center}
Parameters: 
\begin{itemize}
\item \emph{expr1} and \emph{expr2} represent boolean expressions
\end{itemize}
\noindent Description: \begin{itemize}

\item \textbf{$||$} evaluates to the boolean OR of the two
   boolean expressions \emph{expr1} and \emph{expr2}. \textbf{$||$} evaluates to 
   true iff at least one of \emph{expr1} or \emph{expr2} evaluate to true.
\end{itemize}
\noindent Example 1: 
\begin{center}\begin{minipage}{15cm}\begin{Verbatim}[frame=single]
> false || false;
false
\end{Verbatim}
\end{minipage}\end{center}
\noindent Example 2: 
\begin{center}\begin{minipage}{15cm}\begin{Verbatim}[frame=single]
> (1 == exp(0)) || (0 == log(1));
true
\end{Verbatim}
\end{minipage}\end{center}
See also: \textbf{$\&\&$} (\ref{laband}), \textbf{!} (\ref{labnot})

\subsection{parse}
\label{labparse}
\noindent Name: \textbf{parse}\\
parses an expression contained in a string\\
\noindent Usage: 
\begin{center}
\textbf{parse}(\emph{string}) : \textsf{string} $\rightarrow$ \textsf{function} $|$ \textsf{error}
\\ 
\end{center}
Parameters: 
\begin{itemize}
\item \emph{string} represents a character sequence
\end{itemize}
\noindent Description: \begin{itemize}

\item \textbf{parse}(\emph{string}) parses the character sequence \emph{string} containing
   an expression built on constants and base functions.
    
   If the character sequence does not contain a well-defined expression,
   a warning is displayed indicating a syntax error and \textbf{parse} returns
   a \textbf{error} of type \textsf{error}.
\end{itemize}
\noindent Example 1: 
\begin{center}\begin{minipage}{15cm}\begin{Verbatim}[frame=single]
> parse("exp(x)");
exp(x)
\end{Verbatim}
\end{minipage}\end{center}
\noindent Example 2: 
\begin{center}\begin{minipage}{15cm}\begin{Verbatim}[frame=single]
> verbosity = 1!;
> parse("5 + * 3");
Warning: syntax error, unexpected MULTOKEN. Will try to continue parsing (expect
ing ";"). May leak memory.
Warning: the string "5 + * 3" could not be parsed by the miniparser.
Warning: at least one of the given expressions or a subexpression is not correct
ly typed
or its evaluation has failed because of some error on a side-effect.
error
\end{Verbatim}
\end{minipage}\end{center}
See also: \textbf{execute} (\ref{labexecute}), \textbf{readfile} (\ref{labreadfile})

\subsection{perturb}
\label{labperturb}
\noindent Name: \textbf{perturb}\\
indicates random perturbation of sampling points for \textbf{externalplot}\\

\noindent Usage: 
\begin{center}
\textbf{perturb} : \textsf{perturb}\\
\end{center}
\noindent Description: \begin{itemize}

\item The use of \textbf{perturb} in the command \textbf{externalplot} enables the addition
   of some random noise around each sampling point in \textbf{externalplot}.
    
   See \textbf{externalplot} for details.
\end{itemize}
\noindent Example 1: 
\begin{center}\begin{minipage}{15cm}\begin{Verbatim}[frame=single]
> bashexecute("gcc -fPIC -c externalplotexample.c");
> bashexecute("gcc -shared -o externalplotexample externalplotexample.o -lgmp -l
mpfr");
> externalplot("./externalplotexample",relative,exp(x),[-1/2;1/2],12,perturb);
\end{Verbatim}
\end{minipage}\end{center}
See also: \textbf{externalplot} (\ref{labexternalplot}), \textbf{absolute} (\ref{lababsolute}), \textbf{relative} (\ref{labrelative}), \textbf{bashexecute} (\ref{labbashexecute})

\subsection{ pi }
\noindent Name: \textbf{pi}\\
the constant Pi.\\

\noindent Description: \begin{itemize}

\item \textbf{pi} is the constant Pi, defined as half the period of sine and cosine.

\item In Sollya, \textbf{pi} is considered as a 0-ary function. This way, the constant 
   is not evaluated at the time of its definition but at the time of its use. For 
   instance, when you define a constant or a function relating to Pi, the current
   precision at the time of the definition does not matter. What is important is 
   the current precision when you evaluate the function or the constant value.

\item Remark that when you define an interval, the bounds are first evaluated and 
   then the interval is defined. In this case, \textbf{pi} will be evaluated as any 
   other constant value at the definition time of the interval, thus using the 
   current precision at this time.
\end{itemize}
\noindent Example 1: 
\begin{center}\begin{minipage}{14.8cm}\begin{Verbatim}[frame=single]
   > verbosity=1!; prec=12!;
   > a = 2*pi;
   > a;
   Warning: rounding has happened. The value displayed is a faithful rounding of the true result.
   0.62832e1
   > prec=20!;
   > a;
   Warning: rounding has happened. The value displayed is a faithful rounding of the true result.
   0.62831879e1
\end{Verbatim}
\end{minipage}\end{center}
\noindent Example 2: 
\begin{center}\begin{minipage}{14.8cm}\begin{Verbatim}[frame=single]
   > prec=12!;
   > d = [pi; 5];
   > d;
   [0.31406e1;5]
   > prec=20!;
   > d;
   [0.31406e1;5]
\end{Verbatim}
\end{minipage}\end{center}
See also: \textbf{cos}, \textbf{sin}

\subsection{plot}
\label{labplot}
\noindent Name: \textbf{plot}\\
plots one or several functions\\

\noindent Usage: 
\begin{center}
\textbf{plot}(\emph{f1}, ... ,\emph{fn}, \emph{I}) : (\textsf{function}, ... ,\textsf{function}, \textsf{range}) $\rightarrow$ \textsf{void}\\
\textbf{plot}(\emph{f1}, ... ,\emph{fn}, \emph{I}, \textbf{file}, \emph{name}) : (\textsf{function}, ... ,\textsf{function}, \textsf{range}, \textbf{file}, \textsf{string}) $\rightarrow$ \textsf{void}\\
\textbf{plot}(\emph{f1}, ... ,\emph{fn}, \emph{I}, \textbf{postscript}, \emph{name}) : (\textsf{function}, ... ,\textsf{function}, \textsf{range}, \textbf{postscript}, \textsf{string}) $\rightarrow$ \textsf{void}\\
\textbf{plot}(\emph{f1}, ... ,\emph{fn}, \emph{I}, \textbf{postscriptfile}, \emph{name}) : (\textsf{function}, ... ,\textsf{function}, \textsf{range}, \textbf{postscriptfile}, \textsf{string}) $\rightarrow$ \textsf{void}\\
\textbf{plot}(\emph{L}, \emph{I}) : (\textsf{list}, \textsf{range}) $\rightarrow$ \textsf{void}\\
\textbf{plot}(\emph{L}, \emph{I}, \textbf{file}, \emph{name}) : (\textsf{list}, \textsf{range}, \textbf{file}, \textsf{string}) $\rightarrow$ \textsf{void}\\
\textbf{plot}(\emph{L}, \emph{I}, \textbf{postscript}, \emph{name}) : (\textsf{list}, \textsf{range}, \textbf{postscript}, \textsf{string}) $\rightarrow$ \textsf{void}\\
\textbf{plot}(\emph{L}, \emph{I}, \textbf{postscriptfile}, \emph{name}) : (\textsf{list}, \textsf{range}, \textbf{postscriptfile}, \textsf{string}) $\rightarrow$ \textsf{void}\\
\end{center}
Parameters: 
\begin{itemize}
\item \emph{f1}, ..., \emph{fn} are functions to be plotted.
\item \emph{L} is a list of functions to be plotted.
\item \emph{I} is the interval where the functions have to be plotted.
\item \emph{name} is a string representing the name of a file.
\end{itemize}
\noindent Description: \begin{itemize}

\item This command plots one or several functions \emph{f1}, ... ,\emph{fn} on an interval \emph{I}.
   Functions can be either given as parameters of \textbf{plot} or as a list \emph{L}
   which elements are functions.
   Functions are plotted on the same graphic with different colors.

\item If \emph{L} contains an element that is not a function (or a constant), an error
   occurs.

\item \textbf{plot} relies on the value of global variable \textbf{points}. Let $n$ be the 
   value of this variable. The algorithm is the following: each function is 
   evaluated at $n$ evenly distributed points in \emph{I}. At each point, the 
   computed value is a faithful rounding of the exact value with a sufficiently
   big precision. Each point is finally plotted.
   This avoid numerical artefacts such as critical cancellations.

\item You can save the graphic either as a data file or as a postscript file.

\item If you use argument \textbf{file} with a string \emph{name}, Sollya will save a data file
   called name.dat and a gnuplot directives file called name.p. Invoking gnuplot
   on name.p will plots datas stored in name.dat.

\item If you use argument \textbf{postscript} with a string \emph{name}, Sollya will save a 
   postscript file called name.eps representing your graphic.

\item If you use argument \textbf{postscriptfile} with a string \emph{name}, Sollya will 
   produce the corresponding name.dat, name.p and name.eps.

\item This command uses gnuplot to produce the final graphic.
   If your terminal is not graphic (typically if you use Sollya by 
   ssh without -X)
   gnuplot should be able to detect it and produce an ASCII-art version on the
   standard output. If it is not the case, you can either store the graphic in a
   postscript file to view it locally, or use \textbf{asciiplot} command.

\item If every function is constant, \textbf{plot} will not plot them but just display
   their value.

\item If the interval is reduced to a single point, \textbf{plot} will just display the
   value of the functions at this point.
\end{itemize}
\noindent Example 1: 
\begin{center}\begin{minipage}{15cm}\begin{Verbatim}[frame=single]
> plot(sin(x),0,cos(x),[-Pi,Pi]);
\end{Verbatim}
\end{minipage}\end{center}
\noindent Example 2: 
\begin{center}\begin{minipage}{15cm}\begin{Verbatim}[frame=single]
> plot(sin(x),0,cos(x),[-Pi,Pi],postscriptfile,"plotSinCos");
\end{Verbatim}
\end{minipage}\end{center}
\noindent Example 3: 
\begin{center}\begin{minipage}{15cm}\begin{Verbatim}[frame=single]
> plot(exp(0), sin(1), [0;1]);
1
0.841470984807896506652502321630298999622563060798373
\end{Verbatim}
\end{minipage}\end{center}
\noindent Example 4: 
\begin{center}\begin{minipage}{15cm}\begin{Verbatim}[frame=single]
> plot(sin(x), cos(x), [1;1]);
0.841470984807896506652502321630298999622563060798373
0.540302305868139717400936607442976603732310420617923
\end{Verbatim}
\end{minipage}\end{center}
See also: \textbf{externalplot} (\ref{labexternalplot}), \textbf{asciiplot} (\ref{labasciiplot}), \textbf{file} (\ref{labfile}), \textbf{postscript} (\ref{labpostscript}), \textbf{postscriptfile} (\ref{labpostscriptfile})

\subsection{ plus }
\noindent Name: \textbf{$+$}\\
addition function\\

\noindent Usage: 
\begin{center}
\emph{function1} \textbf{$+$} \emph{function2} : (\textsf{function}, \textsf{function}) $\rightarrow$ \textsf{function}\\
\end{center}
Parameters: 
\begin{itemize}
\item \emph{function1} and \emph{function2} represent functions
\end{itemize}
\noindent Description: \begin{itemize}

\item \textbf{$+$} represents the addition (function) on reals. 
   The expression \emph{function1} \textbf{$+$} \emph{function2} stands for
   the function composed of the addition function and the two
   functions \emph{function1} and \emph{function2}.
\end{itemize}
\noindent Example 1: 
\begin{center}\begin{minipage}{15cm}\begin{Verbatim}[frame=single]
> 1 + 2;
3
\end{Verbatim}
\end{minipage}\end{center}
\noindent Example 2: 
\begin{center}\begin{minipage}{15cm}\begin{Verbatim}[frame=single]
> x + 2;
2 + x
\end{Verbatim}
\end{minipage}\end{center}
\noindent Example 3: 
\begin{center}\begin{minipage}{15cm}\begin{Verbatim}[frame=single]
> x + x;
x * 2
\end{Verbatim}
\end{minipage}\end{center}
\noindent Example 4: 
\begin{center}\begin{minipage}{15cm}\begin{Verbatim}[frame=single]
> diff(sin(x) + exp(x));
cos(x) + exp(x)
\end{Verbatim}
\end{minipage}\end{center}
See also: \textbf{$-$}, \textbf{$*$}, \textbf{/}, \textbf{\^}

\subsection{points}
\label{labpoints}
\noindent Name: \textbf{points}\\
controls the number of points chosen by Sollya in certain commands.\\

\noindent Description: \begin{itemize}

\item \textbf{points} is a global variable. Its value represents the number of points
   used in numerical algorithms of Sollya (namely \textbf{dirtyinfnorm},
   \textbf{dirtyintegral}, \textbf{dirtyfindzeros}, \textbf{plot}).
\end{itemize}
\noindent Example 1: 
\begin{center}\begin{minipage}{15cm}\begin{Verbatim}[frame=single]
> f=x^2*sin(1/x);
> points=10;
The number of points has been set to 10.
> dirtyfindzeros(f, [0;1]);
[|0, 0.31830988618379067153776752674502872406891929148091789|]
> points=100;
The number of points has been set to 100.
> dirtyfindzeros(f, [0;1]);
[|0, 0.244853758602915901182898097496175941591476378062242989e-1, 0.353677651315
322968375297251938920804521021434978790232e-1, 0.4547284088339866736253821810643
2674866988470211558935e-1, 0.530516476972984452562945877908381206781532152468192
029e-1, 0.636619772367581343075535053490057448137838582961835781e-1, 0.774999999
999999999999999999999999999999999999991344519e-1, 0.1061032953945968905125891755
81676241356306430493638406, 0.15915494309189533576888376337251436203445964574045
8945, 0.31830988618379067153776752674502872406891929148091789|]
\end{Verbatim}
\end{minipage}\end{center}
See also: \textbf{dirtyinfnorm} (\ref{labdirtyinfnorm}), \textbf{dirtyintegral} (\ref{labdirtyintegral}), \textbf{dirtyfindzeros} (\ref{labdirtyfindzeros}), \textbf{plot} (\ref{labplot})

\subsection{ postscript }
\noindent Name: \textbf{postscript}\\
special value for commands \textbf{plot} and \textbf{externalplot}\\

\noindent Description: \begin{itemize}

\item \textbf{postscript} is a special value used in commands \textbf{plot} and \textbf{externalplot} to save
   the result of the command in a postscript file.

\item As any value it can be affected to a variable and stored in lists.
\end{itemize}
\noindent Example 1: 
\begin{center}\begin{minipage}{15cm}\begin{Verbatim}[frame=single]
> savemode=postscript;
> name="plotSinCos";
> plot(sin(x),0,cos(x),[-Pi,Pi],savemode, name);
\end{Verbatim}
\end{minipage}\end{center}
See also: \textbf{externalplot}, \textbf{plot}, \textbf{file}, \textbf{postscriptfile}

\subsection{postscriptfile}
\label{labpostscriptfile}
\noindent Name: \textbf{postscriptfile}\\
special value for commands 	extbf{plot} and 	extbf{externalplot}\\
\noindent Description: \begin{itemize}

\item \textbf{postscriptfile} is a special value used in commands \textbf{plot} and \textbf{externalplot} to save
   the result of the command in a data file and a postscript file.

\item As any value it can be affected to a variable and stored in lists.
\end{itemize}
\noindent Example 1: 
\begin{center}\begin{minipage}{15cm}\begin{Verbatim}[frame=single]
> savemode=postscriptfile;
> name="plotSinCos";
> plot(sin(x),0,cos(x),[-Pi,Pi],savemode, name);
\end{Verbatim}
\end{minipage}\end{center}
See also: \textbf{externalplot} (\ref{labexternalplot}), \textbf{plot} (\ref{labplot}), \textbf{file} (\ref{labfile}), \textbf{postscript} (\ref{labpostscript})

\subsection{ power }
\noindent Name: \textbf{\^}\\
power function\\

\noindent Usage: 
\begin{center}
\emph{function1} \textbf{\^} \emph{function2} : (\textsf{function}, \textsf{function}) $\rightarrow$ \textsf{function}\\
\end{center}
Parameters: 
\emph{function1} and \emph{function2} represent functions\\

\noindent Description: \begin{itemize}

\item \textbf{\^} represents the power (function) on reals. 
   The expression \emph{function1} \textbf{\^} \emph{function2} stands for
   the function composed of the power function and the two
   functions \emph{function1} and \emph{function2}, where \emph{function1} is
   the base and \emph{function2} the exponent.
   If \emph{function2} is a constant integer, \textbf{\^} is defined
   on negative values of \emph{function1}. Otherwise \textbf{\^}
   is defined as exp(y * log(x)).
\end{itemize}
\noindent Example 1: 
\begin{center}\begin{minipage}{14.8cm}\begin{Verbatim}[frame=single]
   > 5 ^ 2;
   25
\end{Verbatim}
\end{minipage}\end{center}
\noindent Example 2: 
\begin{center}\begin{minipage}{14.8cm}\begin{Verbatim}[frame=single]
   > x ^ 2;
   x^2
\end{Verbatim}
\end{minipage}\end{center}
\noindent Example 3: 
\begin{center}\begin{minipage}{14.8cm}\begin{Verbatim}[frame=single]
   > 3 ^ (-5);
   0.411522633744855967078189300411522633744855967078186e-2
\end{Verbatim}
\end{minipage}\end{center}
\noindent Example 4: 
\begin{center}\begin{minipage}{14.8cm}\begin{Verbatim}[frame=single]
   > (-3) ^ (-2.5);
   @NaN@
\end{Verbatim}
\end{minipage}\end{center}
\noindent Example 5: 
\begin{center}\begin{minipage}{14.8cm}\begin{Verbatim}[frame=single]
   > diff(sin(x) ^ exp(x));
   sin(x)^exp(x) * ((cos(x) * exp(x)) / sin(x) + exp(x) * log(sin(x)))
\end{Verbatim}
\end{minipage}\end{center}
See also: \textbf{$+$}, \textbf{$-$}, \textbf{$*$}, \textbf{/}

\subsection{powers}
\label{labpowers}
\noindent Name: \textbf{powers}\\
special value for global state \textbf{display}\\
\noindent Description: \begin{itemize}

\item \textbf{powers} is a special value used for the global state \textbf{display}.  If
   the global state \textbf{display} is equal to \textbf{powers}, all data will be
   output in dyadic notation with numbers displayed in a Maple and
   PARI/GP compatible format.
    
   As any value it can be affected to a variable and stored in lists.
\end{itemize}
See also: \textbf{decimal} (\ref{labdecimal}), \textbf{dyadic} (\ref{labdyadic}), \textbf{hexadecimal} (\ref{labhexadecimal}), \textbf{binary} (\ref{labbinary})

\subsection{prec}
\label{labprec}
\noindent Name: \textbf{prec}\\
controls the precision used in numerical computations.\\
\noindent Description: \begin{itemize}

\item \\textbf{prec} is a global variable. Its value represents the precision of the \n   floating-point format used in numerical computations.\n
\item Many commands try to adapt their working precision in order to have \n   approximately $n$ correct bits in output, where $n$ is the value of \\textbf{prec}.\n\end{itemize}
\noindent Example 1: 
\begin{center}\begin{minipage}{15cm}\begin{Verbatim}[frame=single]
\end{Verbatim}
\end{minipage}\end{center}

\subsection{precision}
\label{labprecision}
\noindent Name: \textbf{precision}\\
returns the precision necessary to represent a number.\\
\noindent Usage: 
\begin{center}
\textbf{precision}(\emph{x}) : \textsf{constant} $\rightarrow$ \textsf{integer}
\\ 
\end{center}
Parameters: 
\begin{itemize}
\item \emph{x} is a dyadic number.
\end{itemize}
\noindent Description: \begin{itemize}

\item \textbf{precision}(x) is by definition $\vert x \vert$ if x equals 0, NaN, or Inf.

\item If \emph{x} is not zero, it can be uniquely written as $x = m \cdot 2^e$ where
   $m$ is an odd integer and $e$ is an integer. \textbf{precision}(x) returns the number
   of bits necessary to write $m$ (e.g. $\lceil \log_2(m) \rceil$).
\end{itemize}
\noindent Example 1: 
\begin{center}\begin{minipage}{15cm}\begin{Verbatim}[frame=single]
> a=round(Pi,20,RN);
> precision(a);
19
> m=mantissa(a);
> ceil(log2(m));
19
\end{Verbatim}
\end{minipage}\end{center}
See also: \textbf{mantissa} (\ref{labmantissa}), \textbf{exponent} (\ref{labexponent})

\subsection{.:}
\label{labprepend}
\noindent Name: \textbf{.:}\\
add an element at the beginning of a list.\\
\noindent Usage: 
\begin{center}
\emph{x}\textbf{.:}\emph{L} : (\textsf{any type}, \textsf{list}) $\rightarrow$ \textsf{list}\\
\end{center}
Parameters: 
\begin{itemize}
\item \emph{x} is an object of any type.
\item \emph{L} is a list (possibly empty).
\end{itemize}
\noindent Description: \begin{itemize}

\item \textbf{.:} adds the element \emph{x} at the beginning of the list \emph{L}.

\item Note that since \emph{x} may be of any type, it can be in particular a list.
\end{itemize}
\noindent Example 1: 
\begin{center}\begin{minipage}{15cm}\begin{Verbatim}[frame=single]
> 1.:[|2,3,4|];
[|1, 2, 3, 4|]
\end{Verbatim}
\end{minipage}\end{center}
\noindent Example 2: 
\begin{center}\begin{minipage}{15cm}\begin{Verbatim}[frame=single]
> [|1,2,3|].:[|4,5,6|];
[|[|1, 2, 3|], 4, 5, 6|]
\end{Verbatim}
\end{minipage}\end{center}
\noindent Example 3: 
\begin{center}\begin{minipage}{15cm}\begin{Verbatim}[frame=single]
> 1.:[||];
[|1|]
\end{Verbatim}
\end{minipage}\end{center}
See also: \textbf{:.} (\ref{labappend}), \textbf{@} (\ref{labconcat})

\subsection{print}
\label{labprint}
\noindent Name: \textbf{print}\\
prints an expression\\

\noindent Usage: 
\begin{center}
\textbf{print}(\emph{expr1},...,\emph{exprn}) : (\textsf{any type},..., \textsf{any type}) $\rightarrow$ \textsf{void}\\
\textbf{print}(\emph{expr1},...,\emph{exprn}) $>$ \emph{filename} : (\textsf{any type},..., \textsf{any type}, \textsf{string}) $\rightarrow$ \textsf{void}\\
\textbf{print}(\emph{expr1},...,\emph{exprn}) $>>$ \emph{filename} : (\textsf{any type},...,\textsf{any type}, \textsf{string}) $\rightarrow$ \textsf{void}\\
\end{center}
Parameters: 
\begin{itemize}
\item \emph{expr} represents an expression
\item \emph{filename} represents a character sequence indicating a file name
\end{itemize}
\noindent Description: \begin{itemize}

\item \textbf{print}(\emph{expr1},...,\emph{exprn}) prints the expressions \emph{expr1} through
   \emph{exprn} separated by spaces and followed by a newline.
    
   If a second argument \emph{filename} is given after a single  "$>$", the
   displaying is not output on the standard output of Sollya but if in
   the file \emph{filename} that get newly created or overwritten. If a double
    "$>>$" is given, the output will be appended to the file \emph{filename}.
    
   The global variables \textbf{display}, \textbf{midpointmode} and \textbf{fullparentheses} have
   some influence on the formatting of the output (see \textbf{display},
   \textbf{midpointmode} and \textbf{fullparentheses}).
    
   Remark that if one of the expressions \emph{expri} given in argument is of
   type \textsf{string}, the character sequence \emph{expri} evaluates to is
   displayed. However, if \emph{expri} is of type \textsf{list} and this list
   contains a variable of type \textsf{string}, the expression for the list
   is displayed, i.e.  all character sequences get displayed surrounded
   by quotes ('"'). Nevertheless, escape sequences used upon defining
   character sequences are interpreted immediately.
\end{itemize}
\noindent Example 1: 
\begin{center}\begin{minipage}{15cm}\begin{Verbatim}[frame=single]
> print(x + 2 + exp(sin(x))); 
x + 2 + exp(sin(x))
> print("Hello","world");
Hello world
> print("Hello","you", 4 + 3, "other persons.");
Hello you 7 other persons.
\end{Verbatim}
\end{minipage}\end{center}
\noindent Example 2: 
\begin{center}\begin{minipage}{15cm}\begin{Verbatim}[frame=single]
> print("Hello");
Hello
> print([|"Hello"|]);
[|"Hello"|]
> s = "Hello";
> print(s,[|s|]);
> t = "Hello\tyou";
> print(t,[|t|]);
\end{Verbatim}
\end{minipage}\end{center}
\noindent Example 3: 
\begin{center}\begin{minipage}{15cm}\begin{Verbatim}[frame=single]
> print(x + 2 + exp(sin(x))) > "foo.sol";
> readfile("foo.sol");
x + 2 + exp(sin(x))

\end{Verbatim}
\end{minipage}\end{center}
\noindent Example 4: 
\begin{center}\begin{minipage}{15cm}\begin{Verbatim}[frame=single]
> print(x + 2 + exp(sin(x))) >> "foo.sol";
\end{Verbatim}
\end{minipage}\end{center}
\noindent Example 5: 
\begin{center}\begin{minipage}{15cm}\begin{Verbatim}[frame=single]
> display = decimal;
Display mode is decimal numbers.
> a = evaluate(sin(pi * x), 0.25);
> b = evaluate(sin(pi * x), [0.25; 0.25 + 1b-50]);
> print(a);
0.70710678118654752440084436210484903928483593768847
> display = binary;
Display mode is binary numbers.
> print(a);
1.011010100000100111100110011001111111001110111100110010010000100010110010111110
11000100110110011011101010100101010111110100111110001110101101111011000001011101
010001_2 * 2^(-1)
> display = hexadecimal;
Display mode is hexadecimal numbers.
> print(a);
0xb.504f333f9de6484597d89b3754abe9f1d6f60ba88p-4
> display = dyadic;
Display mode is dyadic numbers.
> print(a);
33070006991101558613323983488220944360067107133265b-165
> display = powers;
Display mode is dyadic numbers in integer-power-of-2 notation.
> print(a);
33070006991101558613323983488220944360067107133265 * 2^(-165)
> display = decimal;
Display mode is decimal numbers.
> midpointmode = off;
Midpoint mode has been deactivated.
> print(b);
[0.70710678118654752440084436210484903928483593768844;0.707106781186549497437217
82517557347782646274417048]
> midpointmode = on;
Midpoint mode has been activated.
> print(b);
0.7071067811865~4/5~
> display = dyadic;
Display mode is dyadic numbers.
> print(b);
[2066875436943847413332748968013809022504194195829b-161;165350034955508254441962
37019385936414432675156571b-164]
> display = decimal;
Display mode is decimal numbers.
> autosimplify = off;
Automatic pure tree simplification has been deactivated.
> fullparentheses = off;
Full parentheses mode has been deactivated.
> print(x + x * ((x + 1) + 1));
x + x * (x + 1 + 1)
> fullparentheses = on;
Full parentheses mode has been activated.
> print(x + x * ((x + 1) + 1));
x + (x * ((x + 1) + 1))
\end{Verbatim}
\end{minipage}\end{center}
See also: \textbf{write} (\ref{labwrite}), \textbf{printexpansion} (\ref{labprintexpansion}), \textbf{printhexa} (\ref{labprinthexa}), \textbf{printfloat} (\ref{labprintfloat}), \textbf{printxml} (\ref{labprintxml}), \textbf{readfile} (\ref{labreadfile}), \textbf{autosimplify} (\ref{labautosimplify}), \textbf{display} (\ref{labdisplay}), \textbf{midpointmode} (\ref{labmidpointmode}), \textbf{fullparentheses} (\ref{labfullparentheses}), \textbf{evaluate} (\ref{labevaluate})

\subsection{printdouble}
\label{labprintdouble}
\noindent Name: \textbf{printdouble}\\
\phantom{aaa}prints a constant value as a hexadecimal double precision number\\[0.2cm]
\noindent Library name:\\
\verb|   void sollya_lib_printdouble(sollya_obj_t)|\\[0.2cm]
\noindent Usage: 
\begin{center}
\textbf{printdouble}(\emph{constant}) : \textsf{constant} $\rightarrow$ \textsf{void}\\
\end{center}
Parameters: 
\begin{itemize}
\item \emph{constant} represents a constant
\end{itemize}
\noindent Description: \begin{itemize}

\item Prints a constant value as a hexadecimal number on 16 hexadecimal
   digits. The hexadecimal number represents the integer equivalent to
   the 64 bit memory representation of the constant considered as a
   double precision number.
    
   If the constant value does not hold on a double precision number, it
   is first rounded to the nearest double precision number before
   displayed. A warning is displayed in this case.
\end{itemize}
\noindent Example 1: 
\begin{center}\begin{minipage}{15cm}\begin{Verbatim}[frame=single]
> printdouble(3);
0x4008000000000000
\end{Verbatim}
\end{minipage}\end{center}
\noindent Example 2: 
\begin{center}\begin{minipage}{15cm}\begin{Verbatim}[frame=single]
> prec=100!;
> verbosity = 1!;
> printdouble(exp(5));
Warning: the given expression is not a constant but an expression to evaluate. A
 faithful evaluation to 100 bits will be used.
Warning: rounding down occurred before printing a value as a double.
0x40628d389970338f
\end{Verbatim}
\end{minipage}\end{center}
See also: \textbf{printsingle} (\ref{labprintsingle}), \textbf{printexpansion} (\ref{labprintexpansion}), \textbf{double} (\ref{labdouble})

\subsection{printexpansion}
\label{labprintexpansion}
\noindent Name: \textbf{printexpansion}\\
\phantom{aaa}prints a polynomial in Horner form with its coefficients written as a expansions of double precision numbers\\[0.2cm]
\noindent Library name:\\
\verb|   void sollya_lib_printexpansion(sollya_obj_t)|\\[0.2cm]
\noindent Usage: 
\begin{center}
\textbf{printexpansion}(\emph{polynomial}) : \textsf{function} $\rightarrow$ \textsf{void}\\
\end{center}
Parameters: 
\begin{itemize}
\item \emph{polynomial} represents the polynomial to be printed
\end{itemize}
\noindent Description: \begin{itemize}

\item The command \textbf{printexpansion} prints the polynomial \emph{polynomial} in Horner form
   writing its coefficients as expansions of double precision
   numbers. The double precision numbers themselves are displayed in
   hexadecimal memory notation (see \textbf{printdouble}). 
    
   If some of the coefficients of the polynomial \emph{polynomial} are not
   floating-point constants but constant expressions, they are evaluated
   to floating-point constants using the global precision \textbf{prec}.  If a
   rounding occurs in this evaluation, a warning is displayed.
    
   If the exponent range of double precision is not sufficient to display
   all the mantissa bits of a coefficient, the coefficient is displayed
   rounded and a warning is displayed.
    
   If the argument \emph{polynomial} does not a polynomial, nothing but a
   warning or a newline is displayed. Constants can be displayed using
   \textbf{printexpansion} since they are polynomials of degree $0$.
\end{itemize}
\noindent Example 1: 
\begin{center}\begin{minipage}{15cm}\begin{Verbatim}[frame=single]
> printexpansion(roundcoefficients(taylor(exp(x),5,0),[|DD...|]));
0x3ff0000000000000 + x * (0x3ff0000000000000 + x * (0x3fe0000000000000 + x * ((0
x3fc5555555555555 + 0x3c65555555555555) + x * ((0x3fa5555555555555 + 0x3c4555555
5555555) + x * (0x3f81111111111111 + 0x3c01111111111111)))))
\end{Verbatim}
\end{minipage}\end{center}
\noindent Example 2: 
\begin{center}\begin{minipage}{15cm}\begin{Verbatim}[frame=single]
> printexpansion(remez(exp(x),5,[-1;1]));
(0x3ff0002eec90e5a6 + 0x3c9ea6a6a0087757 + 0xb8eb3e644ef44998) + x * ((0x3ff0002
8358fd3ac + 0x3c8ffa7d96c95f7a + 0xb91da9809b13dd54 + 0x35c0000000000000) + x * 
((0x3fdff2d7e6a9fea5 + 0x3c74460e4c0e4fe2 + 0x38fcd1b6b4e85bb0 + 0x3590000000000
000) + x * ((0x3fc54d6733b4839e + 0x3c6654e4d8614a44 + 0xb905c7a26b66ea92 + 0xb5
98000000000000) + x * ((0x3fa66c209b7150a8 + 0x3c34b1bba8f78092 + 0xb8c75f6eb90d
ae02 + 0x3560000000000000) + x * (0x3f81e554242ab128 + 0xbc23e920a76e760c + 0x38
c0589c2cae6caf + 0x3564000000000000)))))
\end{Verbatim}
\end{minipage}\end{center}
\noindent Example 3: 
\begin{center}\begin{minipage}{15cm}\begin{Verbatim}[frame=single]
> verbosity = 1!;
> prec = 3500!;
> printexpansion(pi);
(0x400921fb54442d18 + 0x3ca1a62633145c07 + 0xb92f1976b7ed8fbc + 0x35c4cf98e80417
7d + 0x32631d89cd9128a5 + 0x2ec0f31c6809bbdf + 0x2b5519b3cd3a431b + 0x27e8158536
f92f8a + 0x246ba7f09ab6b6a9 + 0xa0eedd0dbd2544cf + 0x1d779fb1bd1310ba + 0x1a1a63
7ed6b0bff6 + 0x96aa485fca40908e + 0x933e501295d98169 + 0x8fd160dbee83b4e0 + 0x8c
59b6d799ae131c + 0x08f6cf70801f2e28 + 0x05963bf0598da483 + 0x023871574e69a459 + 
0x8000000005702db3 + 0x8000000000000000)
Warning: the expansion is not complete because of the limited exponent range of 
double precision.
Warning: rounding occurred while printing.
\end{Verbatim}
\end{minipage}\end{center}
See also: \textbf{printdouble} (\ref{labprintdouble}), \textbf{horner} (\ref{labhorner}), \textbf{print} (\ref{labprint}), \textbf{prec} (\ref{labprec}), \textbf{remez} (\ref{labremez}), \textbf{taylor} (\ref{labtaylor}), \textbf{roundcoefficients} (\ref{labroundcoefficients}), \textbf{fpminimax} (\ref{labfpminimax}), \textbf{implementpoly} (\ref{labimplementpoly})

\subsection{printsingle}
\label{labprintsingle}
\noindent Name: \textbf{printsingle}\\
\phantom{aaa}prints a constant value as a hexadecimal single precision number\\[0.2cm]
\noindent Library name:\\
\verb|   void sollya_lib_printsingle(sollya_obj_t)|\\[0.2cm]
\noindent Usage: 
\begin{center}
\textbf{printsingle}(\emph{constant}) : \textsf{constant} $\rightarrow$ \textsf{void}\\
\end{center}
Parameters: 
\begin{itemize}
\item \emph{constant} represents a constant
\end{itemize}
\noindent Description: \begin{itemize}

\item Prints a constant value as a hexadecimal number on 8 hexadecimal
   digits. The hexadecimal number represents the integer equivalent to
   the 32 bit memory representation of the constant considered as a
   single precision number.
    
   If the constant value does not hold on a single precision number, it
   is first rounded to the nearest single precision number before it is
   displayed. A warning is displayed in this case.
\end{itemize}
\noindent Example 1: 
\begin{center}\begin{minipage}{15cm}\begin{Verbatim}[frame=single]
> printsingle(3);
0x40400000
\end{Verbatim}
\end{minipage}\end{center}
\noindent Example 2: 
\begin{center}\begin{minipage}{15cm}\begin{Verbatim}[frame=single]
> prec=100!;
> verbosity = 1!;
> printsingle(exp(5));
Warning: the given expression is not a constant but an expression to evaluate. A
 faithful evaluation to 100 bits will be used.
Warning: rounding down occurred before printing a value as a single.
0x431469c5
\end{Verbatim}
\end{minipage}\end{center}
See also: \textbf{printdouble} (\ref{labprintdouble}), \textbf{single} (\ref{labsingle})

\subsection{printxml}
\label{labprintxml}
\noindent Name: \textbf{printxml}\\
prints an expression as an MathML-Content-Tree\\

\noindent Usage: 
\begin{center}
\textbf{printxml}(\emph{expr}) : \textsf{function} $\rightarrow$ \textsf{void}\\
\textbf{printxml}(\emph{expr}) $>$ \emph{filename} : (\textsf{function}, \textsf{string}) $\rightarrow$ \textsf{void}\\
\textbf{printxml}(\emph{expr}) $>$ $>$ \emph{filename} : (\textsf{function}, \textsf{string}) $\rightarrow$ \textsf{void}\\
\end{center}
Parameters: 
\begin{itemize}
\item \emph{expr} represents a functional expression
\item \emph{filename} represents a character sequence indicating a file name
\end{itemize}
\noindent Description: \begin{itemize}

\item \textbf{printxml}(\emph{expr}) prints the functional expression \emph{expr} as a tree of
   MathML Content Definition Markups. This XML tree can be re-read in
   external tools or by usage of the \textbf{readxml} command.
   If a second argument \emph{filename} is given after a single $>$, the
   MathML tree is not output on the standard output of Sollya but if in
   the file \emph{filename} that get newly created or overwritten. If a double
   $>$ $>$ is given, the output will be appended to the file \emph{filename}.
\end{itemize}
\noindent Example 1: 
\begin{center}\begin{minipage}{15cm}\begin{Verbatim}[frame=single]
> printxml(x + 2 + exp(sin(x)));

<?xml version="1.0" encoding="UTF-8"?>
<!-- generated by sollya: http://sollya.gforge.inria.fr/ -->
<!-- syntax: printxml(...);   exemple: printxml(x^2-2*x+5); -->
<?xml-stylesheet type="text/xsl" href="http://perso.ens-lyon.fr/nicolas.jourdan/
mathmlc2p-web.xsl"?>
<?xml-stylesheet type="text/xsl" href="mathmlc2p-web.xsl"?>
<!-- This stylesheet allows direct web browsing of MathML-c XML files (http:// o
r file://) -->

<math xmlns="http://www.w3.org/1998/Math/MathML">
<semantics>
<annotation-xml encoding="MathML-Content">
<lambda>
<bvar><ci> x </ci></bvar>
<apply>
<apply>
<plus/>
<apply>
<plus/>
<ci> x </ci>
<cn type="integer" base="10"> 2 </cn>
</apply>
<apply>
<exp/>
<apply>
<sin/>
<ci> x </ci>
</apply>
</apply>
</apply>
</apply>
</lambda>
</annotation-xml>
<annotation encoding="sollya/text">(x + 1b1) + exp(sin(x))</annotation>
</semantics>
</math>

\end{Verbatim}
\end{minipage}\end{center}
\noindent Example 2: 
\begin{center}\begin{minipage}{15cm}\begin{Verbatim}[frame=single]
> printxml(x + 2 + exp(sin(x))) > "foo.xml";
\end{Verbatim}
\end{minipage}\end{center}
\noindent Example 3: 
\begin{center}\begin{minipage}{15cm}\begin{Verbatim}[frame=single]
> printxml(x + 2 + exp(sin(x))) >> "foo.xml";
\end{Verbatim}
\end{minipage}\end{center}
See also: \textbf{readxml} (\ref{labreadxml}), \textbf{print} (\ref{labprint}), \textbf{write} (\ref{labwrite})

\subsection{proc}
\label{labproc}
\noindent Name: \textbf{proc}\\
defines a \sollya procedure\\
\noindent Usage: 
\begin{center}
\textbf{proc}(\emph{formal parameter1}, \emph{formal parameter2},..., \emph{formal parameter n}) \textbf{begin} \emph{procedure body} \textbf{end} : \textsf{void} $\rightarrow$ \textsf{procedure}\\
\textbf{proc}(\emph{formal parameter1}, \emph{formal parameter2},..., \emph{formal parameter n}) \textbf{begin} \emph{procedure body} \textbf{return} \emph{expression}; \textbf{end} : \textsf{any type} $\rightarrow$ \textsf{procedure}\\
\textbf{proc}(\emph{formal list parameter} = ...) \textbf{begin} \emph{procedure body} \textbf{end} : \textsf{void} $\rightarrow$ \textsf{procedure}\\
\textbf{proc}(\emph{formal list parameter} = ...) \textbf{begin} \emph{procedure body} \textbf{return} \emph{expression}; \textbf{end} : \textsf{any type} $\rightarrow$ \textsf{procedure}\\
\end{center}
Parameters: 
\begin{itemize}
\item \emph{formal parameter1}, \emph{formal parameter2} through \emph{formal parameter n} represent identifiers used as formal parameters
\item \emph{formal list parameter} represents an identifier used as a formal parameter for the list of an arbitrary number of parameters
\item \emph{procedure body} represents the imperative statements in the body of the procedure
\item \emph{expression} represents the expression \textbf{proc} shall evaluate to
\end{itemize}
\noindent Description: \begin{itemize}

\item The \textbf{proc} keyword allows for defining procedures in the \sollya
   language. These procedures are common \sollya objects that can be
   applied to actual parameters after definition. Upon such an
   application, the \sollya interpreter applies the actual parameters to
   the formal parameters \emph{formal parameter1} through \emph{formal parameter n} (resp. builds up the list of arguments and applies it to the list \emph{formal list parameter}) 
   and executes the \emph{procedure body}. The procedure applied to actual
   parameters evaluates then to the expression \emph{expression} in the
   \textbf{return} statement after the \emph{procedure body} or to \textbf{void}, if no return
   statement is given (i.e. a \textbf{return} \textbf{void} statement is implicitly
   given).

\item \sollya procedures defined by \textbf{proc} have no name. They can be bound
   to an identifier by assigning the procedure object a \textbf{proc}
   expression produces to an identifier. However, it is possible to use
   procedures without giving them any name. For instance, \sollya
   procedures, i.e. procedure objects, can be elements of lists. They can
   even be given as an argument to other internal \sollya procedures. See
   also \textbf{procedure} on this subject.

\item Upon definition of a \sollya procedure using \textbf{proc}, no type check
   is performed. More precisely, the statements in \emph{procedure body} are
   merely parsed but not interpreted upon procedure definition with
   \textbf{proc}. Type checks are performed once the procedure is applied to
   actual parameters or to \textbf{void}. At this time, if the procedure was defined using several different formal 
   parameters \emph{formal parameter 1} through \emph{formal parameter n}, it is checked whether the
   number of actual parameters corresponds to the number of formal
   parameters. If the procedure was defined using the syntax for a procedure with 
   an arbitrary number of parameters by giving a \emph{formal list parameter}, the number of actual arguments is not checked but only 
   a list \emph{formal list parameter} of appropriate length is built up. Type checks are further performed upon execution of each
   statement in \emph{procedure body} and upon evaluation of the expression
   \emph{expression} to be returned.
    
   Procedures defined by \textbf{proc} containing a \textbf{quit} or \textbf{restart} command
   cannot be executed (i.e. applied). Upon application of a procedure,
   the \sollya interpreter checks beforehand for such a statement. If one
   is found, the application of the procedure to its arguments evaluates
   to \textbf{error}. A warning is displayed. Remark that in contrast to other
   type or semantic correctness checks, this check is really performed
   before interpreting any other statement in the body of the procedure.

\item Through the \textbf{var} keyword it is possible to declare local
   variables and thus to have full support of recursive procedures. This
   means a procedure defined using \textbf{proc} may contain in its \emph{procedure body} 
   an application of itself to some actual parameters: it suffices
   to assign the procedure (object) to an identifier with an appropriate
   name.

\item \sollya procedures defined using \textbf{proc} may return other
   procedures. Further \emph{procedure body} may contain assignments of
   locally defined procedure objects to identifiers. See \textbf{var} for the
   particular behaviour of local and global variables.

\item The expression \emph{expression} returned by a procedure is evaluated with
   regard to \sollya commands, procedures and external
   procedures. Simplification may be performed.  However, an application
   of a procedure defined by \textbf{proc} to actual parameters evaluates to the
   expression \emph{expression} that may contain the free global variable or
   that may be composed.
\end{itemize}
\noindent Example 1: 
\begin{center}\begin{minipage}{15cm}\begin{Verbatim}[frame=single]
> succ = proc(n) { return n + 1; };
> succ(5);
6
> 3 + succ(0);
4
> succ;
proc(n)
begin
nop;
return (n) + (1);
end
\end{Verbatim}
\end{minipage}\end{center}
\noindent Example 2: 
\begin{center}\begin{minipage}{15cm}\begin{Verbatim}[frame=single]
> add = proc(m,n) { var res; res := m + n; return res; };
> add(5,6);
11
> add;
proc(m, n)
begin
var res;
res := (m) + (n);
return res;
end
> verbosity = 1!;
> add(3);
Warning: at least one of the given expressions or a subexpression is not correct
ly typed
or its evaluation has failed because of some error on a side-effect.
error
> add(true,false);
Warning: at least one of the given expressions or a subexpression is not correct
ly typed
or its evaluation has failed because of some error on a side-effect.
Warning: the given expression or command could not be handled.
error
\end{Verbatim}
\end{minipage}\end{center}
\noindent Example 3: 
\begin{center}\begin{minipage}{15cm}\begin{Verbatim}[frame=single]
> succ = proc(n) { return n + 1; };
> succ(5);
6
> succ(x);
1 + x
\end{Verbatim}
\end{minipage}\end{center}
\noindent Example 4: 
\begin{center}\begin{minipage}{15cm}\begin{Verbatim}[frame=single]
> hey = proc() { print("Hello world."); };
> hey();
Hello world.
> print(hey());
Hello world.
void
> hey;
proc()
begin
print("Hello world.");
return void;
end
\end{Verbatim}
\end{minipage}\end{center}
\noindent Example 5: 
\begin{center}\begin{minipage}{15cm}\begin{Verbatim}[frame=single]
> fac = proc(n) { var res; if (n == 0) then res := 1 else res := n * fac(n - 1);
 return res; };
> fac(5);
120
> fac(11);
39916800
> fac;
proc(n)
begin
var res;
if (n) == (0) then
res := 1
else
res := (n) * (fac((n) - (1)));
return res;
end
\end{Verbatim}
\end{minipage}\end{center}
\noindent Example 6: 
\begin{center}\begin{minipage}{15cm}\begin{Verbatim}[frame=single]
> myprocs = [| proc(m,n) { return m + n; }, proc(m,n) { return m - n; } |];
> (myprocs[0])(5,6);
11
> (myprocs[1])(5,6);
-1
> succ = proc(n) { return n + 1; };
> pred = proc(n) { return n - 1; };
> applier = proc(p,n) { return p(n); };
> applier(succ,5);
6
> applier(pred,5);
4
\end{Verbatim}
\end{minipage}\end{center}
\noindent Example 7: 
\begin{center}\begin{minipage}{15cm}\begin{Verbatim}[frame=single]
> verbosity = 1!;
> myquit = proc(n) { print(n); quit; };
> myquit;
proc(n)
begin
print(n);
quit;
return void;
end
> myquit(5);
Warning: a quit or restart command may not be part of a procedure body.
The procedure will not be executed.
Warning: an error occurred while executing a procedure.
Warning: the given expression or command could not be handled.
error
\end{Verbatim}
\end{minipage}\end{center}
\noindent Example 8: 
\begin{center}\begin{minipage}{15cm}\begin{Verbatim}[frame=single]
> printsucc = proc(n) { var succ; succ = proc(n) { return n + 1; }; print("Succe
ssor of",n,"is",succ(n)); };
> printsucc(5);
Successor of 5 is 6
\end{Verbatim}
\end{minipage}\end{center}
\noindent Example 9: 
\begin{center}\begin{minipage}{15cm}\begin{Verbatim}[frame=single]
> makeadd = proc(n) { var add; print("n =",n); add = proc(m,n) { return n + m; }
; return add; };
> makeadd(4);
n = 4
proc(m, n)
begin
nop;
return (n) + (m);
end
> (makeadd(4))(5,6);
n = 4
11
\end{Verbatim}
\end{minipage}\end{center}
\noindent Example 10: 
\begin{center}\begin{minipage}{15cm}\begin{Verbatim}[frame=single]
> sumall = proc(l = ...) { var acc, i; acc = 0; for i in l do acc = acc + i; ret
urn acc; };
> sumall;
proc(l = ...)
begin
var acc, i;
acc = 0;
for i in l do
acc = (acc) + (i);
return acc;
end
> sumall();
0
> sumall(2);
2
> sumall(2,5);
7
> sumall(2,5,7,9,16);
39
> sumall @ [|1,...,10|];
55
\end{Verbatim}
\end{minipage}\end{center}
See also: \textbf{return} (\ref{labreturn}), \textbf{externalproc} (\ref{labexternalproc}), \textbf{void} (\ref{labvoid}), \textbf{quit} (\ref{labquit}), \textbf{restart} (\ref{labrestart}), \textbf{var} (\ref{labvar}), \textbf{@} (\ref{labconcat})

\subsection{ procedure }
\noindent Name: \textbf{procedure}\\
defines and assigns a Sollya procedure\\

\noindent Usage: 
\begin{center}
\textbf{procedure} \emph{identifier}(\emph{formal parameter1}, \emph{formal parameter2},..., \emph{formal parameter n}) \textbf{begin} \emph{procedure body} \textbf{end} : \textsf{void} $\rightarrow$ \textsf{void}\\
\textbf{procedure} \emph{identifier}(\emph{formal parameter1}, \emph{formal parameter2},..., \emph{formal parameter n}) \textbf{begin} \emph{procedure body} \textbf{return} \emph{expression}; \textbf{end} : \textsf{any type} $\rightarrow$ \textsf{void}\\
\end{center}
Parameters: 
\begin{itemize}
\item \emph{identifier} represents the name of the procedure to be defined and assigned
\item \emph{formal parameter1}, \emph{formal parameter2} through \emph{formal parameter n} represent identifiers used as formal parameters
\item \emph{procedure body} represents the imperative statements in the body of the procedure
\item \emph{expression} represents the expression \textbf{procedure} shall evaluate to
\end{itemize}
\noindent Description: \begin{itemize}

\item The \textbf{procedure} keyword allows for defining and assigning procedures in
   the Sollya language. It is an abbreviation to a procedure definition
   using \textbf{proc} with the same formal parameters, procedure body and
   return-expression followed by an assignment of the procedure (object)
   to the identifier \emph{identifier}. In particular, all rules concerning
   local variables declared using the \textbf{var} keyword apply for \textbf{procedure}.
\end{itemize}
\noindent Example 1: 
\begin{center}\begin{minipage}{15cm}\begin{Verbatim}[frame=single]
> procedure succ(n) { return n + 1; };
> succ(5);
6
> 3 + succ(0);
4
> succ;
proc(n)
begin
nop;
return (n) + (1);
end
\end{Verbatim}
\end{minipage}\end{center}
See also: \textbf{proc}, \textbf{var}

\subsection{QD}
\label{labqd}
\noindent Name: \textbf{QD}\\
short form for \textbf{quad}\\
See also: \textbf{quad} (\ref{labquad})

\subsection{quad}
\label{labquad}
\noindent Names: \textbf{quad}, \textbf{QD}\\
rounding to the nearest IEEE 754 quad (binary128).\\
\noindent Description: \begin{itemize}

\item \textbf{quad} is both a function and a constant.

\item As a function, it rounds its argument to the nearest IEEE 754 quad precision (i.e. IEEE754-2008 binary128) number.
   Subnormal numbers are supported as well as standard numbers: it is the real
   rounding described in the standard.

\item As a constant, it symbolizes the quad precision format. It is used in 
   contexts when a precision format is necessary, e.g. in the commands 
   \textbf{round} and \textbf{roundcoefficients}. In is not supported for \textbf{implementpoly}.
   See the corresponding help pages for examples.
\end{itemize}
\noindent Example 1: 
\begin{center}\begin{minipage}{15cm}\begin{Verbatim}[frame=single]
> display=binary!;
> QD(0.1);
1.100110011001100110011001100110011001100110011001100110011001100110011001100110
011001100110011001100110011001101_2 * 2^(-4)
> QD(4.17);
1.000010101110000101000111101011100001010001111010111000010100011110101110000101
000111101011100001010001111010111_2 * 2^(2)
> QD(1.011_2 * 2^(-16493));
1.1_2 * 2^(-16493)
\end{Verbatim}
\end{minipage}\end{center}
See also: \textbf{halfprecision} (\ref{labhalfprecision}), \textbf{single} (\ref{labsingle}), \textbf{double} (\ref{labdouble}), \textbf{doubleextended} (\ref{labdoubleextended}), \textbf{doubledouble} (\ref{labdoubledouble}), \textbf{tripledouble} (\ref{labtripledouble}), \textbf{roundcoefficients} (\ref{labroundcoefficients}), \textbf{implementpoly} (\ref{labimplementpoly}), \textbf{round} (\ref{labround}), \textbf{printsingle} (\ref{labprintsingle})

\subsection{quit}
\label{labquit}
\noindent Name: \textbf{quit}\\
quits \sollya\\
\noindent Usage: 
\begin{center}
\textbf{quit} : \textsf{void} $\rightarrow$ \textsf{void}
\\ 
\end{center}
\noindent Description: \begin{itemize}

\item The command \textbf{quit}, when executed abandons the execution of a \sollya
   script and leaves the \sollya interpreter unless the \textbf{quit} command 
   is executed in a \sollya script read into a main \sollya script by
   \textbf{execute} or $\#$include.
    
   Upon exiting the \sollya interpreter, all state is thrown away, all
   memory is deallocated, all bound libraries are unbound and the
   temporary files produced by \textbf{plot} and \textbf{externalplot} are deleted.
    
   If the \textbf{quit} command does not lead to the abandon of the \sollya
   interpreter, a warning is displayed.
\end{itemize}
\noindent Example 1: 
\begin{center}\begin{minipage}{15cm}\begin{Verbatim}[frame=single]
> quit;
\end{Verbatim}
\end{minipage}\end{center}
See also: \textbf{restart} (\ref{labrestart}), \textbf{execute} (\ref{labexecute}), \textbf{plot} (\ref{labplot}), \textbf{externalplot} (\ref{labexternalplot})

\subsection{ range }
\noindent Name: \textbf{range}\\
keyword representing a \textsf{range} type \\

\noindent Usage: 
\begin{center}
\textbf{range} : \textsf{type type}\\
\end{center}
\noindent Description: \begin{itemize}

\item \textbf{range} represents the \textsf{range} type for declarations
   of external procedures by means of \textbf{externalproc}.
   Remark that in contrast to other indicators, type indicators like
   \textbf{range} cannot be handled outside the \textbf{externalproc} context.  In
   particular, they cannot be assigned to variables.
\end{itemize}
See also: \textbf{externalproc}, \textbf{boolean}, \textbf{constant}, \textbf{function}, \textbf{integer}, \textbf{list of}, \textbf{string}

\subsection{rationalapprox}
\label{labrationalapprox}
\noindent Name: \textbf{rationalapprox}\\
returns a fraction close to a given number.\\
\noindent Usage: 
\begin{center}
\textbf{rationalapprox}(\emph{x},\emph{n}) : (\textsf{constant}, \textsf{integer}) $\rightarrow$ \textsf{function}\\
\end{center}
Parameters: 
\begin{itemize}
\item \emph{x} is a number to approximate.
\item \emph{n} is a integer (representing a format).
\end{itemize}
\noindent Description: \begin{itemize}

\item \textbf{rationalapprox}(\emph{x},\emph{n}) returns a constant function of the form $a/b$ where $a$ and $b$ are
   integers. The value $a/b$ is an approximation of \emph{x}. The quality of this 
   approximation is determined by the parameter \emph{n} that indicates the number of
   correct bits that $a/b$ should have.

\item The command is not safe in the sense that it is not ensured that the error 
   between $a/b$ and \emph{x} is less than $2^{-n}$.

\item The following algorithm is used: \emph{x} is first rounded downwards and upwards to
   a format of \emph{n} bits, thus obtaining an interval $[x_l,\,x_u]$. This interval is then
   developped into a continued fraction as far as the representation is the same
   for every elements of $[x_l,\,x_u]$. The corresponding fraction is returned.

\item Since rational numbers are not a primitive object of \sollya, the fraction is
   returned as a constant function. This can be quite amazing, because \sollya
   immediately simplifies a constant function by evaluating it when the constant
   has to be displayed.
   To avoid this, you can use \textbf{print} (that displays the expression representing
   the constant and not the constant itself) or the commands \textbf{numerator} 
   and \textbf{denominator}.
\end{itemize}
\noindent Example 1: 
\begin{center}\begin{minipage}{15cm}\begin{Verbatim}[frame=single]
> pi10 = rationalapprox(Pi,10);
> pi50 = rationalapprox(Pi,50);
> pi100 = rationalapprox(Pi,100);
> print( pi10, ": ", simplify(floor(-log2(abs(pi10-Pi)/Pi))), "bits." );
22 / 7 :  11 bits.
> print( pi50, ": ", simplify(floor(-log2(abs(pi50-Pi)/Pi))), "bits." );
90982559 / 28960648 :  50 bits.
> print( pi100, ": ", simplify(floor(-log2(abs(pi100-Pi)/Pi))), "bits." );
4850225745369133 / 1543874804974140 :  101 bits.
\end{Verbatim}
\end{minipage}\end{center}
\noindent Example 2: 
\begin{center}\begin{minipage}{15cm}\begin{Verbatim}[frame=single]
> a=0.1;
> b=rationalapprox(a,4);
> numerator(b); denominator(b);
1
10
> print(simplify(floor(-log2(abs((b-a)/a)))), "bits.");
166 bits.
\end{Verbatim}
\end{minipage}\end{center}
See also: \textbf{print} (\ref{labprint}), \textbf{numerator} (\ref{labnumerator}), \textbf{denominator} (\ref{labdenominator}), \textbf{rationalmode} (\ref{labrationalmode})

\subsection{rationalmode}
\label{labrationalmode}
\noindent Name: \textbf{rationalmode}\\
\phantom{aaa}global variable controlling if rational arithmetic is used or not.\\[0.2cm]
\noindent Library names:\\
\verb|   void sollya_lib_set_rationalmode_and_print(sollya_obj_t)|\\
\verb|   void sollya_lib_set_rationalmode(sollya_obj_t)|\\
\verb|   sollya_obj_t sollya_lib_get_rationalmode()|\\[0.2cm]
\noindent Usage: 
\begin{center}
\textbf{rationalmode} = \emph{activation value} : \textsf{on$|$off} $\rightarrow$ \textsf{void}\\
\textbf{rationalmode} = \emph{activation value} ! : \textsf{on$|$off} $\rightarrow$ \textsf{void}\\
\textbf{rationalmode} : \textsf{on$|$off}\\
\end{center}
Parameters: 
\begin{itemize}
\item \emph{activation value} controls if rational arithmetic should be used or not
\end{itemize}
\noindent Description: \begin{itemize}

\item \textbf{rationalmode} is a global variable. When its value is \textbf{off}, which is the default,
   \sollya will not use rational arithmetic to simplify expressions. All computations,
   including the evaluation of constant expressions given on the \sollya prompt,
   will be performed using floating-point and interval arithmetic. Constant expressions
   will be approximated by floating-point numbers, which are in most cases faithful 
   roundings of the expressions, when shown at the prompt. 

\item When the value of the global variable \textbf{rationalmode} is \textbf{on}, \sollya will use 
   rational arithmetic when simplifying expressions. Constant expressions, given 
   at the \sollya prompt, will be simplified to rational numbers and displayed 
   as such when they are in the set of the rational numbers. Otherwise, flaoting-point
   and interval arithmetic will be used to compute a floating-point approximation,
   which is in most cases a faithful rounding of the constant expression.
\end{itemize}
\noindent Example 1: 
\begin{center}\begin{minipage}{15cm}\begin{Verbatim}[frame=single]
> rationalmode=off!;
> 19/17 + 3/94;
1.1495619524405506883604505632040050062578222778473
> rationalmode=on!;
> 19/17 + 3/94;
1837 / 1598
\end{Verbatim}
\end{minipage}\end{center}
\noindent Example 2: 
\begin{center}\begin{minipage}{15cm}\begin{Verbatim}[frame=single]
> rationalmode=off!;
> exp(19/17 + 3/94);
3.15680977395514136754709208944824276340328162814418
> rationalmode=on!;
> exp(19/17 + 3/94);
3.15680977395514136754709208944824276340328162814418
\end{Verbatim}
\end{minipage}\end{center}
See also: \textbf{on} (\ref{labon}), \textbf{off} (\ref{laboff}), \textbf{numerator} (\ref{labnumerator}), \textbf{denominator} (\ref{labdenominator}), \textbf{simplify} (\ref{labsimplify}), \textbf{rationalapprox} (\ref{labrationalapprox}), \textbf{autosimplify} (\ref{labautosimplify})

\subsection{RD}
\label{labrd}
\noindent Name: \textbf{RD}\\
constant representing rounding-downwards mode.\\
\noindent Description: \begin{itemize}

\item \textbf{RD} is used in command \textbf{round} to specify that the value $x$ must be rounded
   to the greatest floating-point number $y$ such that $y \le x$.
\end{itemize}
\noindent Example 1: 
\begin{center}\begin{minipage}{15cm}\begin{Verbatim}[frame=single]
> display=binary!;
> round(Pi,20,RD);
1.1001001000011111101_2 * 2^(1)
\end{Verbatim}
\end{minipage}\end{center}
See also: \textbf{RZ} (\ref{labrz}), \textbf{RU} (\ref{labru}), \textbf{RN} (\ref{labrn}), \textbf{round} (\ref{labround})

\subsection{readfile}
\label{labreadfile}
\noindent Name: \textbf{readfile}\\
\phantom{aaa}reads the content of a file into a string variable\\[0.2cm]
\noindent Usage: 
\begin{center}
\textbf{readfile}(\emph{filename}) : \textsf{string} $\rightarrow$ \textsf{string}\\
\end{center}
Parameters: 
\begin{itemize}
\item \emph{filename} represents a character sequence indicating a file name
\end{itemize}
\noindent Description: \begin{itemize}

\item \textbf{readfile} opens the file indicated by \emph{filename}, reads it and puts its
   contents in a character sequence of type \textsf{string} that is returned.
    
   If the file indicated by \emph{filename} cannot be opened for reading, a
   warning is displayed and \textbf{readfile} evaluates to an \textbf{error} variable of
   type \textsf{error}.
\end{itemize}
\noindent Example 1: 
\begin{center}\begin{minipage}{15cm}\begin{Verbatim}[frame=single]
> print("Hello world") > "myfile.txt";
> t = readfile("myfile.txt"); 
> t;
Hello world

\end{Verbatim}
\end{minipage}\end{center}
\noindent Example 2: 
\begin{center}\begin{minipage}{15cm}\begin{Verbatim}[frame=single]
> verbosity=1!;
> readfile("afile.txt");
Warning: the file "(null)" could not be opened for reading.
Warning: at least one of the given expressions or a subexpression is not correct
ly typed
or its evaluation has failed because of some error on a side-effect.
error
\end{Verbatim}
\end{minipage}\end{center}
See also: \textbf{parse} (\ref{labparse}), \textbf{execute} (\ref{labexecute}), \textbf{write} (\ref{labwrite}), \textbf{print} (\ref{labprint}), \textbf{bashexecute} (\ref{labbashexecute}), \textbf{error} (\ref{laberror})

\subsection{readxml}
\label{labreadxml}
\noindent Name: \textbf{readxml}\\
reads an expression written as a MathML-Content-Tree in a file\\

\noindent Usage: 
\begin{center}
\textbf{readxml}(\emph{filename}) : \textsf{string} $\rightarrow$ \textsf{function} $|$ \textsf{error}\\
\end{center}
Parameters: 
\begin{itemize}
\item \emph{filename} represents a character sequence indicating a file name
\end{itemize}
\noindent Description: \begin{itemize}

\item \textbf{readxml}(\emph{filename}) reads the first occurrence of a lambda
   application with one bounded variable on applications of the supported
   basic functions in file \emph{filename} and returns it as a Sollya
   functional expression.
    
   If the file \emph{filename} does not contain a valid MathML-Content tree,
   \textbf{readxml} tries to find an "annotation encoding" markup of type
   "sollya/text". If this annotation contains a character sequence
   parseable by \textbf{parse}, \textbf{readxml} returns that expression.  Otherwise
   \textbf{readxml} displays a warning and returns an \textbf{error} variable of type
   \textsf{error}.
\end{itemize}
\noindent Example 1: 
\begin{center}\begin{minipage}{15cm}\begin{Verbatim}[frame=single]
> readxml("readxmlexample.xml");
2 + x + exp(sin(x))
\end{Verbatim}
\end{minipage}\end{center}
See also: \textbf{printxml} (\ref{labprintxml}), \textbf{readfile} (\ref{labreadfile}), \textbf{parse} (\ref{labparse})

\subsection{relative}
\label{labrelative}
\noindent Name: \textbf{relative}\\
indicates a relative error for 	extbf{externalplot} or 	extbf{fpminimax}\\
\noindent Usage: 
\begin{center}
\textbf{relative} : \textsf{absolute$|$relative}
\end{center}
\noindent Description: \begin{itemize}

\item The use of \textbf{relative} in the command \textbf{externalplot} indicates that during
   plotting in \textbf{externalplot} a relative error is to be considered.
    
   See \textbf{externalplot} for details.
   Used with \textbf{fpminimax}, \textbf{relative} indicates that \textbf{fpminimax} must try to minimize
   the relative error.
    
   See \textbf{fpminimax} for details.
\end{itemize}
\noindent Example 1: 
\begin{center}\begin{minipage}{15cm}\begin{Verbatim}[frame=single]
> bashexecute("gcc -fPIC -c externalplotexample.c");
> bashexecute("gcc -shared -o externalplotexample externalplotexample.o -lgmp -l
mpfr");
> externalplot("./externalplotexample",absolute,exp(x),[-1/2;1/2],12,perturb);
\end{Verbatim}
\end{minipage}\end{center}
See also: \textbf{externalplot} (\ref{labexternalplot}), \textbf{fpminimax} (\ref{labfpminimax}), \textbf{absolute} (\ref{lababsolute}), \textbf{bashexecute} (\ref{labbashexecute})

\subsection{remez}
\label{labremez}
\noindent Name: \textbf{remez}\\
\phantom{aaa}computes the minimax of a function on an interval.\\[0.2cm]
\noindent Library names:\\
\verb|   sollya_obj_t sollya_lib_remez(sollya_obj_t, sollya_obj_t, sollya_obj_t, ...)|\\
\verb|   sollya_obj_t sollya_lib_v_remez(sollya_obj_t, sollya_obj_t, sollya_obj_t,|\\
\verb|                                   va_list)|\\[0.2cm]
\noindent Usage: 
\begin{center}
\textbf{remez}(\emph{f}, \emph{n}, \emph{range}, \emph{w}, \emph{quality}, \emph{bounds}) : (\textsf{function}, \textsf{integer}, \textsf{range}, \textsf{function}, \textsf{constant}, \textsf{range}) $\rightarrow$ \textsf{function}\\
\textbf{remez}(\emph{f}, \emph{L}, \emph{range}, \emph{w}, \emph{quality}, \emph{bounds}) : (\textsf{function}, \textsf{list}, \textsf{range}, \textsf{function}, \textsf{constant}, \textsf{range}) $\rightarrow$ \textsf{function}\\
\end{center}
Parameters: 
\begin{itemize}
\item \emph{f} is the function to be approximated
\item \emph{n} is the degree of the polynomial that must approximate \emph{f}
\item \emph{L} is a list of integers or a list of functions and indicates the basis for the approximation of \emph{f}
\item \emph{range} is the interval where the function must be approximated
\item \emph{w} (optional) is a weight function. Default is 1.
\item \emph{quality} (optional) is a parameter that controls the quality of the returned polynomial \emph{p}, with respect to the exact minimax $p^\star$. Default is 1e-5.
\item \emph{bounds} (optional) is a parameter that allows the user to make the algorithm stop earlier, whenever a given accuracy is reached or a given accuracy is proved unreachable. Default is $[0,\,+\infty]$.
\end{itemize}
\noindent Description: \begin{itemize}

\item \textbf{remez} computes an approximation of the function $f$ with respect to
   the weight function $w$ on the interval \emph{range}. More precisely, it
   searches $p$ such that $\|pw-f\|_{\infty}$ is
   (almost) minimal among all $p$ of a certain form. The norm is
   the infinity norm, e.g. $\|g\|_{\infty} = \max \{|g(x)|, x \in \mathrm{range}\}.$

\item If $w=1$ (the default case), it consists in searching the best
   polynomial approximation of $f$ with respect to the absolute error.
   If $f=1$ and $w$ is of the form $1/g$, it consists in
   searching the best polynomial approximation of $g$ with respect to the
   relative error.

\item If $n$ is given, $p$ is searched among the polynomials with degree not
   greater than $n$.
   If \emph{L} is given and is a list of integers, $p$ is searched as a linear
   combination of monomials $X^k$ where $k$ belongs to \emph{L}.
   In the case when \emph{L} is a list of integers, it may contain ellipses but
   cannot be end-elliptic.
   If \emph{L} is given and is a list of functions $g_k$, $p$ is searched as a
   linear combination of the $g_k$. In that case \emph{L} cannot contain ellipses.
   It is the user responsability to check that the $g_k$ are linearly independent
   over the interval \emph{range}. Moreover, the functions $w\cdot g_k$ must be at least
   twice differentiable over \emph{range}. If these conditions are not fulfilled, the
   algorithm might fail or even silently return a result as if it successfully
   found the minimax, though the returned $p$ is not optimal.

\item The polynomial is obtained by a convergent iteration called Remez'
   algorithm (and an extension of this algorithm, due to Stiefel).
   The algorithm computes a sequence $p_1,\dots ,p_k,\dots$
   such that $e_k = \|p_k w-f\|_{\infty}$ converges towards
   the optimal value $e$. The algorithm is stopped when the relative error
   between $e_k$ and $e$ is less than \emph{quality}.

\item The optional argument \emph{bounds} is an interval $[\varepsilon_\ell,\,\varepsilon_u]$
   with the following behavior:\begin{itemize}
     \item if, during the algorithm, we manage to prove that $\varepsilon_u$ is
       unreachable, we stop the algorithm returning the last computed
       polynomial.
     \item if, during the algorithm, we obtain a polynomial with an error smaller
       than $\varepsilon_\ell$, we stop the algorithm returning that polynomial.
     \item otherwise we loop until we find an optimal polynomial with the required
       quality, as usual.\end{itemize}
   Examples of use:\\
     $[0,\,+\infty]$ (compute the optimal polynomial with the required quality)\\
     $[\varepsilon_u]$ (stops as soon as a polynomial achieving $\varepsilon_u$ is
                   obtained or as soon as such a polynomial is proved not to
                   exist).\\
     $[0,\,\varepsilon_u]$ (finds the optimal polynomial, but provided that its error
                      is smaller than $\varepsilon_u$).\\
     $[\varepsilon_\ell,\,+\infty]$ (stops as soon as a polynomial achieving
                             $\varepsilon_\ell$ is obtained. If such a polynomial
                             does not exist, returns the optimal polynomial).
\end{itemize}
\noindent Example 1: 
\begin{center}\begin{minipage}{15cm}\begin{Verbatim}[frame=single]
> p = remez(exp(x),5,[0;1]);
> degree(p);
5
> dirtyinfnorm(p-exp(x),[0;1]);
1.12956981510961487071711938292660776072226345893629e-6
\end{Verbatim}
\end{minipage}\end{center}
\noindent Example 2: 
\begin{center}\begin{minipage}{15cm}\begin{Verbatim}[frame=single]
> p = remez(1,[|0,2,4,6,8|],[0,Pi/4],1/cos(x));
> canonical=on!;
> p;
0.99999999994393732180959690352543887130348096061124 + -0.4999999957155685776877
20530637215446709494672222587 * x^2 + 4.1666613233473633009941059480570275870113
220089059e-2 * x^4 + -1.3886529147145693651355523880319714051047635695061e-3 * x
^6 + 2.4372679177224179934800328511009205218114284220126e-5 * x^8
\end{Verbatim}
\end{minipage}\end{center}
\noindent Example 3: 
\begin{center}\begin{minipage}{15cm}\begin{Verbatim}[frame=single]
> p1 = remez(exp(x),5,[0;1],default,1e-5);
> p2 = remez(exp(x),5,[0;1],default,1e-10);
> p3 = remez(exp(x),5,[0;1],default,1e-15);
> dirtyinfnorm(p1-exp(x),[0;1]);
1.12956981510961487071711938292660776072226345893629e-6
> dirtyinfnorm(p2-exp(x),[0;1]);
1.12956980227478675612619255125474525171079325793124e-6
> dirtyinfnorm(p3-exp(x),[0;1]);
1.12956980227478675612619255125474525171079325793124e-6
\end{Verbatim}
\end{minipage}\end{center}
\noindent Example 4: 
\begin{center}\begin{minipage}{15cm}\begin{Verbatim}[frame=single]
> L = [|exp(x), sin(x), cos(x)-1, sin(x^3)|];
> g = (2^x-1)/x;
> p1 = remez(g, L, [-1/16;1/16]);
> p2 = remez(g, 3, [-1/16;1/16]);
> dirtyinfnorm(p1 - g, [-1/16;1/16]);
9.8841323829271038137685646777951687620288462194745e-8
> dirtyinfnorm(p2 - g, [-1/16;1/16]);
2.54337800593461418356437401152248866818783932027105e-9
\end{Verbatim}
\end{minipage}\end{center}
\noindent Example 5: 
\begin{center}\begin{minipage}{15cm}\begin{Verbatim}[frame=single]
> f = sin(x);
> I = [-3b-5;-1b-1074];
> time(popt = remez(1, [|1, 3, 4, 5, 7, 8, 9|], I, 1/f));
0.56320499999999999989758192597832930914591997861862
> time(p1 = remez(1, [|1, 3, 4, 5, 7, 8, 9|], I, 1/f, default, [0, 1b-73]));
0.354154999999999999969399477883769122854573652148247
> time(p2 = remez(1, [|1, 3, 4, 5, 7, 8, 9|], I, 1/f, default, [3b-72, +@Inf@]))
;
0.448221999999999999976865727724373300588922575116158
> dirtyinfnorm(popt/f-1, I);
2.06750931454112835098093903810531156576504665659064e-22
> dirtyinfnorm(p1/f-1, I);
2.49711266837493110470637913808914046704452778960875e-22
> dirtyinfnorm(p2/f-1, I);
5.4567247553615435246376977231253834265248756996947e-22
> 1b-73;
1.05879118406787542383540312584955245256423950195312e-22
> 3b-72;
6.3527471044072525430124187550973147153854370117187e-22
\end{Verbatim}
\end{minipage}\end{center}
See also: \textbf{dirtyinfnorm} (\ref{labdirtyinfnorm}), \textbf{infnorm} (\ref{labinfnorm}), \textbf{fpminimax} (\ref{labfpminimax}), \textbf{guessdegree} (\ref{labguessdegree}), \textbf{taylorform} (\ref{labtaylorform}), \textbf{taylor} (\ref{labtaylor})

\subsection{rename}
\label{labrename}
\noindent Name: \textbf{rename}\\
rename the free variable.\\

\noindent Usage: 
\begin{center}
\textbf{rename}(\emph{ident1},\emph{ident2}) : \textsf{void}\\
\end{center}
Parameters: 
\begin{itemize}
\item \emph{ident1} is the current name of the free variable.
\item \emph{ident2} is a fresh name.
\end{itemize}
\noindent Description: \begin{itemize}

\item \textbf{rename} lets one change the name of the free variable. \sollya can handle only
   one free variable at a time. The first time in a session that an unbound name 
   is used in a context where it can be interpreted as a free variable, the name
   is used to represent the free variable of \sollya. In the following, this name
   can be changed using \textbf{rename}.

\item Be careful: if \emph{ident2} has been set before, its value will be lost. Use the 
   command \textbf{isbound} to know if \emph{ident2} is already used or not.

\item If \emph{ident1} is not the current name of the free variable, an error occurs.

\item If \textbf{rename} is used at a time when the name of the free variable has not been 
   defined, \emph{ident1} is just ignored and the name of the free variable is 
   set to \emph{ident2}.
\end{itemize}
\noindent Example 1: 
\begin{center}\begin{minipage}{15cm}\begin{Verbatim}[frame=single]
> f=sin(x);
> f;
sin(x)
> rename(x,y);
> f;
sin(y)
\end{Verbatim}
\end{minipage}\end{center}
\noindent Example 2: 
\begin{center}\begin{minipage}{15cm}\begin{Verbatim}[frame=single]
> a=1;
> f=sin(x);
> rename(x,a);
> a;
a
> f;
sin(a)
\end{Verbatim}
\end{minipage}\end{center}
\noindent Example 3: 
\begin{center}\begin{minipage}{15cm}\begin{Verbatim}[frame=single]
> verbosity=1!;
> f=sin(x);
> rename(y,z);
Warning: the current free variable is named "x" and not "y". Can only rename the
 free variable.
The last command will have no effect.
\end{Verbatim}
\end{minipage}\end{center}
\noindent Example 4: 
\begin{center}\begin{minipage}{15cm}\begin{Verbatim}[frame=single]
> rename(x,y);
> isbound(x);
false
> isbound(y);
true
\end{Verbatim}
\end{minipage}\end{center}
See also: \textbf{isbound} (\ref{labisbound})

\subsection{restart}
\label{labrestart}
\noindent Name: \textbf{restart}\\
brings Sollya back to its initial state\\

\noindent Usage: 
\begin{center}
\textbf{restart} : \textsf{void} $\rightarrow$ \textsf{void}\\
\end{center}
\noindent Description: \begin{itemize}

\item The command \textbf{restart} brings Sollya back to its initial state.  All
   current state is abandoned, all libraries unbound and all memory freed.
   The \textbf{restart} command has no effect when executed inside a Sollya
   script read into a main Sollya script using \textbf{execute}. It is executed
   in a Sollya script included by a $\#$include macro.
   Using the \textbf{restart} command in nested elements of imperative
   programming like for or while loops is possible. Since in most cases
   abandoning the current state of Sollya means altering a loop
   invariant, warnings of the impossibility of continuing a loop may
   follow unless the state is rebuilt.
\end{itemize}
\noindent Example 1: 
\begin{center}\begin{minipage}{15cm}\begin{Verbatim}[frame=single]
> print(exp(x));
exp(x)
> a = 3;
> restart;
The tool has been restarted.
> print(x);
x
> a;
Warning: the identifier "a" is neither assigned to, nor bound to a library funct
ion nor equal to the current free variable.
Will interpret "a" as "x".
x
\end{Verbatim}
\end{minipage}\end{center}
\noindent Example 2: 
\begin{center}\begin{minipage}{15cm}\begin{Verbatim}[frame=single]
> print(exp(x));
exp(x)
> for i from 1 to 10 do {
>     print(i);
>     if (i == 5) then restart;
> };
1
2
3
4
5
The tool has been restarted.
Warning: the tool has been restarted inside a for loop.
The for loop will no longer be executed.
\end{Verbatim}
\end{minipage}\end{center}
\noindent Example 3: 
\begin{center}\begin{minipage}{15cm}\begin{Verbatim}[frame=single]
> print(exp(x));
exp(x)
> a = 3;
> for i from 1 to 10 do {
>     print(i);
>     if (i == 5) then {
>         restart;
>         i = 7;
>     };
> };
1
2
3
4
5
The tool has been restarted.
8
9
10
> print(x);
x
> a;
Warning: the identifier "a" is neither assigned to, nor bound to a library funct
ion nor equal to the current free variable.
Will interpret "a" as "x".
x
\end{Verbatim}
\end{minipage}\end{center}
See also: \textbf{quit} (\ref{labquit}), \textbf{execute} (\ref{labexecute})

\subsection{return}
\label{labreturn}
\noindent Name: \textbf{return}\\
indicates an expression to be returned in a procedure\\

\noindent Usage: 
\begin{center}
\textbf{return} \emph{expression} : \textsf{void}\\
\end{center}
Parameters: 
\begin{itemize}
\item \emph{expression} represents the expression to be returned
\end{itemize}
\noindent Description: \begin{itemize}

\item The keyword \textbf{return} allows for returning the (evaluated) expression
   \emph{expression} at the end of a begin-end-block ({}-block) used as a
   Sollya procedure body. See \textbf{proc} for further details concerning
   Sollya procedure definitions.
    
   Statements for returning expressions using \textbf{return} are only possible
    at the end of a begin-end-block used as a Sollya procedure
    body. Only one \textbf{return} statement can be given per begin-end-block.

\item If at the end of a procedure definition using \textbf{proc} no \textbf{return}
   statement is given, a \textbf{return} \textbf{void} statement is implicitely
   added. Procedures, i.e. procedure objects, when printed out in Sollya
   defined with an implicit \textbf{return} \textbf{void} statement are displayed with
   this statement explicited.
\end{itemize}
\noindent Example 1: 
\begin{center}\begin{minipage}{15cm}\begin{Verbatim}[frame=single]
> succ = proc(n) { var res; res := n + 1; return res; };
> succ(5);
6
> succ;
proc(n)
begin
var res;
res := (n) + (1);
return res;
end
\end{Verbatim}
\end{minipage}\end{center}
\noindent Example 2: 
\begin{center}\begin{minipage}{15cm}\begin{Verbatim}[frame=single]
> hey = proc(s) { print("Hello",s); };
> hey("world");
Hello world
> hey;
proc(s)
begin
print("Hello", s);
return void;
end
\end{Verbatim}
\end{minipage}\end{center}
See also: \textbf{proc} (\ref{labproc}), \textbf{void} (\ref{labvoid})

\subsection{revert}
\label{labrevert}
\noindent Name: \textbf{revert}\\
reverts a list.\\

\noindent Usage: 
\begin{center}
\textbf{revert}(\emph{L}) : \textsf{list} $\rightarrow$ \textsf{list}\\
\end{center}
Parameters: 
\begin{itemize}
\item \emph{L} is a list.
\end{itemize}
\noindent Description: \begin{itemize}

\item \textbf{revert}(\emph{L}) returns the same list, but with its elements in reverse order.

\item If \emph{L} is an end-elliptic list, \textbf{revert} will fail with an error.
\end{itemize}
\noindent Example 1: 
\begin{center}\begin{minipage}{15cm}\begin{Verbatim}[frame=single]
> revert([| |]);
[| |]
\end{Verbatim}
\end{minipage}\end{center}
\noindent Example 2: 
\begin{center}\begin{minipage}{15cm}\begin{Verbatim}[frame=single]
> revert([|2,3,5,2,1,4|]);
[|4, 1, 2, 5, 3, 2|]
\end{Verbatim}
\end{minipage}\end{center}

\subsection{rn}
\label{labrn}
\noindent Name: \textbf{RN}\\
constant representing rounding-to-nearest mode.\\

\noindent Description: \begin{itemize}

\item \textbf{RN} is used in command \textbf{round} to specify that the value must be rounded
   to the nearest representable floating-point number.
\end{itemize}
\noindent Example 1: 
\begin{center}\begin{minipage}{15cm}\begin{Verbatim}[frame=single]
> display=binary!;
> round(Pi,20,RN);
1.100100100001111111_2 * 2^(1)
\end{Verbatim}
\end{minipage}\end{center}
See also: \textbf{RD} (\ref{labrd}), \textbf{RU} (\ref{labru}), \textbf{RZ} (\ref{labrz}), \textbf{round} (\ref{labround})

\subsection{round}
\label{labround}
\noindent Name: \textbf{round}\\
rounds a number to a floating-point format.\\
\noindent Usage: 
\begin{center}
\textbf{round}(\emph{x},\emph{n},\emph{mode}) : (\textsf{constant}, \textsf{integer}, \textbf{RD} $|$ \textbf{RU} $|$ \textbf{RN} $|$ \textbf{RZ}) $\rightarrow$ \textsf{constant}\\
\textbf{round}(\emph{x},\emph{format},\emph{mode}) : (\textsf{constant}, \textsf{D$|$double$|$DD$|$doubledouble$|$DE$|$doubleextended$|$TD$|$tripledouble}, \textbf{RD} $|$ \textbf{RU} $|$ \textbf{RN} $|$ \textbf{RZ}) $\rightarrow$ \textsf{constant}\\
\end{center}
Parameters: 
\begin{itemize}
\item \emph{x} is a constant to be rounded.
\item \emph{n} is the precision of the target format.
\item \emph{format} is the name of a supported floating-point format.
\item \emph{mode} is the desired rounding mode.
\end{itemize}
\noindent Description: \begin{itemize}

\item If used with an integer parameter \\emph{n}, \\textbf{round}(\\emph{x},\\emph{n},\\emph{mode}) rounds \\emph{x} to a floating-point number with \n   precision \\emph{n}, according to rounding-mode \\emph{mode}. \n
\item If used with a format parameter \\emph{format}, \\textbf{round}(\\emph{x},\\emph{format},\\emph{mode}) rounds \\emph{x} to a floating-point number in the \n   floating-point format \\emph{format}, according to rounding-mode \\emph{mode}. \n
\item Subnormal numbers are not handled are handled only if a \\emph{format} parameter is given\n   that is different from \\textbf{doubleextended}. The range of possible exponents is the \n   range used for all numbers represented in \\sollya (e.g. basically the range \n   used in the library MPFR). \n\end{itemize}
\noindent Example 1: 
\begin{center}\begin{minipage}{15cm}\begin{Verbatim}[frame=single]
\end{Verbatim}
\end{minipage}\end{center}
\noindent Example 2: 
\begin{center}\begin{minipage}{15cm}\begin{Verbatim}[frame=single]
\end{Verbatim}
\end{minipage}\end{center}
\noindent Example 3: 
\begin{center}\begin{minipage}{15cm}\begin{Verbatim}[frame=single]
\end{Verbatim}
\end{minipage}\end{center}
See also: \textbf{RN} (\ref{labrn}), \textbf{RD} (\ref{labrd}), \textbf{RU} (\ref{labru}), \textbf{RZ} (\ref{labrz}), \textbf{single} (\ref{labsingle}), \textbf{double} (\ref{labdouble}), \textbf{doubleextended} (\ref{labdoubleextended}), \textbf{doubledouble} (\ref{labdoubledouble}), \textbf{tripledouble} (\ref{labtripledouble}), \textbf{roundcoefficients} (\ref{labroundcoefficients}), \textbf{roundcorrectly} (\ref{labroundcorrectly}), \textbf{printhexa} (\ref{labprinthexa}), \textbf{printfloat} (\ref{labprintfloat})

\subsection{ roundcoefficients }
\noindent Name: \textbf{roundcoefficients}\\
rounds the coefficients of a polynomial to classical formats.\\

\noindent Usage: 
\begin{center}
\textbf{roundcoefficients}(\emph{p},\emph{L}) : (\textsf{function}, \textsf{list}) $\rightarrow$ \textsf{function}\\
\end{center}
Parameters: 
\emph{p} is a function. Usually a polynomial.\\
\emph{L} is a list of formats.\\

\noindent Description: \begin{itemize}

\item If \emph{p} is a polynomial and \emph{L} a list of floating-point formats, 
   \textbf{roundcoefficients}(\emph{p},\emph{L}) rounds each coefficient of \emph{p} to the corresponding format
   in \emph{L}.

\item If \emph{p} is not a polynomial, \textbf{roundcoefficients} does not do anything.

\item If \emph{L} contains other elements than \textbf{D}, \textbf{double}, \textbf{DD}, \textbf{doubledouble}, \textbf{TD} and
   \textbf{tripledouble}, an error occurs.

\item The coefficients in \emph{p} corresponding to $X^i$ is rounded to the 
   format L[i]. If \emph{L} does not contain enough elements
   (e.g. if \textbf{length}(L) < \textbf{degree}(p)+1), a warning is displayed. However, the
   coefficients corresponding to an element of \emph{L} are rounded. The last 
   coefficients (that do not have a corresponding element in \emph{L}) are kept with
   their own precision.
   If \emph{L} contains too much elements, the last useless elements are ignored.
   In particular \emph{L} may be end-elliptic in which case \textbf{roundcoefficients} has the 
   natural behavior.
\end{itemize}
\noindent Example 1: 
\begin{center}\begin{minipage}{14.8cm}\begin{Verbatim}[frame=single]
   > p=exp(1) + x*(exp(2) + x*exp(3));
   > display=binary!;
   > roundcoefficients(p,[|DD,D,D|]);
   1.010110111111000010101000101100010100010101110110100101010011010101011111101110001010110001000000010011101_2 * 2^(1) + x * (1.110110001110011001001011100011010100110111011010111_2 * 2^(2) + x * 1.010000010101111001011011111101101111101100010000011_2 * 2^(4))
   > roundcoefficients(p,[|DD,D...|]);
   1.010110111111000010101000101100010100010101110110100101010011010101011111101110001010110001000000010011101_2 * 2^(1) + x * (1.110110001110011001001011100011010100110111011010111_2 * 2^(2) + x * 1.010000010101111001011011111101101111101100010000011_2 * 2^(4))
\end{Verbatim}
\end{minipage}\end{center}
\noindent Example 2: 
\begin{center}\begin{minipage}{14.8cm}\begin{Verbatim}[frame=single]
   > f=sin(exp(1)*x);
   > display=binary!;
   > f;
   sin(x * 1.0101101111110000101010001011000101000101011101101001010100110101010111111011100010101100010000000100111001111010011110011110001110110001011100111000101100000111101_2 * 2^(1))
   > roundcoefficients(f,[|D...|]);
   sin(x * 1.0101101111110000101010001011000101000101011101101001010100110101010111111011100010101100010000000100111001111010011110011110001110110001011100111000101100000111101_2 * 2^(1))
\end{Verbatim}
\end{minipage}\end{center}
\noindent Example 3: 
\begin{center}\begin{minipage}{14.8cm}\begin{Verbatim}[frame=single]
   > p=exp(1) + x*(exp(2) + x*exp(3));
   > verbosity=1!;
   > display=binary!;
   > roundcoefficients(p,[|DD,D|]);
   Warning: the number of the given formats does not correspond to the degree of the given polynomial.
   Warning: the 0th coefficient of the given polynomial does not evaluate to a floating-point constant without any rounding.
   Will evaluate the coefficient in the current precision in floating-point before rounding to the target format.
   Warning: the 1th coefficient of the given polynomial does not evaluate to a floating-point constant without any rounding.
   Will evaluate the coefficient in the current precision in floating-point before rounding to the target format.
   Warning: rounding may have happened.
   1.010110111111000010101000101100010100010101110110100101010011010101011111101110001010110001000000010011101_2 * 2^(1) + x * (1.110110001110011001001011100011010100110111011010111_2 * 2^(2) + x * 1.01000001010111100101101111110110111110110001000001011111001011010100101111011111110001010011011101000100110000111010001110010000010110000101100000111001011100101001_2 * 2^(4))
\end{Verbatim}
\end{minipage}\end{center}
See also: \textbf{double}, \textbf{doubledouble}, \textbf{tripledouble}

\subsection{roundcorrectly}
\label{labroundcorrectly}
\noindent Name: \textbf{roundcorrectly}\\
rounds an approximation range correctly to some precision\\
\noindent Usage: 
\begin{center}
\textbf{roundcorrectly}(\emph{range}) : \textsf{range} $\rightarrow$ \textsf{constant}
\end{center}
Parameters: 
\begin{itemize}
\item \emph{range} represents a range in which an exact value lies
\end{itemize}
\noindent Description: \begin{itemize}

\item Let \emph{range} be a range of values, determined by some approximation
   process, safely bounding an unknown value $v$. The command
   \textbf{roundcorrectly}(\emph{range}) determines a precision such that for this precision,
   rounding to the nearest any value in \emph{range} yields to the same
   result, i.e. to the correct rounding of $v$.
    
   If no such precision exists, a warning is displayed and \textbf{roundcorrectly}
   evaluates to NaN.
\end{itemize}
\noindent Example 1: 
\begin{center}\begin{minipage}{15cm}\begin{Verbatim}[frame=single]
> printbinary(roundcorrectly([1.010001_2; 1.0101_2]));
1.01_2
> printbinary(roundcorrectly([1.00001_2; 1.001_2]));
1_2
\end{Verbatim}
\end{minipage}\end{center}
\noindent Example 2: 
\begin{center}\begin{minipage}{15cm}\begin{Verbatim}[frame=single]
> roundcorrectly([-1; 1]);
@NaN@
\end{Verbatim}
\end{minipage}\end{center}
See also: \textbf{round} (\ref{labround})

\subsection{suppresswarnings}
\label{labsuppresswarnings}
\noindent Name: \textbf{suppressroundingwarnings}\\
global variable controlling whether or not there is a warning when roundings occur.\\

\noindent Description: \begin{itemize}

\item \textbf{suppressroundingwarnings} is a global variable. When its value is \textbf{off}, warnings are
   emitted in appropriate verbosity modes (see \textbf{verbosity}) when roundings
   occur.  When its value is \textbf{on}, these warnings are suppressed.

\item This mode depends on a verbosity of at least 1. See
   \textbf{verbosity} for more details.

\item Default is \textbf{off} when the standard input is a terminal and
   \textbf{on} when Sollya input is read from a file.
\end{itemize}
\noindent Example 1: 
\begin{center}\begin{minipage}{15cm}\begin{Verbatim}[frame=single]
> suppressroundingwarnings = off;
Rounding warning suppression has been deactivated.
> exp(0.1);
0.110517091807564762481170782649024666822454719473754e1
> suppressroundingwarnings = on;
Rounding warning suppression has been activated.
> exp(0.1);
0.110517091807564762481170782649024666822454719473754e1
\end{Verbatim}
\end{minipage}\end{center}
See also: \textbf{on} (\ref{labon}), \textbf{off} (\ref{laboff}), \textbf{verbosity} (\ref{labverbosity}), \textbf{midpointmode} (\ref{labmidpointmode})

\subsection{RU}
\label{labru}
\noindent Name: \textbf{RU}\\
constant representing rounding-upwards mode.\\
\noindent Description: \begin{itemize}

\item \\textbf{RU} is used in command \\textbf{round} to specify that the value $x$ must be rounded\n   to the smallest floating-point number $y$ such that $x \\le y$.\n\end{itemize}
\noindent Example 1: 
\begin{center}\begin{minipage}{15cm}\begin{Verbatim}[frame=single]
\end{Verbatim}
\end{minipage}\end{center}
See also: \textbf{RZ} (\ref{labrz}), \textbf{RD} (\ref{labrd}), \textbf{RN} (\ref{labrn}), \textbf{round} (\ref{labround})

\subsection{rz}
\label{labrz}
\noindent Name: \textbf{RZ}\\
constant representing rounding-to-zero mode.\\
\noindent Description: \begin{itemize}

\item \textbf{RZ} is used in command \textbf{round} to specify that the value must be rounded
   to the closest floating-point number towards zero. It just consists in 
   truncating the value to the desired format.
\end{itemize}
\noindent Example 1: 
\begin{center}\begin{minipage}{15cm}\begin{Verbatim}[frame=single]
> display=binary!;
> round(Pi,20,RZ);
1.1001001000011111101_2 * 2^(1)
\end{Verbatim}
\end{minipage}\end{center}
See also: \textbf{RD} (\ref{labrd}), \textbf{RU} (\ref{labru}), \textbf{RN} (\ref{labrn}), \textbf{round} (\ref{labround})

\subsection{searchgal}
\label{labsearchgal}
\noindent Name: \textbf{searchgal}\\
searches for a preimage of a function such that the rounding the image commits an error smaller than a constant\\

\noindent Usage: 
\begin{center}
\textbf{searchgal}(\emph{function}, \emph{start}, \emph{preimage precision}, \emph{steps}, \emph{format}, \emph{error bound}) : (\textsf{function}, \textsf{constant}, \textsf{integer}, \textsf{integer}, \textsf{D$|$double$|$DD$|$doubledouble$|$DE$|$doubleextended$|$TD$|$tripledouble}, \textsf{constant}) $\rightarrow$ \textsf{list}\\
\textbf{searchgal}(\emph{list of functions}, \emph{start}, \emph{preimage precision}, \emph{steps}, \emph{list of format}, \emph{list of error bounds}) : (\textsf{list}, \textsf{constant}, \textsf{integer}, \textsf{integer}, \textsf{list}, \textsf{list}) $\rightarrow$ \textsf{list}\\
\end{center}
Parameters: 
\begin{itemize}
\item \emph{function} represents the function to be considered
\item \emph{start} represents a value around which the search is to be performed
\item \emph{preimage precision} represents the precision (discretization) for the eligible preimage values
\item \emph{steps} represents the log2 of the number of search steps to be performed
\item \emph{format} represents the format the image of the function is to be rounded to
\item \emph{error bound} represents a upper bound on the relative rounding error when rounding the image
\item \emph{list of functions} represents the functions to be considered
\item \emph{list of formats} represents the respective formats the images of the functions are to be rounded to
\item \emph{list of error bounds} represents a upper bound on the relative rounding error when rounding the image
\end{itemize}
\noindent Description: \begin{itemize}

\item The command \textbf{searchgal} searches for a preimage $z$ of a function
   \emph{function} or a list of functions \emph{list of functions} such that
   $z$ is a floating-point number with \emph{preimage precision}
   significant mantissa bits and the image $y$ of the function,
   respectively each image $y_i$ of the functions, rounds to
   format \emph{format} respectively to the corresponding format in \emph{list of format} 
   with a relative rounding error less than \emph{error bound}
   respectively the corresponding value in \emph{list of error bounds}. During
   this search, at most 2 raised to \emph{steps} attempts are made. The search
   starts with a preimage value equal to \emph{start}. This value is then
   increased and decreased by $1$ ulp in precision \emph{preimage precision} 
   until a value is found or the step limit is reached.
    
   If the search finds an appropriate preimage $z$, \textbf{searchgal}
   evaluates to a list containing this value. Otherwise, \textbf{searchgal}
   evaluates to an empty list.
\end{itemize}
\noindent Example 1: 
\begin{center}\begin{minipage}{15cm}\begin{Verbatim}[frame=single]
> searchgal(log(x),2,53,15,DD,1b-112);
[| |]
> searchgal(log(x),2,53,18,DD,1b-112);
[|2.0000000000384972054234822280704975128173828125|]
\end{Verbatim}
\end{minipage}\end{center}
\noindent Example 2: 
\begin{center}\begin{minipage}{15cm}\begin{Verbatim}[frame=single]
> f = exp(x);
> s = searchgal(f,2,53,18,DD,1b-112);
> if (s != [||]) then {
>    v = s[0];
>    print("The rounding error is 2^(",evaluate(log2(abs(DD(f)/f - 1)),v),")");
> } else print("No value found");
The rounding error is 2^( -1.12106878438809380148206984258358542322113874177832e
2 )
\end{Verbatim}
\end{minipage}\end{center}
\noindent Example 3: 
\begin{center}\begin{minipage}{15cm}\begin{Verbatim}[frame=single]
> searchgal([|sin(x),cos(x)|],1,53,15,[|D,D|],[|1b-62,1b-60|]);
[|1.00000000000159494639717649988597258925437927246094|]
\end{Verbatim}
\end{minipage}\end{center}
See also: \textbf{round} (\ref{labround}), \textbf{double} (\ref{labdouble}), \textbf{doubledouble} (\ref{labdoubledouble}), \textbf{tripledouble} (\ref{labtripledouble}), \textbf{evaluate} (\ref{labevaluate}), \textbf{worstcase} (\ref{labworstcase})

\subsection{SG}
\label{labsg}
\noindent Name: \textbf{SG}\\
\phantom{aaa}short form for \textbf{single}\\[0.2cm]
See also: \textbf{single} (\ref{labsingle})

\subsection{simplify}
\label{labsimplify}
\noindent Name: \textbf{simplify}\\
simplifies an expression representing a function\\
\noindent Usage: 
\begin{center}
\textbf{simplify}(\emph{function}) : \textsf{function} $\rightarrow$ \textsf{function}
\\ 
\end{center}
Parameters: 
\begin{itemize}
\item \emph{function} represents the expression to be simplified
\end{itemize}
\noindent Description: \begin{itemize}

\item The command \textbf{simplify} simplifies constant subexpressions of the
   expression given in argument representing the function
   \emph{function}. Those constant subexpressions are evaluated in using
   floating-point arithmetic with the global precision \textbf{prec}.
\end{itemize}
\noindent Example 1: 
\begin{center}\begin{minipage}{15cm}\begin{Verbatim}[frame=single]
> print(simplify(sin(pi * x)));
sin(3.14159265358979323846264338327950288419716939937508 * x)
> print(simplify(erf(exp(3) + x * log(4))));
erf(2.00855369231876677409285296545817178969879078385544e1 + x * 1.3862943611198
906188344642429163531361510002687205)
\end{Verbatim}
\end{minipage}\end{center}
\noindent Example 2: 
\begin{center}\begin{minipage}{15cm}\begin{Verbatim}[frame=single]
> prec = 20!;
> t = erf(0.5);
> s = simplify(erf(0.5));
> prec = 200!;
> t;
0.5204998778130465376827466538919645287364515757579637000588058
> s;
0.52050018310546875
\end{Verbatim}
\end{minipage}\end{center}
See also: \textbf{simplifysafe} (\ref{labsimplifysafe}), \textbf{autosimplify} (\ref{labautosimplify}), \textbf{prec} (\ref{labprec}), \textbf{evaluate} (\ref{labevaluate})

\subsection{simplifysafe}
\label{labsimplifysafe}
\noindent Name: \textbf{simplifysafe}\\
simplifies an expression representing a function\\
\noindent Usage: 
\begin{center}
\textbf{simplifysafe}(\emph{function}) : \textsf{function} $\rightarrow$ \textsf{function}\\
\end{center}
Parameters: 
\begin{itemize}
\item \emph{function} represents the expression to be simplified
\end{itemize}
\noindent Description: \begin{itemize}

\item The command \\textbf{simplifysafe} simplifies the expression given in argument\n   representing the function \\emph{function}.  The command \\textbf{simplifysafe} does not\n   endanger the safety of computations even in \\sollya's floating-point\n   environment: the function returned is mathematically equal to the\n   function \\emph{function}. \n    \n   Remark that the simplification provided by \\textbf{simplifysafe} is not perfect:\n   they may exist simpler equivalent expressions for expressions returned\n   by \\textbf{simplifysafe}.\n\end{itemize}
\noindent Example 1: 
\begin{center}\begin{minipage}{15cm}\begin{Verbatim}[frame=single]
\end{Verbatim}
\end{minipage}\end{center}
\noindent Example 2: 
\begin{center}\begin{minipage}{15cm}\begin{Verbatim}[frame=single]
\end{Verbatim}
\end{minipage}\end{center}
\noindent Example 3: 
\begin{center}\begin{minipage}{15cm}\begin{Verbatim}[frame=single]
\end{Verbatim}
\end{minipage}\end{center}
See also: \textbf{simplify} (\ref{labsimplify}), \textbf{autosimplify} (\ref{labautosimplify}), \textbf{rationalmode} (\ref{labrationalmode})

\subsection{sin}
\label{labsin}
\noindent Name: \textbf{sin}\\
the sine function.\\

\noindent Description: \begin{itemize}

\item \textbf{sin} is the usual sine function.

\item It is defined for every real number x.
\end{itemize}
See also: \textbf{asin} (\ref{labasin}), \textbf{cos} (\ref{labcos}), \textbf{tan} (\ref{labtan})

\subsection{single}
\label{labsingle}
\noindent Names: \textbf{single}, \textbf{SG}\\
rounding to the nearest IEEE 754 single (binary32).\\
\noindent Description: \begin{itemize}

\item \textbf{single} is both a function and a constant.

\item As a function, it rounds its argument to the nearest IEEE 754 single precision (i.e. IEEE754-2008 binary32) number.
   Subnormal numbers are supported as well as standard numbers: it is the real
   rounding described in the standard.

\item As a constant, it symbolizes the single precision format. It is used in 
   contexts when a precision format is necessary, e.g. in the commands 
   \textbf{round} and \textbf{roundcoefficients}. In is not supported for \textbf{implementpoly}.
   See the corresponding help pages for examples.
\end{itemize}
\noindent Example 1: 
\begin{center}\begin{minipage}{15cm}\begin{Verbatim}[frame=single]
> display=binary!;
> SG(0.1);
1.10011001100110011001101_2 * 2^(-4)
> SG(4.17);
1.000010101110000101001_2 * 2^(2)
> SG(1.011_2 * 2^(-1073));
0
\end{Verbatim}
\end{minipage}\end{center}
See also: \textbf{double} (\ref{labdouble}), \textbf{doubleextended} (\ref{labdoubleextended}), \textbf{doubledouble} (\ref{labdoubledouble}), \textbf{tripledouble} (\ref{labtripledouble}), \textbf{roundcoefficients} (\ref{labroundcoefficients}), \textbf{implementpoly} (\ref{labimplementpoly}), \textbf{round} (\ref{labround}), \textbf{printfloat} (\ref{labprintfloat})

\subsection{sinh}
\label{labsinh}
\noindent Name: \textbf{sinh}\\
the hyperbolic sine function.\\

\noindent Description: \begin{itemize}

\item \textbf{sinh} is the usual hyperbolic sine function: $\sinh(x) = \frac{e^x - e^{-x}}{2}$.

\item It is defined for every real number x.
\end{itemize}
See also: \textbf{asinh} (\ref{labasinh}), \textbf{cosh} (\ref{labcosh}), \textbf{tanh} (\ref{labtanh})

\subsection{sort}
\label{labsort}
\noindent Name: \textbf{sort}\\
\phantom{aaa}sorts a list of real numbers.\\[0.2cm]
\noindent Library name:\\
\verb|   sollya_obj_t sollya_lib_sort(sollya_obj_t)|\\[0.2cm]
\noindent Usage: 
\begin{center}
\textbf{sort}(\emph{L}) : \textsf{list} $\rightarrow$ \textsf{list}\\
\end{center}
Parameters: 
\begin{itemize}
\item \emph{L} is a list.
\end{itemize}
\noindent Description: \begin{itemize}

\item If \emph{L} contains only constant values, \textbf{sort}(\emph{L}) returns the same list, but
   sorted in increasing order.

\item If \emph{L} contains at least one element that is not a constant, the command fails 
   with a type error.

\item If \emph{L} is an end-elliptic list, \textbf{sort} will fail with an error.
\end{itemize}
\noindent Example 1: 
\begin{center}\begin{minipage}{15cm}\begin{Verbatim}[frame=single]
> sort([| |]);
[| |]
> sort([|2,3,5,2,1,4|]);
[|1, 2, 2, 3, 4, 5|]
\end{Verbatim}
\end{minipage}\end{center}

\subsection{sqrt}
\label{labsqrt}
\noindent Name: \textbf{sqrt}\\
square root.\\
\noindent Description: \begin{itemize}

\item \textbf{sqrt} is the square root, e.g. the inverse of the function square: $\sqrt{y}$
   is the unique positive $x$ such that $x^2=y$.

\item It is defined only for $x$ in $[0;+\infty]$.
\end{itemize}

\subsection{string}
\label{labstring}
\noindent Name: \textbf{string}\\
\phantom{aaa}keyword representing a \textsf{string} type \\[0.2cm]
\noindent Usage: 
\begin{center}
\textbf{string} : \textsf{type type}\\
\end{center}
\noindent Description: \begin{itemize}

\item \textbf{string} represents the \textsf{string} type for declarations
   of external procedures by means of \textbf{externalproc}.
    
   Remark that in contrast to other indicators, type indicators like
   \textbf{string} cannot be handled outside the \textbf{externalproc} context.  In
   particular, they cannot be assigned to variables.
\end{itemize}
See also: \textbf{externalproc} (\ref{labexternalproc}), \textbf{boolean} (\ref{labboolean}), \textbf{constant} (\ref{labconstant}), \textbf{function} (\ref{labfunction}), \textbf{integer} (\ref{labinteger}), \textbf{list of} (\ref{lablistof}), \textbf{range} (\ref{labrange}), \textbf{object} (\ref{labobject})

\subsection{subpoly}
\label{labsubpoly}
\noindent Name: \textbf{subpoly}\\
\phantom{aaa}restricts the monomial basis of a polynomial to a list of monomials\\[0.2cm]
\noindent Library name:\\
\verb|   sollya_obj_t sollya_lib_subpoly(sollya_obj_t, sollya_obj_t)|\\[0.2cm]
\noindent Usage: 
\begin{center}
\textbf{subpoly}(\emph{polynomial}, \emph{list}) : (\textsf{function}, \textsf{list}) $\rightarrow$ \textsf{function}\\
\end{center}
Parameters: 
\begin{itemize}
\item \emph{polynomial} represents the polynomial the coefficients are taken from
\item \emph{list} represents the list of monomials to be taken
\end{itemize}
\noindent Description: \begin{itemize}

\item \textbf{subpoly} extracts the coefficients of a polynomial \emph{polynomial} and builds up a
   new polynomial out of those coefficients associated to monomial degrees figuring in
   the list \emph{list}. 
    
   If \emph{polynomial} represents a function that is not a polynomial, subpoly returns 0.
    
   If \emph{list} is a list that is end-elliptic, let be $j$ the last value explicitly specified
   in the list. All coefficients of the polynomial associated to monomials greater or
   equal to $j$ are taken.
\end{itemize}
\noindent Example 1: 
\begin{center}\begin{minipage}{15cm}\begin{Verbatim}[frame=single]
> p = taylor(exp(x),5,0);
> s = subpoly(p,[|1,3,5|]);
> print(p);
1 + x * (1 + x * (0.5 + x * (1 / 6 + x * (1 / 24 + x / 120))))
> print(s);
x * (1 + x^2 * (1 / 6 + x^2 / 120))
\end{Verbatim}
\end{minipage}\end{center}
\noindent Example 2: 
\begin{center}\begin{minipage}{15cm}\begin{Verbatim}[frame=single]
> p = remez(atan(x),10,[-1,1]);
> subpoly(p,[|1,3,5...|]);
x * (0.99986632941452949026018468446163586361700915018231 + x^2 * (-0.3303047850
2455936362667794059988443130926433421739 + x^2 * (0.1801592931781875646289423703
7824735129130095574422 + x * (2.284558411542478828511250156535857664242985696307
19e-9 + x * (-8.5156349064111377895500552996061844977507560037484e-2 + x * (-2.7
1756340962775019916818769239340943524383018921799e-9 + x * (2.084511343071147293
73239910549169872454686955894998e-2 + x * 1.108898611811290576571996643868266300
81793400489512e-9)))))))
\end{Verbatim}
\end{minipage}\end{center}
\noindent Example 3: 
\begin{center}\begin{minipage}{15cm}\begin{Verbatim}[frame=single]
> subpoly(exp(x),[|1,2,3|]);
0
\end{Verbatim}
\end{minipage}\end{center}
See also: \textbf{roundcoefficients} (\ref{labroundcoefficients}), \textbf{taylor} (\ref{labtaylor}), \textbf{remez} (\ref{labremez}), \textbf{fpminimax} (\ref{labfpminimax}), \textbf{implementpoly} (\ref{labimplementpoly})

\subsection{substitute}
\label{labsubstitute}
\noindent Name: \textbf{substitute}\\
replace the occurences of the free variable in an expression.\\

\noindent Usage: 
\begin{center}
\textbf{substitute}(\emph{f},\emph{g}) : (\textsf{function}, \textsf{function}) $\rightarrow$ \textsf{function}\\
\textbf{substitute}(\emph{f},\emph{t}) : (\textsf{function}, \textsf{constant}) $\rightarrow$ \textsf{constant}\\
\end{center}
Parameters: 
\begin{itemize}
\item \emph{f} is a function.
\item \emph{g} is a function.
\item \emph{t} is a real number.
\end{itemize}
\noindent Description: \begin{itemize}

\item \textbf{substitute}(\emph{f}, \emph{g}) produces the function $(f \circ g) : x \mapsto f(g(x))$.

\item \textbf{substitute}(\emph{f}, \emph{t}) is the constant $f(t)$. Note that the constant is
   represented by its expression until it has been evaluated (exactly the same
   way as if you type the expression \emph{f} replacing instances of the free variable 
   by \emph{t}).

\item If \emph{f} is stored in a variable \emph{F}. It is absolutely equivalent to 
   writing \emph{F(g)} or \emph{F(t)}.
\end{itemize}
\noindent Example 1: 
\begin{center}\begin{minipage}{15cm}\begin{Verbatim}[frame=single]
> f=sin(x);
> g=cos(x);
> substitute(f,g);
sin(cos(x))
> f(g);
sin(cos(x))
\end{Verbatim}
\end{minipage}\end{center}
\noindent Example 2: 
\begin{center}\begin{minipage}{15cm}\begin{Verbatim}[frame=single]
> a=1;
> f=sin(x);
> substitute(f,a);
0.841470984807896506652502321630298999622563060798373
> f(a);
0.841470984807896506652502321630298999622563060798373
\end{Verbatim}
\end{minipage}\end{center}

\subsection{sup}
\label{labsup}
\noindent Name: \textbf{sup}\\
gives the upper bound of an interval.\\
\noindent Usage: 
\begin{center}
\textbf{sup}(\emph{I}) : \textsf{range} $\rightarrow$ \textsf{constant}
\textbf{sup}(\emph{x}) : \textsf{constant} $\rightarrow$ \textsf{constant}
\end{center}
Parameters: 
\begin{itemize}
\item \emph{I} is an interval.
\item \emph{x} is a real number.
\end{itemize}
\noindent Description: \begin{itemize}

\item Returns the upper bound of the interval \emph{I}. Each bound of an interval has its 
   own precision, so this command is exact, even if the current precision is too 
   small to represent the bound.

\item When called on a real number \emph{x}, \textbf{sup} considers it as an interval formed
   of a single point: [x, x]. In other words, \textbf{sup} behaves like the identity.
\end{itemize}
\noindent Example 1: 
\begin{center}\begin{minipage}{15cm}\begin{Verbatim}[frame=single]
> sup([1;3]);
3
> sup(5);
5
\end{Verbatim}
\end{minipage}\end{center}
\noindent Example 2: 
\begin{center}\begin{minipage}{15cm}\begin{Verbatim}[frame=single]
> display=binary!;
> I=[0; 0.111110000011111_2];
> sup(I);
1.11110000011111_2 * 2^(-1)
> prec=12!;
> sup(I);
1.11110000011111_2 * 2^(-1)
\end{Verbatim}
\end{minipage}\end{center}
See also: \textbf{inf} (\ref{labinf}), \textbf{mid} (\ref{labmid})

\subsection{supnorm}
\label{labsupnorm}
\noindent Name: \textbf{supnorm}\\
computes an interval bounding the supremum norm of an approximation error (absolute or relative) between a given polynomial and a function.\\
\noindent Usage: 
\begin{center}
\textbf{supnorm}(\emph{p}, \emph{f}, \emph{I}, \emph{errorType}, \emph{accuracy}) : (\textsf{function}, \textsf{function}, \textsf{range}, \textsf{absolute$|$relative}, \textsf{constant}) $\rightarrow$ \textsf{range}\\
\end{center}
Parameters: 
\begin{itemize}
\item \emph{p} is a polynomial.
\item \emph{f} is a function.
\item \emph{I} is an interval.
\item \emph{errorType} is the type of error to be considered: \textbf{absolute} or \textbf{relative} (see details below).
\item \emph{accuracy} is a constant that controls the relative tightness of the interval returned. 
\end{itemize}
\noindent Description: \begin{itemize}

\item \textbf{supnorm}(\emph{p}, \emph{f}, \emph{I}, \emph{errorType}) computes an interval bound $r = [\ell,\,u]$ for the supremum norm of the error function $\varepsilon_{absolute}=p-f$ or $\varepsilon_{relative}=p/f-1$ (depending on errorType), over the interval $I$, s.t.  $\max_{x \in I} \{|\varepsilon(x)|\} \subseteq r$ and  $0 \le \frac{u-\ell}{\ell} \le$ accuracy. If the interval $r$ is returned, it is guaranteed to contain the supremum norm value and to satisfy the required quality. In some rare cases, roughly speaking if the function is too complicated, our algorithm will simply fail, but it never lies; a corresponing error message is given. 

\item The algorithm used for this command is quite complex to be explained here. 
   Please find a complete description in the following article:\\
      Sylvain Chevillard, John Harrison, Mioara Joldes, Christoph Lauter\\
      Efficient and accurate computation of upper bounds of approximation errors\\
      Journal of Theoretical Computer Science (TCS)(2010)\\
      LIP Research Report number RRLIP2010-2\\
      http://prunel.ccsd.cnrs.fr/ensl-00445343/fr/\\

\item Roughly speaking, \textbf{supnorm} is based on using \textbf{taylorform} to obtain a higher degree polynomial approximation for \emph{f}. Since \textbf{taylorform} does not guarantee that by increasing the degree of the aproximation, the remainder bound will become smaller, we can not guarantee that the supremum norm can get as accurate as desired by the user. However, by splitting the initial interval \emph{I} a potential better eclosure of the remainder is obtained. We do the splitting until the remainder is sufficiently small or the width of the resulting interval is too small, depending on the global variable \textbf{diam}. An approximation of at most $16$ times the degree of \emph{p} is considered for each interval.

\item In practical cases, the algorithm should be able to automatically handle removable discontinuities that relative errors might have. This means that usually, if \emph{f} vanishes at a point $x_0$ in the interval considered, the approximation polynomial \emph{p} is designed such that it vanishes also at the same point with a multiplicity large enough. Hence, although \emph{f} vanishes, $\varepsilon_{relative}=p/f-1$ may be defined by continuous extension at such points $x_0$, called removable discontinuities (see Example $3$).
\end{itemize}
\noindent Example 1: 
\begin{center}\begin{minipage}{15cm}\begin{Verbatim}[frame=single]
> p = remez(exp(x), 5, [-1;1]);
> midpointmode=on!;
> supnorm(p, exp(x), [-1;1], absolute, 2^(-40));
0.452055210438~2/7~e-4
\end{Verbatim}
\end{minipage}\end{center}
\noindent Example 2: 
\begin{center}\begin{minipage}{15cm}\begin{Verbatim}[frame=single]
> prec=200!;
> d = [1;2];
> f = exp(cos(x)^2 + 1);
> p = remez(1,15,d,1/f,1e-40);
> theta=1b-60;
> prec=default!;
> mode=relative;
> supnorm(p,f,d,mode,theta);
[3.0893006200251428571621794052259682827831072375319e-14;3.089300620025142859757
9848990873269730291252477793e-14]
\end{Verbatim}
\end{minipage}\end{center}
\noindent Example 3: 
\begin{center}\begin{minipage}{15cm}\begin{Verbatim}[frame=single]
> mode=relative;
> theta=1b-135;
> d = [-1b-2;1b-2];
> f = expm1(x);
> p = x * (1 +  x * ( 2097145 * 2^(-22) + x * ( 349527 * 2^(-21) + x * (87609 * 
2^(-21) + x * 4369 * 2^(-19))))); 
> theta=1b-40;
> supnorm(p,f,d,mode,theta);
[9.8349131972210814932434890154829866537600224773996e-8;9.8349131972297467696435
93575914432178700294360461e-8]
\end{Verbatim}
\end{minipage}\end{center}
See also: \textbf{dirtyinfnorm} (\ref{labdirtyinfnorm}), \textbf{infnorm} (\ref{labinfnorm}), \textbf{checkinfnorm} (\ref{labcheckinfnorm}), \textbf{absolute} (\ref{lababsolute}), \textbf{relative} (\ref{labrelative}), \textbf{taylorform} (\ref{labtaylorform}), \textbf{autodiff} (\ref{labautodiff}), \textbf{numberroots} (\ref{labnumberroots}), \textbf{diam} (\ref{labdiam})

\subsection{tail}
\label{labtail}
\noindent Name: \textbf{tail}\\
gives the tail of a list.\\

\noindent Usage: 
\begin{center}
\textbf{tail}(\emph{L}) : \textsf{list} $\rightarrow$ \textsf{list}\\
\end{center}
Parameters: 
\begin{itemize}
\item \emph{L} is a list.
\end{itemize}
\noindent Description: \begin{itemize}

\item \textbf{tail}(\emph{L}) returns the list \emph{L} without its first element.

\item If \emph{L} is empty, the command will fail with an error.

\item \textbf{tail} can also be used with end-elliptic lists. In this case, the result of
   \textbf{tail} is also an end-elliptic list.
\end{itemize}
\noindent Example 1: 
\begin{center}\begin{minipage}{15cm}\begin{Verbatim}[frame=single]
> tail([|1,2,3|]);
[|2, 3|]
> tail([|1,2...|]);
[|2...|]
\end{Verbatim}
\end{minipage}\end{center}
See also: \textbf{head} (\ref{labhead})

\subsection{tan}
\label{labtan}
\noindent Name: \textbf{tan}\\
\phantom{aaa}the tangent function.\\[0.2cm]
\noindent Library names:\\
\verb|   sollya_obj_t sollya_lib_tan(sollya_obj_t)|\\
\verb|   sollya_obj_t sollya_lib_build_function_tan(sollya_obj_t)|\\
\verb|   #define SOLLYA_TAN(x) sollya_lib_build_function_tan(x)|\\[0.2cm]
\noindent Description: \begin{itemize}

\item \textbf{tan} is the tangent function, defined by $\tan(x) = \sin(x)/\cos(x)$.

\item It is defined for every real number $x$ that is not of the form $n\pi + \pi/2$ where $n$ is an integer.
\end{itemize}
See also: \textbf{atan} (\ref{labatan}), \textbf{cos} (\ref{labcos}), \textbf{sin} (\ref{labsin})

\subsection{tanh}
\label{labtanh}
\noindent Name: \textbf{tanh}\\
the hyperbolic tangent function.\\
\noindent Description: \begin{itemize}

\item \textbf{tanh} is the hyperbolic tangent function, defined by $\tanh(x) = \sinh(x)/\cosh(x)$.

\item It is defined for every real number $x$.
\end{itemize}
See also: \textbf{atanh} (\ref{labatanh}), \textbf{cosh} (\ref{labcosh}), \textbf{sinh} (\ref{labsinh})

\subsection{taylor}
\label{labtaylor}
\noindent Name: \textbf{taylor}\\
computes a Taylor expansion of a function in a point\\
\noindent Usage: 
\begin{center}
\textbf{taylor}(\emph{function}, \emph{degree}, \emph{point}) : (\textsf{function}, \textsf{integer}, \textsf{constant}) $\rightarrow$ \textsf{function}\\
\end{center}
Parameters: 
\begin{itemize}
\item \emph{function} represents the function to be expanded
\item \emph{degree} represents the degree of the expansion to be delivered
\item \emph{point} represents the point in which the function is to be developped
\end{itemize}
\noindent Description: \begin{itemize}

\item The command \\textbf{taylor} returns an expression that is a Taylor expansion\n   of function \\emph{function} in point \\emph{point} having the degree \\emph{degree}.\n    \n   Let $f$ be the function \\emph{function}, $t$ be the point \\emph{point} and\n   $n$ be the degree \\emph{degree}. Then, \\textbf{taylor}(\\emph{function},\\emph{degree},\\emph{point}) \n   evaluates to an expression mathematically equal to \n   $$\\sum\\limits_{i=0}^n \\frac{f^{(i)}\\left(t\\right)}{i!} \\left(x - t \\right)^i$$\n    \n   Remark that \\textbf{taylor} evaluates to $0$ if the degree \\emph{degree} is negative.\n\end{itemize}
\noindent Example 1: 
\begin{center}\begin{minipage}{15cm}\begin{Verbatim}[frame=single]
\end{Verbatim}
\end{minipage}\end{center}
\noindent Example 2: 
\begin{center}\begin{minipage}{15cm}\begin{Verbatim}[frame=single]
\end{Verbatim}
\end{minipage}\end{center}
\noindent Example 3: 
\begin{center}\begin{minipage}{15cm}\begin{Verbatim}[frame=single]
\end{Verbatim}
\end{minipage}\end{center}
See also: \textbf{remez} (\ref{labremez})

\subsection{taylorform}
\label{labtaylorform}
\noindent Name: \textbf{taylorform}\\
computes a rigorous polynomial approximation (polynomial, interval error bound) for a function, based on Taylor expansions\\
\noindent Usage: 
\begin{center}
\textbf{taylorform}(\emph{f}, \emph{n}, \emph{$x_0$} \emph{I}, \emph{errorType}) : (\textsf{function}, \textsf{integer}, \textsf{constant}, \textsf{range}, \textsf{absolute$|$relative}) $\rightarrow$ \textsf{list}\\
\textbf{taylorform}(\emph{f}, \emph{n}, \emph{$x_0$} \emph{I}, \emph{errorType}) : (\textsf{function}, \textsf{integer}, \textsf{range}, \textsf{range}, \textsf{absolute$|$relative}) $\rightarrow$ \textsf{list}\\
\end{center}
Parameters: 
\begin{itemize}
\item \emph{f} is the function to be approximated
\item \emph{n} is the order of the Taylor form, meaning $\emph{n}-1$ is the degree of the polynomial that must approximate \emph{f}
\item \emph{$x_0$} is the point (it can be a real number or an interval) where the Taylor exansion of the function is to be considered
\item \emph{I} is the interval over which the function is to be approximated
\item \emph{errorType} is the type of error to be considered. See the detailed description below.
\end{itemize}
\noindent Description: \begin{itemize}

\item \textbf{taylorform} computes an approximation polynomial and an interval error bound for function $f$. More precisely, it 
   returns a list $L = \left[p, \textrm{coeffErrors}, \Delta \right]$ where:
   \begin{itemize}
   \item $p$ is an approximation polynomial of degree $n-1$ which is roughly speaking a numerical Taylor expansion of $f$ at the point $x_0$.
   \item coeffsErrors is a list of $n$ intervals. Each interval coeffsErrors[$i$] contains an enclosure of all the errors accumulated when computing the $i$-th coefficient of $p$.
   \item $\Delta$ is an interval that provides a bound for the approximation error between $p$ and $f$. Its significance depends on the \emph{errorType} considered.
   \end{itemize}

\item Please note that $x_0$ can be an interval. In general, it is meant to be a small interval approximating a non representable value. For instance, if one desires to compute a Taylor approximation at point $\pi$, it is possible to set $x_0$ to the (almost) point-interval $[\pi]$. It is also possible to use a large interval for $x_0$, though it is not obvious to give an intuitive sense to the result of \textbf{taylorform} in that case.

\item More formally, the mathematical property ensured by the algorithm may be stated as follows. For all $xi_0$ in $x_0$, there exist (small) values $\varepsilon_i \in \textrm{coeffsErrors}[i]$ such that:
   \\
   If \emph{errorType} is \textbf{absolute}, $\forall x \in I, \exists \delta \in \Delta,\, f(x)-p(x-xi_0) = \sum\limits_{i=0}^{n-1} \varepsilon_i\, (x-xi_0)^i + \delta$.
   \\
   If \emph{errorType} is \textbf{relative}, $\forall x \in I, \exists \delta \in \Delta,\, f(x)-p(x-xi_0) = \sum\limits_{i=0}^{n-1} \varepsilon_i\, (x-xi_0)^i + \delta\,(x-xi_0)^n$.

\item The polynomial $p$ and the bound  $\Delta$ are obtained using Taylor Models principles.

\item Note: The relative case is especially useful when functions with removable singularities are considered. In such a case, this routine is able to compute a finite remainder bound, provided that the expansion point given is the problematic removable singularity point.

\item Note: the algorithm does not guarantee that by increasing the degree of the approximation, the remainder bound will become smaller. Moreover, it may 
   even become larger due to the dependecy phenomenon present with interval arithmetic. In order to reduce this phenomenon, a possible solution is to split the definition domain $I$ into several smaller intervals. 
\end{itemize}
\noindent Example 1: 
\begin{center}\begin{minipage}{15cm}\begin{Verbatim}[frame=single]
> TL=taylorform(exp(x), 10, 0, [-1,1], absolute);
> p=TL[0];
> Delta=TL[2];
> errors=TL[1];
> p; Delta;
1 + x * (1 + x * (0.5 + x * (0.1666666666666666666666666666666666666666666666666
7 + x * (4.1666666666666666666666666666666666666666666666667e-2 + x * (8.3333333
333333333333333333333333333333333333333333e-3 + x * (1.3888888888888888888888888
8888888888888888888888889e-3 + x * (1.984126984126984126984126984126984126984126
98412698e-4 + x * (2.4801587301587301587301587301587301587301587301587e-5 + x * 
(2.75573192239858906525573192239858906525573192239859e-6 + x * 2.755731922398589
0652557319223985890652557319223986e-7)))))))))
[-2.31142719641187619441242534182684745832539555102969e-8;2.73126607556424744202
06278018039434042553645532164e-8]
\end{Verbatim}
\end{minipage}\end{center}
\noindent Example 2: 
\begin{center}\begin{minipage}{15cm}\begin{Verbatim}[frame=single]
> TL=taylorform(sin(x)/x, 10, 0, [-1,1], relative);
> p=TL[0];
> Delta=TL[2];
> errors=TL[1];
> p; Delta;
1 + x^2 * (-0.16666666666666666666666666666666666666666666666667 + x^2 * (8.3333
333333333333333333333333333333333333333333333e-3 + x^2 * (-1.9841269841269841269
8412698412698412698412698412698e-4 + x^2 * (2.7557319223985890652557319223985890
6525573192239859e-6 + x^2 * (-2.505210838544171877505210838544171877505210838544
19e-8)))))
[-1.6135797443886066084999806203254010793747502812764e-10;1.61357974438860660849
99806203254010793747502812764e-10]
\end{Verbatim}
\end{minipage}\end{center}

\subsection{taylorrecursions}
\label{labtaylorrecursions}
\noindent Name: \textbf{taylorrecursions}\\
controls the number of recursion steps when applying Taylor's rule.\\
\noindent Usage: 
\begin{center}
\textbf{taylorrecursions} = \emph{n} : \textsf{integer} $\rightarrow$ \textsf{void}\\
\textbf{taylorrecursions} = \emph{n} ! : \textsf{integer} $\rightarrow$ \textsf{void}\\
\textbf{taylorrecursions} : \textsf{integer}\\
\end{center}
Parameters: 
\begin{itemize}
\item \emph{n} represents the number of recursions
\end{itemize}
\noindent Description: \begin{itemize}

\item \\textbf{taylorrecursions} is a global variable. Its value represents the number of steps\n   of recursion that are used when applying Taylor's rule. This rule is applied\n   by the interval evaluator present in the core of \\sollya (and particularly\n   visible in commands like \\textbf{infnorm}).\n
\item To improve the quality of an interval evaluation of a function $f$, in \n   particular when there are problems of decorrelation), the evaluator of \\sollya\n   uses Taylor's rule:  $f([a,b]) \\subseteq f(m) + [a-m,\\,b-m] \\cdot f'([a,\\,b])$ where $m=\\frac{a+b}{2}$.\n   This rule can be applied recursively.\n   The number of step in this recursion process is controlled by \\textbf{taylorrecursions}.\n
\item Setting \\textbf{taylorrecursions} to 0 makes \\sollya use this rule only once;\n   setting it to 1 makes \\sollya use the rule twice, and so on.\n   In particular: the rule is always applied at least once.\n\end{itemize}
\noindent Example 1: 
\begin{center}\begin{minipage}{15cm}\begin{Verbatim}[frame=single]
\end{Verbatim}
\end{minipage}\end{center}

\subsection{TD}
\label{labtd}
\noindent Name: \textbf{TD}\\
\phantom{aaa}short form for \textbf{tripledouble}\\[0.2cm]
See also: \textbf{tripledouble} (\ref{labtripledouble})

\subsection{time}
\label{labtime}
\noindent Name: \textbf{time}\\
procedure for timing \sollya code.\\
\noindent Usage: 
\begin{center}
\textbf{time}(\emph{code}) : \textsf{code} $\rightarrow$ \textsf{constant}\\
\end{center}
Parameters: 
\begin{itemize}
\item \emph{code} is the code to be timed.
\end{itemize}
\noindent Description: \begin{itemize}

\item \textbf{time} permits timing a \sollya instruction, resp. a begin-end block
   of \sollya instructions. The timing value, measured in seconds, is returned
   as a \sollya constant (and not merely displayed as for \textbf{timing}). This 
   permits performing computations of the timing measurement value inside \sollya.

\item The extended \textbf{nop} command permits executing a defined number of
   useless instructions. Taking the ratio of the time needed to execute a
   certain \sollya instruction and the time for executing a \textbf{nop}
   therefore gives a way to abstract from the speed of a particular 
   machine when evaluating an algorithm's performance.
\end{itemize}
\noindent Example 1: 
\begin{center}\begin{minipage}{15cm}\begin{Verbatim}[frame=single]
> t = time(p=remez(sin(x),10,[-1;1]));
> write(t,"s were spent computing p = ",p,"\n");
0.22212799999999999998054681094039608524326467886567s were spent computing p = -
3.3426550293345171908513995127407122194691200059639e-17 + x * (0.999999999736283
59955372011464713121003442988167693 + x * (7.88027518773027866844993437990477324
95568873819693e-16 + x * (-0.166666661386013037032912982196741385680498698107285
 + x * (-5.3734444911159112186289355138557504839692987221233e-15 + x * (8.333303
7186548537651002133031675072810009327877148e-3 + x * (1.337972213892188158841123
41005509831429347230871284e-14 + x * (-1.983448630182774164932681551541589244220
04290239026e-4 + x * (-1.3789116451286674170531616441916183417598709732816e-14 +
 x * (2.6876259495430304684251822024896210963401672262005e-6 + x * 5.02823783500
10211058128384123578805586173782863605e-15)))))))))
\end{Verbatim}
\end{minipage}\end{center}
\noindent Example 2: 
\begin{center}\begin{minipage}{15cm}\begin{Verbatim}[frame=single]
> write(time({ p=remez(sin(x),10,[-1;1]); write("The error is 2^(", log2(dirtyin
fnorm(p-sin(x),[-1;1])), ")\n"); }), "were spent\n");
The error is 2^(log2(2.39602467695631727848641768186659313738474584992648e-11))
0.37303800000000000001466882171285988079034723341465were spent
\end{Verbatim}
\end{minipage}\end{center}
\noindent Example 3: 
\begin{center}\begin{minipage}{15cm}\begin{Verbatim}[frame=single]
> t = time(bashexecute("sleep 10"));
> write(~(t-10),"s of execution overhead.\n");
4.1869999999999997497557302494897157885134220123291e-3s of execution overhead.
\end{Verbatim}
\end{minipage}\end{center}
\noindent Example 4: 
\begin{center}\begin{minipage}{15cm}\begin{Verbatim}[frame=single]
> ratio := time(p=remez(sin(x),10,[-1;1]))/time(nop(10));
> write("This ratio = ", ratio, " should somehow be independent of the type of m
achine.\n");
This ratio = 6.54288235294117646937882465275081490698288944521 should somehow be
 independent of the type of machine.
\end{Verbatim}
\end{minipage}\end{center}
See also: \textbf{timing} (\ref{labtiming}), \textbf{nop} (\ref{labnop})

\subsection{timing}
\label{labtiming}
\noindent Name: \textbf{timing}\\
global variable controlling timing measures in \sollya.\\
\noindent Usage: 
\begin{center}
\textbf{timing} = \emph{activation value} : \textsf{on$|$off} $\rightarrow$ \textsf{void}\\
\textbf{timing} = \emph{activation value} ! : \textsf{on$|$off} $\rightarrow$ \textsf{void}\\
\textbf{timing} : \textsf{on$|$off}\\
\end{center}
Parameters: 
\begin{itemize}
\item \emph{activation value} controls if timing should be performed or not
\end{itemize}
\noindent Description: \begin{itemize}

\item \textbf{timing} is a global variable. When its value is \textbf{on}, the time spent in each 
   command is measured and displayed (for \textbf{verbosity} levels higher than 1).
\end{itemize}
\noindent Example 1: 
\begin{center}\begin{minipage}{15cm}\begin{Verbatim}[frame=single]
> verbosity=1!;
> timing=on;
Timing has been activated.
> p=remez(sin(x),10,[-1;1]);
Information: Remez: computing the matrix spent 1 ms
Information: Remez: computing the quality of approximation spent 12 ms
Information: Remez: computing the matrix spent 1 ms
Information: Remez: computing the quality of approximation spent 11 ms
Information: Remez: computing the matrix spent 2 ms
Information: Remez: computing the quality of approximation spent 12 ms
Information: computing a minimax approximation spent 229 ms
Information: assignment spent 229 ms
Information: full execution of the last parse chunk spent 229 ms
\end{Verbatim}
\end{minipage}\end{center}
See also: \textbf{on} (\ref{labon}), \textbf{off} (\ref{laboff}), \textbf{time} (\ref{labtime})

\subsection{tripledouble}
\label{labtripledouble}
\noindent Names: \textbf{tripledouble}, \textbf{TD}\\
represents a number as the sum of three IEEE doubles.\\
\noindent Description: \begin{itemize}

\item \textbf{tripledouble} is both a function and a constant.

\item As a function, it rounds its argument to the nearest number that can be written
   as the sum of three double precision numbers.

\item The algorithm used to compute \textbf{tripledouble}($x$) is the following: let $x_h$ = \textbf{double}($x$),
   let $x_m$ = \textbf{double}($x-x_h$) and let $x_l$ = \textbf{double}($x-x_h-x_m$). 
   Return the number $x_h+x_m+x_l$. Note that if the
   current precision is not sufficient to represent exactly $x_h+x_m+x_l$, a rounding will
   occur and the result of \textbf{tripledouble}(x) will be useless.

\item As a constant, it symbolizes the triple-double precision format. It is used in 
   contexts when a precision format is necessary, e.g. in the commands 
   \textbf{roundcoefficients} and \textbf{implementpoly}.
   See the corresponding help pages for examples.
\end{itemize}
\noindent Example 1: 
\begin{center}\begin{minipage}{15cm}\begin{Verbatim}[frame=single]
> verbosity=1!;
> a = 1+ 2^(-55)+2^(-115);
> TD(a);
1.00000000000000002775557561562891353466491600711096
> prec=110!;
> TD(a);
Warning: double rounding occurred on invoking the triple-double rounding operato
r.
Try to increase the working precision.
1.000000000000000027755575615628913
\end{Verbatim}
\end{minipage}\end{center}
See also: \textbf{single} (\ref{labsingle}), \textbf{double} (\ref{labdouble}), \textbf{doubleextended} (\ref{labdoubleextended}), \textbf{doubledouble} (\ref{labdoubledouble}), \textbf{roundcoefficients} (\ref{labroundcoefficients}), \textbf{implementpoly} (\ref{labimplementpoly}), \textbf{fpminimax} (\ref{labfpminimax}), \textbf{printexpansion} (\ref{labprintexpansion})

\subsection{true}
\label{labtrue}
\noindent Name: \textbf{true}\\
the boolean value representing the truth.\\
\noindent Description: \begin{itemize}

\item \textbf{true} is the usual boolean value.
\end{itemize}
\noindent Example 1: 
\begin{center}\begin{minipage}{15cm}\begin{Verbatim}[frame=single]
> true && false;
false
> 2>1;
true
\end{Verbatim}
\end{minipage}\end{center}
See also: \textbf{false} (\ref{labfalse}), \textbf{$\&\&$} (\ref{laband}), \textbf{$||$} (\ref{labor})

\subsection{ var }
\noindent Name: \textbf{var}\\
declaration of a local variable in a scope\\

\noindent Usage: 
\begin{center}
\textbf{var} \emph{identifier1}, \emph{identifier2},... , \emph{identifiern} : \textsf{void}\\
\end{center}
Parameters: 
\emph{identifier1}, \emph{identifier2},... , \emph{identifiern} represent variable identifiers\\

\noindent Description: \begin{itemize}

\item The keyword \textbf{var} allows for the declaration of local variables
   \emph{identifier1} through \emph{identifiern} in a begin-end-block ({}-block).
   Once declared as a local variable, an identifier will shadow
   identifiers declared in higher scopes and undeclared identifiers
   available at top-level.
   Variable declarations using \textbf{var} are only possible in the
   beginning of a begin-end-block. Several \textbf{var} statements can be
   given. Once another statement is given in a begin-end-block, no more
   \textbf{var} statements can be given.
   Variables declared by \textbf{var} statements are dereferenced as \textbf{error}
   until they are assigned a value. 
\end{itemize}
\noindent Example 1: 
\begin{center}\begin{minipage}{14.8cm}\begin{Verbatim}[frame=single]
   > exp(x); 
   exp(x)
   > a = 3; 
   > {var a, b; a=5; b=3; {var a; var b; b = true; a = 1; a; b;}; a; b; }; 
   1
   true
   5
   3
   > a;
   3
\end{Verbatim}
\end{minipage}\end{center}
See also: \textbf{error}

\subsection{verbosity}
\label{labverbosity}
\noindent Name: \textbf{verbosity}\\
global variable controlling the quantity of information displayed by commands.\\
\noindent Description: \begin{itemize}

\item \textbf{verbosity} accepts any integer value. At level 0, commands do not display anything
   on standard out. Note that very critical information may however be displayed on
   standard err.

\item Default level is 1. It displays important information such as warnings when 
   roundings happen.

\item For higher levels more information is displayed depending on the command.
\end{itemize}
\noindent Example 1: 
\begin{center}\begin{minipage}{15cm}\begin{Verbatim}[frame=single]
> verbosity=0!;
> 1.2+"toto";
error
> verbosity=1!;
> 1.2+"toto";
Warning: Rounding occurred when converting the constant "1.2" to floating-point 
with 165 bits.
If safe computation is needed, try to increase the precision.
Warning: at least one of the given expressions or a subexpression is not correct
ly typed
or its evaluation has failed because of some error on a side-effect.
error
> verbosity=2!;
> 1.2+"toto";
Warning: Rounding occurred when converting the constant "1.2" to floating-point 
with 165 bits.
If safe computation is needed, try to increase the precision.
Warning: at least one of the given expressions or a subexpression is not correct
ly typed
or its evaluation has failed because of some error on a side-effect.
Information: the expression or a partial evaluation of it has been the following
:
(1.19999999999999999999999999999999999999999999999999) + ("toto")
error
\end{Verbatim}
\end{minipage}\end{center}
See also: \textbf{roundingwarnings} (\ref{labroundingwarnings})

\subsection{void}
\label{labvoid}
\noindent Name: \textbf{void}\\
the functional result of a side-effect or empty argument resp. the corresponding type\\
\noindent Usage: 
\begin{center}
\textbf{void} : \textsf{void} $|$ \textsf{type type}\\
\end{center}
\noindent Description: \begin{itemize}

\item The variable \textbf{void} represents the functional result of a
   side-effect or an empty argument.  It is used only in combination with
   the applications of procedures or identifiers bound through
   \textbf{externalproc} to external procedures.
    
   The \textbf{void} result produced by a procedure or an external procedure
   is not printed at the prompt. However, it is possible to print it out
   in a print statement or in complex data types such as lists.
    
   The \textbf{void} argument is implicit when giving no argument to a
   procedure or an external procedure when applied. It can nevertheless be given
   explicitly.  For example, suppose that foo is a procedure or an
   external procedure with a void argument. Then foo() and foo(void) are
   correct calls to foo. Here, a distinction must be made for procedures having an
   arbitrary number of arguments. In this case, an implicit \textbf{void}
   as the only parameter to a call of such a procedure gets converted into 
   an empty list of arguments, an explicit \textbf{void} gets passed as-is in the
   formal list of parameters the procedure receives.

\item \textbf{void} is used also as a type identifier for
   \textbf{externalproc}. Typically, an external procedure taking \textbf{void} as an
   argument or returning \textbf{void} is bound with a signature \textbf{void} $->$
   some type or some type $->$ \textbf{void}. See \textbf{externalproc} for more
   details.
\end{itemize}
\noindent Example 1: 
\begin{center}\begin{minipage}{15cm}\begin{Verbatim}[frame=single]
> print(void);
void
> void;
\end{Verbatim}
\end{minipage}\end{center}
\noindent Example 2: 
\begin{center}\begin{minipage}{15cm}\begin{Verbatim}[frame=single]
> hey = proc() { print("Hello world."); };
> hey;
proc()
{
print("Hello world.");
return void;
}
> hey();
Hello world.
> hey(void);
Hello world.
> print(hey());
Hello world.
void
\end{Verbatim}
\end{minipage}\end{center}
\noindent Example 3: 
\begin{center}\begin{minipage}{15cm}\begin{Verbatim}[frame=single]
> bashexecute("gcc -fPIC -Wall -c externalprocvoidexample.c");
> bashexecute("gcc -fPIC -shared -o externalprocvoidexample externalprocvoidexam
ple.o");
> externalproc(foo, "./externalprocvoidexample", void -> void);
> foo;
foo(void) -> void
> foo();
Hello from the external world.
> foo(void);
Hello from the external world.
> print(foo());
Hello from the external world.
void
\end{Verbatim}
\end{minipage}\end{center}
\noindent Example 4: 
\begin{center}\begin{minipage}{15cm}\begin{Verbatim}[frame=single]
> procedure blub(l = ...) { print("Argument list:", l); };
> blub(1);
Argument list: [|1|]
> blub();
Argument list: [| |]
> blub(void); 
Argument list: [|void|]
\end{Verbatim}
\end{minipage}\end{center}
See also: \textbf{error} (\ref{laberror}), \textbf{proc} (\ref{labproc}), \textbf{externalproc} (\ref{labexternalproc})

\subsection{worstcase}
\label{labworstcase}
\noindent Name: \textbf{worstcase}\\
searches for hard-to-round\\
\noindent Usage: 
\begin{center}
\textbf{worstcase}(\emph{function}, \emph{preimage precision}, \emph{preimage exponent range}, \emph{image precision}, \emph{error bound}) : (\textsf{function}, \textsf{integer}, \textsf{range}, \textsf{integer}, \textsf{constant}) $\rightarrow$ \textsf{void}
\\ 
\textbf{worstcase}(\emph{function}, \emph{preimage precision}, \emph{preimage exponent range}, \emph{image precision}, \emph{error bound}, \emph{filename}) : (\textsf{function}, \textsf{integer}, \textsf{range}, \textsf{integer}, \textsf{constant}, \textsf{string}) $\rightarrow$ \textsf{void}
\\ 
\end{center}
Parameters: 
\begin{itemize}
\item \emph{function} represents the function to be considered
\item \emph{preimage precision} represents the precision of the preimages
\item \emph{preimage exponent range} represents the exponents in the preimage format
\item \emph{image precision} represents the precision of the format the images are to be rounded to
\item \emph{error bound} represents the upper bound for the search w.r.t. the relative rounding error
\item \emph{filename} represents a character sequence containing a filename
\end{itemize}
\noindent Description: \begin{itemize}

\item The \textbf{worstcase} command is deprecated. It searches hard-to-round cases of
   a function. The command \textbf{searchgal} has a comparable functionality.
\end{itemize}
\noindent Example 1: 
\begin{center}\begin{minipage}{15cm}\begin{Verbatim}[frame=single]
> worstcase(exp(x),24,[1,2],24,1b-26);
prec = 165
x = 1.99999988079071044921875        f(x) = 7.3890552520751953125        eps = 4
.5998601423446695596184695493764120138001954979037e-9 = 2^(-27.695763) 
x = 2        f(x) = 7.38905620574951171875        eps = 1.4456360874967301812222
8379395533417878125150587072e-8 = 2^(-26.043720) 

\end{Verbatim}
\end{minipage}\end{center}
See also: \textbf{round} (\ref{labround}), \textbf{searchgal} (\ref{labsearchgal}), \textbf{evaluate} (\ref{labevaluate})

\subsection{ write }
\noindent Name: \textbf{write}\\
prints an expression without separators\\

\noindent Usage: 
\begin{center}
\textbf{write}(\emph{expr1},...,\emph{exprn}) : (\textsf{any type},..., \textsf{any type}) $\rightarrow$ \textsf{void}\\
\textbf{write}(\emph{expr1},...,\emph{exprn}) $>$ \emph{filename} : (\textsf{any type},..., \textsf{any type}, \textsf{string}) $\rightarrow$ \textsf{void}\\
\textbf{write}(\emph{expr1},...,\emph{exprn}) $>>$ \emph{filename} : (\textsf{any type},...,\textsf{any type}, \textsf{string}) $\rightarrow$ \textsf{void}\\
\end{center}
Parameters: 
\begin{itemize}
\item \emph{expr} represents an expression
\item \emph{filename} represents a character sequence indicating a file name
\end{itemize}
\noindent Description: \begin{itemize}

\item \textbf{write}(\emph{expr1},...,\emph{exprn}) prints the expressions \emph{expr1} through
   \emph{exprn}. The character sequences corresponding to the expressions are
   concatenated without any separator. No newline is displayed at the
   end.  In contrast to \textbf{print}, \textbf{write} expects the user to give all
   separators and newlines explicitely.
   If a second argument \emph{filename} is given after a single "$>$", the
   displaying is not output on the standard output of Sollya but if in
   the file \emph{filename} that get newly created or overwritten. If a double
    "$>>$" is given, the output will be appended to the file \emph{filename}.
   The global variables \textbf{display}, \textbf{midpointmode} and \textbf{fullparentheses} have
   some influence on the formatting of the output (see \textbf{display},
   \textbf{midpointmode} and \textbf{fullparentheses}).
   Remark that if one of the expressions \emph{expri} given in argument is of
   type \textsf{string}, the character sequence \emph{expri} evaluates to is
   displayed. However, if \emph{expri} is of type \textsf{list} and this list
   contains a variable of type \textsf{string}, the expression for the list
   is displayed, i.e.  all character sequences get displayed surrounded
   by quotes ('"'). Nevertheless, escape sequences used upon defining
   character sequences are interpreted immediately.
\end{itemize}
\noindent Example 1: 
\begin{center}\begin{minipage}{15cm}\begin{Verbatim}[frame=single]
> write(x + 2 + exp(sin(x))); 
> write("Hello\n");
x + 2 + exp(sin(x))Hello
> write("Hello","world\n");
Helloworld
> write("Hello","you", 4 + 3, "other persons.\n");
Helloyou7other persons.
\end{Verbatim}
\end{minipage}\end{center}
\noindent Example 2: 
\begin{center}\begin{minipage}{15cm}\begin{Verbatim}[frame=single]
> write("Hello","\n");
Hello
> write([|"Hello"|],"\n");
[|"Hello"|]
> s = "Hello";
> write(s,[|s|],"\n");
Hello[|"Hello"|]
> t = "Hello\tyou";
> write(t,[|t|],"\n");
Hello	you[|"Hello	you"|]
\end{Verbatim}
\end{minipage}\end{center}
\noindent Example 3: 
\begin{center}\begin{minipage}{15cm}\begin{Verbatim}[frame=single]
> write(x + 2 + exp(sin(x))) > "foo.sol";
> readfile("foo.sol");
x + 2 + exp(sin(x))
\end{Verbatim}
\end{minipage}\end{center}
\noindent Example 4: 
\begin{center}\begin{minipage}{15cm}\begin{Verbatim}[frame=single]
> write(x + 2 + exp(sin(x))) >> "foo.sol";
\end{Verbatim}
\end{minipage}\end{center}
See also: \textbf{print}, \textbf{printexpansion}, \textbf{printhexa}, \textbf{printfloat}, \textbf{printxml}, \textbf{readfile}, \textbf{autosimplify}, \textbf{display}, \textbf{midpointmode}, \textbf{fullparentheses}, \textbf{evaluate}



\newpage
\section{Appendix: interval arithmetic philosophy in \sollya}
\label{IntervalArithmeticPhilopshy}

Although it is currently based on the MPFI library, \sollya has its own way of interpreting interval arithmetic when infinities or NaN occur, or when a function is evaluated on an interval containing points out of its domain, etc. This philosophy may differ from the one applied in MPFI. It is also possible that the behavior of \sollya does not correspond to the behavior that one would expect, e.g. as a natural consequence of the IEEE-754 standard.

The topology that we consider is always the usual topology of $\overline{\mathbb{R}} = \mathbb{R} \cup \{-\infty,\,+\infty\}$. For any function, if one of its arguments is empty (respectively NaN), we return empty (respectively NaN).

\subsection{Univariate functions}
Let $f$ be a univariate basic function and $I$ an interval. We denote by $J$ the result of the interval evaluation of $f$ over $I$ in \sollya. If $I$ is completely included in the domain of $f$, $J$ will usually be the smallest interval (at the current precision) containing the exact image $f(I)$. However, in some cases, it may happen that $J$ is not as small as possible. It is guaranteed however, that $J$ tends to $f(I)$ when the precision of the tool tends to infinity.

When $f$ is not defined at some point $x$ but is defined on a neighborhood of $x$, we consider that the ``value'' of $f$ at $x$ is the convex hull of the limit points of $f$ around $x$. For instance, consider the evaluation of $f= \tan$ on $[0,\, \pi]$. It is not defined at $\pi/2$ (and only at this point). The limit points of $f$ around $\pi/2$ are $-\infty$ and $+\infty$, so, we return $[-\infty,\,\infty]$. Another example: $f=\sin$ on $[+\infty]$. The function has no limit at this point, but all points of $[-1, 1]$ are limit points. So, we return $[-1,\,1]$.

Finally, if $I$ contains a subinterval on which $f$ is not defined, we return $[\textrm{NaN},\,\textrm{NaN}]$ (example: $\sqrt{[-1,\,2]}$).

\subsection{Bivariate functions}
Let $f$ be a bivariate function and $I_1$ and $I_2$ be intervals. If $I_1=[x]$ and $I_2=[y]$ are both point-intervals, we return the convex hull of the limit points of $f$ around $(x,\,y)$ if it exists. In particular, if $f$ is defined at $(x,\,y)$ we return its value (or a small interval around it, if it is not exactly representable). As an example $[1]/[+\infty]$ returns $[0]$. Also, $[1]/[0]$ returns $[-\infty,\,+\infty]$ (note that \sollya does not consider signed zeros). If it is not possible to give a meaning to the expression $f(I_1,\,I_2)$, we return NaN: for instance $[0]/[0]$ or $[0]*[+\infty]$.

If one and only one of the intervals is a point-interval (say $I_1 = [x]$), we consider the partial function $g: y \mapsto f(x,y)$ and return the value that would be obtained when evaluating $g$ on $I_2$. For instance, in order to evaluate $[0]/I_2$, we consider the function $g$ defined for every $y \neq 0$ by $g(y)=0/y=0$. Hence, $g(I_2) = [0]$ (even if $I_2$ contains $0$, by the argument of limit-points). In particular, please note that $[0]/[-1,\,1]$ returns $[0]$ even though $[0]/[0]$ returns NaN. This rule even holds when $g$ can only be defined as limit points: for instance, in the case $I_1/[0]$ we consider $g: x \mapsto x/0$. This function cannot be defined \emph{stricto sensu}, but we can give it a meaning by considering $0$ as a limit. Hence $g$ is multivalued and its value is $\{-\infty,\,+\infty\}$ for every $x$. Hence, $I_1/[0]$ returns $[-\infty,\,+\infty]$ when $I_1$ is not a point-interval.

Finally, if neither $I_1$ nor $I_2$ are point-intervals, we try to give a meaning to $f(I_1,\,I_2)$ by an argument of limit-points when possible. For instance $[1,\,2] / [0,\,1]$ returns $[1,\,+\infty]$.

As a special exception to these rules, $[0]^{[0]}$ returns $[1]$.

\newpage
\section{Appendix: the \sollya library}
\label{Libsollya}
 \subsection{Introduction}
The header file of the \sollya library is \texttt{sollya.h}. Its inclusion may provoke the inclusion of other header files, such as \texttt{gmp.h}, \texttt{mpfr.h} or \texttt{mpfi.h}.

The library provides a virtual \sollya session that is perfectly similar to an interactive session: environment variables such as \texttt{verbosity}, \texttt{prec}, \texttt{display}, \texttt{midpointmode}, etc. are maintained and affect the behavior of the library, warning messages are displayed when something is not exact, etc. Please notice that the \sollya library currently is \textbf{not} re-entrant and can only be opened once. A process using the library must hence not be multi-threaded and is limited to one single virtual \sollya session.

In order to get started with the \sollya library, the first thing to do is hence to initialize this virtual session. This is performed with the \verb|sollya_lib_init| function. Accordingly, one should close the session at the end of the program (which has the effect of releasing all the memory used by \sollya). Please notice that \sollya uses its own allocation functions and registers them to \verb|GMP| using the custom allocation functions provided by \verb|GMP|. Particular precautions should hence be taken when using the \sollya library in a program that also registers its own functions to \verb|GMP|: in that case \verb|sollya_lib_init_with_custom_memory_functions| should be used instead of \verb|sollya_lib_init| for initializing the library. This is discussed in Section~\ref{customMemoryFunctions}.

In the usual case when \sollya is used in a program that does not register allocation functions to~\verb|GMP|, a minimal file using the library is hence the following.

\begin{center}\begin{minipage}{15cm}\begin{Verbatim}[frame=single]
#include <sollya.h>

int main(void) {
  sollya_lib_init();

    /* Functions of the library can be called here */

  sollya_lib_close();
  return 0;
}
\end{Verbatim}
\end{minipage}\end{center}

Suppose that this code is saved in a file called \texttt{foo.c}. The compilation is performed as usual without forgetting to link against \texttt{libsollya} (since the libraries \texttt{libgmp}, \texttt{libmpfr} and \texttt{libmpfi} are dependencies of \sollya, it might also be necessary to explicitly link against them):
\begin{center}\begin{minipage}{15cm}\begin{Verbatim}[frame=single]
~/% cc foo.c -c
~/% cc foo.o -o foo -lsollya -lmpfi -lmpfr -lgmp
\end{Verbatim}
\end{minipage}\end{center}

 \subsection{Sollya object data-type}
The library provides a single data type called \texttt{sollya\_obj\_t} that can contain any \sollya object (a \sollya object is anything that can be stored in a variable within the interactive tool. See Section~\ref{sec:data_types} of the present documentation for details). Please notice that \texttt{sollya\_obj\_t} is in fact a pointer type; this has two consequences:
\begin{itemize}
\item \texttt{NULL} is a placeholder that can be used as a \texttt{sollya\_obj\_t} in some contexts. This placeholder is particularly useful as an end marker for functions with a variable number of arguments (see Sections~\ref{creating_lists} and~\ref{library_commands_and_functions}).
\item An assignment with the ``='' sign does not copy an object but only copies the reference to it. To perform a (deep) copy, the \texttt{sollya\_lib\_copy\_obj()} function is available.
\end{itemize}
Except for a few functions for which the contrary is explicitly specified, the following conventions are used:
\begin{itemize}
\item  A function does not touch its arguments. Hence if \texttt{sollya\_lib\_foo} is a function of the library, a call to \texttt{sollya\_lib\_foo(a)} leaves the object referenced by \texttt{a} unchanged (the notable exceptions to that rule are the functions containing \verb|build| in their name, e.g., \texttt{sollya\_lib\_build\_foo}).
\item A function that returns a \texttt{sollya\_obj\_t} creates a new object (this means that memory is dynamically allocated for that object). The memory allocated for that object should manually be cleared when the object is no longer used and all references to it (on the stack) get out of reach, e.g. on a function return: this is performed by the \texttt{sollya\_lib\_clear\_obj()} function.
\end{itemize}

In general, except if the user perfectly knows what they are doing, the following rules should be applied (here \texttt{a} and \texttt{b} are C variables of type \texttt{sollya\_obj\_t}, and \texttt{sollya\_lib\_foo} and \texttt{sollya\_lib\_bar} are functions of the library):
\begin{itemize}
\item One should never write \texttt{a = b}. Instead, use \texttt{a = sollya\_lib\_copy\_obj(b)}.
\item One should never write \texttt{a = sollya\_lib\_foo(a)} because one loses the reference to the object initially referenced by the variable \texttt{a} (which is hence not cleared).
\item One should never chain function calls such as, e.g., \texttt{a = sollya\_lib\_foo(sollya\_lib\_bar(b))} (the reference to the object created by \texttt{sollya\_lib\_bar(b)} would be lost and hence not cleared).
\item A variable \texttt{a} should never be used twice at the left-hand side of the ``='' sign (or as an lvalue in general) without performing \texttt{sollya\_lib\_clear\_obj(a)} in-between.
\item In an assignment of the form ``\texttt{a = ...}'', the right-hand side should always be a function call (i.e., something like \texttt{a = sollya\_lib\_foo(...);}).
\end{itemize}

Please notice that \texttt{sollya\_lib\_close()} clears the memory allocated by the virtual \sollya session but not the objects that have been created and stored in C variables. All the \texttt{sollya\_obj\_t} created by function calls should be cleared manually.

We can now write a simple Hello World program using the \sollya library:
\begin{center}\begin{minipage}{15cm}\begin{Verbatim}[frame=single]
#include <sollya.h>

int main(void) {
  sollya_obj_t s1, s2, s;
  sollya_lib_init();

  s1 = sollya_lib_string("Hello ");
  s2 = sollya_lib_string("World!");
  s = sollya_lib_concat(s1, s2);
  sollya_lib_clear_obj(s1);
  sollya_lib_clear_obj(s2);

  sollya_lib_printf("%b\n", s);
  sollya_lib_clear_obj(s);
  sollya_lib_close();
  return 0;
}
\end{Verbatim}
\end{minipage}\end{center}

A universal function allows the user to execute any expression, as if it were given at the prompt of the \sollya tool, and to get a \texttt{sollya\_obj\_t} containing the result of the evaluation: this function is \texttt{sollya\_lib\_parse\_string("some expression here")}. This is very convenient, and indeed, any script written in the \sollya language, could easily be converted into a C program by intensively using \texttt{sollya\_lib\_parse\_string}. However, this should not be the preferred way if efficiency is targeted because (as its name suggests) this function uses a parser to decompose its argument, then constructs intermediate data structures to store the abstract interpretation of the expression, etc. Low-level functions are provided for efficiently creating \sollya objects; they are detailed in the next Section.

\subsection{Conventions in use in the library}
The library follows some conventions that it is useful to remember:
\begin{itemize}
\item When a function is a direct transposition of a command or function available in the interactive tool, it returns a \verb|sollya_obj_t|. This is true, even when it would sound natural to return, e.g. an \verb|int|. For instance \verb|sollya_lib_get_verbosity()| returns a \verb|sollya_obj_t|, which integer value must then be recovered with \verb|sollya_lib_get_constant_as_int|. This forces the user to declare (and clear afterwards) a temporary \verb|sollya_obj_t| to store the value, but this is the price of homogeneity in the library.
\item When a function returns an integer, this integer generally is a boolean in the usual C meaning, i.e. $0$ represents false and any non-zero value represents true. In many cases, the integer returned by the function indicates a status of success or failure: the convention is ``false means failure'' and ``true means success''. In case of failure, the convention is that the function did not touch any of its arguments.
\item When a function would need to return several things, or when a function would need to return something together with a status of failure or success, the convention is that pointers are given as the first arguments of the function. These pointers shall point to valid addresses where the function will store the results. This can sometimes give obscure signatures, when the function morally returns a pointer and actually takes as argument a pointer to a pointer (this typically happens when the function allocates a segment of memory and should return a pointer to that segment of memory).
\end{itemize}

\subsection{Displaying \sollya objects and numerical values}
Within the interactive tool, the most simple way of displaying the content of a variable or the value of an expression is to write the name of the variable or the expression, followed by the character ``;''. As a result, \sollya evaluates the expression or the variable and displays the result. Alternatively, a set of objects can be displayed the same way, separating the objects with commas. In library mode, the same behavior can be reproduced using the function \verb|void sollya_lib_autoprint(sollya_obj_t, ...)|. Please notice that this function has a variable number of arguments: they are all displayed, until an argument equal to \verb|NULL| is found. The \verb|NULL| argument is mandatory, even if only one object shall be displayed (the function has no other way to know if other arguments follow or not). So, if only one argument should be displayed, the correct function call is \verb|sollya_lib_autoprint(arg, NULL)|. Accordingly, if two arguments should be displayed, the function call is \verb|sollya_lib_autoprint(arg1, arg2, NULL)|, etc. The function \verb|void sollya_lib_v_autoprint(sollya_obj_t, va_list)| is the same, but it takes a \verb|va_list| argument instead of a variable number of arguments.

Further, there is another way of printing formatted strings containing \sollya objects, using a {\tt printf}-like syntax. Eight functions are provided, namely \verb|sollya_lib_printf|, \verb|sollya_lib_v_printf|, \verb|sollya_lib_fprintf|, \verb|sollya_lib_v_fprintf|, \verb|sollya_lib_sprintf|, \verb|sollya_lib_v_sprintf|, \verb|sollya_lib_snprintf| and \verb|sollya_lib_v_snprintf|. Each one of these functions overloads the usual function (respectively, \verb|printf|, \verb|vprintf|, \verb|fprintf|, \verb|vfprintf|, \verb|sprintf|, \verb|vsprintf|, \verb|snprintf| and \verb|vsnprintf|). The full syntax of conversions specifiers supported with the usual functions is handled (please note that the style using '\verb|$|' ---~as in \verb|%3$| or \verb|%*3$|~--- is not handled though. It is not included in the C99 standard anyway). Additionally, the following conversion specifiers are provided:
\begin{itemize}
\item \verb|%b|: corresponds to a \verb|sollya_obj_t| argument.
\item \verb|%v|: corresponds to a \verb|mpfr_t| argument. An optional precision modifier can be applied (e.g \verb|%.5v|).
\item \verb|%w|: corresponds to a \verb|mpfi_t| argument. An optional precision modifier can be applied (e.g \verb|%.5w|).
\item \verb|%r|: corresponds to a \verb|mpq_t| argument. There is no precision modifier support.
\end{itemize}
When one of the above conversion specifiers is used, the corresponding argument is displayed as it would be within the interactive tool: i.e. the way the argument is displayed depends on \sollya environment variables, such as \verb|prec|, \verb|display|, \verb|midpointmode|, etc. When a precision modifier $n$ is used, the argument is first rounded to a binary precision of roughly $\log_2(10)\times n$ bits (i.e. roughly equivalent to $n$ decimal digits) before being displayed. As with traditional \verb|printf|, the precision modifier can be replaced with \verb|*| which causes the precision to be determined by an additional \verb|int| argument. Notice that no width modifier is supported for the \verb|%v| or \verb|%w| conversion.

The \verb|sollya_lib_printf| functions return an integer with the same meaning as the traditional \verb|printf| functions. It indicates the number of characters that have been output (excluding the final \verb|\0| character). Similarly, the conversion specifier \verb|%n| can be used, even together with the \sollya conversion specifiers \verb|%b|, \verb|%v|, \verb|%w| and \verb|%r|. The functions \verb|sollya_lib_snprintf| and \verb|sollya_lib_v_snprintf| will
never write more characters than indicated by their size argument (including the final \verb|\0| character). If the output gets truncated due to this limit, they will return the number of characters (excluding the final \verb|\0| character) that would have been output if there had not been any trunctation. In case of error, all \verb|sollya_lib_printf| functions return a negative value.

\subsection{Creating \sollya objects}
\sollya objects conceptually fall into one of five categories: numerical constants (e.g. $1$ or $1.5$), functional expressions (they might contain numerical constants, e.g., $\sin(\cos(x+1.5))$), other simple objects (intervals, strings, built-in constants such as \texttt{dyadic}, etc.), lists of objects (e.g., \texttt{[|1, "Hello"|]}) and structures (e.g., \verb|{.a = 1; .b = "Hello"}|).

\subsubsection{Numerical constants}
Table~\ref{creating_numerical_constant} lists the different functions available to construct numerical constants. A \sollya constant is always created without rounding (whatever the value of global variable \texttt{prec} is at the moment of the function call): a sufficient precision is always allocated so that the constant is stored exactly. The objects returned by these functions are newly allocated and copies of the argument. For instance, after the instruction \texttt{a = sollya\_lib\_constant(b)}, the user will eventually have to clear \texttt{a} (with \texttt{sollya\_lib\_clear(a)}) and \texttt{b} (with \texttt{mpfr\_clear(b)}).

The function \texttt{sollya\_lib\_constant\_from\_double} (or more conveniently its shortcut \texttt{SOLLYA\_CONST}) is probably the preferred way for constructing numerical constants. As the name indicates it, its argument is a \texttt{double}; however, due to implicit casting in~C, it is perfectly possible to give an \texttt{int} as argument: it will be converted into a \texttt{double} (without rounding if the integer fits on $53$~bits) before being passed to \texttt{SOLLYA\_CONST}. On the contrary, the user should be aware of the fact that if decimal non-integer constants are given, C rules of rounding (to double) are applied, regardless of the setting of the \sollya precision variable \texttt{prec}.

\begin{table}[htp]
\caption{Creating numerical constants (Creates a fresh \texttt{sollya\_obj\_t}. Conversion is always exact)}
\label{creating_numerical_constant}
\begin{center}
  \begin{tabular}{|l|c|l|}
    \hline
 \hfil Type of the argument \hfil & \hfil \phantom{\Large{$A^A$}}Name of the function\phantom{\Large{$A^A$}}\hfil & Shortcut macro \\ \hline
\verb|double| & \verb|sollya_lib_constant_from_double(x)| & \verb|SOLLYA_CONST(x)| \\
\verb|uint64_t| & \verb|sollya_lib_constant_from_uint64(x)| & \verb|SOLLYA_CONST_UI64(x)| \\
\verb|int64_t| & \verb|sollya_lib_constant_from_int64(x)| &  \verb|SOLLYA_CONST_SI64(x)| \\
\verb|int| & \verb|sollya_lib_constant_from_int(x)| & N/A \\
\verb|mpfr_t| & \verb|sollya_lib_constant(x)| & N/A \\
\hline
  \end{tabular}
\end{center}
\end{table}

\subsubsection{Functional expressions}
Functional expressions are built by composition of basic functions with constants and the free mathematical variable. Since it is convenient to build such expressions by chaining function calls, the library provides functions that ``eat up'' their arguments (actually embedding them in a bigger expression). The convention is that functions that eat up their arguments are prefixed by \texttt{sollya\_lib\_build\_}. For the purpose of building expressions, shortcut macros for the corresponding functions exist. They are all listed in Table~\ref{build_expr}.

It is worth mentioning that, although \texttt{SOLLYA\_X\_} and \texttt{SOLLYA\_PI} are used without parentheses (as if they denoted constants), they are in fact function calls that create a new object each time they are used. The absence of parentheses is just more convenient for constructing expressions, such as, e.g. \texttt{SOLLYA\_COS(SOLLYA\_X\_)}.

\begin{table}[htp]
\caption{Building functional expressions (Eats up arguments, embedding them in the returned object.)}
\label{build_expr}
\begin{center}
  \begin{tabular}{|c|l|l|}
    \hline
    Name in the interactive tool & \hfil \phantom{\Large{$A^A$}}Function to build it\phantom{\Large{$A^A$}}\hfil & Shortcut macro \\ \hline
\verb|_x_| & \verb|sollya_lib_build_function_free_variable()| & \verb|SOLLYA_X_|\\
\verb|pi| & \verb|sollya_lib_build_function_pi()| & \verb|SOLLYA_PI|\\
\verb|e1 + e2| & \verb|sollya_lib_build_function_add(e1, e2)| & \verb|SOLLYA_ADD(e1, e2)|\\
\verb|e1 - e2| & \verb|sollya_lib_build_function_sub(e1, e2)| & \verb|SOLLYA_SUB(e1, e2)|\\
\verb|e1 * e2| & \verb|sollya_lib_build_function_mul(e1, e2)| & \verb|SOLLYA_MUL(e1, e2)|\\
\verb|e1 / e2| & \verb|sollya_lib_build_function_div(e1, e2)| & \verb|SOLLYA_DIV(e1, e2)|\\
\verb|pow(e1, e2)| & \verb|sollya_lib_build_function_pow(e1, e2)| & \verb|SOLLYA_POW(e1, e2)|\\
\verb|-e| & \verb|sollya_lib_build_function_neg(e)| & \verb|SOLLYA_NEG(e)|\\
\verb|sqrt(e)| & \verb|sollya_lib_build_function_sqrt(e)| & \verb|SOLLYA_SQRT(e)|\\
\verb|abs(e)| & \verb|sollya_lib_build_function_abs(e)| & \verb|SOLLYA_ABS(e)|\\
\verb|erf(e)| & \verb|sollya_lib_build_function_erf(e)| & \verb|SOLLYA_ERF(e)|\\
\verb|erfc(e)| & \verb|sollya_lib_build_function_erfc(e)| & \verb|SOLLYA_ERFC(e)|\\
\verb|exp(e)| & \verb|sollya_lib_build_function_exp(e)| & \verb|SOLLYA_EXP(e)|\\
\verb|expm1(e)| & \verb|sollya_lib_build_function_expm1(e)| & \verb|SOLLYA_EXPM1(e)|\\
\verb|log(e)| & \verb|sollya_lib_build_function_log(e)| & \verb|SOLLYA_LOG(e)|\\
\verb|log2(e)| & \verb|sollya_lib_build_function_log2(e)| & \verb|SOLLYA_LOG2(e)|\\
\verb|log10(e)| & \verb|sollya_lib_build_function_log10(e)| & \verb|SOLLYA_LOG10(e)|\\
\verb|log1p(e)| & \verb|sollya_lib_build_function_log1p(e)| & \verb|SOLLYA_LOG1P(e)|\\
\verb|sin(e)| & \verb|sollya_lib_build_function_sin(e)| & \verb|SOLLYA_SIN(e)|\\
\verb|cos(e)| & \verb|sollya_lib_build_function_cos(e)| & \verb|SOLLYA_COS(e)|\\
\verb|tan(e)| & \verb|sollya_lib_build_function_tan(e)| & \verb|SOLLYA_TAN(e)|\\
\verb|asin(e)| & \verb|sollya_lib_build_function_asin(e)| & \verb|SOLLYA_ASIN(e)|\\
\verb|acos(e)| & \verb|sollya_lib_build_function_acos(e)| & \verb|SOLLYA_ACOS(e)|\\
\verb|atan(e)| & \verb|sollya_lib_build_function_atan(e)| & \verb|SOLLYA_ATAN(e)|\\
\verb|sinh(e)| & \verb|sollya_lib_build_function_sinh(e)| & \verb|SOLLYA_SINH(e)|\\
\verb|cosh(e)| & \verb|sollya_lib_build_function_cosh(e)| & \verb|SOLLYA_COSH(e)|\\
\verb|tanh(e)| & \verb|sollya_lib_build_function_tanh(e)| & \verb|SOLLYA_TANH(e)|\\
\verb|asinh(e)| & \verb|sollya_lib_build_function_asinh(e)| & \verb|SOLLYA_ASINH(e)|\\
\verb|acosh(e)| & \verb|sollya_lib_build_function_acosh(e)| & \verb|SOLLYA_ACOSH(e)|\\
\verb|atanh(e)| & \verb|sollya_lib_build_function_atanh(e)| & \verb|SOLLYA_ATANH(e)|\\
\verb|D(e)|, \verb|double(e)| & \verb|sollya_lib_build_function_double(e)| & \verb|SOLLYA_D(e)|\\
\verb|SG(e)|, \verb|single(e)| & \verb|sollya_lib_build_function_single(e)| & \verb|SOLLYA_SG(e)|\\
\verb|QD(e)|, \verb|quad(e)| & \verb|sollya_lib_build_function_quad(e)| & \verb|SOLLYA_QD(e)|\\
\verb|HP(e)|, \verb|halfprecision(e)| & \verb|sollya_lib_build_function_halfprecision(e)| & \verb|SOLLYA_HP(e)|\\
\verb|DD(e)|, \verb|doubledouble(e)| & \verb|sollya_lib_build_function_double_double(e)| & \verb|SOLLYA_DD(e)|\\
\verb|TD(e)|, \verb|tripledouble(e)| & \verb|sollya_lib_build_function_triple_double(e)| & \verb|SOLLYA_TD(e)|\\
\verb|DE(e)|, \verb|doubleextended(e)| & \verb|sollya_lib_build_function_doubleextended(e)| & \verb|SOLLYA_DE(e)|\\
\verb|ceil(e)| & \verb|sollya_lib_build_function_ceil(e)| & \verb|SOLLYA_CEIL(e)|\\
\verb|floor(e)| & \verb|sollya_lib_build_function_floor(e)| & \verb|SOLLYA_FLOOR(e)|\\
\verb|nearestint(e)| & \verb|sollya_lib_build_function_nearestint(e)| & \verb|SOLLYA_NEARESTINT(e)|\\
\hline
  \end{tabular}
\end{center}
\end{table}

For each function of the form \verb|sollya_lib_build_function_foo|, there exists a function called \verb|sollya_lib_foo|. There are two differences between them:
\begin{itemize}
\item First, \verb|sollya_lib_foo| does not ``eat up'' its argument. This can sometimes be useful, e.g., if one has an expression stored in a variable \texttt{a} and one wants to build the expression \texttt{exp(a)} without loosing the reference to the expression represented by \texttt{a}.
\item Second, while \verb|sollya_lib_build_function_foo| mechanically constructs an expression, function \verb|sollya_lib_foo| also evaluates it, as far as this is possible without rounding.\\
For instance, after the instructions \verb|a = SOLLYA_CONST(0); b = sollya_lib_exp(a);| the variable \texttt{b} contains the number $1$, whereas it would have contained the expression "\texttt{exp(0)}" if it had been created by \verb|b = sollya_lib_build_function(a)|.
\end{itemize}
Actually, \verb|sollya_lib_foo| has exactly the same behavior as writing an expression at the prompt within the interactive tool. In particular, it is possible to give a range as an argument to \verb|sollya_lib_foo|: the returned object will be the result of the evaluation of function \verb|foo| on that range by interval arithmetic. In contrast, trying to use \verb|sollya_lib_build_function_foo| on a range would result in a typing error.

\subsubsection{Other simple objects}
Other simple objects are created with functions listed in Table~\ref{creating_sollya_obj_t}. The functions with a name of the form \texttt{sollya\_lib\_range\_something} follow the same convention as \texttt{sollya\_lib\_constant}: they build a new object from a copy of their argument, and the conversion is always exact, whatever the value of \texttt{prec} is.

Please note that in the interactive tool, \texttt{D} either denotes the discrete mathematical function that maps a real number to its closest \texttt{double} number, or is used as a symbolic constant to indicate that the \texttt{double} format must be used (as an argument of \texttt{round} for instance). In the library, they are completely distinct objects, the mathematical function being obtained with \texttt{sollya\_lib\_build\_function\_double} and the symbolic constant with \texttt{sollya\_lib\_double\_obj}. The same holds for other formats (\texttt{DD}, \texttt{SG}, etc.)
\begin{table}[htp]
  \caption{Creating \sollya objects from scratch (Returns a new \texttt{sollya\_obj\_t})}
  \label{creating_sollya_obj_t}
  \renewcommand{\footnoterule}{} % suppress the separation line between main text and footnotes.
  \begin{minipage}{\textwidth}
    ~
    \begin{center}
      \begin{tabular}{|c|l|}
    \hline
    Name in the interactive tool & \hfil \phantom{\Large{$A^A$}}Function to create it\phantom{\Large{$A^A$}}\hfil \\ \hline
\verb|on| & \verb|sollya_lib_on();| \\
\verb|off| & \verb|sollya_lib_off();| \\
\verb|dyadic| & \verb|sollya_lib_dyadic();| \\
\verb|powers| & \verb|sollya_lib_powers();| \\
\verb|binary| & \verb|sollya_lib_binary();| \\
\verb|hexadecimal| & \verb|sollya_lib_hexadecimal();| \\
\verb|file| & \verb|sollya_lib_file();| \\
\verb|postscript| & \verb|sollya_lib_postscript();| \\
\verb|postscriptfile| & \verb|sollya_lib_postscriptfile();| \\
\verb|perturb| & \verb|sollya_lib_perturb();| \\
\verb|RD| & \verb|sollya_lib_round_down();| \\
\verb|RU| & \verb|sollya_lib_round_up();| \\
\verb|RZ| & \verb|sollya_lib_round_towards_zero();| \\
\verb|RN| & \verb|sollya_lib_round_to_nearest();| \\
\verb|honorcoeffprec| & \verb|sollya_lib_honorcoeffprec();| \\
\verb|true| & \verb|sollya_lib_true();| \\
\verb|false| & \verb|sollya_lib_false();| \\
\verb|void| & \verb|sollya_lib_void();| \\
\verb|default| & \verb|sollya_lib_default();| \\
\verb|decimal| & \verb|sollya_lib_decimal();| \\
\verb|absolute| & \verb|sollya_lib_absolute();| \\
\verb|relative| & \verb|sollya_lib_relative();| \\
\verb|fixed| & \verb|sollya_lib_fixed();| \\
\verb|floating| & \verb|sollya_lib_floating();| \\
\verb|error| & \verb|sollya_lib_error();| \\
\verb|D, double| & \verb|sollya_lib_double_obj();| \\
\verb|SG, single| & \verb|sollya_lib_single_obj();| \\
\verb|QD, quad| & \verb|sollya_lib_quad_obj();| \\
\verb|HP, halfprecision| & \verb|sollya_lib_halfprecision_obj();| \\
\verb|DE, doubleextended| & \verb|sollya_lib_doubleextended_obj();| \\
\verb|DD, doubledouble| & \verb|sollya_lib_double_double_obj();| \\
\verb|TD, tripledouble| & \verb|sollya_lib_triple_double_obj();| \\
\verb|"Hello"| &  \verb|sollya_lib_string("Hello")| \\
\verb|[1, 3.5]| &  \verb|sollya_lib_range_from_interval(a);|\footnote{\texttt{a} is a \texttt{mpfi\_t} containing the interval $[1, 3.5]$. Conversion is always exact.} \\
\verb|[1, 3.5]| &  \verb|sollya_lib_range_from_bounds(b, c);|\footnote[2]{\texttt{b} and \texttt{c} are \texttt{mpfr\_t} respectively containing the numbers $1$ and $3.5$.  Conversion is always exact.} \\ \hline
  \end{tabular}
\end{center}
\end{minipage}
\end{table}

\subsubsection{Lists}
\label{creating_lists}
There are actually two kinds of lists: regular lists (such as, e.g., \verb#[|1, 2, 3|]#) and semi-infinite lists (such as, e.g. \verb#[|1, 2, ...|]#). Withing the interactive tool, the ellipsis ``\texttt{...}'' can sometimes be used as a shortcut to define regular lists, e.g. \verb#[|1, 2, ..., 10|]#.

In the library, there is no symbol for the ellipsis, and there are two distinct types: one for regular lists and one for semi-infinite lists (called end-elliptic). Defining a regular list with an ellipsis is not possible in the library (except of course with \texttt{sollya\_lib\_parse\_string}).

Constructing regular lists is achieved through three functions:
\begin{itemize}
\item \verb|sollya_obj_t sollya_lib_list(sollya_obj_t[] L, int n)|: this function returns a new object that is a list the elements of which are copies of \verb|L[0]|, \dots, \verb|L[n-1]|.
\item \verb|sollya_obj_t sollya_lib_build_list(sollya_obj_t obj1, ...)|: this function accepts a variable number of arguments. The last one \textbf{must} be \texttt{NULL}. It ``eats up'' its arguments and returns a list containing the objects given as arguments. Since arguments are eaten up, they may be directly produced by function calls, without being stored in variables. A typical use could be
  \begin{center}
    \verb|sollya_lib_build_list(SOLLYA_CONST(1), SOLLYA_CONST(2), SOLLYA_CONST(3), NULL);|
  \end{center}
\item \verb|sollya_obj_t sollya_lib_v_build_list(va_list)|: the same as the previous functions, but with a \texttt{va\_list}.
\end{itemize}

Following the same conventions, end-elliptic lists can be constructed with the following functions:
\begin{itemize}
\item \verb|sollya_obj_t sollya_lib_end_elliptic_list(sollya_obj_t[] L, int n)|.
\item \verb|sollya_obj_t sollya_lib_build_end_elliptic_list(sollya_obj_t obj1, ...)|.
\item \verb|sollya_obj_t sollya_lib_v_build_end_elliptic_list(va_list)|.
\end{itemize}

\subsubsection{Structures}
\label{creating_structures}
\sollya structures are also available in library mode as any other \sollya object. The support for \sollya structures is however minimal and creating them might seem cumbersome\footnote{Users are encouraged to make well-founded feature requests if they feel the need for better support of structures.}. The only function available to create structures is
\begin{center}
\verb|int sollya_lib_create_structure(sollya_obj_t *res, sollya_obj_t s, char *name,|\\
\verb|                                sollya_obj_t val).                             |
\end{center}

This function returns a boolean integer: false means failure, and true means success. Three cases of success are possible. In all cases, the function creates a new object and stores it at the address referred to by \texttt{res}.
\begin{itemize}
\item  If \texttt{s} is \texttt{NULL}: \texttt{*res} is filled with a structure with only one field. This field is named after the string \texttt{name} and contains a copy of the object \texttt{val}.
\item If \texttt{s} is an already existing structure that has a field named after the string \texttt{name}: \texttt{*res} is filled with a newly created structure. This structure is the same as \texttt{s} except that the field corresponding to \texttt{name} contains a copy of \texttt{val}.
\item If \texttt{s} is an already existing structure that does \textbf{not} have a field named after the string \texttt{name}: \texttt{*res} is filled with a newly created structure. This structure is the same as \texttt{s} except that it has been augmented with a field corresponding to \texttt{name} and that contains a copy of \texttt{val}.
\end{itemize}
Please notice that \texttt{s} is not changed by this function: the structure stored in \texttt{*res} is a new one that does not refer to any of the components of \texttt{s}. As a consequence, one should not forget to explicitly clear \texttt{s} as well as \texttt{*res} when they become useless.

\subsection{Getting the type of an object}
Functions are provided that allow the user to test the type of a \sollya object. They are listed in Table~\ref{type_of_an_object}. They all return an \verb|int| interpreted as the boolean result of the test. Please note that from a typing point of view, a mathematical constant and a non-constant functional expression are both functions.

\begin{table}[htp]
  \caption{Testing the type of a \sollya object (Returns non-zero if true, 0 otherwise)}
  \label{type_of_an_object}
  \begin{center}
    \begin{tabular}{|l|}
      \hline
      \verb|sollya_lib_obj_is_function(obj)| \\
      \verb|sollya_lib_obj_is_range(obj)| \\
      \verb|sollya_lib_obj_is_string(obj)| \\
      \verb|sollya_lib_obj_is_list(obj)| \\
      \verb|sollya_lib_obj_is_end_elliptic_list(obj)| \\
      \verb|sollya_lib_obj_is_structure(obj)| \\
      \verb|sollya_lib_obj_is_error(obj)|\\[0.1cm]
      \hline
      \verb|sollya_lib_is_on(obj)|\phantom{\Large{$A^A$}} \\
      \verb|sollya_lib_is_off(obj)| \\
      \verb|sollya_lib_is_dyadic(obj)| \\
      \verb|sollya_lib_is_powers(obj)| \\
      \verb|sollya_lib_is_binary(obj)| \\
      \verb|sollya_lib_is_hexadecimal(obj)| \\
      \verb|sollya_lib_is_file(obj)| \\
      \verb|sollya_lib_is_postscript(obj)| \\
      \verb|sollya_lib_is_postscriptfile(obj)| \\
      \verb|sollya_lib_is_perturb(obj)| \\
      \verb|sollya_lib_is_round_down(obj)| \\
      \verb|sollya_lib_is_round_up(obj)| \\
      \verb|sollya_lib_is_round_towards_zero(obj)| \\
      \verb|sollya_lib_is_round_to_nearest(obj)| \\
      \verb|sollya_lib_is_honorcoeffprec(obj)| \\
      \verb|sollya_lib_is_true(obj)| \\
      \verb|sollya_lib_is_false(obj)| \\
      \verb|sollya_lib_is_void(obj)| \\
      \verb|sollya_lib_is_default(obj)| \\
      \verb|sollya_lib_is_decimal(obj)| \\
      \verb|sollya_lib_is_absolute(obj)| \\
      \verb|sollya_lib_is_relative(obj)| \\
      \verb|sollya_lib_is_fixed(obj)| \\
      \verb|sollya_lib_is_floating(obj)| \\
      \verb|sollya_lib_is_double_obj(obj)| \\
      \verb|sollya_lib_is_single_obj(obj)| \\
      \verb|sollya_lib_is_quad_obj(obj)| \\
      \verb|sollya_lib_is_halfprecision_obj(obj)| \\
      \verb|sollya_lib_is_doubleextended_obj(obj)| \\
      \verb|sollya_lib_is_double_double_obj(obj)| \\
      \verb|sollya_lib_is_triple_double_obj(obj)| \\
      \verb|sollya_lib_is_pi(obj)| \\
      \hline
  \end{tabular}
\end{center}
\end{table}

\subsection{Recovering the value of a range}
If a \verb|sollya_obj_t| is a range, it is possible to recover the values corresponding to the bounds of the range. The range can be recovered either as a \verb|mpfi_t| or as two \verb|mpfr_t| (one per bound). This is achieved with the following conversion functions:
\begin{itemize}
\item \verb|int sollya_lib_get_interval_from_range(mpfi_t res, sollya_obj_t arg)|,
\item \verb|int sollya_lib_get_bounds_from_range(mpfr_t res_left, mpfr_t res_right,|\\
      \verb|                                     sollya_obj_t arg)|.
\end{itemize}
They return a boolean integer: false means failure (i.e., if the \verb|sollya_obj_t| is not a range) and true means success. These functions follow the same conventions as those of the \verb|MPFR| and \verb|MPFI| libraries: the variables \verb|res|, \verb|res_left| and \verb|res_right| must be initialized beforehand, and are used to store the result of the conversion. Also, the functions \verb|sollya_lib_get_something_from_range| \textbf{do not change the internal precision} of \verb|res|, \verb|res_left| and \verb|res_right|. If the internal precision is sufficient to perform the conversion without rounding, then it is guaranteed to be exact. If, on the contrary, the internal precision is not sufficient, the actual bounds of the range stored in \verb|arg| will be rounded at the target precision using a rounding mode that ensures that the inclusion property remains valid, i.e. $\mathtt{arg} \subseteq \mathtt{res}$ (resp. $\mathtt{arg} \subseteq [\mathtt{res\_left}, \mathtt{res\_right}]$).

Function  \verb|int sollya_lib_get_prec_of_range(mp_prec_t *prec, sollya_obj_t arg)| stores at \verb|*prec| a precision that is guaranteed to be sufficient to represent the range stored in \verb|arg| without rounding. The returned value of this function is a boolean that follows the same convention as above. In conclusion, this is an example of a completely safe conversion:

\begin{center}\begin{minipage}{15cm}\begin{Verbatim}[frame=single]
  ...
  mp_prec_t prec;
  mpfr_t a, b;

  if (!sollya_lib_get_prec_of_range(&prec, arg)) {
    sollya_lib_printf("Unexpected error: %b is not a range\n", arg);
  }
  else {
    mpfr_init2(a, prec);
    mpfr_init2(b, prec);
    sollya_lib_get_bounds_from_range(a, b, arg);

    /* Now [a, b] = arg exactly */
  }
  ...
\end{Verbatim}
\end{minipage}\end{center}

\subsection{Recovering the value of a numerical constant or a constant expression}
From a conceptual point of view, a numerical constant is nothing but a very simple constant functional expression. Hence there is no difference in \sollya between the way constants and constant expressions are handled. The functions presented in this Section allow one to recover the value of such constants or constant expressions into usual C data types.

A constant expression being given, three cases are possible:
\begin{itemize}
\item When naively evaluated at the current global precision, the expression always leads to provably exact computations (i.e., at each step of the evaluation, no rounding happens). For instance numerical constants or simple expressions such as $(\exp(0)+5)/16$ fall in this category.
\item The constant expressions would be exactly representable at some precision but this is not straightforward from a naive evaluation at the current global precision. An example would be $\sin(\pi/3)/\sqrt{3}$ or even $1 + 2^{-\textrm{prec}-10}$.
\item Finally, a third possibility is that the value of the expression is not exactly representable at any precision on a binary floating-point number. Possible examples are $\pi$ or $1/10$.
\end{itemize}

From now on, we suppose that \verb|arg| is a \verb|sollya_obj_t| that contains a constant expression (or, as a particular case, a numerical constant). The general scheme followed by the conversion functions is the following: \sollya chooses an initial working precision greater than the target precision. If the value of \verb|arg| is easily proved to be exactly representable at that precision, \sollya first computes this exact value and then rounds it to the nearest number of the target format (ties-to-even). Otherwise, \sollya tries to adapt the working precision automatically in order to ensure that the result of the conversion is one of both floating-point numbers in the target format that are closest the exact value (a faithful rounding). A warning message indicates that the conversion is not exact and that a faithful rounding has been performed. In some cases really hard to evaluate, the algorithm can even fail to find a faithful rounding. In that case, too, a warning message is emitted indicating that the result of the conversion should not be trusted. Let us remark that these messages can be caught instead of being displayed and adapted handling can be provided by the user of the library at each emission of a warning (see Section~\ref{callbacks}).

The conversion functions are the following. They return a boolean integer: false means failure (i.e., \verb|arg| is not a constant expression) and true means success.
\begin{itemize}
\item \verb|int sollya_lib_get_constant_as_double(double *res, sollya_obj_t arg)|
\item \verb|int sollya_lib_get_constant_as_int(int *res, sollya_obj_t arg)|
\item \verb|int sollya_lib_get_constant_as_int64(int64_t *res, sollya_obj_t arg)|
\item \verb|int sollya_lib_get_constant_as_uint64(uint64_t *res, sollya_obj_t arg)|
\item \verb|int sollya_lib_get_constant(mpfr_t res, sollya_obj_t arg)|: the result of the conversion is stored in \verb|res|. Please note that \verb|res| must be initialized beforehand and that its internal precision is not modified by the algorithm.
\end{itemize}

Function  \verb|int sollya_lib_get_prec_of_constant(mp_prec_t *prec, sollya_obj_t arg)| tries to find a precision that would be sufficient to exactly represent the value of \verb|arg| without rounding. If it manages to find such a precision, it stores it at \verb|*prec| and returns true. If it does not manage to find such a precision, or if \verb|arg| is not a constant expression, it returns false and \verb|*prec| is left unchanged.

In conclusion, here is an example of use for converting a constant expression to a \verb|mpfr_t|:

\begin{center}\begin{minipage}{15cm}\begin{Verbatim}[frame=single]
  ...
  mp_prec_t prec;
  mpfr_t a;
  int test = 0;

  test = sollya_lib_get_prec_of_constant(&prec, arg);
  if (test) {
    mpfr_init2(a, prec);
    sollya_lib_get_constant(a, arg); /* Exact conversion */
  }
  else {
    mpfr_init2(a, 165); /* Initialization at some default precision */
    test = sollya_lib_get_constant(a, arg);
    if (!test) {
      sollya_lib_printf("Error: %b is not a constant expression\n", arg);
    }
  }
  ...
\end{Verbatim}
\end{minipage}\end{center}

\subsection{Converting a string from \sollya to C}
If \verb|arg| is a \verb|sollya_obj_t| that contains a string, that string can be recovered using
\begin{center}
\verb|int sollya_lib_get_string(char **res, sollya_obj_t arg)|.
\end{center}
If \verb|arg| really is a string, this function allocates enough memory on the heap to store the corresponding string, it copies the string at that newly allocated place, and sets \verb|*res| so that it points to it. The function returns a boolean integer: false means failure (i.e., \verb|arg| is not a string) and true means success.

Since this function allocates memory on the heap, this memory should manually be cleared by the user with \verb|sollya_lib_free| once it becomes useless.

\subsection{Converting a \sollya list to a C array}
The function that allows user to recover a C array of \verb|sollya_obj_t| objects from a \sollya list \verb|arg|~is:\\
\begin{center}
\verb|int sollya_lib_get_list_elements(sollya_obj_t **L, int *n, int *end_ell,|\\
\verb|                                 sollya_obj_t arg).                      |
\end{center}
Three cases are possible:
\begin{itemize}
\item If \verb|arg| is a regular list of length $N$, the function allocates memory on the heap for $N$ \verb|sollya_obj_t|, sets \verb|*L| so that it points to that memory segment, and copies each of the elements $N$ of \verb|arg| to \verb|(*L)[0]|, \dots, \verb|(*L)[N-1]|. Finally, it sets \verb|*n| to $N$, \verb|*end_ell| to zero and returns true. A particular case is when \verb|arg| is the empty list: everything is the same except that no memory is allocated and \verb|*L| is left unchanged.
\item If \verb|arg| is an end-elliptic list containing $N$ elements plus the ellipsis. The function allocates memory on the heap for $N$ \verb|sollya_obj_t|, sets \verb|*L| so that it points to that memory segment, and copies each of the elements $N$ of \verb|arg| at \verb|(*L)[0]|, \dots, \verb|(*L)[N-1]|. Finally, it sets \verb|*n| to $N$, \verb|*end_ell| to a non-zero value and returns true. The only difference between a regular list and an end-elliptic list containing the same elements is hence that \verb|*end_ell| is set to a non-zero value in the latter.
\item If \verb|arg| is neither a regular nor an end-elliptic list, \verb|*L|, \verb|*n| and \verb|*end_ell| are left unchanged and the function returns false.
\end{itemize}

\subsection{Recovering the contents of a \sollya structure}
If \verb|arg| is a \verb|sollya_obj_t| that contains a structure, the contents of a given field can be recovered using
\begin{center}
\verb|int sollya_lib_get_element_in_structure(sollya_obj_t *res, char *name,|\\
\verb|                                        sollya_obj_t arg).             |
\end{center}
If \verb|arg| really is a structure and if that structure has a field named after the string \verb|name|, this function copies the contents of that field into the \sollya object \verb|*res|. The function returns a boolean integer: false means failure (i.e., if \verb|arg| is not a structure or if it does not have a field named after \verb|name|) and true means success.

It is also possible to get all the field names and their contents. This is achieved through the function
\begin{center}
\verb|int sollya_lib_get_structure_elements(char ***names, sollya_obj_t **objs, int *n,|\\
\verb|                                      sollya_obj_t arg).                          |
\end{center}
If \verb|arg| really is a structure, say with $N$ fields called ``fieldA'', \dots, ``fieldZ'', this functions sets \verb|*n| to~$N$, allocates and fills an array of $N$ strings and sets \verb|*names| so that it points to that segment of memory (hence \verb|(*names)[0]| is the string ``fieldA'', \dots, \verb|(*names)[N-1]| is the string ``fieldZ''). Moreover, it allocates memory for $N$ \verb|sollya_obj_t|, sets \verb|*objs| so that it points on that memory segment, and copies the contents of each of the $N$ fields at \verb|(*objs)[0]|, \dots, \verb|(*objs)[N-1]|. Finally it returns true. If \verb|arg| is not a structure, the function simply returns false without doing anything. Please note that since \verb|*names| and \verb|*objs| point to memory segments that have been dynamically allocated, they should manually be cleared by the user with \verb|sollya_lib_free| once they become useless.

\subsection{Decomposing a functional expression}
If a \texttt{sollya\_obj\_t} contains a functional expression, one can decompose the expression tree using the following functions. These functions all return a boolean integer: true in case of success (i.e., if the \texttt{sollya\_obj\_t} argument really contains a functional expression) and false otherwise.

\begin{table}[htp]
\caption{List of values defined in type \texttt{sollya\_base\_function\_t}}
\label{list_of_sollya_base_function_t}
\begin{center}
  \begin{tabular}{|c|c|c|}
    \hline
  \phantom{\Large{$A^A$}}  \verb|SOLLYA_BASE_FUNC_COS|\phantom{\Large{$A^A$}} &  \verb|SOLLYA_BASE_FUNC_DOUBLE| & \verb|SOLLYA_BASE_FUNC_LOG| \\
  \verb|SOLLYA_BASE_FUNC_ACOS|  &  \verb|SOLLYA_BASE_FUNC_DOUBLEDOUBLE|      & \verb|SOLLYA_BASE_FUNC_LOG_2| \\
  \verb|SOLLYA_BASE_FUNC_ACOSH| &  \verb|SOLLYA_BASE_FUNC_DOUBLEEXTENDED|    & \verb|SOLLYA_BASE_FUNC_LOG_10|\\
  \verb|SOLLYA_BASE_FUNC_COSH|  &  \verb|SOLLYA_BASE_FUNC_TRIPLEDOUBLE|      & \verb|SOLLYA_BASE_FUNC_LOG_1P| \\
  \verb|SOLLYA_BASE_FUNC_SIN|   &  \verb|SOLLYA_BASE_FUNC_HALFPRECISION|     & \verb|SOLLYA_BASE_FUNC_EXP| \\
  \verb|SOLLYA_BASE_FUNC_ASIN|  &  \verb|SOLLYA_BASE_FUNC_SINGLE|            & \verb|SOLLYA_BASE_FUNC_EXP_M1| \\
  \verb|SOLLYA_BASE_FUNC_ASINH| &  \verb|SOLLYA_BASE_FUNC_QUAD|              & \verb|SOLLYA_BASE_FUNC_NEG|  \\
  \verb|SOLLYA_BASE_FUNC_SINH|  &  \verb|SOLLYA_BASE_FUNC_FLOOR|             & \verb|SOLLYA_BASE_FUNC_SUB| \\
  \verb|SOLLYA_BASE_FUNC_TAN|   &  \verb|SOLLYA_BASE_FUNC_CEIL|              & \verb|SOLLYA_BASE_FUNC_ADD| \\
  \verb|SOLLYA_BASE_FUNC_ATAN|  &  \verb|SOLLYA_BASE_FUNC_NEARESTINT|        & \verb|SOLLYA_BASE_FUNC_MUL| \\
  \verb|SOLLYA_BASE_FUNC_ATANH| &  \verb|SOLLYA_BASE_FUNC_LIBRARYCONSTANT|   & \verb|SOLLYA_BASE_FUNC_DIV|   \\
  \verb|SOLLYA_BASE_FUNC_TANH|  &  \verb|SOLLYA_BASE_FUNC_LIBRARYFUNCTION|   & \verb|SOLLYA_BASE_FUNC_POW|\\
  \verb|SOLLYA_BASE_FUNC_ERF|   &  \verb|SOLLYA_BASE_FUNC_PROCEDUREFUNCTION| & \verb|SOLLYA_BASE_FUNC_SQRT|    \\
  \verb|SOLLYA_BASE_FUNC_ERFC|  &  \verb|SOLLYA_BASE_FUNC_FREE_VARIABLE|     & \verb|SOLLYA_BASE_FUNC_PI|     \\
  \verb|SOLLYA_BASE_FUNC_ABS|   &  \verb|SOLLYA_BASE_FUNC_CONSTANT|          & \\
\hline
  \end{tabular}
\end{center}
\end{table}
\begin{itemize}
\item \verb|int sollya_lib_get_function_arity(int *n, sollya_obj_t f)|: it stores the arity of the head function in \texttt{f} at the address referred to by \texttt{n}. Currently, the mathematical functions handled in \sollya are at most dyadic. Mathematical constants are considered as 0-adic functions.
\item \verb|int sollya_lib_get_head_function(sollya_base_function_t *type, sollya_obj_t f)|:\\
it stores the type of \texttt{f} at the address referred to by \texttt{type}. The \texttt{sollya\_base\_function\_t} is an enum type listing all possible cases (see Table~\ref{list_of_sollya_base_function_t}).
\item \verb|int sollya_lib_get_subfunctions(sollya_obj_t f, int *n, ...)|: let us denote by \texttt{g\_1}, \dots, \texttt{g\_k} the arguments following the argument \texttt{n}. They must be of type \verb|sollya_obj_t *|. The function stores the arity of \texttt{f} at the address referred to by \texttt{n}. Suppose that \texttt{f} contains an expression of the form $f_0(f_1,\dots,f_s)$. For each $i$ from $1$ to $s$, the expression corresponding to $f_i$ is stored at the address referred to by \texttt{g\_i}, unless one of the \texttt{g\_i} is \texttt{NULL} in which case the function returns when encountering it. In practice, it means that the user should always put \texttt{NULL} as last argument, in order to prevent the case when they would not provide enough variables \texttt{g\_i}. They can check afterwards that they provided enough variables by checking the value contained at the address referred to by \texttt{n}. If the user does not put \texttt{NULL} as last argument and do not provide enough variables \texttt{g\_i}, the algorithm will continue storing arguments at random places in the memory (on the contrary, providing more arguments than necessary does not harm: useless arguments are simply ignored and left unchanged). In the case when $f_0$ is a library function, a library constant or a procedure function, and if the user provides a non-\texttt{NULL} argument \texttt{g\_t} after \texttt{g\_s}, additionnal information is returned in the remaining arguments:
  \begin{itemize}
  \item If $f_0$ is a library function, a sollya object corresponding to the expression $f_0(x)$ is stored at the address referred to by \texttt{g\_t}. Please notice that $f_0$ might not be itself directly bound to a library code: for instance, $f_0$ could be the derivative of some function defined by a library bounding.
  \item If $f_0$ is a procedure function, TODO.
  \item If $f_0$ is a library constant, TODO.
  \end{itemize}
Please note that the objects that have been stored in variables \texttt{g\_i} must manually be cleared once they become useless.

\item \verb|int sollya_lib_v_get_subfunctions(sollya_obj_t f, int *n, va_list va)|: the same as the previous function, but with a \texttt{va\_list} argument.
\item \verb|int sollya_lib_decompose_function(sollya_obj_t f, sollya_base_function_t *type,|\\
      \verb|                                  int *n, ...)|:\\
this function is a all-in-one function equivalent to using \verb|sollya_lib_get_head_function| and \verb|sollya_lib_get_subfunctions| in only one function call.
\item \verb|int sollya_lib_v_decompose_function(sollya_obj_t f, sollya_base_function_t *type,|\\
      \verb|                                    int *n, va_list va)|:\\
the same as the previous function, but with a \texttt{va\_list}.
\end{itemize}

As an example of use of these functions, the following code returns $1$ if \texttt{f} denotes a functional expression made only of constants (i.e., without the free variable), and returns $0$ otherwise:

\begin{center}\begin{minipage}{15cm}\begin{Verbatim}[frame=single]
#include <sollya.h>

/* Note: we suppose that the library has already been initialized */
int is_made_of_constants(sollya_obj_t f) {
  sollya_obj_t tmp1 = NULL;
  sollya_obj_t tmp2 = NULL;
  int n, r, res;
  sollya_base_function_t type;

  r = sollya_lib_decompose_function(f, &type, &n, &tmp1, &tmp2, NULL);
  if (!r) { sollya_lib_printf("Not a mathematical function\n"); res = 0; }
  else if (n >= 3) {
    sollya_lib_printf("Unexpected error: %b has more than two arguments.\n", f);
    res = 0;
  }
  else {
    switch (type) {
      case SOLLYA_BASE_FUNC_FREE_VARIABLE: res = 0; break;
      case SOLLYA_BASE_FUNC_PI: res = 1; break;
      case SOLLYA_BASE_FUNC_CONSTANT: res = 1; break;
      case SOLLYA_BASE_FUNC_LIBRARYCONSTANT: res = 1; break;
      default:
        res = is_made_of_constants(tmp1);
        if ((res) && (n==2)) res = is_made_of_constants(tmp2);
    }
  }

  if (tmp1) sollya_lib_clear_obj(tmp1);
  if (tmp2) sollya_lib_clear_obj(tmp2);

  return res;
}
\end{Verbatim}
\end{minipage}\end{center}

\subsection{Faithfully evaluate a functional expression}
Let us suppose that \verb|f| is a functional expression and \verb|a| is a numerical value or a constant expression. One of the very convenient features of the interactive tool is that the user can simply write \verb|f(a)| at the prompt: the tool automatically adapts its internal precision in order to compute a value that is a faithful rounding (at the current tool precision) of the true value $f(a)$. Sometimes it does not achieve to find a faithful rounding, but in any case, if the result is not proved to be exact, a warning is displayed explaining how confident one should be with respect to the returned value. This feature is made available within the library with the two following functions:
\begin{itemize}
\item \verb|sollya_fp_result_t|\\
      \verb|  sollya_lib_evaluate_function_at_constant_expression(mpfr_t res, sollya_obj_t f,|\\
      \verb|                                                      sollya_obj_t a,|\\
      \verb|                                                      mpfr_t *cutoff)|,
\item \verb|sollya_fp_result_t|\\
      \verb|  sollya_lib_evaluate_function_at_point(mpfr_t res, sollya_obj_t f,|\\
      \verb|                                        mpfr_t a, mpfr_t *cutoff)|.
\end{itemize}
In the former, the argument \verb|a| is any \verb|sollya_obj_t| containing a numerical constant or a constant expression, while in the latter \verb|a| is a constant already stored in a \verb|mpfr_t|. These functions store the result in \verb|res| and return a \verb|sollya_fp_result_t| which is an enum type described in Table~\ref{list_of_sollya_fp_result_t}. In order to understand the role of the \verb|cutoff| parameter and the value returned by the function, it is necessary to describe the algorithm in a nutshell:\\~\\
\rule{\textwidth}{0.5px}
\begin{enumerate}
\item[~]\textbf {Input:} a functional expression \verb|f|, a constant expression \verb|a|, a target precision $q$, a parameter $\varepsilon$.
\item Choose an initial working precision $p$.
\item Evaluate \verb|a| with interval arithmetic, performing the computations at precision $p$.
\item Replace the occurrences of the free variable in \verb|f| by the interval obtained at step 2. Evaluate the resulting expression with interval arithmetic, performing the computations at precision $p$. This yields an interval $I = [x,y]$.
\item Examine the following cases successively ($\mathrm{RN}$ denotes rounding to nearest at precision~$q$):
  \begin{enumerate}
  \item If $\mathrm{RN}(x) = \mathrm{RN}(y)$, set \verb|res| to that value and return.
  \item If $I$ does not contain any floating-point number at precision $q$, set \verb|res| to one of both floating-point numbers enclosing $I$ and return.
  \item If $I$ contains exactly one floating-point number at precision $q$, set \verb|res| to that number and return.
  \item If all numbers in $I$ are smaller than $\varepsilon$ in absolute value, then set \verb|res| to $0$ and return.
  \item If $p$ has already been increased many times, then set \verb|res| to the middle of $I$ and return.
  \item Otherwise, increase $p$ and go back to step 2.
  \end{enumerate}
\end{enumerate}
\vspace{-0.2cm}
\rule{\textwidth}{0.5px}\\[0.7cm]
The target precision $q$ is chosen to be the precision of the \verb|mpfr_t| variable \verb|res|. The parameter $\varepsilon$ corresponds to the parameter \verb|cutoff|. The reason why \verb|cutoff| is a pointer is that, most of the time, the user may not want to provide it, and using a pointer makes it possible to pass \verb|NULL| instead. So, if \verb|NULL| is given, $\varepsilon$ is set to $0$. If \verb|cutoff| is not \verb|NULL|, the absolute value of \verb|*cutoff| is used as value for $\varepsilon$. Using a non-zero value for $\varepsilon$ can be useful when one does not care about the precise value of $f(a)$ whenever its absolute value is below a given threshold. Typically, if one wants to compute the maximum of $|f(a_1)|$, \dots, $|f(a_n)|$, it is not necessary to spend too much effort on the computation of $|f(a_i)|$ if one already knows that it is smaller than $\varepsilon = \max \{|f(a_1)|,\dots,|f(a_{i-1})|\}$.
\begin{table}[htp]
\caption{List of values defined in type \texttt{sollya\_fp\_result\_t}}
\label{list_of_sollya_fp_result_t}
\renewcommand{\footnoterule}{} % suppress the separation line between main text and footnotes.
\begin{minipage}{\textwidth}
\hspace{-1cm}
\begin{tabular}{|l|p{7cm}|}
  \multicolumn{1}{c}{~}\\
    \hline
    \hfil\phantom{\Large{$A^A$}}Value\phantom{\Large{$A^A$}}\hfil & \hfil Meaning\hfil \\ \hline
  \verb|SOLLYA_FP_OBJ_NO_FUNCTION| & \verb|f| is not a functional expression.\phantom{\Large{$A^A$}}\\[0.3cm]
  \verb|SOLLYA_FP_EXPRESSION_NOT_CONSTANT| & \verb|a| is not a constant expression.\\[0.3cm]
 \verb|SOLLYA_FP_FAILURE| & The algorithm ended up at step (e) and $I$ contained NaN. This typically happens when $a$ is not in the definition domain of $f$.\\[0.3cm]
  \verb|SOLLYA_FP_CUTOFF_IS_NAN| & \verb|cutoff| was not \verb|NULL| and the value of \verb|*cutoff| is NaN.\\[0.3cm]
  \verb|SOLLYA_FP_INFINITY| & The algorithm ended up at step (a) with $I$ of the form $[+\infty, +\infty]$ or $[-\infty, -\infty]$. Hence $f(a)$ is proved to be an exact infinity.\\[0.3cm]
  \verb|SOLLYA_FP_PROVEN_EXACT| & The algorithm ended up at step (a) with a finite value and $x = \mathrm{RN}(x) = \mathrm{RN}(y)=y$.\\[0.3cm]
  \verb|SOLLYA_FP_CORRECTLY_ROUNDED_PROVEN_INEXACT| & The algorithm ended up at step (b) with a finite value. and $\mathtt{res} \le x \le y$ or $x \le y \le \mathtt{res}$.\\[0.3cm]
  \verb|SOLLYA_FP_CORRECTLY_ROUNDED| & The algorithm ended up at step (a) with a finite value and $x \le \mathtt{res} \le y$.$~^a$\footnotetext[1]{Please notice that this means that the algorithm did not manage to conclude whether the result is exact or not. However, it might have been able to conclude if the working precision had been increased.}\\[0.3cm]
  \verb|SOLLYA_FP_FAITHFUL_PROVEN_INEXACT| & The algorithm ended up at step (b) with a finite value.\\[0.3cm]
  \verb|SOLLYA_FP_FAITHFUL| & The algorithm ended up at step (c) with a finite value.$~^a$\\[0.3cm]
  \verb|SOLLYA_FP_BELOW_CUTOFF| & The algorithm ended up at step (d).\\[0.3cm]
  \verb|SOLLYA_FP_NOT_FAITHFUL_ZERO_CONTAINED_BELOW_THRESHOLD| & The algorithm ended up at step (e) and $I$ was of the form $[-\delta_1,\,\delta_2]$ where $0 < \delta_i \ll 1$ (below some threshold of the algorithm). This typically happens when $f(a)$ exactly equals zero, but the algorithm does not manage to prove this exact equality.\\[0.3cm]
  \verb|SOLLYA_FP_NOT_FAITHFUL_ZERO_CONTAINED_NOT_BELOW_THRESHOLD| & The algorithm ended up at step (e) with an interval $I$ containing $0$ but too large to fall in the above case.$~^b$\footnotetext[2]{In general, this should be considered as a case of failure and the value stored in \texttt{res} might be completely irrelevant.}\\[0.3cm]
  \verb|SOLLYA_FP_NOT_FAITHFUL_ZERO_NOT_CONTAINED| & The algorithm ended up at step (e) with an interval $I$ that does not contain $0$.$~^b$\\[0.3cm]
  \verb|SOLLYA_FP_NOT_FAITHFUL_INFINITY_CONTAINED| & The algorithm ended up at step (e) and (at least) one of the bounds of $I$ was infinite. This typically happens when the limit of $f(x)$ when $x$ goes to $a$ is infinite.\\[0.3cm]
\hline
  \end{tabular}
\end{minipage}
\end{table}

In the interactive tool, it is also possible to write \verb|f(a)| when \verb|a| contains an interval: \sollya performs the evaluation using an enhanced interval arithmetic, e.g., using L'Hopital's rule to produce finite (yet valid of course) enclosures even in cases when $f$ exhibits removable singularities (for instance $\sin(x)/x$ over an interval containing $0$). This feature is achieved in the library with the function
\begin{center}
\verb|int sollya_lib_evaluate_function_over_interval(mpfi_t res, sollya_obj_t f, mpfi_t a).|
\end{center}

This function returns a boolean integer: false means failure (i.e., \verb|f| is not a functional expression), in which case \verb|res| is left unchanged, and true means success, in which case \verb|res| contains the result of the evaluation. The function might succeed, and yet \verb|res| might contain something useless such as an unbounded interval or even $[\textrm{NaN},\textrm{NaN}]$ (this happens for instance when \verb|a| contains points that lie in the interior of the complement of the definition domain of \verb|f|). It is the user's responsibility to check afterwards whether the computed interval is bounded, unbounded or NaN.

\subsection{Name of the free variable}
The default name for the free variable is the same in the library and in the interactive tool: it is \texttt{\_x\_}. In the interactive tool, this name is automatically changed at the first use of an undefined symbol. Accordingly in library mode, if an object is defined by \texttt{sollya\_lib\_parse\_string} with an expression containing an undefined symbol, that symbol will become the free variable name if it has not already been changed before. But what if one does not use \texttt{sollya\_lib\_parse\_string} (because it is not efficient) but one wants to change the name of the free variable? The name can be changed with \texttt{sollya\_lib\_name\_free\_variable("some\_name")}.

It is possible to get the current name of the free variable with \texttt{sollya\_lib\_get\_free\_variable\_name()}. This function returns a \texttt{char *} containing the current name of the free variable. Please note that this \texttt{char *} is dynamically allocated on the heap and should be cleared after its use with \texttt{sollya\_lib\_free()} (see below).

\subsection{Commands and functions}
\label{library_commands_and_functions}
Besides some exceptions, every command and every function available in the \sollya interactive tool has its equivalent (with a very close syntax) in the library. Section~\ref{commandsAndFunctions} of the present documentation gives the library syntax as well as the interactive tool syntax of each commands and functions. The same information is available within the interactive tool by typing \texttt{help some\_command}. So if one knows the name of a command or function in the interactive tool, it is easy to recover its library name and signature.

A particular point is worth mentioning: some functions of the tool such as \texttt{remez} for instance have a variable number of arguments. For instance, one might call \texttt{remez(exp(x), 4, [0,1])} or \texttt{remez(1, 4, [0,1], 1/exp(x))}. This feature is rendered in the C library by the use of variadic functions (functions with an arbitrary number of arguments), as they are permitted by the C standard. The notable difference is that there must \textbf{always be an explicit NULL argument} at the end of the function call. Hence one can write \texttt{sollya\_lib\_remez(a, b, c, NULL)} or \texttt{sollya\_lib\_remez(a, b, c, d, NULL)}. It is very easy to forget the \texttt{NULL} argument and use for instance \texttt{sollya\_lib\_remez(a, b, c)}. This is \textbf{completely wrong} because the memory will be read until a \texttt{NULL} pointer is found. In the best case, this will lead to an error or a result obviously wrong, but it could also lead to subtle, not-easy-to-debug errors. The user is advised to be particularly careful with respect to this point.

Each command or function accepting a variable number of arguments comes in a \texttt{sollya\_lib\_v\_} version accepting a \texttt{va\_list} parameter containing the list of optional arguments. For instance, one might write a function that takes as arguments a function $f$, an interval $I$, optionally a weight function $w$, optionally a quality parameter $q$. That function would display the minimax obtained when approximating $f$ over $I$ (possibly with weight $w$ and quality $q$) by polynomials of degree $n=2$ to $20$. So, that function would get a variable number of arguments (i.e. a \texttt{va\_list} in fact) and pass them straight to remez. In that case, one needs to use the \texttt{v\_remez} version, as the following code shows:

\begin{center}\begin{minipage}{15cm}\begin{Verbatim}[frame=single]
#include <sollya.h>
#include <stdarg.h>

/* Note: we suppose that the library has already been initialized */
void my_function(sollya_obj_t f, sollya_obj_t I, ...) {
  sollya_obj_t n, res;
  int i;
  va_list va;

  for(i=2;i<=20;i++) {
    n = SOLLYA_CONST(i);
    va_start(va, I);
    res = sollya_lib_v_remez(f, n, I, va);
    sollya_lib_printf("Approximation of degree %b is %b\n", n, res);
    va_end(va);
    sollya_lib_clear_obj(n);
    sollya_lib_clear_obj(res);
  }

  return;
}
\end{Verbatim}
\end{minipage}\end{center}

\subsection{Warning messages in library mode}
\label{callbacks}
The philosophy of \sollya is ``whenever something is not exact, explicitly warn about that''. This is a nice feature since this ensures that the user always perfectly knows the degree of confidence they can have in a result (is it exact? or only faithful? or even purely numerical, without any warranty?) However, it is sometimes desirable to hide some (or all) of these messages. This is especially true in library mode where messages coming from \sollya are intermingled with the messages of the main program. The library hence provides a specific mechanism to catch all messages emitted by the \sollya core and handle each of them specifically: installation of a callback for messages.

Before describing the principle of the message callback, it seems appropriate to recall that several mechanisms are available in the interactive tool to filter the messages emitted by \sollya. These mechanisms are also available in library mode for completeness. When a message is emitted, it has two characteristics: a verbosity level and an id (a number uniquely identifying the message). After it has been emitted, it passes through the following steps where it can be filtered. If it has not been filtered (and only in this case) it is displayed.
\begin{enumerate}
\item If the verbosity level of the message if greater than the value of the environment variable \verb|verbosity|, it is filtered.
\item If the environment variable \verb|roundingwarnings| is set to \verb|off| and if the message informs the user that a rounding occurred, it is filtered.
\item If the id of the message has been registered with the \verb|suppressmessage| command, the message is filtered.
\item If a message callback has been installed and if the message has not been previously filtered, it is handled by the callback, which decides to filter it or display it.
\end{enumerate}

A message callback is a function of the form \verb|int my_callback(sollya_msg_t msg)|. It receives as input an object representing the message, performs whatever treatment seems appropriate and returns an integer interpreted as a boolean. If the returned value is false, the message is not displayed. If, on the contrary, the returned value is true, the message is displayed as usual. By default, no callback is installed and all messages are displayed. To install a callback, use \verb|sollya_lib_install_msg_callback(my_callback)|. Please remember that, if a message is filtered because of one of the three other mechanisms, it will never be transmitted to the callback. Hence, in library mode, if one wants to catch every single message through the callback, one should set the value of \verb|verbosity| to \verb|MAX_INT|, set \verb|roundingwarnings| to \verb|on| (this is the default anyway) and one should not use the \verb|suppressmessage| mechanism.

It is possible to come back to the default behavior, using \verb|sollya_lib_uninstall_msg_callback()|. Please notice that callbacks do not stack over each other: i.e., if some callback \verb|callback1| is already installed, and one installs another one \verb|callback2|, then the effect of \verb|sollya_lib_uninstall_msg_callback()| is to come back to the default behavior, \textbf{and not} to come back to callback \verb|callback1|.

Both \verb|sollya_lib_install_msg_callback| and \verb|sollya_lib_uninstall_msg_callback| return an integer interpreted as a boolean: false means failure and true means success.

It is possible to get the currently installed callback using \verb|sollya_lib_get_msg_callback()|. This function returns a function pointer to the current callback --- i.e., a \verb|int (*)(sollya_msg_t)| --- if a callback is installed, or \verb|NULL| if no callback is currently installed.

Currently the type \verb|sollya_msg_t| has only two accessors:
\begin{itemize}
\item \verb|int sollya_lib_get_msg_id(sollya_msg_t msg)| returns an integer that identifies the type of the message. The message types are listed in the file \verb|sollya-messages.h|. Please note that this file not only lists the possible identifiers but only defines meaningful names to each possible message number (e.g., \verb|SOLLYA_MSG_UNDEFINED_ERROR| is an alias for the number $2$ but is more meaningful to understand what the message is about). It is recommended to use these names instead of numerical values.
\item \verb|char *sollya_lib_msg_to_text(sollya_msg_t msg)| returns a generic string briefly summarizing the contents of the message. Please note that this \verb|char *| is dynamically allocated on the heap and should manually be cleared with \verb|sollya_lib_free| when it becomes useless.
\end{itemize}
In the future, other accessors could be added (to get the verbosity level at which the message has been emitted, to get data associated with the message, etc.) The developers of \sollya are open to suggestions and feature requests on this subject.

As an illustration let us give a few examples of possible use of callbacks:
\paragraph{Example 1:} A callback that filters everything.
\begin{center}
\begin{minipage}{15cm}\begin{Verbatim}[frame=single]
int hide_everything(sollya_msg_t msg) {
  return 0;
}
\end{Verbatim}
\end{minipage}\end{center}

\paragraph{Example 2:} filter everything but the messages indicating that a comparison is uncertain.
\begin{center}
\begin{minipage}{15cm}\begin{Verbatim}[frame=single]
int keep_comparison_warnings(sollya_msg_t msg) {
  switch(sollya_lib_get_msg_id(msg)) {
    case SOLLYA_MSG_TEST_RELIES_ON_FP_RESULT_THAT_IS_NOT_FAITHFUL:
    case SOLLYA_MSG_TEST_RELIES_ON_FP_RESULT:
    case SOLLYA_MSG_TEST_RELIES_ON_FP_RESULT_FAITHFUL_BUT_UNDECIDED:
    case SOLLYA_MSG_TEST_RELIES_ON_FP_RESULT_FAITHFUL_BUT_NOT_REAL:
      return 1;
    default:
      return 0;
  }
}
\end{Verbatim}
\end{minipage}\end{center}

\paragraph{Example 3:} ensuring perfect silence for a particular function call (uses the callback defined in Example~1).
\begin{center}
\begin{minipage}{15cm}\begin{Verbatim}[frame=single]
...
int (*)(sollya_msg_t) old_callback = sollya_lib_get_msg_callback();
sollya_lib_uninstall_msg_callback();
sollya_lib_install_msg_callback(hide_everything);
  /* Here takes place the function call that must be completely silent */
sollya_lib_uninstall_msg_callback();
if (old_callback) sollya_lib_install_msg_callback(old_callback);
...
\end{Verbatim}
\end{minipage}\end{center}

\paragraph{Example 4:} changing the value of some flag when some message is emitted.
\begin{center}
\begin{minipage}{15cm}\begin{Verbatim}[frame=single]
int flag_double_rounding = 0;

int set_flag_on_problem(sollya_msg_t msg) {
  switch(sollya_lib_get_msg_id(msg)) {
    case SOLLYA_MSG_DOUBLE_ROUNDING_ON_CONVERSION:
      flag  = 1;
  }
  return 1;
}

...

int main() {
  ...
  sollya_lib_init();
  sollya_lib_install_msg_callback(set_flag_on_problem);
  ...
}

\end{Verbatim}
\end{minipage}\end{center}

More involved examples are possible: for instance, instead of setting a flag, it is possible to keep in some variable what the last message was. One may even implement a stack mechanism and store the messages in a stack, in order to handle them later. One may also decide to raise an exception flag on a particular message, etc.

\subsection{Using \sollya in a program that has its own allocation functions}
\label{customMemoryFunctions}
\sollya uses its own allocation functions: as a consequence, pointers that have been allocated by \sollya functions must be freed using \verb|sollya_lib_free| instead of the usual \verb|free| function. Another consequence is that \sollya registers its own allocation functions to the \verb|GMP| library, using the mechanism provided by \verb|GMP|, so that \verb|GMP| also uses \sollya allocation functions behind the scene, when the user performs a call to, e.g., \verb|mpz_init|, \verb|mpfr_init2|, etc.

In general, this is completely harmless and the user might even not notice it. However, this is a problem if \sollya is used in a program that also uses its own allocation functions and that has already registered these functions to \verb|GMP|. Actually:
\begin{itemize}
\item If the main program has already registered allocation functions to \verb|GMP| and if \sollya is naively initialized with \verb|sollya_lib_init()|, \sollya will register its own allocation functions, thus overriding the previously registered functions.
\item If the user initializes first \sollya, and then registers its own allocation functions to \verb|GMP|, the exact opposite happens: \sollya allocation functions are overridden by those of the user, and this will likely cause \sollya to crash (or worst, silently behave not reliably).
\end{itemize}

In order to solve this issue, \sollya provides a chaining mechanism that we are now going to describe. The idea is the following: suppose that the main program should use a function \verb|custom_malloc|. The user should not use \verb|mp_set_memory_functions| as usual, but should instead initialize \sollya with the initializing function described above. This will cause \sollya to register an allocation function \verb|sollya_lib_malloc| to \verb|GMP|. This function overloads \verb|custom_malloc|: when called, it uses \verb|custom_malloc| to perform the actual allocation and does nothing else but some internal accounting and verification for that allocation. To repeat, the actual allocation is done by \verb|custom_malloc|; hence from the point of view of the user, the mechanism is completely transparent and equivalent to directly registering \verb|custom_malloc| to \verb|GMP|. The same holds for all other allocation functions: in particular, this is true for \verb|free| as well: if a function \verb|custom_free| is given at the initialization of \sollya, then the function \verb|sollya_lib_free| eventually uses \verb|custom_free| to free the memory. 

The initialization function providing this mechanism is:
\begin{center}
\verb|int sollya_lib_init_with_custom_memory_functions(                   |\\
\verb|         void *(*custom_malloc)(size_t),                            |\\
\verb|         void *(*custom_calloc)(size_t, size_t),                    |\\
\verb|         void *(*custom_realloc)(void *, size_t),                   |\\
\verb|         void (*custom_free)(void *),                               |\\
\verb|         void *(*custom_realloc_with_size)(void *, size_t, size_t),|\\
\verb|         void (*custom_free_with_size)(void *, size_t)).            |
\end{center}
None of the arguments is mandatory: if the user does not want to provide an argument, they may use \verb|NULL| as a placeholder for that argument. In that case, the corresponding \sollya default function will be used. Indeed, the default initializing function \verb|sollya_lib_init()| is just an alias to \verb|sollya_lib_init_with_custom_memory_functions(NULL, NULL, NULL, NULL, NULL, NULL)|.

Please notice, that if \verb|custom_malloc| is provided, then the function \verb|sollya_lib_malloc| will be defined as an overloaded version of \verb|custom_malloc|. Hence, \verb|custom_malloc| will eventually be used for all the allocations performed by \sollya (including the allocation of memory for its own purpose). This is true also for \verb|custom_calloc|, \verb|custom_realloc| and \verb|custom_free|. However, this is not the case for \verb|custom_realloc_with_size| and \verb|custom_free_with_size|: these functions are only required for the registration to \verb|GMP| and are not used by \sollya itself (except of course when \sollya allocates function through a call to a \verb|GMP|, \verb|MPFR| or \verb|MPFI| function). Thus, to sum up:
\begin{itemize}
\item If the user only wants to register their own functions to \verb|GMP| through \sollya, they  only need to provide \verb|custom_malloc|, \verb|custom_realloc_with_size| and \verb|custom_free_with_size| at the initialization of \sollya (actually an overloaded version will be registered to \verb|GMP| but this is transparent for the user, as explained above).
\item If the user also wants \sollya to use their custom allocation functions for all allocations of memory by \sollya, then they also need to provide \verb|custom_calloc|, \verb|custom_realloc| and \verb|custom_free|.
\end{itemize}

Of course, even if the user registers \verb|custom_malloc|, \verb|custom_free|, etc., at the initialization of \sollya, they stay free to use them for their own allocation needs: only allocations performed by \verb|GMP| (and consequently \verb|MPFR| and \verb|MPFI|) and allocations performed by \sollya have to use the chaining mechanism. However, for the convenience of the user, the library also provides access to the allocation functions of \sollya. They are the following:
\begin{itemize}
\item \verb|void sollya_lib_free(void *)|
\item \verb|void *sollya_lib_malloc(size_t)|
\item \verb|void *sollya_lib_calloc(size_t, size_t)|
\item \verb|void *sollya_lib_realloc(void *, size_t)|.
\end{itemize}
No access to the overloaded version of \verb|custom_realloc_with_size| and \verb|custom_free_with_size| is provided, but if the user really wants to retrieve them, they can do it with \verb|mp_get_memory_functions| since they are registered to \verb|GMP|.

\end{document}
