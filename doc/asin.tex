\subsection{asin}
\label{labasin}
\noindent Name: \textbf{asin}\\
\phantom{aaa}the arcsine function.\\[0.2cm]
\noindent Library names:\\
\verb|   sollya_obj_t sollya_lib_asin(sollya_obj_t)|\\
\verb|   sollya_obj_t sollya_lib_build_function_asin(sollya_obj_t)|\\
\verb|   #define SOLLYA_ASIN(x) sollya_lib_build_function_asin(x)|\\[0.2cm]
\noindent Description: \begin{itemize}

\item \textbf{asin} is the inverse of the function \textbf{sin}: \textbf{asin}($y$) is the unique number 
   $x \in [-\pi/2; \pi/2]$ such that \textbf{sin}($x$)=$y$.

\item It is defined only for $y \in [-1;1]$.
\end{itemize}
See also: \textbf{sin} (\ref{labsin})
