\subsection{exponent}
\label{labexponent}
\noindent Name: \textbf{exponent}\\
returns the scaled binary exponent of a number.\\
\noindent Usage: 
\begin{center}
\textbf{exponent}(\emph{x}) : \textsf{constant} $\rightarrow$ \textsf{integer}\\
\end{center}
Parameters: 
\begin{itemize}
\item \emph{x} is a dyadic number.
\end{itemize}
\noindent Description: \begin{itemize}

\item \\textbf{exponent}(x) is by definition 0 if $x$ is one of 0, NaN, or Inf.\n
\item If \\emph{x} is not zero, it can be uniquely written as $x = m \\cdot 2^e$ where\n   $m$ is an odd integer and $e$ is an integer. \\textbf{exponent}($x$) returns $e$. \n\end{itemize}
\noindent Example 1: 
\begin{center}\begin{minipage}{15cm}\begin{Verbatim}[frame=single]
\end{Verbatim}
\end{minipage}\end{center}
See also: \textbf{mantissa} (\ref{labmantissa}), \textbf{precision} (\ref{labprecision})
