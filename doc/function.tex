\subsection{function}
\label{labfunction}
\noindent Name: \textbf{function}\\
\phantom{aaa}keyword for declaring a procedure-based function or a keyword representing a \textsf{function} type \\[0.2cm]
\noindent Usage: 
\begin{center}
\textbf{function}(\emph{procedure})  : \textsf{procedure} $\rightarrow$ \textsf{function}\\
\textbf{function} : \textsf{type type}\\
\end{center}
Parameters: 
\begin{itemize}
\item \emph{procedure} is a procedure of type (\textsf{range}, \textsf{integer}, \textsf{integer}) $\rightarrow$ \textsf{range}
\end{itemize}
\noindent Description: \begin{itemize}

\item For the sake of safety and mathematical consistency, \sollya
   distinguishes clearly between functions, seen in the mathematical
   sense of the term, i.e. mappings, and procedures, seen in the sense
   Computer Science gives to functions, i.e. pieces of code that compute
   results for arguments following an algorithm. In some cases however,
   it is interesting to use such Computer Science procedures as
   realisations of mathematical functions, e.g. in order to plot them or
   even to perform polynomial approximation on them. The \textbf{function} keyword
   allows for such a transformation of a \sollya procedure into a \sollya
   function. 

\item The procedure to be used as a function through \textbf{function}(\emph{procedure})
   must be of type (\textsf{range}, \textsf{integer}, \textsf{integer})
   $\rightarrow$ \textsf{range}. This means it must take in argument
   an interval $X$, a degree of differentiation $n$ and a
   working precision $p$. It must return in result an interval
   $Y$ encompassing the image $f^{(n)}(X)$ of the
   $n$-th derivative of the implemented function $f$,
   i.e. $f^{(n)}(X) \subseteq Y$. In order to allow
   \sollya's algorithms to work properly, the procedure must ensure that,
   whenever $(p, \textrm{diam}(X))$ tends to $(+\infty,\,0)$,
   the computed over-estimated bounding $Y$ tends to the actual image $f^{(n)}(X)$.

\item The user must be aware that they are responsible of the correctness
   of the procedure. If, for some $n$ and $X$, \emph{procedure} returns an interval $Y$
   such that $f^{(n)}(X) \not\subseteq Y$, \textbf{function} will successfully
   return a function without any complain, but this function might behave
   inconsistently in further computations.

\item For cases when the procedure does not have the correct signature or
   does not return a finite interval as a result \textbf{function}(\emph{procedure})
   evaluates to Not-A-Number (resp. to an interval of Not-A-Numbers for
   interval evaluation).

\item \textbf{function} also represents the \textsf{function} type for declarations
   of external procedures by means of \textbf{externalproc}.
    
   Remark that in contrast to other indicators, type indicators like
   \textbf{function} cannot be handled outside the \textbf{externalproc} context.  In
   particular, they cannot be assigned to variables.
\end{itemize}
\noindent Example 1: 
\begin{center}\begin{minipage}{15cm}\begin{Verbatim}[frame=single]
> procedure EXP(X,n,p) {
        var res, oldPrec;
        oldPrec = prec;
        prec = p!;
        
        res = exp(X);
        
        prec = oldPrec!;
        return res;
  };
> f = function(EXP);
> f(1);
2.71828182845904523536028747135266249775724709369998
> exp(1);
2.71828182845904523536028747135266249775724709369998
> f(x + 3);
(function(proc(X, n, p)
{
var res, oldPrec;
oldPrec = prec;
prec = p!;
res = exp(X);
prec = oldPrec!;
return res;
}))(3 + x)
> diff(f);
diff(function(proc(X, n, p)
{
var res, oldPrec;
oldPrec = prec;
prec = p!;
res = exp(X);
prec = oldPrec!;
return res;
}))
> (diff(f))(0);
1
> g = f(sin(x));
> g(17);
0.382358169993866834026905546416556413595734583420876
> diff(g);
(diff(function(proc(X, n, p)
{
var res, oldPrec;
oldPrec = prec;
prec = p!;
res = exp(X);
prec = oldPrec!;
return res;
})))(sin(x)) * cos(x)
> (diff(g))(1);
1.25338076749344683697237458088447611474812675164344
> p = remez(f,3,[-1/2;1/2]);
> p;
0.99967120901420646830315493949039176881764871951832 + x * (0.999737029835711401
34762682913614052309208076875596 + x * (0.51049729360282624921622721654643510358
3073053437 + x * 0.169814324607133287588897694747370380479108785868016))
\end{Verbatim}
\end{minipage}\end{center}
See also: \textbf{proc} (\ref{labproc}), \textbf{library} (\ref{lablibrary}), \textbf{procedure} (\ref{labprocedure}), \textbf{externalproc} (\ref{labexternalproc}), \textbf{boolean} (\ref{labboolean}), \textbf{constant} (\ref{labconstant}), \textbf{integer} (\ref{labinteger}), \textbf{list of} (\ref{lablistof}), \textbf{range} (\ref{labrange}), \textbf{string} (\ref{labstring})
