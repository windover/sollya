\subsection{$\sim$}
\label{labapprox}
\noindent Name: \textbf{$\sim$}\\
floating-point evaluation of a constant expression\\
\noindent Usage: 
\begin{center}
\textbf{$\sim$} \emph{expression} : \textsf{function} $\rightarrow$ \textsf{constant}\\
\textbf{$\sim$} \emph{something} : \textsf{any type} $\rightarrow$ \textsf{any type}\\
\end{center}
Parameters: 
\begin{itemize}
\item \emph{expression} stands for an expression that is a constant
\item \emph{something} stands for some language element that is not a constant expression
\end{itemize}
\noindent Description: \begin{itemize}

\item \\textbf{$\\sim$} \\emph{expression} evaluates the \\emph{expression} that is a constant\n   term to a floating-point constant. The evaluation may involve a\n   rounding. If \\emph{expression} is not a constant, the evaluated constant is\n   a faithful rounding of \\emph{expression} with \\textbf{precision} bits, unless the\n   \\emph{expression} is exactly $0$ as a result of cancellation. In the\n   latter case, a floating-point approximation of some (unknown) accuracy\n   is returned.\n
\item \\textbf{$\\sim$} does not do anything on all language elements that are not a\n   constant expression.  In other words, it behaves like the identity\n   function on any type that is not a constant expression. It can hence\n   be used in any place where one wants to be sure that expressions are\n   simplified using floating-point computations to constants of a known\n   precision, regardless of the type of actual language elements.\n
\item \\textbf{$\\sim$} \\textbf{error} evaluates to error and provokes a warning.\n
\item \\textbf{$\\sim$} is a prefix operator not requiring parentheses. Its\n   precedence is the same as for the unary $+$ and $-$\n   operators. It cannot be repeatedly used without brackets.\n\end{itemize}
\noindent Example 1: 
\begin{center}\begin{minipage}{15cm}\begin{Verbatim}[frame=single]
\end{Verbatim}
\end{minipage}\end{center}
\noindent Example 2: 
\begin{center}\begin{minipage}{15cm}\begin{Verbatim}[frame=single]
\end{Verbatim}
\end{minipage}\end{center}
\noindent Example 3: 
\begin{center}\begin{minipage}{15cm}\begin{Verbatim}[frame=single]
\end{Verbatim}
\end{minipage}\end{center}
\noindent Example 4: 
\begin{center}\begin{minipage}{15cm}\begin{Verbatim}[frame=single]
\end{Verbatim}
\end{minipage}\end{center}
\noindent Example 5: 
\begin{center}\begin{minipage}{15cm}\begin{Verbatim}[frame=single]
\end{Verbatim}
\end{minipage}\end{center}
\noindent Example 6: 
\begin{center}\begin{minipage}{15cm}\begin{Verbatim}[frame=single]
\end{Verbatim}
\end{minipage}\end{center}
See also: \textbf{evaluate} (\ref{labevaluate}), \textbf{prec} (\ref{labprec}), \textbf{error} (\ref{laberror})
