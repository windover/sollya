\subsection{display}
\label{labdisplay}
\noindent Name: \textbf{display}\\
sets or inspects the global variable specifying number notation\\
\noindent Usage: 
\begin{center}
\textbf{display} = \emph{notation value} : \textsf{decimal$|$binary$|$dyadic$|$powers$|$hexadecimal} $\rightarrow$ \textsf{void}\\
\textbf{display} = \emph{notation value} ! : \textsf{decimal$|$binary$|$dyadic$|$powers$|$hexadecimal} $\rightarrow$ \textsf{void}\\
\textbf{display} : \textsf{decimal$|$binary$|$dyadic$|$powers$|$hexadecimal}\\
\end{center}
Parameters: 
\begin{itemize}
\item \emph{notation value} represents a variable of type \textsf{decimal$|$binary$|$dyadic$|$powers$|$hexadecimal}
\end{itemize}
\noindent Description: \begin{itemize}

\item An assignment \\textbf{display} = \\emph{notation value}, where \\emph{notation value} is\n   one of \\textbf{decimal}, \\textbf{dyadic}, \\textbf{powers}, \\textbf{binary} or \\textbf{hexadecimal}, activates\n   the corresponding notation for output of values in \\textbf{print}, \\textbf{write} or\n   at the \\sollya prompt.\n    \n   If the global notation variable \\textbf{display} is \\textbf{decimal}, all numbers will\n   be output in scientific decimal notation.  If the global notation\n   variable \\textbf{display} is \\textbf{dyadic}, all numbers will be output as dyadic\n   numbers with Gappa notation.  If the global notation variable \\textbf{display}\n   is \\textbf{powers}, all numbers will be output as dyadic numbers with a\n   notation compatible with Maple and PARI/GP.  If the global notation\n   variable \\textbf{display} is \\textbf{binary}, all numbers will be output in binary\n   notation.  If the global notation variable \\textbf{display} is \\textbf{hexadecimal},\n   all numbers will be output in C99/ IEEE754-2008 notation.  All output\n   notations can be parsed back by \\sollya, inducing no error if the input\n   and output precisions are the same (see \\textbf{prec}).\n    \n   If the assignment \\textbf{display} = \\emph{notation value} is followed by an\n   exclamation mark, no message indicating the new state is\n   displayed. Otherwise the user is informed of the new state of the\n   global mode by an indication.\n\end{itemize}
\noindent Example 1: 
\begin{center}\begin{minipage}{15cm}\begin{Verbatim}[frame=single]
\end{Verbatim}
\end{minipage}\end{center}
See also: \textbf{print} (\ref{labprint}), \textbf{write} (\ref{labwrite}), \textbf{decimal} (\ref{labdecimal}), \textbf{dyadic} (\ref{labdyadic}), \textbf{powers} (\ref{labpowers}), \textbf{binary} (\ref{labbinary}), \textbf{hexadecimal} (\ref{labhexadecimal}), \textbf{prec} (\ref{labprec})
