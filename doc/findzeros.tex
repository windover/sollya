\subsection{ findzeros }
\noindent Name: \textbf{findzeros}\\
gives a list of intervals containing all zeros of a function on an interval.\\

\noindent Usage: 
\begin{center}
\textbf{findzeros}(\emph{f},\emph{I}) : (\textsf{function}, \textsf{range}) $\rightarrow$ \textsf{list}\\
\end{center}
Parameters: 
\begin{itemize}
\item \emph{f} is a function.
\item \emph{I} is an interval.
\end{itemize}
\noindent Description: \begin{itemize}

\item \textbf{findzeros}(\emph{f},\emph{I}) returns a list of intervals \emph{I1}, ... ,\emph{In} such that, for 
   every zero $z$ of $f$, there exists some $k$ such that $z \in I_k$.

\item The list may contain intervals \emph{Ik} that do not contain any zero of \emph{f}.
   An interval \emph{Ik} may contain many zeros of \emph{f}.

\item This command is ment for cases when safety is critical. If you want to be sure
   not to forget any zero, use \textbf{findzeros}. However, if you just want to know 
   numerical values for the zeros of \emph{f}, \textbf{dirtyfindzeros} should be quite 
   satisfactory and a lot faster.

\item If $\delta$ denotes the value of global variable \textbf{diam}, the algorithm ensures
   that for each $k$, $|I_k| \le \delta \cdot |I|$.

\item The algorithm used is basically a bisection algorithm. It is the same algorithm
   that the one used for \textbf{infnorm}. See the help page of this command for more 
   details. In short, the behavior of the algorithm depends on global variables
   \textbf{prec}, \textbf{diam}, \textbf{taylorrecursions} and \textbf{hopitalrecursions}.
\end{itemize}
\noindent Example 1: 
\begin{center}\begin{minipage}{15cm}\begin{Verbatim}[frame=single]
> findzeros(sin(x),[-5;5]);
[|[-0.314208984375e1;-0.3140869140625e1], [-0.1220703125e-2;0.1220703125e-2], [0
.3140869140625e1;0.314208984375e1]|]
> diam=1e-10!;
> findzeros(sin(x),[-5;5]);
[|[-0.314159265370108187198638916015625e1;-0.3141592652536928653717041015625e1],
 [-0.116415321826934814453125e-8;0.116415321826934814453125e-8], [0.314159265253
6928653717041015625e1;0.314159265370108187198638916015625e1]|]
\end{Verbatim}
\end{minipage}\end{center}
See also: \textbf{dirtyfindzeros}, \textbf{infnorm}, \textbf{prec}, \textbf{diam}, \textbf{taylorrecursions}, \textbf{hopitalrecursions}
