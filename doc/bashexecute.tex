\subsection{bashexecute}
\label{labbashexecute}
\noindent Name: \textbf{bashexecute}\\
executes a shell command.\\
\noindent Usage: 
\begin{center}
\textbf{bashexecute}(\emph{command}) : \textsf{string} $\rightarrow$ \textsf{void}\\
\end{center}
Parameters: 
\begin{itemize}
\item \emph{command} is a command to be interpreted by the shell.
\end{itemize}
\noindent Description: \begin{itemize}

\item \\textbf{bashexecute}(\\emph{command}) lets the shell interpret \\emph{command}. It is useful to execute\n   some external code within \\sollya.\n
\item \\textbf{bashexecute} does not return anything. It just executes its argument. However, if\n   \\emph{command} produces an output in a file, this result can be imported in \\sollya\n   with help of commands like \\textbf{execute}, \\textbf{readfile} and \\textbf{parse}.\n\end{itemize}
\noindent Example 1: 
\begin{center}\begin{minipage}{15cm}\begin{Verbatim}[frame=single]
\end{Verbatim}
\end{minipage}\end{center}
See also: \textbf{execute} (\ref{labexecute}), \textbf{readfile} (\ref{labreadfile}), \textbf{parse} (\ref{labparse})
