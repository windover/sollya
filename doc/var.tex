\subsection{var}
\label{labvar}
\noindent Name: \textbf{var}\\
declaration of a local variable in a scope\\
\noindent Usage: 
\begin{center}
\textbf{var} \emph{identifier1}, \emph{identifier2},... , \emph{identifiern} : \textsf{void}\\
\end{center}
Parameters: 
\begin{itemize}
\item \emph{identifier1}, \emph{identifier2},... , \emph{identifiern} represent variable identifiers
\end{itemize}
\noindent Description: \begin{itemize}

\item The keyword \textbf{var} allows for the declaration of local variables
   \emph{identifier1} through \emph{identifiern} in a begin-end-block ($\lbrace \rbrace$-block).
   Once declared as a local variable, an identifier will shadow
   identifiers declared in higher scopes and undeclared identifiers
   available at top-level.
    
   Variable declarations using \textbf{var} are only possible in the
   beginning of a begin-end-block. Several \textbf{var} statements can be
   given. Once another statement is given in a begin-end-block, no more
   \textbf{var} statements can be given.
    
   Variables declared by \textbf{var} statements are dereferenced as \textbf{error}
   until they are assigned a value. 
\end{itemize}
\noindent Example 1: 
\begin{center}\begin{minipage}{15cm}\begin{Verbatim}[frame=single]
> exp(x); 
exp(x)
> a = 3; 
> {var a, b; a=5; b=3; {var a; var b; b = true; a = 1; a; b;}; a; b; }; 
1
true
5
3
> a;
3
\end{Verbatim}
\end{minipage}\end{center}
See also: \textbf{error} (\ref{laberror})
