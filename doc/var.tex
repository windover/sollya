\subsection{var}
\label{labvar}
\noindent Name: \textbf{var}\\
declaration of a local variable in a scope\\
\noindent Usage: 
\begin{center}
\textbf{var} \emph{identifier1}, \emph{identifier2},... , \emph{identifiern} : \textsf{void}\\
\end{center}
Parameters: 
\begin{itemize}
\item \emph{identifier1}, \emph{identifier2},... , \emph{identifiern} represent variable identifiers
\end{itemize}
\noindent Description: \begin{itemize}

\item The keyword \\textbf{var} allows for the declaration of local variables\n   \\emph{identifier1} through \\emph{identifiern} in a begin-end-block ($\\lbrace \\rbrace$-block).\n   Once declared as a local variable, an identifier will shadow\n   identifiers declared in higher scopes and undeclared identifiers\n   available at top-level.\n    \n   Variable declarations using \\textbf{var} are only possible in the\n   beginning of a begin-end-block. Several \\textbf{var} statements can be\n   given. Once another statement is given in a begin-end-block, no more\n   \\textbf{var} statements can be given.\n    \n   Variables declared by \\textbf{var} statements are dereferenced as \\textbf{error}\n   until they are assigned a value. \n\end{itemize}
\noindent Example 1: 
\begin{center}\begin{minipage}{15cm}\begin{Verbatim}[frame=single]
\end{Verbatim}
\end{minipage}\end{center}
See also: \textbf{error} (\ref{laberror})
