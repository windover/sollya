\subsection{plot}
\label{labplot}
\noindent Name: \textbf{plot}\\
plots one or several functions\\
\noindent Usage: 
\begin{center}
\textbf{plot}(\emph{f1}, ... ,\emph{fn}, \emph{I}) : (\textsf{function}, ... ,\textsf{function}, \textsf{range}) $\rightarrow$ \textsf{void}\\
\textbf{plot}(\emph{f1}, ... ,\emph{fn}, \emph{I}, \textbf{file}, \emph{name}) : (\textsf{function}, ... ,\textsf{function}, \textsf{range}, \textbf{file}, \textsf{string}) $\rightarrow$ \textsf{void}\\
\textbf{plot}(\emph{f1}, ... ,\emph{fn}, \emph{I}, \textbf{postscript}, \emph{name}) : (\textsf{function}, ... ,\textsf{function}, \textsf{range}, \textbf{postscript}, \textsf{string}) $\rightarrow$ \textsf{void}\\
\textbf{plot}(\emph{f1}, ... ,\emph{fn}, \emph{I}, \textbf{postscriptfile}, \emph{name}) : (\textsf{function}, ... ,\textsf{function}, \textsf{range}, \textbf{postscriptfile}, \textsf{string}) $\rightarrow$ \textsf{void}\\
\textbf{plot}(\emph{L}, \emph{I}) : (\textsf{list}, \textsf{range}) $\rightarrow$ \textsf{void}\\
\textbf{plot}(\emph{L}, \emph{I}, \textbf{file}, \emph{name}) : (\textsf{list}, \textsf{range}, \textbf{file}, \textsf{string}) $\rightarrow$ \textsf{void}\\
\textbf{plot}(\emph{L}, \emph{I}, \textbf{postscript}, \emph{name}) : (\textsf{list}, \textsf{range}, \textbf{postscript}, \textsf{string}) $\rightarrow$ \textsf{void}\\
\textbf{plot}(\emph{L}, \emph{I}, \textbf{postscriptfile}, \emph{name}) : (\textsf{list}, \textsf{range}, \textbf{postscriptfile}, \textsf{string}) $\rightarrow$ \textsf{void}\\
\end{center}
Parameters: 
\begin{itemize}
\item \emph{f1}, ..., \emph{fn} are functions to be plotted.
\item \emph{L} is a list of functions to be plotted.
\item \emph{I} is the interval where the functions have to be plotted.
\item \emph{name} is a string representing the name of a file.
\end{itemize}
\noindent Description: \begin{itemize}

\item This command plots one or several functions \\emph{f1}, ... ,\\emph{fn} on an interval \\emph{I}.\n   Functions can be either given as parameters of \\textbf{plot} or as a list \\emph{L}\n   which elements are functions.\n   The functions are drawn on the same plot with different colors.\n
\item If \\emph{L} contains an element that is not a function (or a constant), an error\n   occurs.\n
\item \\textbf{plot} relies on the value of global variable \\textbf{points}. Let $n$ be the \n   value of this variable. The algorithm is the following: each function is \n   evaluated at $n$ evenly distributed points in \\emph{I}. At each point, the \n   computed value is a faithful rounding of the exact value with a sufficiently\n   high precision. Each point is finally plotted.\n   This should avoid numerical artefacts such as critical cancellations.\n
\item You can save the function plot either as a data file or as a postscript file.\n
\item If you use argument \\textbf{file} with a string \\emph{name}, \\sollya will save a data file\n   called name.dat and a gnuplot directives file called name.p. Invoking gnuplot\n   on name.p will plot the data stored in name.dat.\n
\item If you use argument \\textbf{postscript} with a string \\emph{name}, \\sollya will save a \n   postscript file called name.eps representing your plot.\n
\item If you use argument \\textbf{postscriptfile} with a string \\emph{name}, \\sollya will \n   produce the corresponding name.dat, name.p and name.eps.\n
\item This command uses gnuplot to produce the final plot.\n   If your terminal is not graphic (typically if you use \\sollya through \n   ssh without -X)\n   gnuplot should be able to detect that and produce an ASCII-art version on the\n   standard output. If it is not the case, you can either store the plot in a\n   postscript file to view it locally, or use \\textbf{asciiplot} command.\n
\item If every function is constant, \\textbf{plot} will not plot them but just display\n   their value.\n
\item If the interval is reduced to a single point, \\textbf{plot} will just display the\n   value of the functions at this point.\n\end{itemize}
\noindent Example 1: 
\begin{center}\begin{minipage}{15cm}\begin{Verbatim}[frame=single]
\end{Verbatim}
\end{minipage}\end{center}
\noindent Example 2: 
\begin{center}\begin{minipage}{15cm}\begin{Verbatim}[frame=single]
\end{Verbatim}
\end{minipage}\end{center}
\noindent Example 3: 
\begin{center}\begin{minipage}{15cm}\begin{Verbatim}[frame=single]
\end{Verbatim}
\end{minipage}\end{center}
\noindent Example 4: 
\begin{center}\begin{minipage}{15cm}\begin{Verbatim}[frame=single]
\end{Verbatim}
\end{minipage}\end{center}
See also: \textbf{externalplot} (\ref{labexternalplot}), \textbf{asciiplot} (\ref{labasciiplot}), \textbf{file} (\ref{labfile}), \textbf{postscript} (\ref{labpostscript}), \textbf{postscriptfile} (\ref{labpostscriptfile}), \textbf{points} (\ref{labpoints})
