\subsection{fixed}
\label{labfixed}
\noindent Name: \textbf{fixed}\\
indicates that fixed-point formats should be used for \textbf{fpminimax}\\
\noindent Usage: 
\begin{center}
\textbf{fixed} : \textsf{fixed$|$floating}
\\ 
\end{center}
\noindent Description: \begin{itemize}

\item The use of \textbf{fixed} in the command \textbf{fpminimax} indicates that the list of
   formats given as argument is to be considered to be a list of fixed-point
   formats.
   See \textbf{fpminimax} for details.
\end{itemize}
\noindent Example 1: 
\begin{center}\begin{minipage}{15cm}\begin{Verbatim}[frame=single]
> fpminimax(cos(x),6,[|32,32,32,32,32,32,32|],[-1;1],fixed);
0.9999997480772435665130615234375 + x^2 * (-0.4999928693287074565887451171875 + 
x^2 * (4.163351492024958133697509765625e-2 + x^2 * (-1.3382239267230033874511718
75e-3)))
\end{Verbatim}
\end{minipage}\end{center}
See also: \textbf{fpminimax} (\ref{labfpminimax}), \textbf{floating} (\ref{labfloating})
