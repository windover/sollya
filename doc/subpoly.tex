\subsection{subpoly}
\label{labsubpoly}
\noindent Name: \textbf{subpoly}\\
restricts the monomial basis of a polynomial to a list of monomials\\
\noindent Usage: 
\begin{center}
\textbf{subpoly}(\emph{polynomial}, \emph{list}) : (\textsf{function}, \textsf{list}) $\rightarrow$ \textsf{function}\\
\end{center}
Parameters: 
\begin{itemize}
\item \emph{polynomial} represents the polynomial the coefficients are taken from
\item \emph{list} represents the list of monomials to be taken
\end{itemize}
\noindent Description: \begin{itemize}

\item \\textbf{subpoly} extracts the coefficients of a polynomial \\emph{polynomial} and builds up a\n   new polynomial out of those coefficients associated to monomial degrees figuring in\n   the list \\emph{list}. \n    \n   If \\emph{polynomial} represents a function that is not a polynomial, subpoly returns 0.\n    \n   If \\emph{list} is a list that is end-elliptic, let be $j$ the last value explicitly specified\n   in the list. All coefficients of the polynomial associated to monomials greater or\n   equal to $j$ are taken.\n\end{itemize}
\noindent Example 1: 
\begin{center}\begin{minipage}{15cm}\begin{Verbatim}[frame=single]
\end{Verbatim}
\end{minipage}\end{center}
\noindent Example 2: 
\begin{center}\begin{minipage}{15cm}\begin{Verbatim}[frame=single]
\end{Verbatim}
\end{minipage}\end{center}
\noindent Example 3: 
\begin{center}\begin{minipage}{15cm}\begin{Verbatim}[frame=single]
\end{Verbatim}
\end{minipage}\end{center}
See also: \textbf{roundcoefficients} (\ref{labroundcoefficients}), \textbf{taylor} (\ref{labtaylor}), \textbf{remez} (\ref{labremez})
