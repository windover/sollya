\subsection{proc}
\label{labproc}
\noindent Name: \textbf{proc}\\
defines a \sollya procedure\\
\noindent Usage: 
\begin{center}
\textbf{proc}(\emph{formal parameter1}, \emph{formal parameter2},..., \emph{formal parameter n}) \textbf{begin} \emph{procedure body} \textbf{end} : \textsf{void} $\rightarrow$ \textsf{procedure}\\
\textbf{proc}(\emph{formal parameter1}, \emph{formal parameter2},..., \emph{formal parameter n}) \textbf{begin} \emph{procedure body} \textbf{return} \emph{expression}; \textbf{end} : \textsf{void} $\rightarrow$ \textsf{procedure}\\
\textbf{proc}(\emph{formal list parameter} = ...) \textbf{begin} \emph{procedure body} \textbf{end} : \textsf{void} $\rightarrow$ \textsf{procedure}\\
\textbf{proc}(\emph{formal list parameter} = ...) \textbf{begin} \emph{procedure body} \textbf{return} \emph{expression}; \textbf{end} : \textsf{void} $\rightarrow$ \textsf{procedure}\\
\end{center}
Parameters: 
\begin{itemize}
\item \emph{formal parameter1}, \emph{formal parameter2} through \emph{formal parameter n} represent identifiers used as formal parameters
\item \emph{formal list parameter} represents an identifier used as a formal parameter for the list of an arbitrary number of parameters
\item \emph{procedure body} represents the imperative statements in the body of the procedure
\item \emph{expression} represents the expression \textbf{proc} shall evaluate to
\end{itemize}
\noindent Description: \begin{itemize}

\item The \\textbf{proc} keyword allows for defining procedures in the \\sollya\n   language. These procedures are common \\sollya objects that can be\n   applied to actual parameters after definition. Upon such an\n   application, the \\sollya interpreter applies the actual parameters to\n   the formal parameters \\emph{formal parameter1} through \\emph{formal parameter n}\n   (resp. builds up the list of arguments and applies it to the list\n   \\emph{formal list parameter}) and executes the \\emph{procedure body}. The\n   procedure applied to actual parameters evaluates then to the\n   expression \\emph{expression} in the \\textbf{return} statement after the <procedure\n   body> or to \\textbf{void}, if no return statement is given (i.e. a \\textbf{return}\n   \\textbf{void} statement is implicitly given).\n
\item \\sollya procedures defined by \\textbf{proc} have no name. They can be bound\n   to an identifier by assigning the procedure object a \\textbf{proc}\n   expression produces to an identifier. However, it is possible to use\n   procedures without giving them any name. For instance, \\sollya\n   procedures, i.e. procedure objects, can be elements of lists. They can\n   even be given as an argument to other internal \\sollya procedures. See\n   also \\textbf{procedure} on this subject.\n
\item Upon definition of a \\sollya procedure using \\textbf{proc}, no type check\n   is performed. More precisely, the statements in \\emph{procedure body} are\n   merely parsed but not interpreted upon procedure definition with\n   \\textbf{proc}. Type checks are performed once the procedure is applied to\n   actual parameters or to \\textbf{void}. At this time, if the procedure was\n   defined using several different formal parameters \\emph{formal parameter 1}\n   through \\emph{formal parameter n}, it is checked whether the number of\n   actual parameters corresponds to the number of formal parameters. If\n   the procedure was defined using the syntax for a procedure with an\n   arbitrary number of parameters by giving a \\emph{formal list parameter},\n   the number of actual arguments is not checked but only a list <formal\n   list parameter> of appropriate length is built up. Type checks are\n   further performed upon execution of each statement in \\emph{procedure body}\n   and upon evaluation of the expression \\emph{expression} to be returned.\n    \n   Procedures defined by \\textbf{proc} containing a \\textbf{quit} or \\textbf{restart} command\n   cannot be executed (i.e. applied). Upon application of a procedure,\n   the \\sollya interpreter checks beforehand for such a statement. If one\n   is found, the application of the procedure to its arguments evaluates\n   to \\textbf{error}. A warning is displayed. Remark that in contrast to other\n   type or semantic correctness checks, this check is really performed\n   before interpreting any other statement in the body of the procedure.\n
\item Through the \\textbf{var} keyword it is possible to declare local\n   variables and thus to have full support of recursive procedures. This\n   means a procedure defined using \\textbf{proc} may contain in its \\emph{procedure body} \n   an application of itself to some actual parameters: it suffices\n   to assign the procedure (object) to an identifier with an appropriate\n   name.\n
\item \\sollya procedures defined using \\textbf{proc} may return other\n   procedures. Further \\emph{procedure body} may contain assignments of\n   locally defined procedure objects to identifiers. See \\textbf{var} for the\n   particular behaviour of local and global variables.\n
\item The expression \\emph{expression} returned by a procedure is evaluated with\n   regard to \\sollya commands, procedures and external\n   procedures. Simplification may be performed.  However, an application\n   of a procedure defined by \\textbf{proc} to actual parameters evaluates to the\n   expression \\emph{expression} that may contain the free global variable or\n   that may be composed.\n\end{itemize}
\noindent Example 1: 
\begin{center}\begin{minipage}{15cm}\begin{Verbatim}[frame=single]
\end{Verbatim}
\end{minipage}\end{center}
\noindent Example 2: 
\begin{center}\begin{minipage}{15cm}\begin{Verbatim}[frame=single]
\end{Verbatim}
\end{minipage}\end{center}
\noindent Example 3: 
\begin{center}\begin{minipage}{15cm}\begin{Verbatim}[frame=single]
\end{Verbatim}
\end{minipage}\end{center}
\noindent Example 4: 
\begin{center}\begin{minipage}{15cm}\begin{Verbatim}[frame=single]
\end{Verbatim}
\end{minipage}\end{center}
\noindent Example 5: 
\begin{center}\begin{minipage}{15cm}\begin{Verbatim}[frame=single]
\end{Verbatim}
\end{minipage}\end{center}
\noindent Example 6: 
\begin{center}\begin{minipage}{15cm}\begin{Verbatim}[frame=single]
\end{Verbatim}
\end{minipage}\end{center}
\noindent Example 7: 
\begin{center}\begin{minipage}{15cm}\begin{Verbatim}[frame=single]
\end{Verbatim}
\end{minipage}\end{center}
\noindent Example 8: 
\begin{center}\begin{minipage}{15cm}\begin{Verbatim}[frame=single]
\end{Verbatim}
\end{minipage}\end{center}
\noindent Example 9: 
\begin{center}\begin{minipage}{15cm}\begin{Verbatim}[frame=single]
\end{Verbatim}
\end{minipage}\end{center}
\noindent Example 10: 
\begin{center}\begin{minipage}{15cm}\begin{Verbatim}[frame=single]
\end{Verbatim}
\end{minipage}\end{center}
See also: \textbf{return} (\ref{labreturn}), \textbf{externalproc} (\ref{labexternalproc}), \textbf{void} (\ref{labvoid}), \textbf{quit} (\ref{labquit}), \textbf{restart} (\ref{labrestart}), \textbf{var} (\ref{labvar}), \textbf{@} (\ref{labconcat})
