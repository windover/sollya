\subsection{rationalapprox}
\label{labrationalapprox}
\noindent Name: \textbf{rationalapprox}\\
returns a fraction close to a given number.\\
\noindent Usage: 
\begin{center}
\textbf{rationalapprox}(\emph{x},\emph{n}) : (\textsf{constant}, \textsf{integer}) $\rightarrow$ \textsf{function}\\
\end{center}
Parameters: 
\begin{itemize}
\item \emph{x} is a number to approximate.
\item \emph{n} is a integer (representing a format).
\end{itemize}
\noindent Description: \begin{itemize}

\item \\textbf{rationalapprox}(\\emph{x},\\emph{n}) returns a constant function of the form $a/b$ where $a$ and $b$ are\n   integers. The value $a/b$ is an approximation of \\emph{x}. The quality of this \n   approximation is determined by the parameter \\emph{n} that indicates the number of\n   correct bits that $a/b$ should have.\n
\item The command is not safe in the sense that it is not ensured that the error \n   between $a/b$ and \\emph{x} is less than $2^{-n}$.\n
\item The following algorithm is used: \\emph{x} is first rounded downwards and upwards to\n   a format of \\emph{n} bits, thus obtaining an interval $[x_l,\\,x_u]$. This interval is then\n   developped into a continued fraction as far as the representation is the same\n   for every elements of $[x_l,\\,x_u]$. The corresponding fraction is returned.\n
\item Since rational numbers are not a primitive object of \\sollya, the fraction is\n   returned as a constant function. This can be quite amazing, because \\sollya\n   immediately simplifies a constant function by evaluating it when the constant\n   has to be displayed.\n   To avoid this, you can use \\textbf{print} (that displays the expression representing\n   the constant and not the constant itself) or the commands \\textbf{numerator} \n   and \\textbf{denominator}.\n\end{itemize}
\noindent Example 1: 
\begin{center}\begin{minipage}{15cm}\begin{Verbatim}[frame=single]
\end{Verbatim}
\end{minipage}\end{center}
\noindent Example 2: 
\begin{center}\begin{minipage}{15cm}\begin{Verbatim}[frame=single]
\end{Verbatim}
\end{minipage}\end{center}
See also: \textbf{print} (\ref{labprint}), \textbf{numerator} (\ref{labnumerator}), \textbf{denominator} (\ref{labdenominator})
