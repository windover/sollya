\subsection{$-$}
\label{labminus}
\noindent Name: \textbf{$-$}\\
subtraction function\\
\noindent Usage: 
\begin{center}
\emph{function1} \textbf{$-$} \emph{function2} : (\textsf{function}, \textsf{function}) $\rightarrow$ \textsf{function}\\
\emph{interval1} \textbf{$-$} \emph{interval2} : (\textsf{range}, \textsf{range}) $\rightarrow$ \textsf{range}\\
\emph{interval1} \textbf{$-$} \emph{constant} : (\textsf{range}, \textsf{constant}) $\rightarrow$ \textsf{range}\\
\emph{interval1} \textbf{$-$} \emph{constant} : (\textsf{constant}, \textsf{range}) $\rightarrow$ \textsf{range}\\
\textbf{$-$} \emph{function1} : \textsf{function} $\rightarrow$ \textsf{function}\\
\textbf{$-$} \emph{interval1} : \textsf{range} $\rightarrow$ \textsf{range}\\
\end{center}
Parameters: 
\begin{itemize}
\item \emph{function1} and \emph{function2} represent functions
\item \emph{interval1} and \emph{interval2} represent intervals (ranges)
\item \emph{constant} represents a constant or constant expression
\end{itemize}
\noindent Description: \begin{itemize}

\item \\textbf{$-$} represents the subtraction (function) on reals. \n   The expression \\emph{function1} \\textbf{$-$} \\emph{function2} stands for\n   the function composed of the subtraction function and the two\n   functions \\emph{function1} and \\emph{function2}, where \\emph{function1} is \n   the subtrahend and \\emph{function2} the subtractor.\n
\item \\textbf{$-$} can be used for interval arithmetic on intervals\n   (ranges). \\textbf{$-$} will evaluate to an interval that safely\n   encompasses all images of the subtraction function with arguments varying\n   in the given intervals.  Any combination of intervals with intervals\n   or constants (resp. constant expressions) is supported. However, it is\n   not possible to represent families of functions using an interval as\n   one argument and a function (varying in the free variable) as the\n   other one.\n
\item \\textbf{$-$} stands also for the negation function.\n\end{itemize}
\noindent Example 1: 
\begin{center}\begin{minipage}{15cm}\begin{Verbatim}[frame=single]
\end{Verbatim}
\end{minipage}\end{center}
\noindent Example 2: 
\begin{center}\begin{minipage}{15cm}\begin{Verbatim}[frame=single]
\end{Verbatim}
\end{minipage}\end{center}
\noindent Example 3: 
\begin{center}\begin{minipage}{15cm}\begin{Verbatim}[frame=single]
\end{Verbatim}
\end{minipage}\end{center}
\noindent Example 4: 
\begin{center}\begin{minipage}{15cm}\begin{Verbatim}[frame=single]
\end{Verbatim}
\end{minipage}\end{center}
\noindent Example 5: 
\begin{center}\begin{minipage}{15cm}\begin{Verbatim}[frame=single]
\end{Verbatim}
\end{minipage}\end{center}
\noindent Example 6: 
\begin{center}\begin{minipage}{15cm}\begin{Verbatim}[frame=single]
\end{Verbatim}
\end{minipage}\end{center}
See also: \textbf{$+$} (\ref{labplus}), \textbf{$*$} (\ref{labmult}), \textbf{/} (\ref{labdivide}), \textbf{$\mathbf{\hat{~}}$} (\ref{labpower})
