\subsection{ taylor }
\noindent Name: \textbf{taylor}\\
computes a Taylor expansion of a function in a point\\

\noindent Usage: 
\begin{center}
\textbf{taylor}(\emph{function}, \emph{degree}, \emph{point}) : (\textsf{function}, \textsf{integer}, \textsf{constant}) $\rightarrow$ \textsf{function}\\
\end{center}
Parameters: 
\emph{function} represents the function to be expanded\\
\emph{degree} represents the degree of the expansion to be delivered\\
\emph{point} represents the point in which the function is to be developped\\

\noindent Description: \begin{itemize}

\item The command \textbf{taylor} returns an expression that is a Taylor expansion
   of function \emph{function} in point \emph{point} having the degree \emph{degree}.
   Let $f$ be the function \emph{function}, $t$ be the point \emph{point} and
   $n$ be the degree \emph{degree}. Then, \textbf{taylor}(\emph{function},\emph{degree},\emph{point}) 
   evaluates to an expression mathematically equal to 
   $$\sum\limits_{i=0}^n \frac{f^{(i)}}{i!} \left(x - t \right)^i$$
   Remark that \textbf{taylor} evaluates to $0$ if the degree \emph{degree} is negative.
\end{itemize}
\noindent Example 1: 
\begin{center}\begin{minipage}{14.8cm}\begin{Verbatim}[frame=single]
   > print(taylor(exp(x),5,0));
   1 + x * (1 + x * (0.5 + x * (1 / 6 + x * (1 / 24 + x / 120))))
\end{Verbatim}
\end{minipage}\end{center}
\noindent Example 2: 
\begin{center}\begin{minipage}{14.8cm}\begin{Verbatim}[frame=single]
   > print(taylor(asin(x),7,0));
   x * (1 + x^2 * (1 / 6 + x^2 * (9 / 120 + x^2 * 225 / 5040)))
\end{Verbatim}
\end{minipage}\end{center}
\noindent Example 3: 
\begin{center}\begin{minipage}{14.8cm}\begin{Verbatim}[frame=single]
   > print(taylor(erf(x),6,0));
   x * (1 / sqrt((pi) / 4) + x^2 * ((sqrt((pi) / 4) * 4 / (pi) * (-2)) / 6 + x^2 * (sqrt((pi) / 4) * 4 / (pi) * 12) / 120))
\end{Verbatim}
\end{minipage}\end{center}
See also: \textbf{remez}
