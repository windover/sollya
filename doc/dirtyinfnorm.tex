\subsection{dirtyinfnorm}
\label{labdirtyinfnorm}
\noindent Name: \textbf{dirtyinfnorm}\\
computes a numerical approximation of the infinity norm of a function on an interval.\\
\noindent Usage: 
\begin{center}
\textbf{dirtyinfnorm}(\emph{f},\emph{I}) : (\textsf{function}, \textsf{range}) $\rightarrow$ \textsf{constant}\\
\end{center}
Parameters: 
\begin{itemize}
\item \emph{f} is a function.
\item \emph{I} is an interval.
\end{itemize}
\noindent Description: \begin{itemize}

\item \\textbf{dirtyinfnorm}(\\emph{f},\\emph{I}) computes an approximation of the infinity norm of the \n   given function $f$ on the interval $I$, e.g. $\\max_{x \\in I} \\{|f(x)|\\}$.\n
\item The interval must be bound. If the interval contains one of -Inf or +Inf, the \n   result of \\textbf{dirtyinfnorm} is NaN.\n
\item The result of this command depends on the global variables \\textbf{prec} and \\textbf{points}.\n   Therefore, the returned result is generally a good approximation of the exact\n   infinity norm, with precision \\textbf{prec}. However, the result is generally \n   underestimated and should not be used when safety is critical.\n   Use \\textbf{infnorm} instead.\n
\item The following algorithm is used: let $n$ be the value of variable \\textbf{points}\n   and $t$ be the value of variable \\textbf{prec}.\n   \\begin{itemize}\n   \\item Evaluate $|f|$ at $n$ evenly distributed points in the\n     interval $I$. The evaluation are faithful roundings of the exact\n     results at precision $t$.\n   \\item Whenever the derivative of $f$ changes its sign for two consecutive \n     points, find an approximation $x$ of its zero with precision $t$.\n     Then compute a faithful rounding of $|f(x)|$ at precision $t$.\n   \\item Return the maximum of all computed values.\n   \\end{itemize}\n\end{itemize}
\noindent Example 1: 
\begin{center}\begin{minipage}{15cm}\begin{Verbatim}[frame=single]
\end{Verbatim}
\end{minipage}\end{center}
\noindent Example 2: 
\begin{center}\begin{minipage}{15cm}\begin{Verbatim}[frame=single]
\end{Verbatim}
\end{minipage}\end{center}
\noindent Example 3: 
\begin{center}\begin{minipage}{15cm}\begin{Verbatim}[frame=single]
\end{Verbatim}
\end{minipage}\end{center}
See also: \textbf{prec} (\ref{labprec}), \textbf{points} (\ref{labpoints}), \textbf{infnorm} (\ref{labinfnorm}), \textbf{checkinfnorm} (\ref{labcheckinfnorm})
