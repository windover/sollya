\subsection{parse}
\label{labparse}
\noindent Name: \textbf{parse}\\
parses an expression contained in a string\\

\noindent Usage: 
\begin{center}
\textbf{parse}(\emph{string}) : \textsf{string} $\rightarrow$ \textsf{function} $|$ \textsf{error}\\
\end{center}
Parameters: 
\begin{itemize}
\item \emph{string} represents a character sequence
\end{itemize}
\noindent Description: \begin{itemize}

\item \textbf{parse}(\emph{string}) parses the character sequence \emph{string} containing
   an expression built on constants and base functions.
    
   If the character sequence does not contain a well-defined expression,
   a warning is displayed indicating a syntax error and \textbf{parse} returns
   a \textbf{error} of type \textsf{error}.
\end{itemize}
\noindent Example 1: 
\begin{center}\begin{minipage}{15cm}\begin{Verbatim}[frame=single]
> parse("exp(x)");
exp(x)
\end{Verbatim}
\end{minipage}\end{center}
\noindent Example 2: 
\begin{center}\begin{minipage}{15cm}\begin{Verbatim}[frame=single]
> verbosity = 1!;
> parse("5 + + 3");
8
\end{Verbatim}
\end{minipage}\end{center}
See also: \textbf{execute} (\ref{labexecute}), \textbf{readfile} (\ref{labreadfile})
