\subsection{parse}
\label{labparse}
\noindent Name: \textbf{parse}\\
parses an expression contained in a string\\
\noindent Usage: 
\begin{center}
\textbf{parse}(\emph{string}) : \textsf{string} $\rightarrow$ \textsf{function} $|$ \textsf{error}\\
\end{center}
Parameters: 
\begin{itemize}
\item \emph{string} represents a character sequence
\end{itemize}
\noindent Description: \begin{itemize}

\item \\textbf{parse}(\\emph{string}) parses the character sequence \\emph{string} containing\n   an expression built on constants and base functions.\n    \n   If the character sequence does not contain a well-defined expression,\n   a warning is displayed indicating a syntax error and \\textbf{parse} returns\n   a \\textbf{error} of type \\textsf{error}.\n
\item The character sequence to be parsed by \\textbf{parse} may contain commands that \n   return expressions, including \\textbf{parse} itself. Those commands get executed after the string has been parsed.\n   \\textbf{parse}(\\emph{string}) will return the expression computed by the commands contained in the character \n   sequence \\emph{string}.\n\end{itemize}
\noindent Example 1: 
\begin{center}\begin{minipage}{15cm}\begin{Verbatim}[frame=single]
\end{Verbatim}
\end{minipage}\end{center}
\noindent Example 2: 
\begin{center}\begin{minipage}{15cm}\begin{Verbatim}[frame=single]
\end{Verbatim}
\end{minipage}\end{center}
\noindent Example 3: 
\begin{center}\begin{minipage}{15cm}\begin{Verbatim}[frame=single]
\end{Verbatim}
\end{minipage}\end{center}
See also: \textbf{execute} (\ref{labexecute}), \textbf{readfile} (\ref{labreadfile}), \textbf{print} (\ref{labprint})
