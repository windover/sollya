\subsection{tripledouble}
\label{labtripledouble}
\noindent Names: \textbf{tripledouble}, \textbf{TD}\\
represents a number as the sum of three IEEE doubles.\\
\noindent Description: \begin{itemize}

\item \\textbf{tripledouble} is both a function and a constant.\n
\item As a function, it rounds its argument to the nearest number that can be written\n   as the sum of three double precision numbers.\n
\item The algorithm used to compute \\textbf{tripledouble}($x$) is the following: let $x_h$ = \\textbf{double}($x$),\n   let $x_m$ = \\textbf{double}($x-x_h$) and let $x_l$ = \\textbf{double}($x-x_h-x_m$). \n   Return the number $x_h+x_m+x_l$. Note that if the\n   current precision is not sufficient to represent exactly $x_h+x_m+x_l$, a rounding will\n   occur and the result of \\textbf{tripledouble}(x) will be useless.\n
\item As a constant, it symbolizes the triple-double precision format. It is used in \n   contexts when a precision format is necessary, e.g. in the commands \n   \\textbf{roundcoefficients} and \\textbf{implementpoly}.\n   See the corresponding help pages for examples.\n\end{itemize}
\noindent Example 1: 
\begin{center}\begin{minipage}{15cm}\begin{Verbatim}[frame=single]
\end{Verbatim}
\end{minipage}\end{center}
See also: \textbf{single} (\ref{labsingle}), \textbf{double} (\ref{labdouble}), \textbf{doubleextended} (\ref{labdoubleextended}), \textbf{doubledouble} (\ref{labdoubledouble}), \textbf{roundcoefficients} (\ref{labroundcoefficients}), \textbf{implementpoly} (\ref{labimplementpoly})
