\subsection{length}
\label{lablength}
\noindent Name: \textbf{length}\\
computes the length of a list or string.\\
\noindent Usage: 
\begin{center}
\textbf{length}(\emph{L}) : \textsf{list} $\rightarrow$ \textsf{integer}\\
\textbf{length}(\emph{s}) : \textsf{string} $\rightarrow$ \textsf{integer}\\
\end{center}
Parameters: 
\begin{itemize}
\item \emph{L} is a list.
\item \emph{s} is a string.
\end{itemize}
\noindent Description: \begin{itemize}

\item \\textbf{length} returns the length of a list or a string, e.g. the number of elements\n   or letters.\n
\item The empty list or string have length 0.\n   If \\emph{L} is an end-elliptic list, \\textbf{length} returns +Inf.\n\end{itemize}
\noindent Example 1: 
\begin{center}\begin{minipage}{15cm}\begin{Verbatim}[frame=single]
\end{Verbatim}
\end{minipage}\end{center}
\noindent Example 2: 
\begin{center}\begin{minipage}{15cm}\begin{Verbatim}[frame=single]
\end{Verbatim}
\end{minipage}\end{center}
\noindent Example 3: 
\begin{center}\begin{minipage}{15cm}\begin{Verbatim}[frame=single]
\end{Verbatim}
\end{minipage}\end{center}
\noindent Example 4: 
\begin{center}\begin{minipage}{15cm}\begin{Verbatim}[frame=single]
\end{Verbatim}
\end{minipage}\end{center}
