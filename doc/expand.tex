\subsection{expand}
\label{labexpand}
\noindent Name: \textbf{expand}\\
expands polynomial subexpressions\\
\noindent Usage: 
\begin{center}
\textbf{expand}(\emph{function}) : \textsf{function} $\rightarrow$ \textsf{function}\\
\end{center}
Parameters: 
\begin{itemize}
\item \emph{function} represents a function
\end{itemize}
\noindent Description: \begin{itemize}

\item \\textbf{expand}(\\emph{function}) expands all polynomial subexpressions in function\n   \\emph{function} as far as possible. Factors of sums are multiplied out,\n   power operators with constant positive integer exponents are replaced\n   by multiplications and divisions are multiplied out, i.e. denomiators\n   are brought at the most interior point of expressions.\n\end{itemize}
\noindent Example 1: 
\begin{center}\begin{minipage}{15cm}\begin{Verbatim}[frame=single]
\end{Verbatim}
\end{minipage}\end{center}
\noindent Example 2: 
\begin{center}\begin{minipage}{15cm}\begin{Verbatim}[frame=single]
\end{Verbatim}
\end{minipage}\end{center}
\noindent Example 3: 
\begin{center}\begin{minipage}{15cm}\begin{Verbatim}[frame=single]
\end{Verbatim}
\end{minipage}\end{center}
See also: \textbf{simplify} (\ref{labsimplify}), \textbf{simplifysafe} (\ref{labsimplifysafe}), \textbf{horner} (\ref{labhorner})
