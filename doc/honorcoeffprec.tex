\subsection{honorcoeffprec}
\label{labhonorcoeffprec}
\noindent Name: \textbf{honorcoeffprec}\\
indicates the (forced) honoring the precision of the coefficients in \textbf{implementpoly}\\
\noindent Usage: 
\begin{center}
\textbf{honorcoeffprec} : \textsf{honorcoeffprec}\\
\end{center}
\noindent Description: \begin{itemize}

\item Used with command \textbf{implementpoly}, \textbf{honorcoeffprec} makes \textbf{implementpoly} honor
   the precision of the given polynomial. This means if a coefficient
   needs a double-double or a triple-double to be exactly stored,
   \textbf{implementpoly} will allocate appropriate space and use a double-double
   or triple-double operation even if the automatic (heuristic)
   determination implemented in command \textbf{implementpoly} indicates that the
   coefficient could be stored on less precision or, respectively, the
   operation could be performed with less precision. See \textbf{implementpoly}
   for details.
\end{itemize}
\noindent Example 1: 
\begin{center}\begin{minipage}{15cm}\begin{Verbatim}[frame=single]
> verbosity = 1!;
> q = implementpoly(1 - simplify(TD(1/6)) * x^2,[-1b-10;1b-10],1b-60,DD,"p","imp
lementation.c");
Warning: at least one of the coefficients of the given polynomial has been round
ed in a way
that the target precision can be achieved at lower cost. Nevertheless, the imple
mented polynomial
is different from the given one.
> printexpansion(q);
0x3ff0000000000000 + x^2 * 0xbfc5555555555555
> r = implementpoly(1 - simplify(TD(1/6)) * x^2,[-1b-10;1b-10],1b-60,DD,"p","imp
lementation.c",honorcoeffprec);
Warning: the infered precision of the 2th coefficient of the polynomial is great
er than
the necessary precision computed for this step. This may make the automatic dete
rmination
of precisions useless.
> printexpansion(r);
0x3ff0000000000000 + x^2 * (0xbfc5555555555555 + 0xbc65555555555555 + 0xb9055555
55555555)
\end{Verbatim}
\end{minipage}\end{center}
See also: \textbf{implementpoly} (\ref{labimplementpoly}), \textbf{printexpansion} (\ref{labprintexpansion}), \textbf{fpminimax} (\ref{labfpminimax})
