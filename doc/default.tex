\subsection{default}
\label{labdefault}
\noindent Name: \textbf{default}\\
default value for some commands.\\

\noindent Description: \begin{itemize}

\item \textbf{default} is a special value and is replaced by something depending on the 
   context where it is used. It can often be used as a joker, when you want to 
   specify one of the optional parameters of a command and not the others: set 
   the value of uninterresting parameters to \textbf{default}.

\item Global variables can be reset by affecting them the special value \textbf{default}.
\end{itemize}
\noindent Example 1: 
\begin{center}\begin{minipage}{15cm}\begin{Verbatim}[frame=single]
> p = remez(exp(x),5,[0;1],default,1e-5);
> q = remez(exp(x),5,[0;1],1,1e-5);
> p==q;
true
\end{Verbatim}
\end{minipage}\end{center}
\noindent Example 2: 
\begin{center}\begin{minipage}{15cm}\begin{Verbatim}[frame=single]
> prec=?;
165
> prec=200;
The precision has been set to 200 bits.
> prec=default;
The precision has been set to 165 bits.
\end{Verbatim}
\end{minipage}\end{center}
