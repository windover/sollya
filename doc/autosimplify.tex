\subsection{autosimplify}
\label{labautosimplify}
\noindent Name: \textbf{autosimplify}\\
activates, deactivates or inspects the value of the automatic simplification state variable\\
\noindent Usage: 
\begin{center}
\textbf{autosimplify} = \emph{activation value} : \textsf{on$|$off} $\rightarrow$ \textsf{void}\\
\textbf{autosimplify} = \emph{activation value} ! : \textsf{on$|$off} $\rightarrow$ \textsf{void}\\
\textbf{autosimplify} : \textsf{on$|$off}\\
\end{center}
Parameters: 
\begin{itemize}
\item \emph{activation value} represents \textbf{on} or \textbf{off}, i.e. activation or deactivation
\end{itemize}
\noindent Description: \begin{itemize}

\item An assignment \\textbf{autosimplify} = \\emph{activation value}, where \\emph{activation value}\n   is one of \\textbf{on} or \\textbf{off}, activates respectively deactivates the\n   automatic safe simplification of expressions of functions generated by\n   the evaluation of commands or in argument of other commands.\n    \n   \\sollya commands like \\textbf{remez}, \\textbf{taylor} or \\textbf{rationalapprox} sometimes\n   produce expressions that can be simplified. Constant subexpressions\n   can be evaluated to dyadic floating-point numbers, monomials with\n   coefficients $0$ can be eliminated. Further, expressions\n   indicated by the user perform better in many commands when simplified\n   before being passed in argument to a command. When the automatic\n   simplification of expressions is activated, \\sollya automatically\n   performs a safe (not value changing) simplification process on such\n   expressions.\n    \n   The automatic generation of subexpressions can be annoying, in\n   particular if it takes too much time for not enough benefit. Further the\n   user might want to inspect the structure of the expression tree\n   returned by a command. In this case, the automatic simplification\n   should be deactivated.\n    \n   If the assignment \\textbf{autosimplify} = \\emph{activation value} is followed by an\n   exclamation mark, no message indicating the new state is\n   displayed. Otherwise the user is informed of the new state of the\n   global mode by an indication.\n\end{itemize}
\noindent Example 1: 
\begin{center}\begin{minipage}{15cm}\begin{Verbatim}[frame=single]
\end{Verbatim}
\end{minipage}\end{center}
\noindent Example 2: 
\begin{center}\begin{minipage}{15cm}\begin{Verbatim}[frame=single]
\end{Verbatim}
\end{minipage}\end{center}
See also: \textbf{print} (\ref{labprint}), \textbf{prec} (\ref{labprec}), \textbf{points} (\ref{labpoints}), \textbf{diam} (\ref{labdiam}), \textbf{display} (\ref{labdisplay}), \textbf{verbosity} (\ref{labverbosity}), \textbf{canonical} (\ref{labcanonical}), \textbf{taylorrecursions} (\ref{labtaylorrecursions}), \textbf{timing} (\ref{labtiming}), \textbf{fullparentheses} (\ref{labfullparentheses}), \textbf{midpointmode} (\ref{labmidpointmode}), \textbf{hopitalrecursions} (\ref{labhopitalrecursions}), \textbf{remez} (\ref{labremez}), \textbf{rationalapprox} (\ref{labrationalapprox}), \textbf{taylor} (\ref{labtaylor})
