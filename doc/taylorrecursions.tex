\subsection{taylorrecursions}
\label{labtaylorrecursions}
\noindent Name: \textbf{taylorrecursions}\\
controls the number of recursion steps when applying Taylor's rule.\\
\noindent Description: \begin{itemize}

\item \textbf{taylorrecursions} is a global variable. Its value represents the number of steps
   of recursion that are used when applying Taylor's rule. This rule is applied
   by the interval evaluator present in the core of \sollya (and particularly
   visible in commands like \textbf{infnorm}).

\item To improve the quality of an interval evaluation of a function $f$, in 
   particular when there are problems of decorrelation), the evaluator of \sollya
   uses Taylor's rule:  $f([a,b]) \subseteq f(m) + [a-m,\,b-m] \cdot f'([a,\,b])$ where $m=\frac{a+b}{2}$.
   This rule can be applied recursively.
   The number of step in this recursion process is controlled by \textbf{taylorrecursions}.

\item Setting \textbf{taylorrecursions} to 0 makes \sollya use this rule only once;
   setting it to 1 makes \sollya use the rule twice, and so on.
   In particular: the rule is always applied at least once.
\end{itemize}
\noindent Example 1: 
\begin{center}\begin{minipage}{15cm}\begin{Verbatim}[frame=single]
> f=exp(x);
> p=remez(f,3,[0;1]);
> taylorrecursions=0;
The number of recursions for Taylor evaluation has been set to 0.
> evaluate(f-p, [0;1]);
[-0.46839364816303627522963565754743169862357620487739;0.46947781754667086491682
464997088054443583003517779]
> taylorrecursions=1;
The number of recursions for Taylor evaluation has been set to 1.
> evaluate(f-p, [0;1]);
[-0.13813111495387910066337940912697015317218647208804;0.13921528433751369035056
840155041899898444030238844]
\end{Verbatim}
\end{minipage}\end{center}
