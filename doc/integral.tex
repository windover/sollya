\subsection{integral}
\label{labintegral}
\noindent Name: \textbf{integral}\\
computes an interval bounding the integral of a function on an interval.\\
\noindent Usage: 
\begin{center}
\textbf{integral}(\emph{f},\emph{I}) : (\textsf{function}, \textsf{range}) $\rightarrow$ \textsf{range}\\
\end{center}
Parameters: 
\begin{itemize}
\item \emph{f} is a function.
\item \emph{I} is an interval.
\end{itemize}
\noindent Description: \begin{itemize}

\item \\textbf{integral}(\\emph{f},\\emph{I}) returns an interval $J$ such that the exact value of \n   the integral of \\emph{f} on \\emph{I} lies in $J$.\n
\item This command is safe but very inefficient. Use \\textbf{dirtyintegral} if you just want\n   an approximate value.\n
\item The result of this command depends on the global variable \\textbf{diam}.\n   The method used is the following: \\emph{I} is cut into intervals of length not \n   greater then $\\delta \\cdot |I|$ where $\\delta$ is the value\n   of global variable \\textbf{diam}.\n   On each small interval \\emph{J}, an evaluation of \\emph{f} by interval is\n   performed. The result is multiplied by the length of \\emph{J}. Finally all values \n   are summed.\n\end{itemize}
\noindent Example 1: 
\begin{center}\begin{minipage}{15cm}\begin{Verbatim}[frame=single]
\end{Verbatim}
\end{minipage}\end{center}
See also: \textbf{diam} (\ref{labdiam}), \textbf{dirtyintegral} (\ref{labdirtyintegral})
