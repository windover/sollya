\subsection{worstcase}
\label{labworstcase}
\noindent Name: \textbf{worstcase}\\
searches for hard-to-round cases of a function\\
\noindent Usage: 
\begin{center}
\textbf{worstcase}(\emph{function}, \emph{preimage precision}, \emph{preimage exponent range}, \emph{image precision}, \emph{error bound}) : (\textsf{function}, \textsf{integer}, \textsf{range}, \textsf{integer}, \textsf{constant}) $\rightarrow$ \textsf{void}\\
\textbf{worstcase}(\emph{function}, \emph{preimage precision}, \emph{preimage exponent range}, \emph{image precision}, \emph{error bound}, \emph{filename}) : (\textsf{function}, \textsf{integer}, \textsf{range}, \textsf{integer}, \textsf{constant}, \textsf{string}) $\rightarrow$ \textsf{void}\\
\end{center}
Parameters: 
\begin{itemize}
\item \emph{function} represents the function to be considered
\item \emph{preimage precision} represents the precision of the preimages
\item \emph{preimage exponent range} represents the exponents in the preimage format
\item \emph{image precision} represents the precision of the format the images are to be rounded to
\item \emph{error bound} represents the upper bound for the search w.r.t. the relative rounding error
\item \emph{filename} represents a character sequence containing a filename
\end{itemize}
\noindent Description: \begin{itemize}

\item The \\textbf{worstcase} command is deprecated. It searches for hard-to-round cases of\n   a function. The command \\textbf{searchgal} has a comparable functionality.\n\end{itemize}
\noindent Example 1: 
\begin{center}\begin{minipage}{15cm}\begin{Verbatim}[frame=single]
\end{Verbatim}
\end{minipage}\end{center}
See also: \textbf{round} (\ref{labround}), \textbf{searchgal} (\ref{labsearchgal}), \textbf{evaluate} (\ref{labevaluate})
