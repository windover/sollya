\subsection{divide}
\label{labdivide}
\noindent Name: \textbf{/}\\
division function\\
\noindent Usage: 
\begin{center}
\emph{function1} \textbf{/} \emph{function2} : (\textsf{function}, \textsf{function}) $\rightarrow$ \textsf{function}
\end{center}
Parameters: 
\begin{itemize}
\item \emph{function1} and \emph{function2} represent functions
\end{itemize}
\noindent Description: \begin{itemize}

\item \textbf{/} represents the division (function) on reals. 
   The expression \emph{function1} \textbf{/} \emph{function2} stands for
   the function composed of the division function and the two
   functions \emph{function1} and \emph{function2}, where \emph{function1} is
   the numerator and \emph{function2} the denominator.
\end{itemize}
\noindent Example 1: 
\begin{center}\begin{minipage}{15cm}\begin{Verbatim}[frame=single]
> 5 / 2;
2.5
\end{Verbatim}
\end{minipage}\end{center}
\noindent Example 2: 
\begin{center}\begin{minipage}{15cm}\begin{Verbatim}[frame=single]
> x / 2;
x * 0.5
\end{Verbatim}
\end{minipage}\end{center}
\noindent Example 3: 
\begin{center}\begin{minipage}{15cm}\begin{Verbatim}[frame=single]
> x / x;
1
\end{Verbatim}
\end{minipage}\end{center}
\noindent Example 4: 
\begin{center}\begin{minipage}{15cm}\begin{Verbatim}[frame=single]
> 3 / 0;
@NaN@
\end{Verbatim}
\end{minipage}\end{center}
\noindent Example 5: 
\begin{center}\begin{minipage}{15cm}\begin{Verbatim}[frame=single]
> diff(sin(x) / exp(x));
(exp(x) * cos(x) - sin(x) * exp(x)) / exp(x)^2
\end{Verbatim}
\end{minipage}\end{center}
See also: \textbf{$+$} (\ref{labplus}), \textbf{$-$} (\ref{labminus}), \textbf{$*$} (\ref{labmult}), \textbf{\^} (\ref{labpower})
