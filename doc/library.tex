\subsection{library}
\label{lablibrary}
\noindent Name: \textbf{library}\\
binds an external mathematical function to a variable in \sollya\\
\noindent Usage: 
\begin{center}
\textbf{library}(\emph{path}) : \textsf{string} $\rightarrow$ \textsf{function}\\
\end{center}
\noindent Description: \begin{itemize}

\item The command \\textbf{library} lets you extend the set of mathematical\n   functions known to \\sollya.\n   By default, \\sollya knows the most common mathematical functions such\n   as \\textbf{exp}, \\textbf{sin}, \\textbf{erf}, etc. Within \\sollya, these functions may be\n   composed. This way, \\sollya should satisfy the needs of a lot of\n   users. However, for particular applications, one may want to\n   manipulate other functions such as Bessel functions, or functions\n   defined by an integral or even a particular solution of an ODE.\n
\item \\textbf{library} makes it possible to let \\sollya know about new functions. In\n   order to let it know, you have to provide an implementation of the\n   function you are interested in. This implementation is a C file containing\n   a function of the form:\n   \\begin{verbatim} int my_ident(mpfi_t result, mpfi_t op, int n)\\end{verbatim}\n   The semantic of this function is the following: it is an implementation of\n   the function and its derivatives in interval arithmetic.\n   \\verb|my_ident(result, I, n)| shall store in \\verb|result| an enclosure \n   of the image set of the $n$-th derivative\n   of the function f over \\verb|I|: $f^{(n)}(I) \\subseteq \\mathrm{result}$.\n
\item The integer value returned by the function implementation currently has no meaning.\n
\item You do not need to provide a working implementation for any \\verb|n|. Most functions\n   of \\sollya requires a relevant implementation only for $f$, $f'$ and $f''$. For higher \n   derivatives, its is not so critical and the implementation may just store \n   $[-\\infty,\\,+\\infty]$ in result whenever $n>2$.\n
\item Note that you should respect somehow MPFI standards in your implementation:\n   \\verb|result| has its own precision and you should perform the \n   intermediate computations so that \\verb|result| is as tight as possible.\n
\item You can include sollya.h in your implementation and use library \n   functionnalities of \\sollya for your implementation. However, this requires to have compiled\n   \\sollya with \\texttt{-fPIC} in order to make the \\sollya executable code position \n   independent and to use a system on with programs, using \\texttt{dlopen} to open\n   dynamic routines can dynamically open themselves.\n
\item To bind your function into \\sollya, you must use the same identifier as the\n   function name used in your implementation file (\\verb|my_ident| in the previous\n   example). Once the function code has been bound to an identifier, you can use a simple assignment\n   to assign the bound identifier to yet another identifier. This way, you may use convenient\n   names inside \\sollya even if your implementation environment requires you to use a less\n   convenient name.\n\end{itemize}
\noindent Example 1: 
\begin{center}\begin{minipage}{15cm}\begin{Verbatim}[frame=single]
\end{Verbatim}
\end{minipage}\end{center}
See also: \textbf{bashexecute} (\ref{labbashexecute}), \textbf{externalproc} (\ref{labexternalproc}), \textbf{externalplot} (\ref{labexternalplot})
