\subsection{ library }
\noindent Name: \textbf{library}\\
binds an external mathematical function to a variable in Sollya\\

\noindent Usage: 
\begin{center}
\textbf{library}(\emph{path}) : \textsf{string} $\rightarrow$ \textsf{function}\\
\end{center}
\noindent Description: \begin{itemize}

\item The command \textbf{library} lets you extends the set of mathematical
   functions known by Sollya.
   By default, Sollya knows the most common mathematical functions such
   as \textbf{exp}, \textbf{sin}, \textbf{erf}, etc. Within Sollya, these functions may be
   composed. This way, Sollya should satisfy the needs of a lot of
   users. However, for particular applications, one may want to
   manipulates other functions such as Bessel functions, or functions
   defined by an integral or even a particular solution of an ODE.

\item \textbf{library} makes it possible to let Sollya know about new functions. In
   order to let it know, you have to provide an implementation of the
   function you are interested with. This implementation is a C file containing
   a function of the form:
   \begin{verbatim} int my_ident(mpfi_t result, mpfi_t op, int n)\end{verbatim}
   The semantic of this function is the following: it is an implementation of
   the function and its derivatives in interval arithmetic.
   \verb|my_ident(result, I, n)| shall store in \verb|result| an enclosure 
   of the image set of the n-th derivative
   of the function f over \verb|I|: $f^{(n)}(I) \subseteq \mathrm{result}$.

\item The integer returned value has no meaning currently.

\item You must not provide a non trivial implementation for any \verb|n|. Most functions
   of Sollya needs a relevant implementation of $f$, $f'$ and $f''$. For higher 
   derivatives, its is not so critical and the implementation may just store 
   $[-\infty,\,+\infty]$ in result whenever $n>2$.

\item Note that you should respect somehow MPFI standards in your implementation:
   \verb|result| has its own precision and you should perform the 
   intermediate computations so that \verb|result| is as tighter as possible.

\item You can include sollya.h in your implementation and use library 
   functionnalities of Sollya for your implementation.

\item To bind your function into Sollya, you must use the same identifier as the
   function name used in your implementation file (\verb|my_ident| in the previous
   example).
\end{itemize}
\noindent Example 1: 
\begin{center}\begin{minipage}{14.8cm}\begin{Verbatim}[frame=single]
   > bashexecute("gcc -fPIC -Wall -c libraryexample.c");
   > bashexecute("gcc -shared -o libraryexample libraryexample.o -lgmp -lmpfr");
   > myownlog = library("./libraryexample");
   > evaluate(log(x), 2);
   0.693147180559945309417232121458176568075500134360248314207
   > evaluate(myownlog(x), 2);
   0.693147180559945309417232121458176568075500134360248314207
\end{Verbatim}
\end{minipage}\end{center}
See also: \textbf{bashexecute}, \textbf{externalproc}, \textbf{externalplot}
