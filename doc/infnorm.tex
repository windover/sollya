\subsection{infnorm}
\label{labinfnorm}
\noindent Name: \textbf{infnorm}\\
computes an interval bounding the infinity norm of a function on an interval.\\
\noindent Usage: 
\begin{center}
\textbf{infnorm}(\emph{f},\emph{I},\emph{filename},\emph{Ilist}) : (\textsf{function}, \textsf{range}, \textsf{string}, \textsf{list}) $\rightarrow$ \textsf{range}\\
\end{center}
Parameters: 
\begin{itemize}
\item \emph{f} is a function.
\item \emph{I} is an interval.
\item \emph{filename} (optional) is the name of the file into a proof will be saved.
\item \emph{IList} (optional) is a list of intervals to be excluded.
\end{itemize}
\noindent Description: \begin{itemize}

\item \textbf{infnorm}(\emph{f},\emph{range}) computes an interval bounding the infinity norm of the 
   given function $f$ on the interval $I$, e.g. computes an interval $J$
   such that $\max_{x \in I} \{|f(x)|\} \subseteq J$.

\item If \emph{filename} is given, a proof in English will be produced (and stored in file
   called \emph{filename}) proving that  $\max_{x \in I} \{|f(x)|\} \subseteq J$.

\item If a list \emph{IList} of intervals $I_1, \dots, I_n$ is given, the infinity norm will
   be computed on $I \backslash (I_1 \cup \dots \cup I_n)$.

\item The function \emph{f} is assumed to be at least twice continuous on \emph{I}. More 
   generally, if \emph{f} is $\mathcal{C}^k$, global variables \textbf{hopitalrecursions} and
   \textbf{taylorrecursions} must have values not greater than $k$.  

\item If the interval is reduced to a single point, the result of \textbf{infnorm} is an 
   interval containing the exact absolute value of \emph{f} at this point.

\item If the interval is not bound, the result will be $[0,\,+\infty]$ 
   which is correct but perfectly useless. \textbf{infnorm} is not meant to be used with 
   infinite intervals.

\item The result of this command depends on the global variables \textbf{prec}, \textbf{diam},
   \textbf{taylorrecursions} and \textbf{hopitalrecursions}. The contribution of each variable is 
   not easy even to analyse.
   \begin{itemize}
   \item  The algorithm uses interval arithmetic with precision \textbf{prec}. The
     precision should thus be set high enough to ensure that no critical
     cancellation will occur.
   \item  When an evaluation is performed on an interval $[a,\,b]$, if the result
     is considered being too large, the interval is split into $[a,\,\frac{a+b}{2}]$
     and $[\frac{a+b}{2},\,b]$ and so on recursively. This recursion step
     is  not performed if the $(b-a) < \delta \cdot |I|$ where $\delta$ is the value
     of variable \textbf{diam}. In other words, \textbf{diam} controls the minimum length of an
     interval during the algorithm.
   \item  To perform the evaluation of a function on an interval, Taylor's rule is
     applied, e.g. $f([a,b]) \subseteq f(m) + [a-m,\,b-m] \cdot f'([a,\,b])$
     where $m=\frac{a+b}{2}$. This rule is recursively applied $n$ times
     where $n$ is the value of variable \textbf{taylorrecursions}. Roughly speaking,
     the evaluations will avoid decorrelation up to order $n$.
   \item  When a function of the form $\frac{g}{h}$ has to be evaluated on an
     interval $[a,\,b]$ and when $g$ and $h$ vanish at a same point
     $z$ of the interval, the ratio may be defined even if the expression
     $\frac{g(z)}{h(z)}=\frac{0}{0}$ does not make any sense. In this case, L'Hopital's rule
     may be used and $\left(\frac{g}{h}\right)([a,\,b]) \subseteq \left(\frac{g'}{h'}\right)([a,\,b])$.
     Since the same can occur with the ratio $\frac{g'}{h'}$, the rule is applied
     recursively. The variable \textbf{hopitalrecursions} controls the number of 
     recursion steps.
   \end{itemize}

\item The algorithm used for this command is quite complex to be explained here. 
   Please find a complete description in the following article:\\
        S. Chevillard and C. Lauter\\
        A certified infinity norm for the implementation of elementary functions\\
        LIP Research Report number RR2007-26\\
        http://prunel.ccsd.cnrs.fr/ensl-00119810\\
\end{itemize}
\noindent Example 1: 
\begin{center}\begin{minipage}{15cm}\begin{Verbatim}[frame=single]
> infnorm(exp(x),[-2;3]);
[2.00855369231876677409285296545817178969879078385537e1;2.0085536923187667740928
5296545817178969879078385544e1]
\end{Verbatim}
\end{minipage}\end{center}
\noindent Example 2: 
\begin{center}\begin{minipage}{15cm}\begin{Verbatim}[frame=single]
> infnorm(exp(x),[-2;3],"proof.txt");
[2.00855369231876677409285296545817178969879078385537e1;2.0085536923187667740928
5296545817178969879078385544e1]
\end{Verbatim}
\end{minipage}\end{center}
\noindent Example 3: 
\begin{center}\begin{minipage}{15cm}\begin{Verbatim}[frame=single]
> infnorm(exp(x),[-2;3],[| [0;1], [2;2.5] |]);
[2.00855369231876677409285296545817178969879078385537e1;2.0085536923187667740928
5296545817178969879078385544e1]
\end{Verbatim}
\end{minipage}\end{center}
\noindent Example 4: 
\begin{center}\begin{minipage}{15cm}\begin{Verbatim}[frame=single]
> infnorm(exp(x),[-2;3],"proof.txt", [| [0;1], [2;2.5] |]);
[2.00855369231876677409285296545817178969879078385537e1;2.0085536923187667740928
5296545817178969879078385544e1]
\end{Verbatim}
\end{minipage}\end{center}
\noindent Example 5: 
\begin{center}\begin{minipage}{15cm}\begin{Verbatim}[frame=single]
> infnorm(exp(x),[1;1]);
[2.71828182845904523536028747135266249775724709369989;2.718281828459045235360287
47135266249775724709369998]
\end{Verbatim}
\end{minipage}\end{center}
\noindent Example 6: 
\begin{center}\begin{minipage}{15cm}\begin{Verbatim}[frame=single]
> infnorm(exp(x), [log(0);log(1)]);
[0;@Inf@]
\end{Verbatim}
\end{minipage}\end{center}
See also: \textbf{prec} (\ref{labprec}), \textbf{diam} (\ref{labdiam}), \textbf{hopitalrecursions} (\ref{labhopitalrecursions}), \textbf{dirtyinfnorm} (\ref{labdirtyinfnorm}), \textbf{checkinfnorm} (\ref{labcheckinfnorm})
