\subsection{prec}
\label{labprec}
\noindent Name: \textbf{prec}\\
controls the precision used in numerical computations.\\

\noindent Description: \begin{itemize}

\item \textbf{prec} is a global variable. Its value represents the precision of the 
   floating-point format used in numerical computations.

\item Many commands try to adapt their intern precision in order to have 
   approximately $n$ correct bits in output, where $n$ is the value of \textbf{prec}.
\end{itemize}
\noindent Example 1: 
\begin{center}\begin{minipage}{15cm}\begin{Verbatim}[frame=single]
> display=binary!;
> prec=50;
The precision has been set to 50 bits.
> dirtyinfnorm(exp(x),[1;2]);
1.110110001110011001001011100011010100110111011011_2 * 2^(2)
> prec=100;
The precision has been set to 100 bits.
> dirtyinfnorm(exp(x),[1;2]);
1.110110001110011001001011100011010100110111011010110111001100001100111010001110
11101000100000011011_2 * 2^(2)
\end{Verbatim}
\end{minipage}\end{center}
