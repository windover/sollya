\subsection{degree}
\label{labdegree}
\noindent Name: \textbf{degree}\\
gives the degree of a polynomial.\\
\noindent Usage: 
\begin{center}
\textbf{degree}(\emph{f}) : \textsf{function} $\rightarrow$ \textsf{integer}\\
\end{center}
Parameters: 
\begin{itemize}
\item \emph{f} is a function (usually a polynomial).
\end{itemize}
\noindent Description: \begin{itemize}

\item If \\emph{f} is a polynomial, \\textbf{degree}(\\emph{f}) returns the degree of \\emph{f}.\n
\item Contrary to the usage, \\sollya considers that the degree of the null polynomial\n   is 0.\n
\item If \\emph{f} is a function that is not a polynomial, \\textbf{degree}(\\emph{f}) returns -1.\n\end{itemize}
\noindent Example 1: 
\begin{center}\begin{minipage}{15cm}\begin{Verbatim}[frame=single]
\end{Verbatim}
\end{minipage}\end{center}
\noindent Example 2: 
\begin{center}\begin{minipage}{15cm}\begin{Verbatim}[frame=single]
\end{Verbatim}
\end{minipage}\end{center}
See also: \textbf{coeff} (\ref{labcoeff})
