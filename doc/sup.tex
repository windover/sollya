\subsection{sup}
\label{labsup}
\noindent Name: \textbf{sup}\\
\phantom{aaa}gives the upper bound of an interval.\\[0.2cm]
\noindent Library name:\\
\verb|   sollya_obj_t sollya_lib_sup(sollya_obj_t)|\\[0.2cm]
\noindent Usage: 
\begin{center}
\textbf{sup}(\emph{I}) : \textsf{range} $\rightarrow$ \textsf{constant}\\
\textbf{sup}(\emph{x}) : \textsf{constant} $\rightarrow$ \textsf{constant}\\
\end{center}
Parameters: 
\begin{itemize}
\item \emph{I} is an interval.
\item \emph{x} is a real number.
\end{itemize}
\noindent Description: \begin{itemize}

\item Returns the upper bound of the interval \emph{I}. Each bound of an interval has its 
   own precision, so this command is exact, even if the current precision is too 
   small to represent the bound.

\item When called on a real number \emph{x}, \textbf{sup} behaves like the identity.
\end{itemize}
\noindent Example 1: 
\begin{center}\begin{minipage}{15cm}\begin{Verbatim}[frame=single]
> sup([1;3]);
3
> sup(5);
5
\end{Verbatim}
\end{minipage}\end{center}
\noindent Example 2: 
\begin{center}\begin{minipage}{15cm}\begin{Verbatim}[frame=single]
> display=binary!;
> I=[0; 0.111110000011111_2];
> sup(I);
1.11110000011111_2 * 2^(-1)
> prec=12!;
> sup(I);
1.11110000011111_2 * 2^(-1)
\end{Verbatim}
\end{minipage}\end{center}
See also: \textbf{inf} (\ref{labinf}), \textbf{mid} (\ref{labmid}), \textbf{max} (\ref{labmax}), \textbf{min} (\ref{labmin})
