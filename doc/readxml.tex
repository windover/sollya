\subsection{readxml}
\label{labreadxml}
\noindent Name: \textbf{readxml}\\
reads an expression written as a MathML-Content-Tree in a file\\
\noindent Usage: 
\begin{center}
\textbf{readxml}(\emph{filename}) : \textsf{string} $\rightarrow$ \textsf{function} $|$ \textsf{error}\\
\end{center}
Parameters: 
\begin{itemize}
\item \emph{filename} represents a character sequence indicating a file name
\end{itemize}
\noindent Description: \begin{itemize}

\item \\textbf{readxml}(\\emph{filename}) reads the first occurrence of a lambda\n   application with one bounded variable on applications of the supported\n   basic functions in file \\emph{filename} and returns it as a \\sollya\n   functional expression.\n    \n   If the file \\emph{filename} does not contain a valid MathML-Content tree,\n   \\textbf{readxml} tries to find an "annotation encoding" markup of type\n   "sollya/text". If this annotation contains a character sequence\n   that can be parsed by \\textbf{parse}, \\textbf{readxml} returns that expression.  Otherwise\n   \\textbf{readxml} displays a warning and returns an \\textbf{error} variable of type\n   \\textsf{error}.\n\end{itemize}
\noindent Example 1: 
\begin{center}\begin{minipage}{15cm}\begin{Verbatim}[frame=single]
\end{Verbatim}
\end{minipage}\end{center}
See also: \textbf{printxml} (\ref{labprintxml}), \textbf{readfile} (\ref{labreadfile}), \textbf{parse} (\ref{labparse})
