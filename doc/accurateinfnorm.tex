\subsection{accurateinfnorm}
\label{labaccurateinfnorm}
\noindent Name: \textbf{accurateinfnorm}\\
computes a faithful rounding of the infinity norm of a function \\
\noindent Usage: 
\begin{center}
\textbf{accurateinfnorm}(\emph{function},\emph{range},\emph{constant}) : (\textsf{function}, \textsf{range}, \textsf{constant}) $\rightarrow$ \textsf{constant}\\
\textbf{accurateinfnorm}(\emph{function},\emph{range},\emph{constant},\emph{exclusion range 1},...,\emph{exclusion range n}) : (\textsf{function}, \textsf{range}, \textsf{constant}, \textsf{range}, ..., \textsf{range}) $\rightarrow$ \textsf{constant}\\
\end{center}
Parameters: 
\begin{itemize}
\item \emph{function} represents the function whose infinity norm is to be computed
\item \emph{range} represents the infinity norm is to be considered on
\item \emph{constant} represents the number of bits in the significant of the result
\item \emph{exclusion range 1} through \emph{exclusion range n} represent ranges to be excluded 
\end{itemize}
\noindent Description: \begin{itemize}

\item The command \\textbf{accurateinfnorm} computes an upper bound to the infinity norm of\n   function \\emph{function} in \\emph{range}. This upper bound is the least\n   floating-point number greater than the value of the infinity norm that\n   lies in the set of dyadic floating point numbers having \\emph{constant}\n   significant mantissa bits. This means the value \\textbf{accurateinfnorm} evaluates to\n   is at the time an upper bound and a faithful rounding to \\emph{constant}\n   bits of the infinity norm of function \\emph{function} on range \\emph{range}.\n    \n   If given, the fourth and further arguments of the command \\textbf{accurateinfnorm},\n   \\emph{exclusion range 1} through \\emph{exclusion range n} the infinity norm of\n   the function \\emph{function} is not to be considered on.\n\end{itemize}
\noindent Example 1: 
\begin{center}\begin{minipage}{15cm}\begin{Verbatim}[frame=single]
\end{Verbatim}
\end{minipage}\end{center}
\noindent Example 2: 
\begin{center}\begin{minipage}{15cm}\begin{Verbatim}[frame=single]
\end{Verbatim}
\end{minipage}\end{center}
See also: \textbf{infnorm} (\ref{labinfnorm}), \textbf{dirtyinfnorm} (\ref{labdirtyinfnorm}), \textbf{checkinfnorm} (\ref{labcheckinfnorm}), \textbf{remez} (\ref{labremez}), \textbf{diam} (\ref{labdiam})
