\subsection{ absolute }
\noindent Name: \textbf{perturb}\\
indicates an absolute error for \textbf{externalplot}\\

\noindent Usage: 
\begin{center}
\textbf{perturb} : \textsf{absolute|relative}\\
\end{center}
\noindent Description: \begin{itemize}

\item The use of \textbf{perturb} in the command \textbf{externalplot} indicates that during
   plotting in \textbf{externalplot} an absolute error is to be considered.
   See \textbf{externalplot} for details.
\end{itemize}
\noindent Example 1: 
\begin{center}\begin{minipage}{14.8cm}\begin{Verbatim}[frame=single]
   > bashexecute("gcc -fPIC -c externalplotexample.c");
   > bashexecute("gcc -shared -o externalplotexample externalplotexample.o -lgmp -lmpfr");
   > externalplot("./externalplotexample",absolute,exp(x),[-1/2;1/2],12,perturb);
\end{Verbatim}
\end{minipage}\end{center}
See also: \textbf{externalplot}, \textbf{relative}, \textbf{bashexecute}
