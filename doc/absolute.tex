\subsection{absolute}
\label{lababsolute}
\noindent Name: \textbf{absolute}\\
indicates an absolute error for \textbf{externalplot} or \textbf{fpminimax}\\
\noindent Usage: 
\begin{center}
\textbf{absolute} : \textsf{absolute$|$relative}\\
\end{center}
\noindent Description: \begin{itemize}

\item The use of \\textbf{absolute} in the command \\textbf{externalplot} indicates that during\n   plotting in \\textbf{externalplot} an absolute error is to be considered.\n    \n   See \\textbf{externalplot} for details.\n
\item Used with \\textbf{fpminimax}, \\textbf{absolute} indicates that \\textbf{fpminimax} must try to minimize\n   the absolute error.\n    \n   See \\textbf{fpminimax} for details.\n\end{itemize}
\noindent Example 1: 
\begin{center}\begin{minipage}{15cm}\begin{Verbatim}[frame=single]
\end{Verbatim}
\end{minipage}\end{center}
See also: \textbf{externalplot} (\ref{labexternalplot}), \textbf{fpminimax} (\ref{labfpminimax}), \textbf{relative} (\ref{labrelative}), \textbf{bashexecute} (\ref{labbashexecute})
