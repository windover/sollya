\subsection{pi}
\label{labpi}
\noindent Name: \textbf{pi}\\
the constant $\pi$.\\
\noindent Description: \begin{itemize}

\item \textbf{pi} is the constant $\pi$, defined as half the period of sine and cosine.

\item In \sollya, \textbf{pi} is considered as a 0-ary function. This way, the constant 
   is not evaluated at the time of its definition but at the time of its use. For 
   instance, when you define a constant or a function relating to $\pi$, the current
   precision at the time of the definition does not matter. What is important is 
   the current precision when you evaluate the function or the constant value.

\item Remark that when you define an interval, the bounds are first evaluated and 
   then the interval is defined. In this case, \textbf{pi} will be evaluated as any 
   other constant value at the definition time of the interval, thus using the 
   current precision at this time.
\end{itemize}
\noindent Example 1: 
\begin{center}\begin{minipage}{15cm}\begin{Verbatim}[frame=single]
> verbosity=1!; prec=12!;
> a = 2*pi;
> a;
Warning: rounding has happened. The value displayed is a faithful rounding of th
e true result.
6.283
> prec=20!;
> a;
Warning: rounding has happened. The value displayed is a faithful rounding of th
e true result.
6.283187
\end{Verbatim}
\end{minipage}\end{center}
\noindent Example 2: 
\begin{center}\begin{minipage}{15cm}\begin{Verbatim}[frame=single]
> prec=12!;
> d = [pi; 5];
> d;
[3.1406;5]
> prec=20!;
> d;
[3.140625;5]
\end{Verbatim}
\end{minipage}\end{center}
See also: \textbf{cos} (\ref{labcos}), \textbf{sin} (\ref{labsin})
