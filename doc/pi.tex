\subsection{pi}
\label{labpi}
\noindent Name: \textbf{pi}\\
the constant $\pi$.\\
\noindent Description: \begin{itemize}

\item \\textbf{pi} is the constant $\\pi$, defined as half the period of sine and cosine.\n
\item In \\sollya, \\textbf{pi} is considered a 0-ary function. This way, the constant \n   is not evaluated at the time of its definition but at the time of its use. For \n   instance, when you define a constant or a function relating to $\\pi$, the current\n   precision at the time of the definition does not matter. What is important is \n   the current precision when you evaluate the function or the constant value.\n
\item Remark that when you define an interval, the bounds are first evaluated and \n   then the interval is defined. In this case, \\textbf{pi} will be evaluated as any \n   other constant value at the definition time of the interval, thus using the \n   current precision at this time.\n\end{itemize}
\noindent Example 1: 
\begin{center}\begin{minipage}{15cm}\begin{Verbatim}[frame=single]
\end{Verbatim}
\end{minipage}\end{center}
\noindent Example 2: 
\begin{center}\begin{minipage}{15cm}\begin{Verbatim}[frame=single]
\end{Verbatim}
\end{minipage}\end{center}
See also: \textbf{cos} (\ref{labcos}), \textbf{sin} (\ref{labsin})
