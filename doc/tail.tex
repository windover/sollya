\subsection{tail}
\label{labtail}
\noindent Name: \textbf{tail}\\
gives the tail of a list.\\

\noindent Usage: 
\begin{center}
\textbf{tail}(\emph{L}) : \textsf{list} $\rightarrow$ \textsf{any type}\\
\end{center}
Parameters: 
\begin{itemize}
\item \emph{L} is a list.
\end{itemize}
\noindent Description: \begin{itemize}

\item \textbf{tail}(\emph{L}) returns the list \emph{L} without its first element.

\item If \emph{L} is empty, the command will fail with an error.

\item \textbf{tail} can also be used with end-elliptic lists. In this case, the result of
   \textbf{tail} is also an end-elliptic list.
\end{itemize}
\noindent Example 1: 
\begin{center}\begin{minipage}{15cm}\begin{Verbatim}[frame=single]
> tail([|1,2,3|]);
[|2, 3|]
> tail([|1,2...|]);
[|2...|]
\end{Verbatim}
\end{minipage}\end{center}
See also: \textbf{head} (\ref{labhead})
