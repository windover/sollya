\subsection{nop}
\label{labnop}
\noindent Name: \textbf{nop}\\
no operation\\
\noindent Usage: 
\begin{center}
\textbf{nop} : \textsf{void} $\rightarrow$ \textsf{void}\\
\textbf{nop}() : \textsf{void} $\rightarrow$ \textsf{void}\\
\textbf{nop}(\emph{n}) : \textsf{integer} $\rightarrow$ \textsf{void}\\
\end{center}
\noindent Description: \begin{itemize}

\item The command \\textbf{nop} does nothing. This means it is an explicit parse\n   element in the \\sollya language that finally does not produce any\n   result or side-effect.\n
\item The command \\textbf{nop} may take an optional positive integer argument \\emph{n}. The argument controls how much (useless) integer additions \\sollya performs while doing nothing. \n   With this behaviour, \\textbf{nop} can be used for calibration of timing tests.\n
\item The keyword \\textbf{nop} is implicit in some procedure\n   definitions. Procedures without imperative body get parsed as if they\n   had an imperative body containing one \\textbf{nop} statement.\n\end{itemize}
\noindent Example 1: 
\begin{center}\begin{minipage}{15cm}\begin{Verbatim}[frame=single]
\end{Verbatim}
\end{minipage}\end{center}
\noindent Example 2: 
\begin{center}\begin{minipage}{15cm}\begin{Verbatim}[frame=single]
\end{Verbatim}
\end{minipage}\end{center}
\noindent Example 3: 
\begin{center}\begin{minipage}{15cm}\begin{Verbatim}[frame=single]
\end{Verbatim}
\end{minipage}\end{center}
See also: \textbf{proc} (\ref{labproc})
