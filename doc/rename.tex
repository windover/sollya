\subsection{rename}
\label{labrename}
\noindent Name: \textbf{rename}\\
rename the free variable.\\
\noindent Usage: 
\begin{center}
\textbf{rename}(\emph{ident1},\emph{ident2}) : \textsf{void}\\
\end{center}
Parameters: 
\begin{itemize}
\item \emph{ident1} is the current name of the free variable.
\item \emph{ident2} is a fresh name.
\end{itemize}
\noindent Description: \begin{itemize}

\item \textbf{rename} permits a change of the name of the free variable. \sollya can handle only
   one free variable at a time. The first time in a session that an unbound name 
   is used in a context where it can be interpreted as a free variable, the name
   is used to represent the free variable of \sollya. In the following, this name
   can be changed using \textbf{rename}.

\item Be careful: if \emph{ident2} has been set before, its value will be lost. Use the 
   command \textbf{isbound} to know if \emph{ident2} is already used or not.

\item If \emph{ident1} is not the current name of the free variable, an error occurs.

\item If \textbf{rename} is used at a time when the name of the free variable has not been 
   defined, \emph{ident1} is just ignored and the name of the free variable is 
   set to \emph{ident2}.
\end{itemize}
\noindent Example 1: 
\begin{center}\begin{minipage}{15cm}\begin{Verbatim}[frame=single]
> f=sin(x);
> f;
sin(x)
> rename(x,y);
> f;
sin(y)
\end{Verbatim}
\end{minipage}\end{center}
\noindent Example 2: 
\begin{center}\begin{minipage}{15cm}\begin{Verbatim}[frame=single]
> a=1;
> f=sin(x);
> rename(x,a);
> a;
a
> f;
sin(a)
\end{Verbatim}
\end{minipage}\end{center}
\noindent Example 3: 
\begin{center}\begin{minipage}{15cm}\begin{Verbatim}[frame=single]
> verbosity=1!;
> f=sin(x);
> rename(y,z);
Warning: the current free variable is named "x" and not "y". Can only rename the
 free variable.
The last command will have no effect.
\end{Verbatim}
\end{minipage}\end{center}
\noindent Example 4: 
\begin{center}\begin{minipage}{15cm}\begin{Verbatim}[frame=single]
> rename(x,y);
> isbound(x);
false
> isbound(y);
true
\end{Verbatim}
\end{minipage}\end{center}
See also: \textbf{isbound} (\ref{labisbound})
