\subsection{concat}
\label{labconcat}
\noindent Name: \textbf{@}\\
concatenates two lists or strings or applies a list as arguments to a procedure\\
\noindent Usage: 
\begin{center}
\emph{L1}\textbf{@}\emph{L2} : (\textsf{list}, \textsf{list}) $\rightarrow$ \textsf{list}
\\ 
\emph{string1}\textbf{@}\emph{string2} : (\textsf{string}, \textsf{string}) $\rightarrow$ \textsf{string}
\\ 
\emph{proc}\textbf{@}\emph{L1} : (\textsf{string}, \textsf{string}) $\rightarrow$ \textsf{string}
\\ 
\end{center}
Parameters: 
\begin{itemize}
\item \emph{L1} and \emph{L2} are two lists.
\item \emph{string1} and \emph{string2} are two strings.
\item \emph{proc} is a procedure.
\end{itemize}
\noindent Description: \begin{itemize}

\item In its first usage form, \textbf{@} concatenates two lists or strings.

\item In its second usage form, \textbf{@} applies the elements of a list as
   arguments to a procedure.
\end{itemize}
\noindent Example 1: 
\begin{center}\begin{minipage}{15cm}\begin{Verbatim}[frame=single]
> [|1,...,3|]@[|7,8,9|];
[|1, 2, 3, 7, 8, 9|]
\end{Verbatim}
\end{minipage}\end{center}
\noindent Example 2: 
\begin{center}\begin{minipage}{15cm}\begin{Verbatim}[frame=single]
> "Hello "@"World!";
Hello World!
\end{Verbatim}
\end{minipage}\end{center}
\noindent Example 3: 
\begin{center}\begin{minipage}{15cm}\begin{Verbatim}[frame=single]
> procedure cool(a,b,c) { 
> write(a,", ", b," and ",c," are cool guys.
> ");
> };
> cool @ [| "Christoph", "Mioara", "Sylvain" |];
Christoph, Mioara and Sylvain are cool guys.
\end{Verbatim}
\end{minipage}\end{center}
See also: \textbf{.:} (\ref{labprepend}), \textbf{:.} (\ref{labappend}), \textbf{procedure} (\ref{labprocedure})
