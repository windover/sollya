\subsection{ concat }
\noindent Name: \textbf{@}\\
concatenates two lists or strings.\\

\noindent Usage: 
\begin{center}
\emph{L1}\textbf{@}\emph{L2} : (\textsf{list}, \textsf{list}) $\rightarrow$ \textsf{list}\\
\emph{string1}\textbf{@}\emph{string2} : (\textsf{string}, \textsf{string}) $\rightarrow$ \textsf{string}\\
\end{center}
Parameters: 
\begin{itemize}
\item \emph{L1} and \emph{L2} are two lists.
\item \emph{string1} and \emph{string2} are two strings.
\end{itemize}
\noindent Description: \begin{itemize}

\item \textbf{@} concatenates two lists or strings.
\end{itemize}
\noindent Example 1: 
\begin{center}\begin{minipage}{15cm}\begin{Verbatim}[frame=single]
> [|1,...,3|]@[|7,8,9|];
[|1, 2, 3, 7, 8, 9|]
\end{Verbatim}
\end{minipage}\end{center}
\noindent Example 2: 
\begin{center}\begin{minipage}{15cm}\begin{Verbatim}[frame=single]
> "Hello "@"World!";
Hello World!
\end{Verbatim}
\end{minipage}\end{center}
See also: \textbf{.:}, \textbf{:.}
