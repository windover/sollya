\subsection{ mid }
\noindent Name: \textbf{mid}\\
gives the middle of an interval.\\

\noindent Usage: 
\begin{center}
\textbf{mid}(\emph{I}) : \textsf{range} $\rightarrow$ \textsf{constant}\\
\textbf{mid}(\emph{x}) : \textsf{constant} $\rightarrow$ \textsf{constant}\\
\end{center}
Parameters: 
\begin{itemize}
\item \emph{I} is an interval.
\item \emph{x} is a real number.
\end{itemize}
\noindent Description: \begin{itemize}

\item Returns the middle of the interval \emph{I}. If the middle is not exactly 
   representable at the current precision, the value is rounded.

\item When called on a real number \emph{x}, \textbf{mid} considers it as an interval formed
   of a single point: [x, x]. In other words, \textbf{mid} behaves like the identity.
\end{itemize}
\noindent Example 1: 
\begin{center}\begin{minipage}{15cm}\begin{Verbatim}[frame=single]
> mid([1;3]);
2
> mid(17);
17
\end{Verbatim}
\end{minipage}\end{center}
See also: \textbf{inf}, \textbf{sup}
